%% Dissertação de Mestrado - EESC-SEL-USP
%% 
%% Nome: Rodrigo de Barros Vimieiro
%% E-mail: rodrigo.vimieiro@gmail.com
%%
%% Universidade de São Paulo - São Carlos
%% Laboratório de Visão Computacional - LAVI
%%
%% Data: 13/11/2018


% ------------------------------------------------------------------------
% ------------------------------------------------------------------------
% eesc: Modelo de Trabalho Acadêmico (tese de doutorado, dissertação de
% mestrado e trabalhos monográficos em geral) em conformidade com 
% ABNT NBR 14724:2011. Esta classe estende as funcionalidades da classe
% abnTeX2 elaborada de forma a adequar os parâmetros exigidos pelas 
% normas USP e do departamento de elétrica da Escola de Engenharia 
% de São Carlos - USP.
% ------------------------------------------------------------------------
% ------------------------------------------------------------------------

% ------------------------------------------------------------------------
% Opções:
% 	tesedr:     Formata documento para tese de doutorado
%	qualidr:    Formata documento para qualificação de doutorado
% 	dissertmst: Formata documento para dissertação de mestrado
% 	qualimst:   Formata documento para qualificação de mestrado
% ------------------------------------------------------------------------
\documentclass[dissertmst]{eesc}

% ---
% PACOTES
% ---

\newcommand{\rv}[1]{{\textcolor[rgb]{.5,.7,.15}{\small[\textbf{Rodrigo}: #1]}}}	%(Rodrigo)
\newcommand{\RV}[2]{{\textcolor[rgb]{.5,.7,.15}{\uwave{#1 }\small[\sout{#2}]}}} %(Rodrigo)

% ---
% Pacotes fundamentais 
% ---
\usepackage{cmap}				% Mapear caracteres especiais no PDF
\usepackage{lmodern}			% Usa a fonte Latin Modern			
\usepackage{makeidx}           	% Cria o indice
\usepackage{hyperref}  			% Controla a formação do índice
\usepackage{lastpage}			% Usado pela Ficha catalográfica
\usepackage{indentfirst}		% Indenta o primeiro parágrafo de cada seção.
\usepackage{nomencl} 			% Lista de simbolos
\usepackage{graphicx}			% Inclusão de gráficos
\usepackage{subfig}				% Inclusão de figura side-by-side (Rodrigo)
\usepackage{amsmath}			% Formulas matematicas (Rodrigo)
\usepackage{gensymb}			% Para colocar \degree (Rodrigo)
\usepackage[algo2e, portuguese, boxruled]{algorithm2e}	% Para colocar pseudocódigo (Rodrigo)
\newcommand{\overbar}[1]{\mkern 1.5mu\overline{\mkern-1.5mu#1\mkern-1.5mu}\mkern 1.5mu} 	% Rodrigo: Overbar 
\usepackage{amssymb}  % Rodrigo
\usepackage{multirow}  % Rodrigo

% ---

% ---
% Pacotes adicionais, usados apenas no âmbito do Modelo eesc
% ---
\usepackage[printonlyused]{acronym}
\usepackage[table]{xcolor}
% ---


% ---
% Informações de dados para CAPA e FOLHA DE ROSTO
% ---
%
% Título:
%	1. Título em português
%	2. Título em inglês
\titulo{Ferramenta para reconstrução de imagens de tomossíntese mamária e sua aplicação na análise do ruído em imagens reconstruídas}{Digital breast tomosynthesis reconstruction toolbox and its application on the noise analysis in the reconstructed slices}
%
% Autor:
%	1. Nome completo do autor
%	2. Formato de nome para bibliografia
\autor{Rodrigo de Barros Vimieiro}{de Barros Vimieiro, Rodrigo}
%
% Cidade
\local{São Carlos}
% Ano de defesa
\data{2019}
% Área de concentração da pesquisa
\areaconcentracao{Processamento de Sinais e Instrumentação}
% Nome do orientador
\orientador{Marcelo Andrade da Costa Vieira}
% Nome do coorientador
%\coorientador{}
% ---

% ---
% compila o indice
% ---
\makeindex
% ---

% ---
% Compila a lista de abreviaturas e siglas
% ---
\makenomenclature
% ---

% ---
% Inserir ficha catalográfica
%
% Caso o comando \inserirfichacatalografica seja definido, a ficha catalográfica
% será inserida atrás da folha de rosto. Caso contrário a página será deixada em
% branco.
%
% CUIDADO: Esta opção deve ser preenchida antes do comando \maketitle
% ---
\inserirfichacatalografica{docs/fichaCatalografica_Rodrigo.pdf}
% ---

% ---
% Inserir folha de aprovação
%
% Caso o comando \inserirfolhaaprovacao seja definido, a a folha de aprovação
% será inserida. Além disso, conforme Resolução CoPGr 5890, as informações 
% de rodapé são inseridas apropriadamente na folha de rosto.
%
% CUIDADO: Esta opção deve ser preenchida antes do comando \maketitle
% ---
\inserirfolhaaprovacao{docs/folhaAprovacao.pdf}
% ---

% ----
% Início do documento
% ----

\begin{document}

% ----------------------------------------------------------
% ELEMENTOS PRÉ-TEXTUAIS
% ----------------------------------------------------------
\pretextual

% ---
% Insere Capa, Folha de rosto, Ficha catalográfica (se inserida)
% e folha de aprovação (se inserida).
% ---
\maketitle

%%%%%%%%%%%%%%%%%%%%%%%%%%%%%%%%%%%%%%%%%%%%%%%%%%%%%%%%%%%%%%%%%%%%%%%%%%%%%%%%%%%%%%%%%%%%%%%%%%%%%%%%%%%%%%%%%%%%%%%%%%%%%%													  Dedicatória															%
%%%%%%%%%%%%%%%%%%%%%%%%%%%%%%%%%%%%%%%%%%%%%%%%%%%%%%%%%%%%%%%%%%%%%%%%%%%%%%%%%%%%%%%%%%%%%%%%%%%%%%%%%%%%%%%%%%%%%%%%%%%%%
\imprimirdedicatoria{Este trabalho é dedicado a Deus\\
   e aos meus pais, Erika e José Ronaldo.}

%%%%%%%%%%%%%%%%%%%%%%%%%%%%%%%%%%%%%%%%%%%%%%%%%%%%%%%%%%%%%%%%%%%%%%%%%%%%%%%%%%%%%%%%%%%%%%%%%%%%%%%%%%%%%%%%%%%%%%%%%%%%%%													 Agradecimentos															%
%%%%%%%%%%%%%%%%%%%%%%%%%%%%%%%%%%%%%%%%%%%%%%%%%%%%%%%%%%%%%%%%%%%%%%%%%%%%%%%%%%%%%%%%%%%%%%%%%%%%%%%%%%%%%%%%%%%%%%%%%%%%%
\imprimiragradecimentos{

Julgo que esse capítulo seja o mais importante do trabalho.  Não há como descrever em palavras a gratidão que eu tenho por todas as pessoas que me auxiliaram, não só para a elaboração da pesquisa, mas também o apoio pessoal no geral. Nesse mundo ninguém faz nada sozinho e tudo gira em torno de cooperações; sendo assim, eu só tenho a agradecer. 

Primeiramente a Deus e ao mestre Jesus pela oportunidade concedida aos meus estudos e a saúde que disponho para enfrentar as lutas do dia a dia. 

Aos meus pais Erika e José Ronaldo e ao meu irmão Júnior pelo apoio e presença sempre constante em minha vida. 

A todos os meus familiares, sem exceção, que essa singela citação possa demonstrar minha enorme gratidão a cada um. 

A Débora, por todo amor, força e companheirismo nessa jornada durante a qual estive em São Carlos. 

A todos da república, onde vivi o período da pesquisa, pelo companheirismo na convivência do dia a dia. 

Aos amigos da Casa do Caminho pela força e engrandecimento espiritual. 

Em especial, ao professor Marcelo por ter me acolhido de maneira paternal no laboratório e ter tido toda a paciência ao transmitir-me os conhecimentos. Grande parte desse trabalho só foi possível devido a excepcional orientação que tive o privilégio de ter. 

A todos os membros e ex-membros do laboratório, que ao compartilhar as informações, enriquecem o trabalho de todos. O meu agradecimento a cada um pela enorme contribuição tanto pessoal quanto profissional. 

Ao pessoal do Hospital de Amor, de Barretos, por todo o apoio na pesquisa, em especial ao Renato por toda paciência comigo. 

Aos órgãos públicos, especificamente o Conselho Nacional de Desenvolvimento Científico e Tecnológico (Cnpq), a Fundação de Amparo à Pesquisa do Estado de São Paulo (Fapesp) e a Universidade de São Paulo (USP) pelo apoio financeiro. 

Por fim, que Deus possa nos fortalecer e amparar, para que cada vez mais possamos contribuir para a obra divina como um todo.

}
%%%%%%%%%%%%%%%%%%%%%%%%%%%%%%%%%%%%%%%%%%%%%%%%%%%%%%%%%%%%%%%%%%%%%%%%%%%%%%%%%%%%%%%%%%%%%%%%%%%%%%%%%%%%%%%%%%%%%%%%%%%%%%													   Epígrafe																%
%%%%%%%%%%%%%%%%%%%%%%%%%%%%%%%%%%%%%%%%%%%%%%%%%%%%%%%%%%%%%%%%%%%%%%%%%%%%%%%%%%%%%%%%%%%%%%%%%%%%%%%%%%%%%%%%%%%%%%%%%%%%%
\imprimirepigrafe{
``Além do trabalho-obrigação que nos remunera de pronto, é necessário nos
atenhamos ao prazer de servir. (...)\\
O trabalho-ação transforma o ambiente.\\
O trabalho-serviço, transforma o homem. (...)\\
Quando começamos a ajudar o próximo, sem aguilhões, matriculamo-nos no acrisolamento da própria alma, entrando em sintonia com a Vida Abundante''.\\
(Francisco Cândido Xavier ditado pelo espírito Emmanuel, Cap. 7, Trabalho, Pensamento e Vida.)
}
%%%%%%%%%%%%%%%%%%%%%%%%%%%%%%%%%%%%%%%%%%%%%%%%%%%%%%%%%%%%%%%%%%%%%%%%%%%%%%%%%%%%%%%%%%%%%%%%%%%%%%%%%%%%%%%%%%%%%%%%%%%%%%												  RESUMO e ABSTRACT															%
%%%%%%%%%%%%%%%%%%%%%%%%%%%%%%%%%%%%%%%%%%%%%%%%%%%%%%%%%%%%%%%%%%%%%%%%%%%%%%%%%%%%%%%%%%%%%%%%%%%%%%%%%%%%%%%%%%%%%%%%%%%%%
% Resumo em português
\begin{resumo}{tomossíntese digital mamária, reconstrução, análise de ruído}
A tomossíntese digital mamária (\textit{Digital Breast Tomosynthesis} - \acs{DBT}) é um exame radiográfico utilizado para o rastreamento do câncer de mama, que busca superar a limitação da sobreposição de tecidos existente na mamografia digital \acs{2D}. Nessa técnica são adquiridas projeções radiográficas em diferentes ângulos, que são processadas para a reconstrução do volume da mama. Um grande desafio é a elaboração dos algoritmos para a reconstrução tomográfica, visto que há um número limitado de projeções adquiridas com baixas doses de radiação, abrangendo uma estreita faixa de ângulo. Outro fator importante é o ruído presente nas imagens, que pode impactar o diagnóstico do câncer pelos radiologistas. Esse trabalho tem como objetivo apresentar uma ferramenta de reconstrução de imagens para \acs{DBT} e fazer um estudo do comportamento do sinal e do ruído nas imagens reconstruídas. Os métodos analíticos de retroprojeção simples e  filtrada, bem como os interativos de máxima verossimilhança e algébricos foram avaliados. A validação dos algoritmos de reconstrução foi feita por meio de métricas objetivas e as imagens reconstruídas foram comparadas com imagens obtidas a partir de um software de reconstrução para \acs{DBT} desenvolvido pelo \ac{FDA}. A partir das análises objetivas e visuais, demonstrou-se o potencial da ferramenta desenvolvida nesse trabalho. O ruído pós-reconstrução foi investigado através da aquisição de imagens de \textit{phantoms} homogêneos, utilizando dois sistemas comerciais de \acs{DBT}. As curvas de valor médio de \textit{pixel}, a variância do ruído e a relação sinal-ruído seguiram o mesmo padrão já demonstrado para as projeções. A análise do espectro de potência do ruído demonstrou que o processo de reconstrução correlaciona o ruído para ambos os equipamentos.

\end{resumo}

\begin{abstract}{digital breast tomosynthesis, reconstruction, noise analysis}
	
\acf{DBT} is a radiographic examination used for breast cancer screening, which seeks to overcome the tissue superposition in 2D digital mammography. In this technique, radiographic projections are acquired at different angles, which are processed for the reconstruction of the breast volume. A major challenge is the elaboration of algorithms for tomographic reconstruction since there are a limited number of projections acquired with low doses of radiation, covering a narrow-angle range. Another important factor is the noise present in this modality that can impact the diagnosis of tumors by radiologists. This work aims to present an image reconstruction toolbox for \acs{DBT} and study the signal and noise behavior in the reconstructed slices. The analytical methods of simple and filtered back-projection, as well as the maximum likelihood and algebraic iterative methods were evaluated. The validation of the reconstruction algorithms was done by objective metrics and the reconstructed images were compared with the images obtained from a reconstruction software for \acs{DBT} developed by the \acf{FDA}. Through the objective and visual analysis, the potential of the toolbox developed in this work was demonstrated. The noise after reconstruction was investigated by means of the acquisition of homogeneous phantom images, using two commercial \acs{DBT} systems. The mean pixel value, the noise variance and the signal-to-noise ratio follow the same curve shape already shown for the projection domain. The analysis of noise power spectrum demonstrated that the process of reconstruction correlates the noise for both systems used.
	

\end{abstract}


%%%%%%%%%%%%%%%%%%%%%%%%%%%%%%%%%%%%%%%%%%%%%%%%%%%%%%%%%%%%%%%%%%%%%%%%%%%%%%%%%%%%%%%%%%%%%%%%%%%%%%%%%%%%%%%%%%%%%%%%%%%%%%												  	DISSERTAÇÃO																%
%%%%%%%%%%%%%%%%%%%%%%%%%%%%%%%%%%%%%%%%%%%%%%%%%%%%%%%%%%%%%%%%%%%%%%%%%%%%%%%%%%%%%%%%%%%%%%%%%%%%%%%%%%%%%%%%%%%%%%%%%%%%%


% ---
% inserir lista de ilustrações
% ---
\listailustracoes
% ---

% ---
% inserir lista de tabelas
% ---
\listatabelas
% ---

\listaalgoritmos

% ---
% inserir lista de abreviaturas e siglas
% ---
\listasiglas{abrev/Abreviaturas}
% ---

% ---
% inserir o sumario
% ---
\sumario
% ---

% ----------------------------------------------------------
% ELEMENTOS TEXTUAIS
% ----------------------------------------------------------
\mainmatter

% ----------------------------------------------------------
% Introdução
% ----------------------------------------------------------

\chapter[Introdução]{Introdução}\label{Introdução}


%%%%%%%%%%%%%%%%%%%%%%%%%%%%%%%%%%%%%%%%%%%%%%%%%%%%%%%%%%%%%%%%%%%%%%%%%%%%%%%%%%%%%%%%%%%%%%%%%%%%%%%%%%%%%%%%%%%%%%%%%%%%%%												 	 Motivação    															%
%%%%%%%%%%%%%%%%%%%%%%%%%%%%%%%%%%%%%%%%%%%%%%%%%%%%%%%%%%%%%%%%%%%%%%%%%%%%%%%%%%%%%%%%%%%%%%%%%%%%%%%%%%%%%%%%%%%%%%%%%%%%%
\section{Motivação}
Segundo a \ac{OMS}, o câncer é uma preocupação crescente da saúde pública no âmbito mundial e requer um aumento de atenção, priorização e financiamento. Ainda, segundo a organização, o câncer é a segunda maior causa de morte em todo o mundo, com crescimento de novos casos de 14,1 milhões em 2012 para 21,6 milhões projetados em 2030. A grande preocupação está nos países em desenvolvimento, nos quais esses números crescem mais rapidamente e o gasto estimado anualmente é de 1,16 trilhão de dólares \cite{oms}.

De acordo com os dados do \ac{INCA}, no Brasil, a estimativa indica a ocorrência de 600 mil novos casos para o biênio 2018-2019. Em mulheres, o tipo mais frequente será o de mama\footnote{ Com exceção do câncer de pele não melanoma.} com previsão de 60 mil casos, sendo esse número 21\% do total para o sexo feminino \cite{inca}.

Grandes esforços são necessários para minimizar essas ocorrências. Dentre as ações recomendadas para a diminuição dos casos em geral, está o diagnóstico precoce, o qual deve ser acessível à toda população. O fornecimento de capacitação para as forças de trabalho e o aperfeiçoamento de dados para o auxílio de tomada de decisão é também de suma importância \cite{oms}. 

O câncer de mama possui uma maior ocorrência na população feminina, excetuando-se os casos de câncer de pele não melanoma, conforme as estatísticas do \ac{INCA}. No ano de 2018 foram estimados aproximadamente 627 mil óbitos no mundo, representando a mais elevada causa de morte por câncer em mulheres \cite{oms}.

Dentre as estratégias para o controle e detecção estão o rastreamento e o diagnóstico precoce. A mamografia é o método preconizado para o rastreamento, porém, outras técnicas são utilizadas como: o autoexame e o exame clínico das mamas, o uso de equipamentos de ressonância nuclear magnética, ultrassonografia, termografia e tomossíntese \cite{inca}. A utilização da mamografia como exame para o rastreamento reduz em aproximadamente 20\% a taxa de mortalidade. Essa técnica consiste em examinar mulheres em determinada faixa etária para identificar o câncer de mama antes mesmo de quaisquer sintomas. O exame com fins de rastreamento é indicado para mulheres na faixa etária de 50 a 69 anos em locais com boas condições no sistema de saúde \cite{oms}.

Apesar de ser amplamente usada para rastreamento, o exame de mamografia possui algumas limitações. A maior e mais conhecida, citada na literatura, é a sobreposição de tecidos, que obscurece possíveis lesões, dentre as quais podem haver aquelas malignas, induzindo um diagnóstico errado \cite{vedantham2015digital}. 

A tomossíntese digital da mama (\textit{Digital Breast Tomosynthesis} - \acs{DBT}) é uma técnica tomográfica de ângulo limitado, desenvolvida para minimizar os problemas relacionados à sobreposição de tecidos da mamografia \acs{2D}. Nesse método, múltiplas projeções de raios X da mama são adquiridas em diferentes ângulos, enquanto o tubo se move em uma trajetória fixa pré-definida. Ao final do exame, as imagens radiográficas são processadas para a reconstrução pseudo-\acs{3D} do volume da mama \cite{vedantham2015digital}.  

Diversos métodos de reconstrução de imagem vêm sendo estudados e comparados \cite{wu2004comparison,zhang2006comparative}. No entanto, a elaboração desses algoritmos para tomossíntese mamária é um grande desafio, uma vez que há um limitado número de projeções que são adquiridas com baixas doses de radiação \cite{wu2004comparison}. A geometria de aquisição varia para cada equipamento de \acs{DBT} \cite{vedantham2015digital}. Dessa forma, não há um consenso sobre qual é o número de projeções ideal, ou o melhor ângulo de aquisição, ou ainda qual o algoritmo de reconstrução que deve ser utilizado para o caso da \acs{DBT} \cite{sechopoulos2009optimization}. As técnicas de reconstrução são comumente divididas, de uma maneira geral, em duas categorias: os métodos analíticos e os iterativos. Dentre estes, pode-se citar os algoritmos de retroprojeção (\textit{Back-projection} - \acs{BP}), retroprojeção filtrada (\textit{Filtered Back-projection} - \acs{FBP}) e os iterativos, sendo o método de \acs{FBP} o mais comum em tomossíntese \cite{michell2018role}.  

Algoritmos de reconstrução iterativa (\textit{Iterative Reconstruction} - \acs{IR}) demandam um alto custo computacional. Devido a esse fato, historicamente, os equipamentos comerciais de tomografia por raios X não utilizam esses métodos. Todavia com o avanço do poder computacional, os métodos iterativos vêm sendo  amplamente utilizados por diversos grupos de pesquisas e fabricantes \cite{wu2003tomographic}. Sistemas comerciais de tomossíntese mamária como, por exemplo os fabricados pela Siemens\footnote{\url{www.healthcare.siemens.com}} e pela Hologic\footnote{\url{www.hologic.com}} utilizam \acs{FBP}, no entanto, algoritmos iterativos já são utilizados pelos equipamentos de \acs{DBT} da \ac{GE}\footnote{\url{www.gehealthcare.com}}. 

Dentre os objetivos dos diversos métodos de reconstrução, a redução de artefatos nas imagens reconstruídas pode ser destacada \cite{hu2008image}. Devido a utilização de um feixe cônico e uma baixa amostragem no domínio da frequência dos sistemas de tomossíntese, os algoritmos de \acs{BP} e \acs{FBP}, quando aplicados, introduzem artefatos de objetos fora do foco em imagens que estão em foco no processo de reconstrução \cite[]{levakhina2013weighted, borges2017metal}. Já os métodos iterativos são capazes de agregar no processo de reconstrução o modelamento físico do sistema em geral, além da possibilidade de incluir restrições à convergência do método a partir de conhecimentos \textit{a priori} \cite{xu2015statistical,levakhina2013weighted}.

Esses modelamentos e restrições vêm sendo incorporados nos processos iterativos de tomossíntese com as recentes pesquisas a fim de aprimorar a qualidade das imagens. Informações \textit{a priori} como a similaridade dos \textit{pixels} e a morfologia da imagem foram incorporadas na reconstrução no trabalho de \citeonline{xu2015statistical}. A redução de artefatos de alta atenuação foi incorporada em um algoritmo iterativo, demonstrada no trabalho de \citeonline[]{levakhina2013weighted}. A correlação do ruído e o respectivo borramento nos detectores indiretos foram estudados por \citeonline[]{zheng2018detector}. Foram dados então os primeiros passos para a criação das reconstruções iterativas baseadas em modelamentos (\textit{Model-Based Iterative Reconstruction }- \acs{MBIR}) aplicadas à \acs{DBT}, que já são extensivamente estudadas para o equipamento de tomografia computadorizada (\textit{Computed Tomography} - \acs{CT}).    

De fato, o ruído tem uma importância muito grande em relação a qualidade das imagens de mamografia que são fornecidas para os radiologistas, tendo em vista a detecção do câncer de mama \cite{haus2000screen,huda2003experimental,ruschin2007dose, saunders2007does, samei2007digital, mackenzie2016relationship}. É extremamente relevante conhecer as fontes dessas perturbações para então modelá-las matematicamente e incorporá-las nos métodos computacionais para os devidos propósitos. Remoção de ruído, redução da dose de radiação e função de restrição para algoritmos de reconstrução iterativa são algumas das aplicações nas quais o entendimento dessas perturbações é importante \cite{wu2012dose, romualdo2013mammographic,borges2016method,borges2017pipeline,borges2017method,mackenzie2017characterisation,zheng2018detector}.

%%%%%%%%%%%%%%%%%%%%%%%%%%%%%%%%%%%%%%%%%%%%%%%%%%%%%%%%%%%%%%%%%%%%%%%%%%%%%%%%%%%%%%%%%%%%%%%%%%%%%%%%%%%%%%%%%%%%%%%%%%%%%%												  	Objetivos	    														%
%%%%%%%%%%%%%%%%%%%%%%%%%%%%%%%%%%%%%%%%%%%%%%%%%%%%%%%%%%%%%%%%%%%%%%%%%%%%%%%%%%%%%%%%%%%%%%%%%%%%%%%%%%%%%%%%%%%%%%%%%%%%%

\section{Objetivos}

Esse trabalho tem como objetivo apresentar uma ferramenta de reconstrução de imagens para a tomossíntese da mama e fazer um estudo do comportamento do sinal e do ruído nas imagens reconstruídas.

Especificamente, o desenvolvimento dessa ferramenta visa ampliar as pesquisas em reconstrução tomográfica da mama, visto que a mesma é disponibilizada de maneira \textit{online} e gratuita. É desejável que o \textit{software} seja capaz de implementar os principais algoritmos de reconstrução de imagens para \acs{DBT}, bem como qualquer geometria de aquisição. Tanto os métodos analíticos quanto os iterativos são incluídos no \textit{software} proposto.

Em trabalhos anteriores, nosso grupo propôs um modelo de função afim para descrever a variância do ruído nas projeções de \acs{DBT}. Neste modelo, o ruído eletrônico é descrito por uma distribuição Gaussiana aditiva e o ruído quântico é descrito por um modelo Gaussiano dependente do sinal com ganho quântico dependente espacialmente \cite{borges2017method,borges2018restoration}. 

Agora, uns dos objetivos desse trabalho é realizar um estudo preliminar do ruído em fatias reconstruídas de \acs{DBT}, a fim de avaliar a adequação desse modelo para imagens pós-reconstrução. Esse modelamento tem a finalidade de fornecer uma base matemática para aplicações que envolvam a remoção de ruído em ambos os domínios, redução de dose de radiação e restrições para algoritmos iterativos de reconstrução.

%%%%%%%%%%%%%%%%%%%%%%%%%%%%%%%%%%%%%%%%%%%%%%%%%%%%%%%%%%%%%%%%%%%%%%%%%%%%%%%%%%%%%%%%%%%%%%%%%%%%%%%%%%%%%%%%%%%%%%%%%%%%%%										     Organização da Monografia														%
%%%%%%%%%%%%%%%%%%%%%%%%%%%%%%%%%%%%%%%%%%%%%%%%%%%%%%%%%%%%%%%%%%%%%%%%%%%%%%%%%%%%%%%%%%%%%%%%%%%%%%%%%%%%%%%%%%%%%%%%%%%%%

\section{Organização da Monografia}

O trabalho é dividido em seu âmbito geral em duas partes, sendo a primeira a respeito do desenvolvimento e validação do \textit{software} de reconstrução e a segunda parte relacionada com a análise e medição do sinal e do ruído em imagens reconstruídas de tomossíntese.

Mais especificamente, o documento é divido em 9 capítulos que visam demostrar todo o trabalho desenvolvido, bem como apresentar os conceitos teóricos fundamentais. 

Primeiramente, no \autoref{Capitulo2}, é feita uma revisão do conteúdo de mamografia digital \acs{2D} e da técnica de tomossíntese da mama. São abordados os fatos históricos, bem como o funcionamento geral de ambas modalidades de exame. São detalhados a física por trás dos equipamentos de \acs{DBT} assim como seus parâmetros geométricos.

Já no \autoref{Capitulo3} é exposto a teoria básica sobre reconstrução, demonstrando os princípios para \acs{2D} e para \acs{3D}. São discutidos os métodos de reconstrução analítica e iterativa. 

No \autoref{Capitulo4}, são expostos os conceitos de ruído em imagens digitais, bem como em exames de mamografia. Formulações matemáticas são feitas para a modelagem do ruído nas projeções e também pós-reconstrução.

Os materiais e métodos utilizados na confecção das duas partes mencionadas são apresentados no \autoref{Capitulo5}  do documento. 

Os resultados obtidos a partir dos experimentos e das avaliações realizadas, bem como as respectivas discussões são descritas no \autoref{Capitulo6}.

Então, no \autoref{Capitulo7}, é feita a conclusão do trabalho e no \autoref{Capitulo8} são apresentados os trabalhos futuros. 

Por fim, no \autoref{Capitulo9} são apontadas as publicações geradas a partir do desenvolvimento desse trabalho.



  

  

\chapter[Mamografia e Tomossíntese]{Mamografia e Tomossíntese}\label{Capitulo2}

%%%%%%%%%%%%%%%%%%%%%%%%%%%%%%%%%%%%%%%%%%%%%%%%%%%%%%%%%%%%%%%%%%%%%%%%%%%%%%%%%%%%%%%%%%%%%%%%%%%%%%%%%%%%%%%%%%%%%%%%%%%%%%											Câncer de Mama e Mamografia														%
%%%%%%%%%%%%%%%%%%%%%%%%%%%%%%%%%%%%%%%%%%%%%%%%%%%%%%%%%%%%%%%%%%%%%%%%%%%%%%%%%%%%%%%%%%%%%%%%%%%%%%%%%%%%%%%%%%%%%%%%%%%%%
\section{Câncer de Mama e Mamografia}

Segundo o \citeonline{inca}, o câncer de mama é o mais frequente em mulheres no mundo todo e também no Brasil, salvo os tipos de pele não melanoma. Ainda, conforme as estatísticas do instituto, para os anos de 2018 e 2019 a previsão é de 60 mil novos casos, representando para as mulheres 21\% do total de casos previstos para estas. 

É estimado que, 28\% dos casos podem ser evitados com práticas de alimentação saudável, atividade física  e adequação do peso corporal. Essas precauções podem ser tomadas a fim de diminuir os fatores de risco da doença que são: o excesso de peso corporal, particularmente após a menopausa, o consumo de bebidas alcoólicas, a terapia de reposição hormonal e a exposição à radiação ionizante \cite{inca}.

O Ministério da Saúde, em conformidade com a \ac{OMS}, recomenda que a mamografia de rastreamento seja feita periódicamente por mulheres na faixa etária entre 50 e 69 anos. Esse exame auxilia na descoberta precoce do câncer, proporcionando um começo de tratamento rápido, diminuído assim as chances de óbito. Estima-se que o uso desse equipamento para o rastreamento reduz em aproximadamente 20\% a taxa de mortalidade da doença \cite{oms}, porém, a mamografia quando aplicada a toda uma população está relacionada a riscos como: resultados incorretos e a exposição aos raios X que podem induzir o câncer na população a longo prazo \cite{yaffe2011risk,inca}. 

%As seções seguintes têm como objetivo detalhar o funcionamento dos equipamentos usados atualmente para rastreamento tanto no Brasil quanto no resto do mundo. Esse capítulo é utilizado como base para o entendimento da evolução dos equipamentos e também para a compreensão dos métodos de formação das imagens detalhadas nas próximas seções.    

%%%%%%%%%%%%%%%%%%%%%%%%%%%%%%%%%%%%%%%%%%%%%%%%%%%%%%%%%%%%%%%%%%%%%%%%%%%%%%%%%%%%%%%%%%%%%%%%%%%%%%%%%%%%%%%%%%%%%%%%%%%%%%													Mamografia																%
%%%%%%%%%%%%%%%%%%%%%%%%%%%%%%%%%%%%%%%%%%%%%%%%%%%%%%%%%%%%%%%%%%%%%%%%%%%%%%%%%%%%%%%%%%%%%%%%%%%%%%%%%%%%%%%%%%%%%%%%%%%%%
\subsection{Mamografia}

O exame de mamografia tem como finalidade proporcionar uma imagem radiográfica da mama, com o intuito de que os profissionais da saúde possam identificar possíveis lesões em meio aos tecidos e estruturas que compõe a mesma (Figura \ref{fig:imgCap2EstruturaMama}), antes mesmo de qualquer sintoma. 

A mama feminina, em sua anatomia é constituída por: lobos, que são glândulas produtoras de leite; ductos, que escoam o leite materno para o mamilo; tecido adiposo e conjuntivo, que revestem os ductos e lobos, dentre outras estruturas. O câncer se caracteriza quando células na mama começam a crescer de maneira descontrolada formando tumores malignos que se espalham por tecidos próximos ou para outras localidades do corpo. Esses tumores podem começar em diversas partes da mama, no entanto, na maioria dos casos, os tumores começam nos ductos ou nas glândulas \cite{americancancersociety2017}.      

\begin{figure}[H]
	\caption{Tecidos e estruturas que compõem uma mama normal.}
	\begin{center}
		\includegraphics[scale=0.7]{imgs/cap2/EstruturaMama.png}
	\end{center}
	\legend{Fonte: Adaptado de \citeonline{americancancersociety2017}.}
	\label{fig:imgCap2EstruturaMama}
\end{figure}


Com o auxílio do exame de mamografia, o câncer de mama pode ser detectado através de quatro indicadores, segundo \citeonline{boone2000handbook}:

\begin{enumerate}
	\item morfologia característica de uma massa tumoral,
	\item presença de depósitos minerais em forma de partículas, denominadas microcalcificações,
	\item distorção arquitetural dos padrões teciduais normais,
	\item assimetria entre regiões correspondentes da imagem esquerda e direita.
\end{enumerate}

A Figura \ref{fig:imgCap2ExameMamografia} ilustra dois exames de mamografia da mesma mama em orientação \ac{CC} e \ac{MLO}, onde são evidenciados os sinais de um possível câncer de mama.

\begin{figure}[H]
	\centering
	
	\caption{Ilustração de dois exames de mamografia \acs{2D} da mama direita na orientação (a) \acs{CC} e (b) \acs{MLO}, na qual as setas em vermelho evidenciam agrupamentos de microcalcificações e nódulos .}
	
	\subfloat[]{\includegraphics[scale=0.5, clip, trim=13cm 0cm 12.2cm 0cm]{imgs/cap2/FFDM_Cli_CC} \label{fig:imgCap2ExameMamografiaCC}}	
	\hfil
	\subfloat[]{\includegraphics[scale=0.5, clip, trim=12.3cm 0cm 12.2cm 0cm]{imgs/cap2/FFDM_Cli_MLO}\label{fig:imgCap2ExameMamografiaMLO}}
	
	\legend{Fonte: do autor, 2019.}
	\label{fig:imgCap2ExameMamografia}
\end{figure}

Os primeiros registros da utilização de radiografia aplicada à mama foram feitos em 1913 com Albert Salomon. Ele demonstrou o espalhamento de tumores através de imagens radiográficas de mamas extraídas, tento suas pesquisas interrompidas em 1933 devido a segunda guerra mundial. Já em 1930, Stafford L. Warren registrou o uso  \textit{in vivo} da mamografia em 119 pacientes. O autor ressaltou em seu trabalho a exatidão das opiniões baseadas em diagnósticos através do uso da mamografia, em contrapartida com opiniões pré-operatórias sem a utilização do exame. Apesar dos registros de seu uso desde o começo do século XX, a mamografia como equipamento dedicado para o rastreamento, não desapontou até os anos de 1960, quando em 1965, Charles Gros em parceria com a empresa Compagnie Générale de Radiologie desenvolveu, na França, o primeiro equipamento restrito ao exame mamografia, ilustrado pelo Figura \ref{fig:imgCap2MamografiaPrimeiroEq} \cite{bassett1988evolution,gold1990highlights}. 

\begin{figure}[H]
	\caption{Primeiro equipamento dedicado à mamografia.}
	\begin{center}
		\includegraphics[scale=0.95]{imgs/cap2/MamografiaPrimeiroEq}
	\end{center}
	\legend{Fonte: \citeonline[p. 12]{gold1990highlights}.}
	\label{fig:imgCap2MamografiaPrimeiroEq}
\end{figure}

A mamografia por filme fotográfico foi amplamente utilizada por inúmeros anos e considerada por muito tempo um equipamento preconizado para o rastreamento do câncer de mama, porém foi sendo substituída devido as suas limitações técnicas como: a curva característica de contraste limitada, a presença de granularidade no filme, a impossibilidade de pós processamento e a difícil detecção de lesões em tecidos moles na presença de tecido glandular denso \cite{karellas2008breast,lewin2001comparison}. 

Para superar essas limitações existentes, foi desenvolvida a mamografia digital de campo total (\textit{Full Field Digital Mammography} - \acs{FFDM})  \cite{nishikawa1987scanned,yaffe1988development}. Esse equipamento possibilita a captura de uma imagem digital através de múltiplos sensores, além de poder armazenar, transmitir e mostrar os exames através da tela de um computador. Com essa evolução houve uma maior precisão em diagnósticos para mulheres na faixa etária de 49 anos ou menos e para outras com mamas densas \cite{vedantham2015digital}. A Figura \ref{fig:imgCap2EsquematicoMamografia} ilustra o esquemático de um mamógrafo e detalha suas partes específicas. 

\begin{figure}[H]
	\caption{Esquemático geral de um equipamento de mamografia atual.}
	\begin{center}
		\includegraphics[scale=0.8]{imgs/cap2/FFDMGeometry}
	\end{center}
	\legend{Fonte: \citeonline[]{guerrero2018}.}
	\label{fig:imgCap2EsquematicoMamografia}
\end{figure}

Apesar de sua grande evolução e ampla utilização, o equipamento de mamografia digital impõe limitações físicas para o diagnóstico dos exames, podendo ser destacada a sobreposição de tecidos \cite{vedantham2015digital}. A mama por si só é constituída de uma estrutura volumétrica em \acs{3D}, mas o equipamento proporciona uma imagem bidimensional, como ilustrado pela Figura \ref{fig:imgCap2ExameMamografia}. Por esse motivo, há a sobreposição de estruturas (Figura \ref{fig:imgCap2MamografiaSobreposicao}), fazendo com que tecidos normais (quadrado vermelho) venham a obscurecer lesões malignas (círculo verde), levando o exame à uma menor sensibilidade na detecção de estruturas que podem ser um indicativo de câncer, altas taxas de falsos-positivos e por fim uma alta taxa de \textit{recall} \footnote{Palavra da língua inglesa que tem como significado, no contexto, de chamar a paciente para realizar um novo exame devido a alguma suspeita ao realizar o diagnóstico.} \cite{roth2014digital}. 

Com o propósito de contornar esses problemas foi desenvolvido o equipamento de \acs{DBT} que é abordado com mais detalhes na próxima seção. 

\begin{figure}[H]
	\caption{Ilustração da sobreposição de tecidos.}
	\begin{center}
		\includegraphics[scale=0.8, clip, trim=13cm 1.5cm 10cm 2.5cm]{imgs/cap2/FFDM}
	\end{center}
	\legend{Fonte: do autor, 2019.}
	\label{fig:imgCap2MamografiaSobreposicao}
\end{figure}    

%%%%%%%%%%%%%%%%%%%%%%%%%%%%%%%%%%%%%%%%%%%%%%%%%%%%%%%%%%%%%%%%%%%%%%%%%%%%%%%%%%%%%%%%%%%%%%%%%%%%%%%%%%%%%%%%%%%%%%%%%%%%%%											Tomossíntese Digital Mamária													%
%%%%%%%%%%%%%%%%%%%%%%%%%%%%%%%%%%%%%%%%%%%%%%%%%%%%%%%%%%%%%%%%%%%%%%%%%%%%%%%%%%%%%%%%%%%%%%%%%%%%%%%%%%%%%%%%%%%%%%%%%%%%%
\section{Tomossíntese Digital Mamária}     

\subsection{Histórico e Atualidade}

Uma radiografia analógica ou até mesmo a digital, é a representação de um corpo anatômico em \acs{3D} por uma simples projeção \acs{2D}. Ela traz consigo a sobreposição de órgãos, tecidos e outras formações do corpo humano, tornando impossível a localização precisa das estruturas internas \cite{levakhina2014three}. 

Essa preocupação era eminente logo após a descoberta dos raios X por Wilhelm Röntgen, em 1895, onde pesquisadores da época já buscavam soluções para a reconstrução volumétrica dos objetos. O conceito de impressão tridimensional de uma cena, todavia, foi introduzido pelo princípio da estereoscopia por Charles Wheatstone, em 1838, muito antes do descobrimento do raio X. Já no começo do século XX, inúmeros cientistas ao redor do mundo relataram trabalhos e pesquisas com o intuito da reconstrução de estruturas internas através dos raios X \cite{dobbins2003digital,levakhina2014three}. 

\begin{figure}[H]
	\caption{Princípio da Tomografia Convencional.}
	\begin{center}
		\includegraphics[scale=0.8, clip, trim=12cm 6.2cm 12cm 1.8cm]{imgs/cap2/TomographyPriciple}
	\end{center}
	\legend{Fonte: do autor, 2019.}
	\label{fig:imgCap2TomographyPriciple}
\end{figure}

No ano de 1921, A. E. M. Bocage descreveu o primeiro sistema tomográfico convencional, onde o tubo de raios X e o filme moviam-se linearmente em direções opostas ao longo do paciente (Figura \ref{fig:imgCap2TomographyPriciple}) com o propósito de gerar um plano em foco do objeto e desfocar as estruturas fora desse plano \cite{hsieh2009computed}. Contudo, um dos grandes problemas relacionados a esta descoberta, é o fato de ser preciso a realização de diversos exames para a obtenção de diferentes fatias do corpo. 

Já no ano de 1932, o pesquisador holandês B. G. Ziedses des Plantes publicou um trabalho \cite{des1932neue} que descrevia a possibilidade da formação de diversos planos do objeto a partir de um número definido de projeções do mesmo. As implementações mais marcantes da teoria de Ziedses des Plantes foram feitas por \citeonline []{garrison1969three} e mais tarde por \citeonline []{miller1971infinite}. Os autores demonstraram o princípio da tomografia convencional discreta por meio da utilização de luz, lentes e espelhos para a reconstrução de um número arbitrário de planos tomográficos do objeto examinado. O termo tomossíntese, entretanto, só ficou conhecido um ano mais tarde com o trabalho de \citeonline []{grant1972tomosynthesis}. Uma descrição mais detalhada sobre a história do desenvolvimento da tomossíntese pode ser encontrada em  \citeonline []{dobbins2003digital,levakhina2014three} e \citeonline []{goodsitt2014history}. 

Embora a descoberta da técnica de tomossíntese tenha sido feita no início do século passado, seu real interesse e sua ampla utilização para exames da mama aconteceram recentemente com o trabalho de  \citeonline []{niklason1997digital} no \textit{Massachusetts General Hospital} na década de 90. Isso se deve ao fato da sua difícil implementação e da escassez de recursos tecnológicos naquela época. 

O avanço da computação, a introdução de detectores digitais de tela plana e a possibilidade da sua rápida leitura foram peças essenciais para a volta da tomossíntese ao foco das pesquisas junto a sua vasta utilização, principalmente no âmbito do rastreamento do câncer de mama \cite{Niklason20185}. 

Hoje essa técnica está se espalhando rapidamente e substituindo os equipamento de mamografia digital, de acordo com \citeonline [p. 6]{Niklason20185}. O gráfico da Figura \ref{fig:imgCap2TimelineDBT} ilustra o número de publicações científicas nos últimos 10 anos com o termo ``\textit{Digital Breast Tomosynthesis}'' na base de dados do site \textit{PubMed}\footnote{\url{www.ncbi.nlm.nih.gov/pubmed/}}, demonstrando um grande aumento de interesse na área ao longo dos anos.

No ano de 2011, o equipamento da empresa \textit{Hologic}\footnote{\url{www.hologic.com/}} foi aprovado pelo órgão regulamentador dos Estados Unidos (\textit{Food and Drug Administration} - \acs{FDA})\footnote{\url{www.fda.gov}} para ser comercializado e utilizado na prática clínica.  
    
 
\begin{figure}[H]
	\caption{Número de publicações científicas nos últimos 10 anos com o termo ``\textit{Digital Breast Tomosynthesis}''.}
	\begin{center}
		\includegraphics[scale=0.6, clip, trim=1.7cm 9.4cm 1.9cm 10.8cm]{imgs/cap2/TimelineDBT}
	\end{center}
	\legend{Fonte: PubMed, 2018.}
	\label{fig:imgCap2TimelineDBT}
\end{figure}

A \acs{DBT} é caracterizada por ser uma técnica tomográfica de ângulo limitado. Nesse exame, múltiplas radiografias, denominadas de projeções, são adquiridas em diferentes ângulos, demonstrado na Figura \ref{fig:imgCap2DBTEstrutura}, enquanto o tubo se move em uma trajetória fixa pré-definida, amenizando assim o problema de sobreposição de tecidos existente no equipamento de \acs{FFDM}. Ao final do conjunto de exposições, as projeções são processadas por um algoritmo e, por fim, é reconstruído o volume pseudo-\acs{3D} da mama, como ilustra a Figura \ref{fig:imgCap2ExameDBTRecon} \cite{vedantham2015digital,michell2018role}. Essas fatias são então apresentadas ao radiologista para serem laudadas, como demonstra a Figura \ref{fig:imgCap2ExameDBT}. 

Resumidamente, a técnica de \acs{DBT} é muito semelhante a mamografia digital comum,  diferenciando-se somente na rotação do tubo de raios X e na utilização de um algoritmo para a reconstrução do volume da mama \cite{michell2018role}.

\begin{figure}[H]
	\caption{Geometria básica de aquisição de um equipamento de \acs{DBT}.}
	\begin{center}
		\includegraphics[scale=0.9, clip, trim=11cm 1.5cm 9.5cm 2.5cm]{imgs/cap2/DBT}
	\end{center}
	\legend{Fonte: do autor, 2019.}
	\label{fig:imgCap2DBTEstrutura}
\end{figure}


\begin{figure}[H]
	\caption{Esquemático geral do procedimento de reconstrução do volume \acs{3D} da mama a partir das projeções de raios X.}
	\begin{center}
		\includegraphics[scale=0.5, clip, trim=3.5cm 1.2cm 8.4cm 1.9cm]{imgs/cap2/Recon}
	\end{center}
	\legend{Fonte: do autor, 2019.}
	\label{fig:imgCap2ExameDBTRecon}
\end{figure}

\begin{figure}[H]
	\caption{Fatias do volume \acs{3D} reconstruídas referente a mama apresentada na Figura \ref{fig:imgCap2ExameMamografia} na sua versão \acs{2D}.}
	\begin{center}
		\includegraphics[scale=0.5, clip, trim=10cm 1cm 10cm 0cm]{imgs/cap2/DBT_Cli}
	\end{center}
	\legend{Fonte: do autor, 2019.}
	\label{fig:imgCap2ExameDBT}
\end{figure}
%são formadas as imagens de recortes da mama paralelas ao detector. 

As variações no processo de aquisição das imagens dessa técnica ocorrem de acordo com cada fabricante, contudo, de uma maneira geral e simplificada, a Figura \ref{fig:imgCap2DBTEstrutura} exemplifica a geometria de aquisição de um equipamento de \acs{DBT}. Nessa figura, o tubo de raios X se move na trajetória de um arco emitindo radiação nas posições de A até C, com angulação de $\theta_{1}$ até $\theta_{3}$ respectivamente. Cada exposição gera uma projeção dos objetos no detector plano, representada na parte inferior da figura.

\subsection{Parâmetros Físicos e Geométricos}\label{ParâmetrosFísicoseGeométricos}

Em geral, os equipamentos de tomossíntese possuem sua estrutura física muito semelhante aos de  mamografia digital. Durante o exame clínico a paciente mantêm-se posicionada de pé, junto ao tubo e ao detector posicionados em uma orientação \ac{CC} ou \ac{MLO} \cite{Niklason20185}. Então, o prato de compressão comprime a mama para realização do exame, com o propósito de deixar os tecidos o mais distribuído possível e reduzir o movimento da paciente durante o exame. Após esse procedimento são realizadas as exposições de raios X em uma determinada faixa de ângulo com um tipo de movimento do tubo preestabelecido, como ilustra a Figura \ref{fig:imgCap2DBTEstrutura1} \cite{baker2011breast}.  

\begin{figure}[H]
	\caption{Geometria de um equipamento de \acs{DBT}.}
	\begin{center}
		\includegraphics[scale=0.75, clip, trim=13cm 4cm 9.5cm 5cm]{imgs/cap2/DBT1}
	\end{center}
	\legend{Fonte: do autor, 2019.}
	\label{fig:imgCap2DBTEstrutura1}
\end{figure}

A movimentação pode ser dada de duas maneiras: \textit{Step-and-shoot} ou contínuo (\textit{continuous tube motion}). No segundo modo, o tubo se move de maneira ininterrupta disparando a radiação em tempos espaçados igualmente ao longo de todo período. Esse método gera uma redução no tempo do exame e na movimentação do paciente, em contrapartida, há um aumento no borramento da imagem devido ao movimento do ponto focal durante os disparos \cite{glick2014system}. 

Já no modo \textit{Step-and-shoot}, o tubo para o movimento totalmente antes de cada exposição e logo em seguida retorna ao seu movimento. Esse processo elimina quase totalmente o borramento relacionado à movimentação, porém ainda existente em pequenas vibrações do tubo devido a sua inércia e seu peso \cite{glick2014system}.

No que se refere à rotação dos detectores, alguns fabricantes optam por sua movimentação ou não. Mantê-lo estacionário reduz sua complexidade mecânica e consequentemente o borramento das imagens. Já sua angulação possibilita a confecção de detectores menores e também uma redução no borramento dos raios X, que são incididos com uma maior angulação nos detectores fixos \cite{glick2014system}. 

Para a angulação do tubo, os equipamentos comercias variam entre 15$\degree$ a 50$\degree$, com um número de 9 a 25 exposições, porém de acordo com \citeonline [p. 25]{glick2014system} não existe ao certo um valor ótimo para um número de projeção nem para a faixa de ângulo. Um estudo de \citeonline []{sechopoulos2009optimization} com 63 combinações de ângulos e projeções, concluiu que os parâmetros que obtiveram melhores resultados no experimento foram com uma faixa total de ângulo de 60$\degree$ com 13 projeções. Ainda segundo os autores, com o intuito de aumentar a resolução na direção Z é necessário maximizar a extensão do ângulo de aquisição, entretanto o aumento no número de projeções não está relacionado com essa melhoria. Segundo \citeonline []{hu2008image}, a diminuição do ângulo além de causar a redução de resolução em Z, produz também um borramento entre estruturas em profundidades diferentes. Todavia \citeonline [p. 7]{Niklason20185} afirma que a utilização de ângulos menores está associada a uma melhor visualização de calcificações na mama. 

Devido a substituição da mamografia digital pela tomossíntese, a preocupação com a manutenção da dose de radiação é eminente. Conforme \citeonline [p. 9]{Hooley20189} a dose cumulativa é aproximadamente a mesma para ambos os exames ou um pouco maior para a \acs{DBT}. Ainda de acordo com os autores, a dose de radiação da combinação de ambos exames é próxima a $2.65 mGy$ (\textit{miligray})\footnote{Quantidade de energia de radiação absorvida em um quilograma de matéria \ref{Hooley20189}.}, onde $1.2 mGy$ e $1.45 mGy$ são oriundos da mamografia e tomossíntese respectivamente. É importante ressaltar que esses valores estão abaixo dos limites estabelecidos pela norma norte-americana \ac{MQSA} e que o valor da dose é dependente do fabricante e do seu respectivo modelo. 

O número de projeções está intimamente ligado com a dose total e com os artefatos na imagem reconstruída. Ainda, em conformidade com \citeonline [p. 7]{Niklason20185}, um baixo número de projeções está associado com a geração de artefatos em planos fora de foco decorrentes de objetos com alta atenuação, como as microcalcificações. Em contrapartida, um elevado número de projeções está limitado ao valor da dose total de radiação associado ao exame. 

Dado uma dose total de $X\,mGy$ com um número $N$ de projeções, a dose individual $X_{i}$ de cada projeção é dada por: $X_{i} = \frac{X}{N}$, ou seja, a dose total é dividida igualmente para cada projeção. Quanto menor a dose por imagem $X_{i}$, menor será sua relação sinal-ruído (\textit{Signal to Noise Ratio} - \acs{SNR}), devido ao ruído ser modelado através de uma distribuição \textit{Poisson}, como detalhado no \autoref{CapituloRuido}. Sendo assim, a \acs{SNR} de cada projeção é inversamente proporcional ao número total de projeções \cite{sechopoulos2009optimization}. Isso resulta ao final em uma degradação na qualidade da imagem reconstruída caso o número de exposições seja elevado.  No entanto, existe um compromisso entre a dose de radiação e a qualidade da imagem exposta ao radiologista, de modo que, do ponto de vista da dosagem de radiação deseja-se minimizar o número de projeções, mas para a qualidade da imagem reconstruída ser boa é preciso aumentar o número de projeções.

Outra consideração a ser feita para os equipamentos de \acs{DBT} é a sua geometria de emissão dos feixes. Diferente de \acs{CT}, a emissão dos feixes de raios X é feita através de uma geometria de meio cone (\textit{half cone-beam}). Esse fato deve ser levado em consideração em ambos procedimentos de projeção e retroprojeção nos algoritmos de reconstrução \cite{wu2004comparison}. 
    
Tomando como base os sistemas atuais, eles se diferem em diversos aspectos como: a geometria física, a angulação do tubo, a movimentação do detector, o número de projeções, o tamanho do \textit{pixel}, o método de aquisição direto ou indireto, os algoritmos de reconstrução, o tempo do exame etc. 

A Tabela \ref{tab:tabCap2SistemasDBT} mostra a relação entre as principais diferenças encontradas nos equipamentos comerciais de \acs{DBT} aprovados pelo \acs{FDA}. É possível observar uma grande variação entre as propriedades de cada modelo especificado. Isso se deve ao fato de que cada fabricante utiliza de suas ferramentas de pesquisa e desenvolvimento para solucionar um determinado problema e cada sistema possui os seus benefícios e suas limitações. 

Os equipamentos em sua essência buscam a melhor forma para obtenção das imagens, sendo assim, visam um bom contraste, menor dose, maior resolução  e uma boa relação sinal-ruído. Para isso, são utilizadas diferentes abordagens na aquisição dessas imagens \cite{vedantham2015digital}. 

\begin{table}[H]
	\centering
	\caption{Caraterísticas dos sistemas  de \acs{DBT}.}
	\label{tab:tabCap2SistemasDBT}
	\begin{tabular}{l|c|c|c}
		\textbf{Fabricantes}                                       &        \textbf{\acs{GE}}        &                 \textbf{Hologic}                 &   \textbf{Siemens}   \\
		[5pt]
		\hline
		\hline
		\rule[-0.5ex]{-3pt}{1ex}
		Modelo &    SenoClaire{\footnotesize\texttrademark}    & Selenia\textsuperscript{\textregistered} Dimensions\textsuperscript{\textregistered} & Mammomat Inspiration \\ \hline
		\rule[-0.5ex]{-3pt}{1ex}
		Número de projeções              &                9                &                        15                        &          25          \\ \hline
		\rule[-0.5ex]{-3pt}{1ex}
		Angulação do tubo                &           25$\degree$           &                   15$\degree$                    &     50$\degree$      \\ \hline
		\rule[-0.5ex]{-3pt}{1ex}
		Angulação do detector            &          Estacionário           &                   4,2$\degree$                   &     Estacionário     \\ \hline
		\rule[-0.5ex]{-3pt}{1ex}
		Movimento do tubo                &     \textit{Step-and-shoot}     &                     Contínuo                     &       Contínuo       \\ \hline
		\rule[-0.5ex]{-3pt}{1ex}
		Tempo do exame                   &               7s                &                       3,7s                       &         25s          \\ \hline
		\rule[-0.5ex]{-3pt}{1ex}
		Tamanho do detector              &             24x30cm             &                     24x29cm                      &       24x30cm        \\ \hline
		\rule[-0.5ex]{-3pt}{1ex}
		Tamanho do \textit{pixel}        &            100$\mu$m            &         70$\mu$m (2x2 \textit{Binning})          &       85$\mu$m       \\ \hline
		\rule[-0.5ex]{-3pt}{1ex}
		Tipo do detector                 &         \acs{a-Si} (Indireto)         &                  \acs{a-Se} (Direto)                   &    \acs{a-Se} (Direto)     \\ \hline
		\rule[-0.5ex]{-3pt}{1ex}
		Método de reconstrução           & Iterativo (ASiR\textsuperscript{\textregistered}) &                    \acs{FBP}                     &      \acs{FBP}       \\ \hline
	\end{tabular}
	\vspace{2ex}
	\legend{Fonte: \citeonline{michell2018role,vedantham2015digital,sechopoulos2013review,baker2011breast}.}
\end{table}


 
    


\chapter[Reconstrução]{Reconstrução de imagens}\label{Capitulo3}

%%%%%%%%%%%%%%%%%%%%%%%%%%%%%%%%%%%%%%%%%%%%%%%%%%%%%%%%%%%%%%%%%%%%%%%%%%%%%%%%%%%%%%%%%%%%%%%%%%%%%%%%%%%%%%%%%%%%%%%%%%%%%%												  Introdução    															%
%%%%%%%%%%%%%%%%%%%%%%%%%%%%%%%%%%%%%%%%%%%%%%%%%%%%%%%%%%%%%%%%%%%%%%%%%%%%%%%%%%%%%%%%%%%%%%%%%%%%%%%%%%%%%%%%%%%%%%%%%%%%%%

Com o objetivo de obter imagens radiográficas feitas através de um exame de tomografia, seja esse advindo de qualquer equipamento, é necessário a aplicação de procedimentos para a reconstrução dessas imagens dado as informações que o equipamento gera. Esses procedimentos envolvem aplicação de técnicas matemáticas, físicas e computacionais. As informações geradas pelo equipamento são comumente denominadas de projeções, pois representam ``sombras'' que o objeto produz ao ser irradiado por feixes de raios X. Essas sombras, ou mais precisamente projeções, são informações das atenuações sofridas pelos raios X quando esses interagem com o tecido biológico ao longo de seu caminho \cite{buzug2008computed,avinash1988principles}.
 
Os métodos de reconstrução atuais destinados à tomossíntese foram primeiramente desenvolvidos para a tomografia computadorizada. Posteriormente, esses foram adaptados às necessidades específicas de cada equipamento e em geral todos os métodos utilizados para \acs{CT} são aplicáveis à tomossíntese, respeitando as devidas particularidades. Contudo o desenvolvimento de algoritmos para tomossíntese é considerado um desafio pelo fato dessa técnica possuir poucas projeções em uma estreita extensão angular e baixa dose de radiação \cite{levakhina2014three,yang2012numerical}.   

%%%%%%%%%%%%%%%%%%%%%%%%%%%%%%%%%%%%%%%%%%%%%%%%%%%%%%%%%%%%%%%%%%%%%%%%%%%%%%%%%%%%%%%%%%%%%%%%%%%%%%%%%%%%%%%%%%%%%%%%%%%%%%											Problemas Inversos	    														%
%%%%%%%%%%%%%%%%%%%%%%%%%%%%%%%%%%%%%%%%%%%%%%%%%%%%%%%%%%%%%%%%%%%%%%%%%%%%%%%%%%%%%%%%%%%%%%%%%%%%%%%%%%%%%%%%%%%%%%%%%%%%%%
\section{Problemas Inversos}

Em geral, reconstruções de imagens tomográficas são consideradas, do ponto de vista matemático, como um problema inverso, ou seja, busca-se a formação da estrutura espacial do objeto a partir das projeções do mesmo em diferentes ângulos \cite{buzug2008computed}. A Figura \ref{fig:imgCap3ProbInver} ilustra o procedimento tomando projeções do objeto por meio de ângulos distintos. A Figura \ref{fig:imgCap3ProbInverA} exemplifica o imageamento do objeto e a formação de suas respectivas projeções, já a Figura \ref{fig:imgCap3ProbInverB} exemplifica o problema inverso para a reconstrução da estrutura espacial do objeto. 

%Forward problem and Inverse problem 

\begin{figure}[H]
	\centering
	
	\caption{Processo de recuperação de imagens 2D através de suas projeções em diferentes ângulos.}
	
	\subfloat[Exemplo de projeções do objeto em diferentes ângulos.]{\includegraphics[scale=0.4]{imgs/cap3/ProbInverA.png}	\label{fig:imgCap3ProbInverA}}
	\hfil
	\subfloat[Exemplo de problema inverso.]{\includegraphics[scale=0.4]{imgs/cap3/ProbInverB.png} \label{fig:imgCap3ProbInverB}}
	\hfil
	
	\legend{Fonte: do autor, 2018.}
	\label{fig:imgCap3ProbInver}
\end{figure}

Em geral, os equipamentos de raios X por transmissão tem como modelo físico básico a formulação de \textit{Beer-Lambert} \cite{zeng2010medical}. Essa fórmula relaciona a função de atenuação $\mu(s)$ do feixe de radiação quando esse interage com um certo material por um determinado caminho. Considerando um feixe de raio X ideal, infinitamente pequeno, monoenergético e sem espalhamento, sua intensidade de energia inicial $I_{0}$ interage com a matéria por um caminho de comprimento $l$ e sua intensidade de energia final $I(l)$ é dada por \eqref{eq:eqCap3BeerLambert1}, segundo \citeonline[p. 13]{levakhina2014three}:

\begin{equation}
 I(l) = I_{0} \, e \, ^{-\int_{0}^{l} \,\mu(s) \, ds}.
\label{eq:eqCap3BeerLambert1}
\end{equation}

Considerando a Equação \ref{eq:eqCap3BeerLambert1}, se tomarmos o logaritmo da razão entre a intensidade final e a inicial com seu sinal negativo e denotarmos essa formulação como $p(l)$, ou seja, $p(l) = -\ln \, \left(\frac{I(l)}{I_{0}}\right)$, tem-se o seguinte resultado:

\begin{equation}
p(l) = {\int_{0}^{l} \,\mu(s) \, ds},
\label{eq:eqCap3BeerLambert2}
\end{equation}

\noindent onde $s$ denota as atenuações ao longo de $l$ e o problema inverso, ou de reconstrução, passa a ser a recuperação da função  $\mu(s)$ tomando um conjunto de integral de linha $p(l)$, segundo \citeonline[p. 33]{levakhina2014three}, como na transformada de Radon detalhada na seção seguinte.  
 
 
%%%%%%%%%%%%%%%%%%%%%%%%%%%%%%%%%%%%%%%%%%%%%%%%%%%%%%%%%%%%%%%%%%%%%%%%%%%%%%%%%%%%%%%%%%%%%%%%%%%%%%%%%%%%%%%%%%%%%%%%%%%%%%											Transformada de Radon    														%
%%%%%%%%%%%%%%%%%%%%%%%%%%%%%%%%%%%%%%%%%%%%%%%%%%%%%%%%%%%%%%%%%%%%%%%%%%%%%%%%%%%%%%%%%%%%%%%%%%%%%%%%%%%%%%%%%%%%%%%%%%%%%%
\section{Transformada de Radon}\label{TransformadaRadon2D}

No ano de 1917, o matemático Johann Radon publicou um estudo detalhando a solução para os ditos problemas inversos em tomografia computadorizada, tornando-se assim a transformada mais importante no aspecto teórico matemático da área atualmente \cite{radon1917uber}. Esse trabalho foi posteriormente traduzido para a língua inglesa \cite{radon1986determination}.

Essencialmente a transformada de Radon direta visa calcular as projeções $P(\rho,\theta)$ de um objeto com coordenadas $f(x,y)$ a partir de determinados ângulos $\theta_{x}$ e feixes $L_{n}$, como ilustra a Figura \ref{fig:imgCap3Radon}. Essas projeções são obtidas através de um conjunto de integrais de linha de $f(x,y)$ ao longo de feixes paralelos  $L_{n}$, como demonstra a Equação \ref{eq:eqCap3Radon1}, onde $d\sigma$ é o incremento ao longo dos feixes. Já a sua transformada inversa é a simples tomada dos valores de projeção e a distribuição desses ao longo do caminho percorrido pelo feixe paralelo. Se tomarmos suficientes projeções do objeto, podemos ao final reconstruir sua estrutura espacial. 

\begin{equation} 
P(\rho,\theta) = {\int_{L}^{} f(x,y) \, d\sigma}.
\label{eq:eqCap3Radon1}
\end{equation}


\begin{figure}[H]
	\caption{Ilustração da Transformada de Radon direta.}
	\begin{center}
		\includegraphics[scale=0.6]{imgs/cap3/Radon.png}
	\end{center}
	\legend{Fonte: do autor, 2018.}
	\label{fig:imgCap3Radon}
\end{figure}

Como pode ser visto na Figura \ref{fig:imgCap3Radon} a projeção está em função das duas variáveis: $\rho$ e $\theta$. Isso pode ser feito, uma vez que, as retas que representam os feixes são parametrizadas na forma polar, onde $\rho$ é a distância entre a reta do feixe e a origem e $\theta$ é o ângulo formado entre o eixo $x$ e $\rho$ \cite{yang2012numerical}. Isto é, sua forma é dada pela seguinte equação:

\begin{equation} 
L_{n} = 
\begin{cases}
x = \rho \, cos(\theta) - \sigma \, sin(\theta)\\ 
y = \rho \, sin(\theta) + \sigma \, cos(\theta)
\end{cases},
\label{eq:eqCap3Radon2}
\end{equation}

\noindent onde $\sigma$ é a distância entre $\rho$ e qualquer ponto na reta. A Figura \ref{fig:imgCap3RadonRhoTheta} ilustra a relação entre essas variáveis.

\begin{figure}[H]
	\caption{Ilustração da relação entre $x$, $y$, $\rho$, $\theta$ e $\sigma$.}
	\begin{center}
		\includegraphics[scale=1.2]{imgs/cap3/RadonRhoTheta.pdf}
	\end{center}
	\legend{Fonte: \citeonline[p. 74]{yang2012numerical}}
	\label{fig:imgCap3RadonRhoTheta}
\end{figure}

Após a parametrização das retas, podemos reescrever a Equação \ref{eq:eqCap3Radon1} em função da Equação \ref{eq:eqCap3Radon2} e então teremos o seguinte resultado:

\begin{equation} 
P(\rho,\theta) = \int_{L}^{} f(x,y) \, d\sigma = {\int_{-\infty}^{+\infty} f\left((\rho \, cos(\theta) - \sigma \, sin(\theta)),(\rho \, sin(\theta) + \sigma \, cos(\theta)\right) \, d\sigma},
\label{eq:eqCap3Radon3}
\end{equation}

\noindent ou fazendo o uso da função \textit{Delta} de \textit{Dirac} $\delta(\rho)$ teremos:

\begin{equation} 
P(\rho,\theta) = {\int_{-\infty}^{+\infty} \, \int_{-\infty}^{+\infty} f(x,y) \, \delta(\rho - (x\,cos(\theta)+y\,sin(\theta)))  \, dxdy},
\label{eq:eqCap3Radon4}
\end{equation} 

\noindent no qual a Equação \ref{eq:eqCap3Radon4} é conhecida como a transformada direta de Radon da função $f(x,y)$. Um maior detalhamento dos equacionamentos pode ser encontrado em \citeonline[p. 73-74]{yang2012numerical}.

%%%%%%%%%%%%%%%%%%%%%%%%%%%%%%%%%%%%%%%%%%%%%%%%%%%%%%%%%%%%%%%%%%%%%%%%%%%%%%%%%%%%%%%%%%%%%%%%%%%%%%%%%%%%%%%%%%%%%%%%%%%%%%											Métodos de Reconstrução    														%
%%%%%%%%%%%%%%%%%%%%%%%%%%%%%%%%%%%%%%%%%%%%%%%%%%%%%%%%%%%%%%%%%%%%%%%%%%%%%%%%%%%%%%%%%%%%%%%%%%%%%%%%%%%%%%%%%%%%%%%%%%%%%%
\section{Métodos de Reconstrução}

Após a formulação da transformada de Radon é possível obter a estrutura espacial do objeto a partir de suficientes projeções. Todo equacionamento, no entanto, é proposto em modo contínuo, fazendo o uso de integrais, por exemplo, e sendo necessário o uso de uma geometria com feixes de raios paralelos. Tendo dito isso, é eminente a necessidade da discretização dessa transformada e adequação ao sistema físico de cada equipamento para implementações práticas \cite{levakhina2014three}. 

As técnicas de projeção e retroprojeção são essenciais nos métodos de reconstrução, pois representam através de um modelo matemático o processo físico de aquisição da imagem. Para os métodos iterativos, a importância é ainda maior pelo fato da utilização de ambos a cada iteração \cite{levakhina2014three}.   

\subsection{Discretização da Imagem}\label{DiscretizaçãodaImagem}

Para a implementação prática da transformada é necessário a representação de um objeto contínuo de uma forma discreta, ou seja, transforma-lo em uma matriz de valores finitos em duas ou três dimensões, dependendo da aplicação. Isso se deve ao fato de que os coeficientes de atenuação ao longo de um objeto, por exemplo uma mama, são dados em uma forma contínua e são dependentes de fatores como: número atômico, densidade e espessura do meio \cite{yang2012numerical}.  

Portanto podemos fazer o uso de \textit{pixels} ou \textit{voxels} para representações em \acs{2D} ou \acs{3D} respectivamente. Além dessas duas representações também é possível a utilização de bolhas, ou \textit{blobs} do inglês. Um maior detalhamento sobre o modelamento dessas técnicas pode ser encontrado em  \citeonline[p. 45-69]{levakhina2014three}. A Figura \ref{fig:imgCap3DiscretizacaoImagem1} demonstra a discretização dos coeficientes de atenuações da mama representados por \textit{voxels} cúbicos, dado uma matriz de $m$ linhas por $n$ colunas e $l$ níveis.   

\begin{figure}[H]
	\caption{Discretização dos coeficientes de atenuações da mama.}
	\begin{center}
		\includegraphics[scale=1.1]{imgs/cap3/DiscretizacaoImg1.pdf}
	\end{center}
	\legend{Fonte: \citeonline[p. 72]{yang2012numerical}}
	\label{fig:imgCap3DiscretizacaoImagem1}
\end{figure}

Se tomarmos qualquer linha dessa matriz, temos a Figura \ref{fig:imgCap3DiscretizacaoImagem2}, ilustrando os coeficientes discretos e os níveis de energia final e inicial da respectiva linha considerando um feixe ideal. Dado isto, pode-se adaptar a Equação \ref{eq:eqCap3BeerLambert1} para a sua forma discreta, demonstrada a seguir:

\begin{equation}
I_{l} = I_{0} \, e \, ^{-\sum_{i=1}^{n} \mu_{i}}.
\label{eq:eqCap3BeerLambertDiscreto1}
\end{equation} 

\begin{figure}[H]
	\caption{Coeficientes de atenuação discretos da primeira linha da matriz mostrada na Figura \ref{fig:imgCap3DiscretizacaoImagem1}.}
	\begin{center}
		\includegraphics[scale=1.3]{imgs/cap3/DiscretizacaoImg2.pdf}
	\end{center}
	\legend{Fonte: \citeonline[p. 72]{yang2012numerical}}
	\label{fig:imgCap3DiscretizacaoImagem2}
\end{figure}

\subsection{Projeção e Retroprojeção 2D}

 Os operadores de projeção, essencialmente, visam projetar um objeto em (\acs{3D} ou \acs{2D}) em um anteparo (\acs{2D} ou \acs{1D}) respectivamente, ou seja, somar as contribuições de cada objeto ao longo do eixo de projeção. A Figura \ref{fig:imgCap3ProjeçãoDireta} ilustra a projeção de objetos em \acs{2D} para sensores em \acs{1D}, tomados ângulos de $0\degree$, $45\degree$, $90\degree$ e $135\degree$. 

\begin{figure}[H]
	\caption{Exemplo de projeção \acs{2D} para \acs{1D}}
	\begin{center}
		\includegraphics[scale=0.55]{imgs/cap3/FP.pdf}
	\end{center}
	\legend{Fonte: do autor, 2018.}
	\label{fig:imgCap3ProjeçãoDireta}
\end{figure}


Já os operadores de retroprojeção atuam no processo inverso, de reprojetar os dados (\acs{2D} ou \acs{1D}) para um espaço (\acs{3D} ou \acs{2D}) respectivamente, isto é, distribuir os dados de projeção ao longo do eixo que foram adquiridos. Assim como na seção anterior, a Figura \ref{fig:imgCap3BPMosaico} ilustra o processo de retroprojeção dos dados, no espaço da imagem, adquiridos pelos sensores da Figura \ref{fig:imgCap3ProjeçãoDireta} nos respectivos ângulos que foram gerados. 

\begin{figure}[H]
	\centering
	
	\caption{Exemplo de retroprojeção \acs{1D} para \acs{2D}. Onde as figuras de (a) até (d) representam o \acs{BP} de $0\degree$, $45\degree$, $90\degree$ e $135\degree$ respectivamente.}
	
	\subfloat[]{\includegraphics[scale=0.2]{imgs/cap3/iRadon_0deg.png}\label{fig:imgCap3BPMosaico0}}
	\hfil
	\subfloat[]{\includegraphics[scale=0.2]{imgs/cap3/iRadon_45deg.png}\label{fig:imgCap3BPMosaico45}}
	\hfil
	\subfloat[]{\includegraphics[scale=0.2]{imgs/cap3/iRadon_90deg.png}\label{fig:imgCap3BPMosaico90}}
	\hfil
	\subfloat[]{\includegraphics[scale=0.2]{imgs/cap3/iRadon_135deg.png}\label{fig:imgCap3BPMosaico135}}
	\hfil
	
	\legend{Fonte: do autor, 2018.}
	\label{fig:imgCap3BPMosaico}
\end{figure}


Se somarmos as contribuições de cada ângulo da Figura \ref{fig:imgCap3BPMosaico} a fim de reconstruir o objeto, obteremos a Figura \ref{fig:imgCap3BPParcial} como resultado. Nota-se que a partir dessa operação ainda não é possível identificar as estruturas espaciais com clareza. Já se considerarmos mais projeções com uma maior faixa de ângulo, como por exemplo $0\degree$ \text{a} $180\degree$, tornam-se visíveis os objetos, porém ainda de uma forma borrada como demostra a Figura \ref{fig:imgCap3BPFinal}. Não há a necessidade de se utilizar $360\degree$, visto que as projeções do primeiro e segundo quadrante serão as mesmas se tomadas em ângulos contidos no terceiro e quarto quadrante \colorbox{pink}{Corrigir}.   

\begin{figure}[H]
	\centering
	
	\caption{Recuperação dos objetos \acs{2D} através da soma de suas retroprojeções em diferentes ângulos. Onde (a) representa a técnica considerando apenas 4 projeções, ilustrado pela Figura \ref{fig:imgCap3BPMosaico}, e (b) representa a utilização de 180 projeções na faixa de $180\degree$.}
	
	\subfloat[]{\includegraphics[scale=0.3]{imgs/cap3/Backprojection_Parcial.png}\label{fig:imgCap3BPParcial}}
	\hfil
	\subfloat[]{\includegraphics[scale=0.3]{imgs/cap3/Backprojection_Final.png}\label{fig:imgCap3BPFinal}}
	\hfil
	
	\legend{Fonte: do autor, 2018.}
	\label{fig:imgCap3BPSoma}
\end{figure}

O método descrito acima é o mais simples encontrado na literatura e é conhecido como \textit{Pixel Driven} (Figura \ref{fig:imgCap3Projetores1}) segundo \citeonline[p. 47]{levakhina2014three}. Em seu operador de projeção direta, esse consiste em projetar o centro de cada \textit{pixel} no detector e o seu valor de atenuação ser repartido entre os detectores vizinhos por meio de métodos de interpolação linear ou outros mais complexos. O mesmo princípio é aplicado para o operador de retroprojeção, no qual os \textit{pixels} da imagem recebem os valores obtidos por sensores vizinhos também através de métodos de interpolação.

Outras técnicas mais sofisticadas também são aplicadas como: \textit{Ray Casting} ou \textit{Ray Driven}, \textit{Distance-Driven} e \textit{Trapezoid footprint} como demonstra a Figura \ref{fig:imgCap3Projetores}. Um maior detalhamento das técnicas mencionadas pode ser encontrado em \citeonline[p. 47-49]{levakhina2014three} ou ainda métodos do estado da arte como demostrado em \citeonline[]{zheng2017segmented}.

\begin{figure}[H]
	\centering
	
	\caption{Projetores utilizados para técnicas de projeção e retroprojeção para modelos de representação utilizando \textit{Pixels}. Onde os diferentes métodos são ilustrados por: (a) \textit{Pixel Driven}, (b) \textit{Ray Casting}, (c) \textit{Distance-Driven} e (d) \textit{Trapezoid footprint}.}
	
	\subfloat[]{\includegraphics[scale=1.1]{imgs/cap3/Projetor1.pdf}\label{fig:imgCap3Projetores1}}
	\hfil
	\subfloat[]{\includegraphics[scale=1.1]{imgs/cap3/Projetor2.pdf}\label{fig:imgCap3Projetores2}}
	\hfil
	\subfloat[]{\includegraphics[scale=1.1]{imgs/cap3/Projetor3.pdf}\label{fig:imgCap3Projetores3}}
	\hfil
	\subfloat[]{\includegraphics[scale=1.1]{imgs/cap3/Projetor4.pdf}\label{fig:imgCap3Projetores4}}
	\hfil
	
	\legend{Fonte: \citeonline[p. 47]{levakhina2014three}}
	\label{fig:imgCap3Projetores}
\end{figure}  
  
\subsection{Projeção e Retroprojeção 3D}\label{ProjeçãoeRetroprojeção3D}

Os problemas para a reconstrução de imagens em tomossíntese são relacionados a estruturas tridimensionais, sendo assim é importante discorrer sobre os operadores de projeção e retroprojeção aplicados a essa técnica. A Figura \ref{fig:imgCap33DGeometry} ilustra como um feixe de raio X é atenuando e quais \textit{voxels} são responsáveis por esta interação. Em geral, os operadores utilizados para \acs{2D} são também atribuídos as geometrias \acs{3D}, porém sendo necessários ajustes dependente de cada técnica e geometria. O processamento é feito então plano por plano, passando por todos os \textit{voxels} daquele nível. 

\begin{figure}[H]
	\caption{Ilustração da geometria de tomossíntese e a atenuação dos feixes pelos \textit{voxels}}
	\begin{center}
		\includegraphics[scale=0.9]{imgs/cap3/3DGeometry.pdf}
	\end{center}
	\legend{Fonte: \citeonline[p. 71]{Duarte2009}}
	\label{fig:imgCap33DGeometry}
\end{figure}

Para a implementação do operador \textit{Pixel Driven} é necessário encontrar a posição específica onde cada \textit{voxel} será projetado. Segundo \citeonline[p. 77]{dobbins2003digital}, os equipamentos de tomossíntese, em geral, são caracterizados por um movimento isocêntrico. Isto porque tubo e detector ou somente o tubo se movem em um arco com o mesmo centro de movimento.  

Um dos tipos de geometria de movimento isocêntrico foi primeiramente apresentado por \citeonline[]{niklason1997digital}, como ilustra a Figura \ref{fig:imgCap3ProjectionGeometry}, onde o detector se mantêm estacionário e o tubo se move em um arco com um determinado centro de rotação. 

\begin{figure}[H]
	\caption{Geometria de movimento isocêntrico apresentado por \citeonline[]{niklason1997digital}.}
	\begin{center}
		\includegraphics[scale=0.4]{imgs/cap3/ProjectionGeometry.png}
	\end{center}
	\legend{Fonte: \citeonline[p. 76]{dobbins2003digital}}
	\label{fig:imgCap3ProjectionGeometry}
\end{figure}

Os passos necessários para a formação da imagem com esse tipo de movimento foram apresentados no trabalho do autor, onde o mesmo descreve as equações para a projeção de um objeto no plano da imagem. Se tomarmos um ponto qualquer com coordenadas $(X,Y,Z)$ e ângulo do tubo $\theta$, temos que a posição $y_{i}$ de sua projeção é dada por:

\begin{equation}
y_{i} = Y \, + \dfrac{ Z(L \, sin(\theta) \, + Y)}{L \, cos(\theta) \, + D - Z} \; \forall \, \theta ,
\label{eq:eqCap3ProjectionY}
\end{equation} 

\noindent e a projeção no eixo X é dada pela equação a seguir:

\begin{equation}
x_{i} = \dfrac{X \, (L \, cos(\theta)\,+\, D)}{L \, cos(\theta) \, + D - Z} \; \forall \, \theta.
\label{eq:eqCap3ProjectionX}
\end{equation} 

Já a \textbf{retroprojeção} dos dados projetados é dada a partir da inversão dessas equações a fim de calcular os valores de  coordenada $(X,Y)$ para cada fatia em $Z$. O procedimento é executado a partir de cada projeção obtida em um determinado ângulo $\theta$ e por fim são somadas a contribuições de cada projeção para cada fatia em $Z$. O fluxograma contido na Figura \ref{fig:imgCap3FluxogramaBP} ilustra o procedimento que deve ser feito para a reconstrução do volume. Em geral a combinação matemática utilizada para os planos retroprojetados é uma simples soma ou média, mas pode-se levar em conta aspectos mais avançados como comentado na seção \ref{ReduçãodeArtefatosdeAltaAtenuação}.

\begin{figure}[H]
	\caption{Fluxograma para o processo de simples retroprojeção para a reconstrução do volume.}
	\begin{center}
		\includegraphics[scale=0.5]{imgs/cap3/FluxogramaBP.pdf}
	\end{center}
	\legend{Fonte: do autor, 2018.}
	\label{fig:imgCap3FluxogramaBP}
\end{figure}

%%%%%%%%%%%%%%%%%%%%%%%%%%%%%%%%%%%%%%%%%%%%%%%%%%%%%%%%%%%%%%%%%%%%%%%%%%%%%%%%%%%%%%%%%%%%%%%%%%%%%%%%%%%%%%%%%%%%%%%%%%%%%%											Reconstrução Analítica    														%
%%%%%%%%%%%%%%%%%%%%%%%%%%%%%%%%%%%%%%%%%%%%%%%%%%%%%%%%%%%%%%%%%%%%%%%%%%%%%%%%%%%%%%%%%%%%%%%%%%%%%%%%%%%%%%%%%%%%%%%%%%%%%%

\subsection{Reconstrução Analítica}

As técnicas utilizadas para a reconstrução em equipamentos de tomossíntese são provenientes de \acs{CT}, pelo fato de já estarem consolidadas e serem vastamente conhecidas na literatura. Isso não é diferente quando o assunto é reconstrução de forma analítica. Todo processo vêm da teoria da transformada inversa de Radon, que foi discutida com mais detalhes no item \ref{TransformadaRadon2D}. É importante notar que devido ao problema de reconstrução ser mal condicionado, tal como uma aquisição incompleta de dados, a resolução de forma analítica é uma solução aproximada e não leva em conta diversos fatores, como o ruído quântico presente nas projeções. A melhor forma de explicar esse método é a partir de duas dimensões e posteriormente sua extensão para a terceira dimensão, como é demonstrado nos itens a seguir \cite{mertelmeier2014filtered,xu2014tomographic}.  


\subsubsection{Retroprojeção Filtrada}\label{RetroprojeçãoFiltrada}

Um grande problema relacionado com a simples retroprojeção dos dados do detector é a soma das baixas frequências que ocorre a partir dos diferentes ângulos (Figura \ref{fig:imgCap3BPMosaico} e \ref{fig:imgCap3BPParcial}). Isso é facilmente observado quando faz-se o uso do \textit{Fourier Slice Theorem}, também conhecido como \textit{Central Slice Theorem}, ou traduzindo para o português como Teorema do Corte de Fourier. O problema dito acima pode ser observado na ilustração abaixo (Figura \ref{fig:imgCap3FourierSliceTheorem}) caso sejam tomados diferentes ângulos de projeção. Aplicando a transformada de Fourier inversa no domínio da frequência em toda a faixa de ângulo, é possível recuperar a estrutura espacial do objeto, porém de forma ``borrada'' devido a sobreposição de baixas frequências. \colorbox{pink}{Alexandre: Essa é a motivação para processamento de sinais, mas há uma justificativa que envolve o Jacobiano da transformada de coordenadas.}    

\begin{figure}[H]
	\caption{Ilustração do Teorema do Corte de Fourier.}
	\begin{center}
		\includegraphics[scale=0.4]{imgs/cap3/FourierSliceTheorem.png}
	\end{center}
	\legend{Fonte: do autor, 2018.}
	\label{fig:imgCap3FourierSliceTheorem}
\end{figure}

Com o propósito de contornar esse problema, é aplicado em cada projeção um filtro de rampa \acs{1D}, conhecido como \textit{Ramp Filter} (\ref{fig:imgCap3FBPFiltersA}), até a frequência de Nyquist $\frac{1}{2 \varDelta x}$, onde $\varDelta x$ é o tamanho espacial do \textit{pixel}. Esse filtro faz com que as baixas frequências sejam atenuadas mantendo somente as altas, porém devido ao seu formato, o ruído que predomina na alta frequência é amplificado. 

\begin{figure}[H]
	\centering
	
	\caption{Ilustração do filtro (a) rampa no domínio da frequência e (b) o mesmo ``janelado'' por uma função Hanning para redução de ruídos. \colorbox{pink}{Resolução baixa, mudar pra .pdf}}
	
	\subfloat[\colorbox{pink}{Alexandre: Passa Alta}]{\includegraphics[scale=0.2]{imgs/cap3/RampFilter.png}\label{fig:imgCap3FBPFiltersA}}
	\hfil
	\subfloat[\colorbox{pink}{Alexandre: Passa banda}]{\includegraphics[scale=0.2]{imgs/cap3/RampFilterHanning.png}\label{fig:imgCap3FBPFiltersB}}
	\hfil
	
	\legend{Fonte: do autor, 2018.}
	\label{fig:imgCap3FBPFilters}
\end{figure}

Devido a esse impasse é aplicado um ``janelamento'' no filtro, como por exemplo: Hamming, Hanning, Shepp-Logan ou cosseno. A Figura \ref{fig:imgCap3FBPFiltersA} ilustra o filtro rampa no domínio da frequência e a Figura \ref{fig:imgCap3FBPFiltersB} demonstra o mesmo após um ``janelamento'' pela função Hanning. Seguido do procedimento de filtragem, os dados são retomados para o domínio do espaço e então são retroprojetados para a formação da estrutura espacial do objeto \cite{xu2014tomographic}. 

Já para a aplicação em tomossíntese, diversas aproximações devem ser feitas, pois essa técnica abrange somente uma pequena extensão de ângulo com poucas projeções quando comparado com equipamentos de \acs{CT}. Outro problema é com a geometria de emissão dos feixes, que são dispostos em formato de meio cone (\textit{Half Cone Beam}) \cite{mertelmeier2014filtered}. 

Tendo em vista todos esses fatores, com uma aproximação da geometria dos feixes, dispondo-os paralelamente, pode-se utilizar o Teorema do Corte de Fourier para analisar as consequências desses problemas e suas devidas soluções. A Figura \ref{fig:imgCap3FourierSliceTheorem3D} ilustra a aproximação feita para a aplicação do teorema. Mais detalhes podem ser encontrados em \citeonline[p. 101-106]{mertelmeier2014filtered}.

\begin{figure}[H]
	\caption{Ilustração do Teorema do Corte de Fourier aproximado para tomossíntese.}
	\begin{center}
		\includegraphics[scale=1.3]{imgs/cap3/FourierSliceTheorem3D.pdf}
	\end{center}
	\legend{Fonte: \citeonline[p. 102]{mertelmeier2014filtered}}
	\label{fig:imgCap3FourierSliceTheorem3D}
\end{figure} 

O algoritmo de \citeonline[]{feldkamp1984practical} é o mais utilizado para tomossíntese da mama. De maneira geral o método apresenta uma aproximação do algoritmo de \acs{FBP} para uma geometria de cone (\textit{Cone-beam}). Esse algoritmo foi implementado em tomossíntese por \citeonline[]{wu2004comparison}. Basicamente, a implementação da técnica de \acs{FBP} para tomossíntese se dá em 5 passos \cite[p. 16]{xu2014tomographic}, como descrito a seguir:   

\begin{enumerate}
	\item Fazer a transformada de Fourier da projeção em cada linha paralela a trajetória do tubo.
	\item Aplicar o filtro de rampa no domínio da frequência.
	\item Aplicar o filtro de ``janelamento'' para redução de ruído e artefatos.
	\item Fazer a transformada inversa de Fourier das projeções filtradas.
	\item Retroprojetar os dados no domínio espacial.   
\end{enumerate} 

%%%%%%%%%%%%%%%%%%%%%%%%%%%%%%%%%%%%%%%%%%%%%%%%%%%%%%%%%%%%%%%%%%%%%%%%%%%%%%%%%%%%%%%%%%%%%%%%%%%%%%%%%%%%%%%%%%%%%%%%%%%%%%											Reconstrução Iterativa    														%
%%%%%%%%%%%%%%%%%%%%%%%%%%%%%%%%%%%%%%%%%%%%%%%%%%%%%%%%%%%%%%%%%%%%%%%%%%%%%%%%%%%%%%%%%%%%%%%%%%%%%%%%%%%%%%%%%%%%%%%%%%%%%%

\subsection{Reconstrução Iterativa}

Como mencionado anteriormente, os algoritmos de \acs{FBP} têm sido os mais utilizados atualmente em tomossíntese da mama \cite{zheng2018detector}, pelo fato de serem rápidos e obterem resultados aproximados considerados bons \cite{das2011penalized}, porém devido à baixa amostragem no domínio da frequência, esses algoritmos introduzem erros na reconstrução e sofrem dificuldade de confecção dos filtros no domínio da frequência \cite{xu2015statistical}.

Os métodos iterativos demandam um alto custo computacional para a resolução dos problemas de reconstrução. Devido a esse fato, os mesmos não eram utilizados por equipamentos comerciais no passado e somente eram alvo de pesquisas científicas. Com o avanço do poder computacional essas técnicas tem chamado a atenção para aplicações em reconstrução de imagens médicas e demonstram ser promissoras para \acs{DBT} \cite{zeng2010medical,zheng2018detector}.

Para a resolução do problema de maneira iterativa é necessário primeiramente a discretização do modelo físico do sistema, como foi dito na seção \ref{DiscretizaçãodaImagem}. Após essa etapa, faz-se o uso de equações lineares para a solução do problema, de acordo com \citeonline[p. 125]{zeng2010medical}. Cada \textit{voxel} é denotado por $f_{j} (j=1,2,...,N)$ e cada projeção por $p_{i} (i=1,2,...,M)$. A Figura \ref{fig:imgCap3SistemaMatriz} ilustra um exemplo da retirada de um plano vertical da Figura \ref{fig:imgCap33DGeometry}, onde cada elemento de detector recebe somente um feixe de raio em cada projeção e cada \textit{voxel} possui tamanho unitário.  

\begin{figure}[H]
	\caption{Ilustração do princípio de reconstrução algébrico através de equações lineares simples.}
	\begin{center}
		\includegraphics[scale=0.55]{imgs/cap3/SistemaMatriz.pdf}
	\end{center}
	\legend{Fonte: Adaptado de \citeonline[p. 202]{buzug2008computed}}
	\label{fig:imgCap3SistemaMatriz}
\end{figure} 

Pode-se então relacionar o valor de cada \textit{voxel} com as projeções através das seguintes equações lineares:

\begin{equation}
\begin{cases}
f_{1} \, + f_{3} \, = p_{1}, \\ 
f_{2} \, + f_{4} \, = p_{2}, \\ 
\sqrt{2} \, f_{1} \, + \sqrt{2} \, f_{4} \, = p_{3}, \\ 
f_{1} \, + f_{2} \, = p_{4}, \\ 
\end{cases}
\label{eq:eqCap3EquacoesLineares}
\end{equation} 

\noindent sendo também possível reescrever o sistema acima na sua forma matricial, como é representado na equação abaixo:

\begin{equation}
P \, = A \, f,
\label{eq:eqCap3MatrizEquacoesLineares1}
\end{equation}

\noindent onde $P = [p_{1},p_{2},p_{3},...,p_{M}]^{T}$ é um vetor coluna que representa o valores do sinograma no espaço de Radon, ou seja as projeções, {$f = [f_{1},f_{2},f_{3},...,f_{N}]^{T}$ é um vetor coluna que representa os coeficientes de atenuação dos \textit{voxels} no espaço tridimensional e $A$ é uma matriz $MxN$, onde seu elemento $a_{ij}$ representa a contribuição que o ``j-ésimo'' \textit{voxel} $f_{j}$ apresenta para a atenuação do feixe de raio X que forma a ``i-ésima'' projeção $p_{i}$, por exemplo o valor de $\sqrt{2}$ em $f_{1}$ e $f_{4}$ para $p_{3}$. Tratando-se da matriz $A$, $M$ significa o número de elementos detectores vezes o número de projeções e $N$ representa o número de \textit{voxels} total do objeto \cite{zeng2010medical,levakhina2014three}.

Como o objetivo da reconstrução é encontrar os coeficientes de atenuação do objeto, deve-se então solucionar $f$. Para isso, se a matriz $A$ é inversível, a imagem reconstruída pode ser facilmente encontrada algebricamente através da equação abaixo:

 \begin{equation}
 f \, = A^{-1} \, P,
 \label{eq:eqCap3MatrizEquacoesLineares2}
 \end{equation}      

\noindent porém na prática esse processo é inviável devido a reconstrução tomográfica ser um problema mal condicionado e ao grande número de equações no sistema \cite{levakhina2014three}. Tendo isso em mente, a melhor alternativa é encontrar uma \textbf{solução aproximada} através de métodos matemáticos de otimização de problemas de uma forma iterativa \cite{buzug2008computed}.  

Para isso, deve-se otimizar uma função $E(x)$, normalmente chamada de função de custo, objetiva ou de energia, com o intuito de encontrar a melhor solução $\hat{x}$, ou a mais aproximada. Visto isso, é necessário minimizar ou maximizar os resultados de $E(x)$, de acordo com as equações a seguir:
 
\begin{equation}
\hat{x} = \underset{x\geq 0}{\arg\min} \; E(x) \;\;\;\; \text{ou} \;\;\;\; \hat{x} = \underset{x\geq 0}{\arg\max} \; E(x).
\label{eq:eqCap3Minimizacao}
\end{equation} 

Segundo \citeonline[p. 206]{buzug2008computed}, uma solução generalizada para a Equação \ref{eq:eqCap3MatrizEquacoesLineares1} é dada pela minimização de \eqref{eq:eqCap3MinimosQuadrados} dita como método dos mínimos quadrados, caracterizando-se por um problema de otimização:

\begin{equation}
\chi^{2} \, = \left\| A \, f \, - \,P \right\|^{2},
\label{eq:eqCap3MinimosQuadrados}
\end{equation} 

\noindent onde $E(x) = \chi^{2}$, ou seja, \eqref{eq:eqCap3MinimosQuadrados} é a função de energia. Outras técnicas são encontradas na literatura para a otimização de funções de custo, tais como métodos de gradientes descendentes, iterativos algébricos e estatísticos, os quais são abordados nos próximos itens.

%%%%%%%%%%%%%%%%%%%%%%%%%%%%%%%%%%%%%%%%%%%%%%%%%%%%%%%%%%%%%%%%%%%%%%%%%%%%%%%%%%%%%%%%%%%%%%%%%%%%%%%%%%%%%%%%%%%%%%%%%%%%%%						               Reconstrução Iterativa (Método Algébrico)    										%
%%%%%%%%%%%%%%%%%%%%%%%%%%%%%%%%%%%%%%%%%%%%%%%%%%%%%%%%%%%%%%%%%%%%%%%%%%%%%%%%%%%%%%%%%%%%%%%%%%%%%%%%%%%%%%%%%%%%%%%%%%%%%%

\subsubsection{Método Algébrico}

Como já foi dito, para a resolver o problema de reconstrução deve-se encontrar a solução para as equações demonstradas em \eqref{eq:eqCap3EquacoesLineares}, porém as aplicações práticas tendem a ter uma enorme quantidade de equações, tornando inviável a resolução por métodos usuais \cite[p. 210]{buzug2008computed}. 

Baseado nisso, um método muito encontrado na literatura é a técnica de reconstrução algébrica (\textit{Algebraic Reconstruction Technique} - \acs{ART}). Esse método visa resolver iterativamente um sistema enorme de equações lineares satisfazendo-as uma a uma, em outras palavras, estima uma imagem que solucione uma equação por vez \cite{rangayyan2004biomedical}. 

Segundo \citeonline[p. 211]{buzug2008computed}, esse método idealmente estima que a solução da imagem, i.e. o vetor {$f = [f_{1},f_{2},f_{3},...,f_{N}]^{T}$, seja um ponto no espaço $N$ dimensional, onde as $M$ equações de \eqref{eq:eqCap3EquacoesLineares} sejam hiperplanos que cruzam esse ponto, dado como solução.

Acompanhando a linha do autor, o exemplo abaixo ilustra como é dada a solução do problema através do método especificado. Para o exemplo, o mesmo é simplificado para $N = 2$, ou seja, são considerados somente dois \textit{pixels} e cada um é interceptado por somente um feixe de raio, tendo como resultado duas projeções $M=2$. Através disso, as equações podem ser modeladas seguindo o mesmo raciocínio de \eqref{eq:eqCap3EquacoesLineares}, demonstradas a seguir:

\begin{equation}
\begin{cases}
a_{11}f_{1} \, + a_{12}f_{2} \, = p_{1} \\ 
a_{21}f_{2} \, + a_{22}f_{2} \, = p_{2} \\  
\end{cases}.
\label{eq:eqCap3ARTEquacoesLineares}
\end{equation}

Cada equação, nesse caso, simboliza uma reta no espaço bidimensional, como demonstra a Figura \ref{fig:imgCap3ART}. Uma estimativa inicial $f^{(0)}$ deve ser feita e então esse vetor deve ser projetado perpendicularmente na reta que representa $p_{1}$ para obter a nova imagem $f^{(1)}$. O processo volta a se repetir, projetando $f^{(1)}$ em $p_{2}$ e assim até convergir para a solução desejada $f = (f_{1},f_{2})^{T}$. \colorbox{pink}{Elias: Corrigir a figura?}


\begin{figure}[H]
	\caption{Ilustração do processo de reconstrução através do método \acs{ART}.}
	\begin{center}
		\includegraphics[scale=1]{imgs/cap3/ART1.pdf}
	\end{center}
	\legend{Fonte: \citeonline[p. 212]{buzug2008computed}}
	\label{fig:imgCap3ART}
\end{figure} 

A equação abaixo demonstra a forma representativa para o cálculo do algoritmo \acs{ART}, retirado de \citeonline[p. 131]{zeng2010medical}: \colorbox{pink}{Elias: Corrigir a equação?}

\begin{equation}
f^{Próximo} = f^{Atual} - Retroprojeção_{raio}\{\dfrac{Projeção_{raio}(f^{Atual})- Medição_{raio}}{Fator\,Normalização}\}.  
\label{eq:eqCap3ARTEquacoesSimbolica}
\end{equation}

Os métodos ditados acima assumem um caso ideal. As projeções dos feixes em sistemas reais sofrem de inconsistências como ruídos e artefatos, bem como na prática o problema é constituído de inúmeras equações. Dispondo disso, a solução para as ditas equações lineares não é encontrada em somente um ponto, mas sim em uma região que possui múltiplas soluções. Nesse caso uma solução aproximada $\hat{f}$ deve ser estimada, no sistema caracterizado como possível e indeterminado \cite[p. 218]{buzug2008computed}. A Figura \ref{fig:imgCap3ARTIndeterminado} exemplifica o caso acima. 

\begin{figure}[H]
	\caption{Ilustração geométrica para a reconstrução de um sistema possível indeterminado.}
	\begin{center}
		\includegraphics[scale=1]{imgs/cap3/ART2.pdf}
	\end{center}
	\legend{Fonte: \citeonline[p. 218]{buzug2008computed}}
	\label{fig:imgCap3ARTIndeterminado}
\end{figure}

\colorbox{pink}{Elias: Corrigir essas afirmações} Variações do método algébrico, dito acima, são encontradas na literatura. O algoritmo que utiliza a técnica de reconstrução iterativa simultânea (\textit{Simultaneous Iterative Reconstruction Technique} - \acs{SIRT}) é uma evolução do método convencional algébrico e visa resolver todo o sistema de equações lineares e atualizar a estimativa da imagem $f$ ao fim de cada iteração, ao contrário do \acs{ART} que atualiza a cada raio \cite{yang2012numerical,zeng2010medical}. Já o método que utiliza a técnica de reconstrução algébrica simultânea (\textit{Simultaneous Algebraic Reconstruction Technique} - \acs{SART}), produz resultados superiores aos ditos anteriormente \cite{levakhina2014three,yang2012numerical}.   

%%%%%%%%%%%%%%%%%%%%%%%%%%%%%%%%%%%%%%%%%%%%%%%%%%%%%%%%%%%%%%%%%%%%%%%%%%%%%%%%%%%%%%%%%%%%%%%%%%%%%%%%%%%%%%%%%%%%%%%%%%%%%%						               Reconstrução Iterativa (Método Estatístico)    										%
%%%%%%%%%%%%%%%%%%%%%%%%%%%%%%%%%%%%%%%%%%%%%%%%%%%%%%%%%%%%%%%%%%%%%%%%%%%%%%%%%%%%%%%%%%%%%%%%%%%%%%%%%%%%%%%%%%%%%%%%%%%%%%

\subsubsection{Método Estatístico}\label{MétodoEstatístico}

Os problemas de reconstrução também podem ser resolvidos por meio de métodos estatísticos que são equivalentes aos problemas de otimização \cite[p. 79]{levakhina2014three}. O algoritmo da máxima verossimilhança (\textit{Maximum Likelihood} - \acs{ML}) é um exemplo de modelamento estatístico. Nesse caso, a função de custo torna-se a função de verossimilhança \cite[p. 77]{levakhina2014three}.

Segundo \citeonline[p. 10]{Fessler2000handbook}, esse método busca estimar um parâmetro, por exemplo os coeficientes de atenuação $(ImgEstimada)$, que maximizam a verossimilhança dado um conjunto de observações $(Medidas)$. Essa estimativa pode ser modelada através da função de verossimilhança representada por meio da equação abaixo e explicada com detalhes a seguir:

\begin{equation}
ImgReconstruida = \underset{ImgEstimada}{\arg\max} \; l(ImgEstimada \mid Medidas),
\label{eq:eqCap3ModeloVerossimilhança}
\end{equation} 

\noindent onde $l(ImgEstimada)$ é a função de verossimilhança a ser otimizada tendo as medidas das projeções.


De acordo com \citeonline[p. 230]{buzug2008computed}, em um equipamento que utiliza raios X de transmissão, o número de \textit{quanta} emitido pelo tubo e recebido pelo detector seguem a distribuição de Poisson, descrita matematicamente por:

\begin{equation}
P(Y_{i} = y_{i} \mid \hat{y_{i}}(f) ) = \dfrac{e^{-\hat{y_{i}}(f) } \, [\hat{y_{i}}(f)] ^{y_{i}}}  {y_{i}!},
\label{eq:eqCap3DistribuicaoPoisson}
\end{equation}

\noindent onde $Y_{i}$ é uma variável aleatória contando o número de \textit{quanta} em cada observação da projeção $y_{i}$. O valor observado $y_{i}$ difere do valor esperado $\hat{y_{i}}(f)$ , pelo fato do detector ser um contador de fótons que obedece a distribuição descrita acima, por radiações espalhadas de Comptom, \textit{crosstalk}\footnote{Interferência entre \textit{pixels} vizinhos que ocorrem em detectores indiretos de \acs{DBT}, causados pela difusão da luz no fósforo ou através dos cintiladores \cite{zheng2018detector}.} entre os detectores e também o tamanho finito dos mesmos contrariando as situações ideais da Equação \ref{eq:eqCap3BeerLambert1} \cite[p. 6]{Fessler2000handbook}. Esse valor esperado pode ser calculado segundo uma aproximação da lei de Beer-Lambert, demonstrada abaixo \cite[p. 9]{Fessler2000handbook}:

\begin{equation}
\hat{y_{i}}(f) = b_{i} \, e \, ^{-\hat{p_{i}}(f)} \, + r_{i},
\label{eq:eqCap3BeerLambertDiscreto2}
\end{equation}

\noindent onde $\hat{p_{i}}(f)$ é a função de atenuação que cada raio sofre, $b_{i}$ é o número médio de \textit{quanta} recebido pelo detector caso não haja nenhum objeto entre o tubo e o detector e $r_{i}$ modela os fatores externos mencionados acima. 

A função $\hat{p_{i}}(f)$ é calculada pela soma dos produtos da contribuição $a_{ij}$ para a atenuação $f_{j}$ de todos os \textit{voxel} $j$ no percurso do feixe de raio $i$, dado por: 


\begin{equation}
\hat{p_{i}}(f) = \sum_{j=1}^{N} a_{ij} \, f_{j}.
\label{eq:eqCap3BeerEsperancaAtenuacao}
\end{equation}

A probabilidade conjunta $P \mid (P = {y_{1},y_{2},y_{3},...,y_{M}})$ das projeções independentes $y_{i}$, dado a esperança $\hat{y_{i}}(f)$, é dado pela seguinte equação:   

\begin{equation}
P(P = p\mid f ) = \prod_{i=1}^{M}  \dfrac{e^{-\hat{y_{i}}(f)} \, [\hat{y_{i}}(f)]^{y_{i}}}  {y_{i}!},
\label{eq:eqCap3DistribuicaoPoissonConjunta}
\end{equation}

\noindent porém segundo \citeonline[p. 225]{buzug2008computed}, não faz sentido analisar a probabilidade de uma projeção $P$, ou seja uma observação, dado o valor esperado $\hat{y_{i}}(f)$ calculado através dos coeficientes de atenuação $f$, já que o intuito da resolução do problema é estimar os valores desses coeficientes $\hat{f}$. Nesse sentido, pode-se fazer o uso da \textbf{função de verossimilhança} para a resolução do problema.

A \textbf{verossimilhança} diz que uma observação qualquer $Y = y$ é uma evidência que favorece a hipótese A sobre a hipótese B, dado as duas hipóteses inicialmente, se a condição abaixo é satisfeita \cite{morettin2010}:

\begin{equation}
p_{A}(y) >p_{B}(y).
\label{eq:eqCap3LeiVerossimilhanca}
\end{equation}

Existe uma diferença entre probabilidade e verossimilhança. Quando em uma função de densidade, e.g. Equação \ref{eq:eqCap3DistribuicaoPoisson}, o parâmetro fixo é a esperança $\hat{y_{i}}(f)$ e o variável são as observações $y_{i}$, essa é denominada de função de probabilidade. Já se a observação é fixa e o parâmetro esperado é variável denomina-se \textbf{função de verossimilhança} \cite{morettin2010}.

Dito esses conceitos junto ao problema encontrado com a Equação \ref{eq:eqCap3DistribuicaoPoissonConjunta}, é plausível a utilização da função de verossimilhança para a resolução do mesmo, tendo como resultado a equação abaixo: 

\begin{equation}
l(f \mid P = p) = \prod_{i=1}^{M}  \dfrac{e^{-\hat{y_{i}}(f)} \, [\hat{y_{i}}(f)]^{y_{i}}}  {y_{i}!},
\label{eq:eqCap3DistribuicaoPoissonConjuntaVerossimilhanca1}
\end{equation} 
  
Sendo assim, a finalidade da função acima é de variar os valores de $f$ buscando o máximo valor de verossimilhança $l$, dado um conjunto de observações $P$ \cite[p. 231]{buzug2008computed}. A busca pelo valor máximo da equação está intimamente ligada com o conceito de verossimilhança ditada acima que busca a melhor hipótese $\hat{f}$ para os valores de $f$ dado o conjunto de observações $P$, como ilustra a figura abaixo. 


\begin{figure}[H]
	\caption{Ilustração de busca pela máxima verossimilhança.}
	\begin{center}
		\includegraphics[scale=0.4]{imgs/cap3/Verossimilhanca.pdf}
	\end{center}
	\legend{Fonte: do autor, 2018.}
	\label{fig:imgCap3Verossimilhanca}
\end{figure}   

Substituindo a Equação \ref{eq:eqCap3BeerLambertDiscreto2} na Equação \ref{eq:eqCap3DistribuicaoPoissonConjuntaVerossimilhanca1} e fazendo o uso da função logaritmo, a fim de simplificar os cálculos e por conveniência, resulta na Equação \ref{eq:eqCap3DistribuicaoPoissonConjuntaVerossimilhanca2}, desprezando constantes independentes de $f$ \cite[p. 10]{Fessler2000handbook}. 

\begin{equation}
L(f \mid P = p) = \log (l(f \mid P = p)) = \sum_{i=1}^{M} \, y_{i} \log (b_{i} \, e \, ^{-\hat{p_{i}}(f)} \, + r_{i}) \,-\,(b_{i} \, e \, ^{-\hat{p_{i}}(f)} \, + r_{i}),
\label{eq:eqCap3DistribuicaoPoissonConjuntaVerossimilhanca2}
\end{equation} 

\noindent onde a equação acima é a log-verossimilhança para \acs{CT} de transmissão e reescrevendo-a de maneira mais simples abaixo, podendo fazer uma analogia com a Equação \ref{eq:eqCap3ModeloVerossimilhança}. \colorbox{pink}{Alexandre: Minimizar ou maximizar? Analisar isso!}

\begin{equation}
\hat{f} = \underset{f\geq 0}{\arg\min} \; L(f \mid P),
\label{eq:eqCap3VerossimilhançaFinal}
\end{equation}

Ainda de acordo com \citeonline[p. 11]{Fessler2000handbook}, devido ao problema de reconstrução tomográfica ser mal condicionado, a maximização de $l(f)$ por si só leva a resultados de imagens ruidosas. Isso porque existe um grande conjunto de mapas de atenuação que se encaixam bem como o resultado da equação. Então, segundo o autor, a função de verossimilhança por si só não consegue encontrar o melhor resultado. Uma das soluções encontradas para esse problema, que é vastamente conhecida na literatura, é a introdução de um termo de regularização na função objetiva, ou seja, incorporar informações já conhecidas para limitar o espaço de soluções de um vasto conjunto para um subconjunto menor. Então a maximização da sua nova forma ``penalizada'' é feita para encontrar a melhor solução contida nesse subconjunto, como é demonstrado abaixo:     

\begin{equation}
\hat{f} = \underset{f\geq 0}{\arg\max} \; \Phi(f), \;\; \text{onde} \;\; \Phi(f) = l(f) - \beta R(f),
\label{eq:eqCap3VerossimilhançaPenalizada}
\end{equation}

\noindent ou simbolicamente:  \colorbox{pink}{Elias: letras acentuadas não estão em itálico}

\begin{equation}
(Função \;a \;posteriori) = (Função \;de \;Verossimilhança) - \beta (Função \;a \;priori).
\label{eq:eqCap3VerossimilhançaPenalizadaSimbolica}
\end{equation}

Nessa notação, a função $R(f)$ modela a restrição do sistema \colorbox{pink}{Elias: Mudar a descrição de R(f)}, o fator $\beta$ controla a força de influência dessa função e $l(f)$ é a função objetiva formulada anteriormente. Os algoritmos que introduzem essas técnicas são conhecidos como: \ac{PML}, métodos \textit{Bayesianos} ou \ac{MAP} \colorbox{pink}{Alexandre: MAP não está certo. Analisar}. 

A formulação de \eqref{eq:eqCap3VerossimilhançaPenalizada} é dada a partir do modelo Bayesiano, ou teorema de Bayes, demonstrado abaixo, que utiliza de conhecimentos de probabilidade \textit{a priori} para a estimativa de parâmetros \textit{a posteriori}, dado as probabilidades condicionais, de acordo com \citeonline[p. 144]{zeng2010medical}.  

\begin{equation}
Prob(f \mid P) = \dfrac{Prob(P \mid f) Prob(f)}{Prob(P)}.
\label{eq:eqCap3TeoremaBayes}
\end{equation}  

Aplicando o logaritmo na equação acima, tem-se \eqref{eq:eqCap3TeoremaBayesLog}, onde o elemento que não depende do coeficiente de atenuação $f$ é desprezado, gerando \eqref{eq:eqCap3TeoremaBayesLogSimplificada}:

\begin{equation}
\log(Prob(f \mid P)) = \log(Prob(P \mid f)) + \log(Prob(f)) - \log(Prob(P)),
\label{eq:eqCap3TeoremaBayesLog}
\end{equation}

\begin{equation}
\log(Prob(f \mid P)) = \log(Prob(P \mid f)) + \log(Prob(f)),
\label{eq:eqCap3TeoremaBayesLogSimplificada}
\end{equation}

\noindent onde o primeiro termo é o modelo estatístico da aquisição física do sistema, i.e. função de máxima verossimilhança \eqref{eq:eqCap3VerossimilhançaFinal}, e o segundo termo é o modelo estatístico que faz a restrição do sistema para que esse possa convergir para um solução apropriada. Simplificadamente tem-se a equação abaixo, sendo possível sua assimilação com a Equação \ref{eq:eqCap3VerossimilhançaPenalizadaSimbolica}.

\begin{equation}
\Phi(f) = l(f) - \beta R(f).
\label{eq:eqCap3TeoremaBayesLogSimplificadaFinal}
\end{equation}

É conhecido da literatura que os mapas de atenuações não variam bruscamente em uma pequena faixa de extensão, ou seja, os níveis de cinza entre os \textit{pixels} vizinhos não alteram significantemente em relação à média local \cite{Fessler2000handbook,buzug2008computed}. Essa informação pode ser considerada como um conhecimento \textit{a priori} sobre a imagem desejada e a partir desse fato, pode ser incorporado uma equação que desestimule a ``rispidez'' sobre a imagem processada, impondo restrições de suavidade. Sendo assim, essa equação pode ser modelada através de funções matemáticas e incorporada como uma restrição para o conjunto de soluções do modelo estatístico.

Um das maneiras mais simples de modelar uma função de restrição de suavidade na imagem é através da discrepância existente entre os valores dos \textit{pixels} vizinhos \cite[p. 12]{Fessler2000handbook}. Esse modelamento pode ser feito de acordo com \citeonline[p. 4]{das2011penalized}:

\begin{equation}
R(f) = \sum_{j=1}^{N}  \sum_{k=1}^{N_{i}} \omega_{jk} \, \psi(f_{j}-f_{k}),
\label{eq:eqCap3DiscrepanciaPixels}
\end{equation}  

\noindent onde $f$ é o coeficiente de atenuação, $\psi$ mede a discrepância entre $f_{j}$ e $f_{k}$, $N_{i}$ são os índices da vizinhança, $\omega_{jk} = \omega_{kj}$ define a influência do ``k-ésimo'' \textit{voxel} no ``j-ésimo'' \textit{voxel}. A influência de cada \textit{voxel} em seu vizinho é definida por: $\omega_{jk} = 1$ para os seis \textit{voxels} vizinhos, sendo dois em cada direção do eixo, $\omega_{jk} = 1/\sqrt{2}$ para os diagonais e $\omega_{jk} = 0$ para qualquer outro. A Figura \ref{fig:imgCap3PesosVizinhanca} exemplifica os pesos dados a cada vizinho, em um espaço \acs{2D}, sendo que esses pesos são normalizados.


\begin{figure}[H]
	\centering
	
	\caption{Pesos normalizados atribuídos para a vizinhança (a) quatro e (b) oito.}
	
	\subfloat[]{\includegraphics[scale=0.8]{imgs/cap3/PesoVizinhancaB.pdf}	\label{fig:imgCap3PesosVizinhancaA}}
	
	\subfloat[]{\includegraphics[scale=0.8]{imgs/cap3/PesoVizinhancaA.pdf} \label{fig:imgCap3PesosVizinhancaB}}
	
	\legend{Fonte: \citeonline[p. 2]{chen2009bayesian}}
	\label{fig:imgCap3PesosVizinhanca}
\end{figure}   



Ainda no contexto da restrição de suavidade, outra técnica utilizada é o modelamento da função de restrição por meio do Campo Aleatório Markoviano (Markov Random Field - \acs{MRF}), através do uso de distribuição de probabilidades que levam em conta o contexto espacial dos \textit{pixels} \cite[p. 71]{salvadeo2013filtragem}.  

Segundo a definição de \citeonline[p. 8]{li2009markov}, a teoria do Campo Aleatório Markoviano é um ramo da teoria de probabilidade que tem como função analisar as \textbf{dependências espaciais} ou contextuais de dado \textbf{fenômeno físico}. 

Com a finalidade de explicar todo o conceito por trás dessa teoria, é necessário a utilização de notações formais que auxiliarão todo o processo de formulação. Todas essas notações estão contidas no Apêndice \ref{ApendiceA:CamposAleatóriosMarkovianos}, junto com conceitos teóricos importantes a serem definidos. Recomenda-se ao leitor investir um tempo analisando esse capítulo a fim de compreender melhor a teoria descrita com detalhes a seguir.    

Dado todos os conceitos apresentados, um campo aleatório $F$ (\ref{ApendiceA:CampoAleatorio}) em $S$ (\ref{ApendiceA:SitesRotulos}) em relação a sua vizinhança $\nu$ (\ref{ApendiceA:sVizinhancaCliques}) é dito ser de Markov quando o campo atende as duas condições ditadas abaixo:  


\begin{enumerate}
	\item \textbf{Positividade}: $P(f) > 0, \;\; \forall\, f \,\in\, \varPsi$,
	\item \textbf{Markovianidade}: $P(f_{i} \mid f_{j},\;\; \forall \; j \in S,\; j \neq i) = P(f_{i}\mid f_{\nu_{i}})$,
\end{enumerate}

\noindent onde $f_{\nu_{i}} = \{f_{i^{'}} \mid i^{'} \in \nu_{i} \} $ é o conjunto de rótulos admitidos para os \textit{sites} na vizinhança de $i$. Em outras palavras, a \textbf{Markovianidade} dita as características locais do campo, onde somente a vizinhança tem influência direta no valor do \textit{site}, ou \textit{pixel} nesse caso. Já a \textbf{positividade}, quando satisfeita, diz que a probabilidade conjunta do campo $P(f)$ é determinada pelas probabilidades condicionais locais do campo \cite{li2009markov}.  

De acordo com \citeonline[p. 4]{won2013stochastic}, os \acs{MRF} em \acs{2D} são uma generalização não-causal das cadeias de Markov \acs{1D} proveniente das análises de sequências. Devido a representação de uma dependência de características locais do campo com a sua vizinhança de um modo não-causal, a probabilidade conjunta não pode ser fatorada\footnote{A fatoração é a conexão entre a probabilidade conjunta e a probabilidade condicional local, segundo \citeonline[p. 18]{won2013stochastic}} em probabilidades das características locais como é feito nas cadeias de Markov \acs{1D}. Tendo isso em mente é necessário a busca de um método que especifique a probabilidade conjunta de um \acs{MRF} por meio de suas características locais, ou seja, probabilidades condicionais locais. 

Isso só foi possível devido a equivalência entre o \acs{MRF} e o Campo Aleatório de Gibbs (\textit{Gibbs Random Field} - \acs{GRF}) dada pelo teorema de \citeonline[]{hammersley1971markov}. Com isso foi possível estabelecer uma relação entre a \textbf{probabilidade conjunta global} (\acs{GRF}) e a \textbf{probabilidade condicional local} (\acs{MRF}) e vice-versa.  

O teorema mencionado acima diz que: \textit{$F$ é um \acs{MRF} em $S$, dado uma vizinhança $\nu$, se e somente se $F$ é um \acs{GRF} em $S$, dado uma vizinhança $\nu$}. Em outras palavras, $F$ é dito ser de Markov somente se a sua probabilidade conjunta $P(f)$ seguir a distribuição de Gibbs (Apêndice \ref{ApendiceB:CamposAleatóriosdeGibbs}).

Diversos modelos distintos de \acs{MRF} podem ser encontrados na literatura \cite{li2009markov,won2013stochastic}, sendo que cada um é utilizado para uma determinada aplicação de acordo com suas funções potenciais. O modelo Gaussiano, denominado \ac{GMRF}, é bastante utilizado em problemas de restauração e reconstrução de imagens \cite{salvadeo2016nonlocal,xu2015statistical,jeng1991compound}. Sua equação de probabilidade condicional local é definida segundo \citeonline[p. 18]{li2009markov}:

\begin{equation}
P(f_{i} \mid f_{\nu_{i}}) = \dfrac{1}{\sqrt{2 \pi \sigma^{2}_{i}}}  e^{-\frac{1}{2\sigma^{2}_{i}} \, [f_{i}-\mu_{i}-\sum_{i^{'}\in \,\nu_{i}}^{} \beta_{i,i^{'}}(f_{i^{'}} - \mu_{i^{'}}) ]^{2} },
\label{eq:eqCap3GMRF1}
\end{equation} 

\noindent onde $\sigma^{2}_{i}$ e $\mu_{i}$ correspondem a variância e a média local respectivamente e $\beta_{i,i^{'}}$ corresponde aos coeficientes de interação dos cliques. 

A probabilidade conjunta é dada de maneira generalizada segundo \citeonline[p. 4]{xu2015statistical}:

\begin{equation}
P(f) \sim \prod_{i}^{N} \prod_{i^{'}}^{\nu_{i}} e^{\rho(\varDelta_{ii^{'}})},
\label{eq:eqCap3GMRF2}
\end{equation} 

\noindent onde: 

\begin{equation}
\rho(\varDelta_{ii^{'}}) = - \omega_{ii^{'}} \dfrac{\varDelta_{ii^{'}}^{2}}{2\sigma^{2}},
\label{eq:eqCap3GMRF3}
\end{equation}

\noindent $\varDelta_{ii^{'}}^{2} = (f_{i}-f_{i^{'}})^{2}$ é a função potencial quadrática e $\omega_{ii^{'}}$ são os pesos da vizinhança, calculados pelo inverso da distância euclidiana entre o \textit{pixel} central e seus vizinhos, dado por:

\begin{equation}
\omega_{ii^{'}} = \dfrac{1}{((x_{i}-x_{i^{'}})^{2}+((y_{i}-y_{i^{'}}))^{2})^{1/2}}.
\label{eq:eqCap3GMRF4}
\end{equation}

No ano de 2005, o trabalho de \citeonline[]{buades2005non} propôs a filtragem de imagens corrompidas por ruído Gaussiano por meio de um filtro da média não-local. Esse filtro explora o fato de que imagens em sua grande maioria possuem um alto grau de redundância. Basicamente, a filtragem é feita através da similaridade entre conjuntos de \textit{pixels}, denominados \textit{patches}, ao invés de ser feita somente com a vizinhança local como era anteriormente, em outras palavras, a remoção do ruído da imagem $I$ se dá através de uma média ponderada $w_{st}$ entre os \textit{patches} similares $P_{s}\, \text{e}\, P_{t} \, \forall \, t \in SW_{s}$. O \textit{patch} $P_{s}$ de um determinado \textit{pixel} $i_{s}$ pode ser caracterizado por uma região quadrada ao entorno daquele \textit{pixel}, e.g. $7x7$. Buscando a redução do custo computacional do método, uma janela de busca $SW$, e.g. $17x17$, é estipulada para a comparação dos \textit{patches}. O valor estimado $\hat{i}_{s}$ do \textit{pixel} livre de ruído é dado por meio de \eqref{eq:eqCap3NLM}, para todo $s$ pertencente a imagem $I$. A Figura \ref{fig:imgCap3Patches} ilustra como é feito o procedimento não-local. 

\begin{equation}
NLM(\hat{i}_{s}) = \sum_{t \,\in\, SW_{s}}^{} w_{st} \cdot i_{t},
\label{eq:eqCap3NLM}
\end{equation}


\begin{figure}[H]
	\caption{Ilustração do procedimento não-local baseado na similaridade entre os \textit{patches}.}
	\begin{center}
		\includegraphics[scale=1.3]{imgs/cap3/Patches.pdf}
	\end{center}
	\legend{Fonte: \citeonline[p. 2]{salvadeo2016nonlocal}}
	\label{fig:imgCap3Patches}
\end{figure}

\noindent onde o peso da similaridade $w_{st}$ entre os \textit{patches} é obtido por meio de \eqref{eq:eqCap3NLMw}, calculado pela distância Euclidiana entre os vetores $P_{s}\, \text{e}\, P_{t}$ de forma normalizada. É importante ressaltar que a soma dos pesos dentro da janela deve somar 1, em outras palavras, $\sum_{t\,\in \,SW_{s}}^{}w_{st} = 1$, o que leva a normalização dada pelo denominador da equação abaixo.

\begin{equation}
w(s,t) = \dfrac{e^{-\|P_{s} - P_{t}\|^{2}/h^{2}}}{\sum_{t\,\in \,SW_{s}}^{}e^{-\|P_{s} - P_{t}\|^{2}/h^{2}}},
\label{eq:eqCap3NLMw}
\end{equation}

\noindent onde $h$ é um parâmetro que controla o nível de borramento do filtro. 

Baseado no conceito não-local de similaridade, o trabalho de \citeonline[]{zhang2017applications} propõe a expansão dessa técnica para problemas inversos tais como: remoção de ruído, redução de borramento, reconstrução e reconstrução com alta resolução. O trabalho de \citeonline[]{salvadeo2016nonlocal} explorou os conceitos de \acs{MRF} não-local para remover ruídos de imagens.

Adicionando o conceito não-local no cálculo dos coeficientes de interação dos cliques da equação de \acs{GMRF}, temos a equação \eqref{eq:eqCap3GMRF1} modificada para:

\begin{equation}
P(f_{i} \mid f_{\nu_{i}}) = \dfrac{1}{\sqrt{2 \pi \sigma^{2}_{i}}}  e^{-\frac{1}{2\sigma^{2}_{i}} \, [f_{i}-\mu_{i}-\sum_{i^{'}\in \,\nu_{i}}^{} \omega(i,i^{'})\beta_{i,i^{'}}(f_{i^{'}} - \mu_{i^{'}}) ]^{2} },
\label{eq:eqCap3GMRFNL}
\end{equation}     

Após a formulação da função objetiva e da restrição do sistema é necessário resolver essa equação para a obtenção da imagem reconstruída. Segundo \citeonline[p. 12]{Fessler2000handbook} não existe uma fórmula fechada para a resolução da Equação \ref{eq:eqCap3TeoremaBayesLogSimplificadaFinal} de forma analítica. Desse modo, o uso de algoritmos iterativos é uma alternativa para a solução do problema. Ainda segundo o autor, um algoritmo iterativo primeiro estima uma imagem inicial $f^{(0)}$ como hipótese e então recursivamente busca uma sequência $f^{(1)}, f^{(2)}, \dots , f^{(n)}$ até convergir para o máximo ou mínimo da função. A Figura \ref{fig:imgCap3ProcedimentoIterativo} ilustra o esquemático de um algoritmo iterativo.


\begin{figure}[H]
	\caption{Esquemático do procedimento geral dos algoritmos de reconstrução iterativa.}
	\begin{center}
		\includegraphics[scale=0.4]{imgs/cap3/ProcedimentoIterativo.png}
	\end{center}
	\legend{Fonte: \citeonline[p. 129]{zeng2010medical}}
	\label{fig:imgCap3ProcedimentoIterativo}
\end{figure}

Diversos algoritmos para a otimização das funções objetivo são encontrados na literatura \cite{Fessler2000handbook,das2011penalized,zeng2010medical,sidky2014iterative,xu2015statistical,zheng2018detector}. No entanto nesse trabalho somente o algoritmo de Máxima Expectativa (\textit{Expectativa Maximization} - \acs{EM}) é discutido.

Considerando a constante $\beta = 0$ da Equação \ref{eq:eqCap3TeoremaBayesLogSimplificadaFinal} elimina-se a informação \textit{a priori} do sistema e então o problema volta a lidar somente com a equação de máxima verossimilhança \eqref{eq:eqCap3DistribuicaoPoissonConjuntaVerossimilhanca2}, ou do inglês \acs{ML}.  

Basicamente, o algoritmo de \acs{EM} é constituído por duas etapas \cite[p. 149]{zeng2010medical}. A primeira é a substituição da variável aleatória $y_{i}$ de \eqref{eq:eqCap3DistribuicaoPoissonConjuntaVerossimilhanca2} por sua expectativa, caracterizando o passo ``E''. O próximo passo é maximizar a nova equação, igualando a derivada de $f_{j}$ a zero, sendo essa etapa caracterizada por ``M''. Daí então que é derivado o algoritmo de \acs{MLEM}. A derivação matemática dos passos apresentados pode ser encontrada em \citeonline[p. 148-150]{zeng2010medical}. A equação matemática do algoritmo de \acs{MLEM} é dada por:  \colorbox{pink}{Elias: EM para emissão?}

\begin{equation}
f_{j}^{Próximo} = \dfrac{f_{j}^{Atual}}{\sum_{i}^{} a_{ij}} \sum_{i}^{} a_{ij} \, \dfrac{y_{i}}{\sum_{j}^{} a_{ij}f_{j}^{Atual}},
\label{eq:eqCap3MLEM}
\end{equation}    

\noindent ou de forma simbólicas:  \colorbox{pink}{Elias: Remover formar simbólica?}

\begin{equation}
f^{Próximo} = f^{Atual} \, \dfrac{Retroprojeção\left\{\dfrac{Medição}{Projeção(f^{Atual})}\right\}}{Retroprojeção\{1\}},
\label{eq:eqCap3MLEMSimbolico}
\end{equation} 

\noindent onde a retroprojeção do vetor 1 no denominador significa uma constante de normalização do volume.

Para incorporar a restrição \acs{MAP} no sistema, o algoritmo exposto acima pode ser modificado para sua forma denominada \textit{Green’s one-step late} \cite[p. 151]{zeng2010medical}, adicionando o termo de penalidade no denominador, demonstrado por: 

 \begin{equation}
 f_{j}^{Próximo} = \dfrac{f_{j}^{Atual}}{\sum_{i}^{} a_{ij} \, + \beta \dfrac{\partial U(F^{Atual})}{ \partial f_{j}^{Atual}}} \sum_{i}^{} a_{ij} \, \dfrac{y_{i}}{\sum_{j}^{} a_{ij}f_{j}^{Atual}},
 \label{eq:eqCap3MLEMMAP}
 \end{equation}
 
 \noindent onde $U(F)$ é a função de energia do modelo \acs{MAP}, dado nesse trabalho por \eqref{eq:eqApendiceBDistribuicaoGibbsU1} ou mais precisamente em \eqref{eq:eqCap3GMRFNL} no modelo não-local \acs{GMRF}.


%%%%%%%%%%%%%%%%%%%%%%%%%%%%%%%%%%%%%%%%%%%%%%%%%%%%%%%%%%%%%%%%%%%%%%%%%%%%%%%%%%%%%%%%%%%%%%%%%%%%%%%%%%%%%%%%%%%%%%%%%%%%%%											Redução de Artefatos Metálicos	  												 %
%%%%%%%%%%%%%%%%%%%%%%%%%%%%%%%%%%%%%%%%%%%%%%%%%%%%%%%%%%%%%%%%%%%%%%%%%%%%%%%%%%%%%%%%%%%%%%%%%%%%%%%%%%%%%%%%%%%%%%%%%%%%%%
\section{Redução de Artefatos de Alta Atenuação} \label{ReduçãodeArtefatosdeAltaAtenuação}

Artefatos inter-planos de objetos com alta atenuação são motivos de pesquisa em \acs{DBT} \cite{hu2008image,levakhina2013weighted,borges2017metal}. Como foi abordado na seção \ref{ProjeçãoeRetroprojeção3D}, a simples retroprojeção calcula as coordenadas de $(X,Y)$ para cada nível das fatias em $Z$ para todas as projeções (\ref{fig:imgCap3wBP2}). 	

\begin{figure}[H]
	\centering
	
	\caption{Ilustração da (a) projeção de objetos em diferentes níveis, seguido da (b) retroprojeção dos mesmos, (c) mostrando como os objetos se dispõem nos planos retroprojetados em um mesmo nível.}
	
	\subfloat[Projeção]{\includegraphics[scale=0.6]{imgs/cap3/BpPonderado/wBP1.png}\label{fig:imgCap3wBP1}}
	\hfil
	\subfloat[Retroprojeção]{\includegraphics[scale=0.6]{imgs/cap3/BpPonderado/wBP2.png}\label{fig:imgCap3wBP2}}
	\hfil
	\subfloat[Foco no processo de retroprojeção]{\includegraphics[scale=0.6]{imgs/cap3/BpPonderado/wBP3.png}\label{fig:imgCap3wBP3}}
		
	\legend{Fonte: \citeonline[p. 3]{levakhina2013weighted}.}
	\label{fig:imgCap3wBPIlustracao}
\end{figure}

A ocorrência desse tipo de artefato é inevitável devido ao fato da técnica de tomossíntese abranger uma estreita faixa de ângulo junto a um número limitado de projeções \cite{hu2008image}. 

O entendimento desse acontecimento fica mais claro a partir da Figura \ref{fig:imgCap3wBP3}, onde no plano em evidência o círculo vermelho aparece três vezes de forma borrada devido a esse objeto não estar localizado naquele plano. Já o triângulo fica em evidência pois o mesmo está localizado no plano de interesse. É interessante notar que a interferência do círculo se dá pelo aparecimento de cópias de acordo com o número de projeções realizadas e essas cópias se localizam na direção em que cada projeção foi adquirida.

De acordo com \citeonline[p. 4]{levakhina2013weighted}, o princípio do borramento se dá pelo fato de que na retroprojeção dos dados, as estruturas que estão contidas no plano de interesse se coincidem e mutuamente contribuem para aparecerem em foco, e.g. voxel $x_{1}$ na Figura \ref{fig:imgCap3wBP3}, já as estruturas que não pertencem àquele plano não se coincidem e então geram cópias borradas de suas estruturas, e.g. voxel $x_{2}$ na Figura \ref{fig:imgCap3wBP3}.

Como mencionado anteriormente, normalmente a combinação matemática (Figura \ref{fig:imgCap3FluxogramaBP}) utilizada para os planos retroprojetados é uma simples soma ou média. Segundo \citeonline[]{borges2017metal}, uma solução para a diminuição dos artefatos de alta atenuação é a realização de uma média ponderada entre os planos retroprojetados.   

No trabalho de \citeonline[]{borges2017metal} foi encontrado um \textit{patch} como sendo o referência entre os planos e a distância euclidiana entre o escolhido e os outros foi então calculada. Baseado nessa distância, foi estipulado um peso para os \textit{patches} de acordo com a função logística \eqref{eq:eqCap3wBPLogistic}. Após a estipulação dos pesos, foi realizado a média ponderada entre os \textit{pixels} centrais daqueles \textit{patches}. A Figura \ref{fig:imgCap3wBPPesos} ilustra o gráfico resultado da aplicação da função logística para o cálculo dos pesos em função da distância, variando o valor da constante M de inclinação da curva.

\begin{equation}
w_{x}(d_{x,y}) = 1 - \dfrac{1}{1 + e^{\frac{4,6}{M}(M-2d_{x,y})}} 
\label{eq:eqCap3wBPLogistic}
\end{equation}

\begin{figure}[H]
	\caption{Gráfico da atribuição dos pesos em relação a distância.}
	\begin{center}		
	\includegraphics[scale=0.6]{imgs/cap3/wBPPesos.pdf}
	\end{center}
	\legend{Fonte: do autor, 2018.}
	\label{fig:imgCap3wBPPesos}
\end{figure} 




	

\chapter[Materiais e Métodos]{Materiais e Métodos}\label{Capitulo4}
 

%%%%%%%%%%%%%%%%%%%%%%%%%%%%%%%%%%%%%%%%%%%%%%%%%%%%%%%%%%%%%%%%%%%%%%%%%%%%%%%%%%%%%%%%%%%%%%%%%%%%%%%%%%%%%%%%%%%%%%%%%%%%%%														Phantom																%
%%%%%%%%%%%%%%%%%%%%%%%%%%%%%%%%%%%%%%%%%%%%%%%%%%%%%%%%%%%%%%%%%%%%%%%%%%%%%%%%%%%%%%%%%%%%%%%%%%%%%%%%%%%%%%%%%%%%%%%%%%%% 
\section{Materiais}

No presente trabalho foram utilizados \textit{phantoms} a fim de simular todos os métodos de reconstrução. Um \textit{phantom} pode ser físico ou virtual e tem como objetivo simular condições reais de exame para que os métodos alvos de pesquisas possam ser estudados e testados de maneira sistemática. Isso se deve ao fato de que exames clínicos, ou seja, com pacientes, não podem ser realizados toda vez que um novo método necessitar ser testado.

\subsection{\textit{Phantom} Virtual}

Uma versão modificada de \citeonline[]{shepp1974fourier} foi usada. O \textit{phantom} originalmente é constituído por uma seção da cabeça que foi desenvolvido para testar o algoritmo proposto pelos autores naquela época. Esse \textit{phantom} simula diversas estruturas anatômicas com diferentes densidades $D$ como: a água ($D=1$) nos ventrículos, massa cinzenta ($D=1,2$), tumores ($D=1,03;1,04$) e para o crânio ($D=2$). A Figura \ref{fig:imgCap4SheppLogan} ilustra o \textit{phantom} de Shepp-Logan em sua versão modificada.

\begin{figure}[H]
	\caption{Ilustração do \textit{phantom} de Shepp-Logan em sua versão modificada.}
	\begin{center}
		\includegraphics[scale=1]{imgs/cap4/SheppLogan.png}
	\end{center}
	\legend{Fonte: do autor, 2018.}
	\label{fig:imgCap4SheppLogan}
\end{figure}   

Como esse trabalho visa a reconstrução de estruturas anatômicas em \acs{3D} pela técnica de tomossíntese, deve-se usar um objeto tri-dimensional para validar as técnicas aplicadas. Tendo isso em mente, foi utilizado um algoritmo\footnote{\url{www.mathworks.com/matlabcentral/fileexchange/9416-3d-shepp-logan-phantom}}, retirado da comunidade \textit{MathWorks}\footnote{\url{www.mathworks.com}}, que gera uma extensão do \textit{phantom} Shepp-Logan em \acs{3D}. A Figura \ref{fig:imgCap4SheppLogan3D} ilustra o \textit{phantom} criado pelo respectivo algoritmo e suas medidas físicas são descritas na seção seguinte.

\begin{figure}[H]
	\caption{Ilustração do \textit{phantom} de Shepp-Logan \acs{3D}.}
	\begin{center}
		\includegraphics[scale=0.3]{imgs/cap4/SheppLogan3D.pdf}
	\end{center}
	\legend{Fonte: do autor, 2018.}
	\label{fig:imgCap4SheppLogan3D}
\end{figure} 

\subsection{\textit{Phantom} Físico}

Já para os testes envolvendo a redução de artefatos de alta atenuação foi usado o \textit{phantom} físico BR3D\footnote{\url{www.cirsinc.com/products/all/51/br3d-breast-imaging-phantom/}} \cite{PhantomBR3D}, como ilustra a Figura \ref{fig:imgCap4BR3D}. Esse \textit{phantom} é constituído de 6 fatias com um material que simula o tecido $100\%$ adiposo e $100\%$ glandular de uma mama. Junto ao tecido estão contidas microcalcificações, fibrilas e massas em formato de esfera. Todas essas estruturas possuem diferentes tamanhos e buscam mimetizar os aspectos reais encontrados em uma mama com possíveis lesões.   

\begin{figure}[H]
	\caption{Ilustração do \textit{phantom} BR3D.}
	\begin{center}
		\includegraphics[scale=0.34]{imgs/cap4/BR3D.jpeg}
	\end{center}
	\legend{Fonte: do autor, 2018.}
	\label{fig:imgCap4BR3D}
\end{figure} 
   

%%%%%%%%%%%%%%%%%%%%%%%%%%%%%%%%%%%%%%%%%%%%%%%%%%%%%%%%%%%%%%%%%%%%%%%%%%%%%%%%%%%%%%%%%%%%%%%%%%%%%%%%%%%%%%%%%%%%%%%%%%%%%%													Geometria																%
%%%%%%%%%%%%%%%%%%%%%%%%%%%%%%%%%%%%%%%%%%%%%%%%%%%%%%%%%%%%%%%%%%%%%%%%%%%%%%%%%%%%%%%%%%%%%%%%%%%%%%%%%%%%%%%%%%%%%%%%%%%% 
\section{Geometria} 

%\subsection{\textit{Phantom} Virtual}
 
Para a aplicação dos métodos no \textit{phantom} virtual foi proposto uma geometria física do sistema que se aproxima de um equipamento de \acs{DBT} comercial. A Tabela \ref{tab:tabCap4ParametrosPhantoms} expõe os parâmetros físicos do sistema que foram utilizados, seguindo as propriedades geométricas de um sistema hipotético de \acs{DBT}. %O leitor pode fazer uma comparação com os valores dos equipamentos comerciais demonstrados na Tabela \ref{tab:tabCap2SistemasDBT}.

A Figura \ref{fig:imgCap4GeometriaProjecao} ilustra a geometria que foi usada para a projeção e retroprojeção, sendo que a Figura \ref{fig:imgCap4GeometriaProjecao1} representa a projeção de número 1, \ref{fig:imgCap4GeometriaProjecao2} de número 9 e \ref{fig:imgCap4GeometriaProjecao3} de número 5, onde os ângulos são: $-1,25\degree$, $1,25\degree$ e $0\degree$ respectivamente. O cubo azul representa espacialmente o \textit{phantom} virtual da Figura \ref{fig:imgCap4SheppLogan3D}, o plano em amarelo a posição da projeção do objeto no detector, o plano pontilhado as extremidades do detector e o ponto vermelho o tubo de raios X. 


\begin{table}[H]
	\footnotesize
	\centering
	\caption{Parâmetros físicos do sistema implementado com \textit{phantom} virtual e do sistema comercial utilizado pelo \textit{phantom} BR3D.}
	\label{tab:tabCap4ParametrosPhantoms}
	\begin{tabular}{l|c|c}
		\textbf{Parâmetro}                            &  \textbf{Shepp-Logan} &  \textbf{BR3D}  \\
		[5pt]
		\hline
		\hline
		\rule[-0.5ex]{0pt}{3ex}
		Tamanho do detector 						  &     28x35cm       &      24x30cm       \\ \hline
		\rule[-0.5ex]{0pt}{3ex}
		Tamanho do \textit{pixel}                     &       1mm        &     100$\mu$m      \\ \hline
		\rule[-0.5ex]{0pt}{3ex}
		Número de projeções                           &        9         &         9          \\ \hline
		\rule[-0.5ex]{0pt}{3ex}
		Angulação do tubo                             &   2,5$\degree$   &    25$\degree$     \\ \hline
		\rule[-0.5ex]{0pt}{3ex}
		Angulação do detector                         &   Estacionário   &    Estacionário    \\ \hline
		\rule[-0.5ex]{0pt}{3ex}
		Tamanho do objeto                             & 128x128x128 u.v. & 2004x1092x129 u.v. \\ \hline
		\rule[-0.5ex]{0pt}{3ex}
		Tamanho do \textit{Voxel}                     &     1x1x10mm     &   0.1x0.1x0.5mm    \\ \hline
		\rule[-0.5ex]{0pt}{3ex}
		Distância fonte para detector                 &      660cm       &       660mm        \\ \hline
		\rule[-0.5ex]{0pt}{3ex}
		Distância detector para centro de rotação     &       40cm       &        40mm        \\ \hline
		\rule[-0.5ex]{0pt}{3ex}
		Tamanho lacuna de ar                          &       22cm       &        22mm        \\ \hline
	\end{tabular}
	\vspace{2ex}
	\legend{Fonte: \citeonline{michell2018role,vedantham2015digital,sechopoulos2013review,baker2011breast}}
\end{table}


\begin{figure}[t!]
	\centering
	
	\caption{Geometria utilizada na aplicação dos métodos, onde (a) representa a projeção de número 1, (b) de número 9 e (c) de número 5, onde os ângulos são: $-1,25\degree$, $1,25\degree$ e $0\degree$ respectivamente.}
	
	\subfloat[]{\includegraphics[scale=.5]{imgs/cap4/Geometria/Proj1.png}\label{fig:imgCap4GeometriaProjecao1}}
	\subfloat[]{\includegraphics[scale=.5]{imgs/cap4/Geometria/Proj9.png}\label{fig:imgCap4GeometriaProjecao2}}
	
	\subfloat[]{\includegraphics[scale=.55]{imgs/cap4/Geometria/Proj5.png}\label{fig:imgCap4GeometriaProjecao3}}
	
	\legend{Fonte: do autor, 2018.}
	\label{fig:imgCap4GeometriaProjecao}
\end{figure}

 
%\subsection{\textit{Phantom} Físico}\label{PhantomReal}

Já para o \textit{phantom} BR3D, as projeções foram realizadas em um equipamento da empresa \acs{GE} modelo Senographe Essential\texttrademark, ilustrado pela Figura \ref{fig:imgCap4EquipamentoGE}. A geometria utilizada pelo equipamento é detalhada também pela Tabela \ref{tab:tabCap4ParametrosPhantoms}.

%\begin{table}[H]
%	\footnotesize
%	\centering
%	\caption{Parâmetros físicos do sistema comercial.}
%	\label{tab:tabCap4ParametrosGESenographeEssential}
%	\begin{tabular}{l|c}
%		\textbf{Parâmetro}                            &   \textbf{Valor}   \\
%		[5pt]
%		\hline
%		\hline
%		\rule[-0.5ex]{0pt}{3ex}
%		Tamanho do detector 						  &      24x30cm       \\ \hline
%		\rule[-0.5ex]{0pt}{3ex}
%		Tamanho do \textit{pixel}                     &     100$\mu$m      \\ \hline
%		\rule[-0.5ex]{0pt}{3ex}
%		Número de projeções                           &         9          \\ \hline
%		\rule[-0.5ex]{0pt}{3ex}
%		Angulação do tubo                             &    25$\degree$     \\ \hline
%		\rule[-0.5ex]{0pt}{3ex}
%		Angulação do detector                         &    Estacionário    \\ \hline
%		\rule[-0.5ex]{0pt}{3ex}
%		Tamanho do objeto                             & 2004x1092x129 u.v. \\ \hline
%		\rule[-0.5ex]{0pt}{3ex}
%		Tamanho do \textit{Voxel}                     &   0.1x0.1x0.5mm    \\ \hline
%		\rule[-0.5ex]{0pt}{3ex}
%		Distância fonte para detector                 &       660mm        \\ \hline
%		\rule[-0.5ex]{0pt}{3ex}
%		Distância detector para centro de rotação     &        40mm        \\ \hline
%		\rule[-0.5ex]{0pt}{3ex}
%		Tamanho lacuna de ar                          &        22mm        \\ \hline
%	\end{tabular}
%	\vspace{2ex}
%	\legend{Fonte: \citeonline{michell2018role,vedantham2015digital,sechopoulos2013review,baker2011breast}}
%\end{table}

\begin{figure}[H]
	\centering
	
	\caption{(a) Equipamento de tomossíntese modelo Senographe Essential\texttrademark \ utilizado para realizar as projeções e (b) o mesmo com o \textit{phantom} BR3D.}
	
	\subfloat[]{\includegraphics[scale=0.08]{imgs/cap4/EquipamentoGE.jpg}\label{fig:imgCap4EquipamentoGEA}}
	\hfil
	\subfloat[]{\includegraphics[scale=.49]{imgs/cap4/BR3Deq.jpeg}\label{fig:imgCap4EquipamentoGEB}}
	
	\legend{Fonte: do autor, 2018.}
	\label{fig:imgCap4EquipamentoGE}
\end{figure} 


%%%%%%%%%%%%%%%%%%%%%%%%%%%%%%%%%%%%%%%%%%%%%%%%%%%%%%%%%%%%%%%%%%%%%%%%%%%%%%%%%%%%%%%%%%%%%%%%%%%%%%%%%%%%%%%%%%%%%%%%%%%%%%														Métodos																%
%%%%%%%%%%%%%%%%%%%%%%%%%%%%%%%%%%%%%%%%%%%%%%%%%%%%%%%%%%%%%%%%%%%%%%%%%%%%%%%%%%%%%%%%%%%%%%%%%%%%%%%%%%%%%%%%%%%%%%%%%%%% 
\section{Métodos} 

Para os métodos de reconstrução, uma \textit{toolbox}\footnote{www.mathworks.com/matlabcentral/fileexchange/35548-3d-cone-beam-ct--cbct--projection-backprojection-fdk--iterative-reconstruction-matlab-examples.} foi utilizada para auxílio da implementação dos métodos propostos. Foram aplicadas as técnicas de simples retroprojeção, retroprojeção filtrada e o método estatístico de máxima verossimilhança. Entretanto antes de implementar essas técnicas é necessário a obtenção das projeções dado o volume do objeto. Em função disso, são discutidos os métodos utilizados para a obtenção delas.  

\subsection{Projeção e Retroprojeção} 

Para a implementação da projeção direta foi utilizado o método de \textit{Pixel Driven} (Figura \ref{fig:imgCap3Projetores1}) descrito nos capítulos anteriores. Para isso então são utilizadas as Equações \ref{eq:eqCap3ProjectionY} e \ref{eq:eqCap3ProjectionX}, considerando a distância entre o detector e o eixo de rotação igual a zero, ou seja, $D=0$. 

 \colorbox{pink}{Reescrever esse parágrafo} De modo que nenhum \textit{pixel} no detector fique sem valor, um processo inverso é feito, ou seja, para cada \textit{pixel} no detector é calculado o respectivo \textit{voxel} do qual se atribui o seu valor. O mesmo é feito para a retroprojeção, onde são calculados para todos os \textit{voxels} qual o respectivo \textit{pixel} que contribui para o seu valor. 


 Os cálculos das projeções e retroprojeções são feitos com o objetivo de converter as coordenadas do mundo, e.g. $(X,Y,Z)$, para as coordenadas da imagem, e.g. $(x_{i},y_{i})$, porém após isso ainda são necessárias as conversões das coordenadas da imagem para as coordenadas em \textit{pixel}, e.g. $(i,j)$. Isso pode ser feito segundo as Equações \ref{eq:eqCap4ConversaoCoordImgparaPixelY} e \ref{eq:eqCap4ConversaoCoordImgparaPixelX}, seguido por uma interpolação linear entre as coordenadas em \textit{pixel} e a imagem a ser projetada ou reprojetada, onde $dx\,\text{e}\,dy$ são os tamanhos dos \textit{pixels} em (mm) para as coordenadas $x\,\text{e}\,y$ respectivamente. A Figura \ref{fig:imgCap4ConversaoCoord} ilustra a relação entre as coordenadas e os pseudocódigos abaixo resumem todo o processo. 
 
\begin{equation}
i = \dfrac{y_{i}}{dy} + y_{0}  
\label{eq:eqCap4ConversaoCoordImgparaPixelY}
\end{equation} 
 
\begin{equation}
 j = -\dfrac{x_{i}}{dx} + x_{0}.  
\label{eq:eqCap4ConversaoCoordImgparaPixelX}
\end{equation} 
 
 
 \begin{figure}[H]
 	\caption{Ilustração da relação entre as coordenadas.}
 	\begin{center}
 		\includegraphics[scale=1]{imgs/cap4/ConversaoCoord.pdf}
 	\end{center}
 	\legend{Fonte: do autor, 2018.}
 	\label{fig:imgCap4ConversaoCoord}
\end{figure} 

\begin{algorithm}[H]
	\label{alg:algProjecao}
	\caption{Projeção}
	\Entrada{$Volume3D$, $Parâmetros$} 
	\Saida{$Projeções$}
	\Inicio{
		
		\Para{cada $Projeção \in Projeções$}{
			$\theta \leftarrow Ângulo(Projeção)$
			
			\Para{cada $Fatia \in Volume3D$}{
				
				Calcular $Y \;\text{e}\; X \; \forall \; (y_{i}, x_{i}) \; \in Projeção$
				
				Calcular $i \;\text{e}\; j \; \forall \; (Y, X)$
				
				$Projeção \leftarrow Projeção + \text{Interpolação}(Fatia\,,\,(i,j)\,)$			
			}
		}
	}
	\Retorna{$Projeções$}
\end{algorithm}


\begin{algorithm}[H]
	\label{alg:algRetroprojecao}
	\caption{Retroprojeção}
	\Entrada{$Projeções$, $Parâmetros$} 
	\Saida{$Volume3D$}
	\Inicio{
		
		\Para{cada $Projeção \in Projeções$}{
			$\theta \leftarrow Ângulo(Projeção)$
			
			\Para{cada $Fatia \in Volume3D$}{
				
				Calcular $y_{i} \;\text{e}\; x_{i} \; \forall \; (X,Y,Z) \; \in Fatia$
				
				Calcular $i \;\text{e}\; j \; \forall \; (y_{i}, x_{i})$
				
				$Fatia \leftarrow Fatia + \text{Interpolação}(Projeção\,,\,(i,j)\,)$			
			}
		}
	}
	\Retorna{$Volume3D$}
\end{algorithm}


Após a aplicação dos passos mencionados acima em todo o volume \acs{3D}, são geradas as projeções da Figura \ref{fig:imgCap4Projecoes}, onde as imagens de (a) até (i) representam as projeções de 1 a 9 respectivamente. A partir dessas projeções é possível fazer a reconstrução do volume aplicando os métodos propostos.
 
\begin{figure}[H]
	\centering
	
	\caption{Projeções geradas a partir da Figura \ref{fig:imgCap4GeometriaProjecao}, onde de (a) até (i) representam as projeções de 1 a 9.}
	
	\subfloat[]{\includegraphics[scale=.45]{imgs/cap4/Proj/1.png}\label{fig:imgCap4Projecao1}}
	\subfloat[]{\includegraphics[scale=.45]{imgs/cap4/Proj/2.png}\label{fig:imgCap4Projecao2}}
	\subfloat[]{\includegraphics[scale=.45]{imgs/cap4/Proj/3.png}\label{fig:imgCap4Projecao3}}
	\subfloat[]{\includegraphics[scale=.45]{imgs/cap4/Proj/4.png}\label{fig:imgCap4Projecao4}}
	\subfloat[]{\includegraphics[scale=.45]{imgs/cap4/Proj/5.png}\label{fig:imgCap4Projecao5}}
	\subfloat[]{\includegraphics[scale=.45]{imgs/cap4/Proj/6.png}\label{fig:imgCap4Projecao6}}
	\subfloat[]{\includegraphics[scale=.45]{imgs/cap4/Proj/7.png}\label{fig:imgCap4Projecao7}}
	\subfloat[]{\includegraphics[scale=.45]{imgs/cap4/Proj/8.png}\label{fig:imgCap4Projecao8}}
	\subfloat[]{\includegraphics[scale=.45]{imgs/cap4/Proj/9.png}\label{fig:imgCap4Projecao9}}
	
	\legend{Fonte: do autor, 2018.}
	\label{fig:imgCap4Projecoes}
\end{figure}

%%%%%%%%%%%%%%%%%%%%%%%%%%%%%%%%%%%%%%%%%%%%%%%%%%%%%%%%%%%%%%%%%%%%%%%%%%%%%%%%%%%%%%%%%%%%%%%%%%%%%%%%%%%%%%%%%%%%%%%%%%%%%%												Reconstrução																%
%%%%%%%%%%%%%%%%%%%%%%%%%%%%%%%%%%%%%%%%%%%%%%%%%%%%%%%%%%%%%%%%%%%%%%%%%%%%%%%%%%%%%%%%%%%%%%%%%%%%%%%%%%%%%%%%%%%%%%%%%%%%
\subsection{Reconstrução de Imagens} 

Após a execução do passo a passo exemplificado anteriormente é possível a aplicação dos métodos de reconstrução. São utilizadas três técnicas para a comparação do melhor resultado: retroprojeção, retroprojeção filtrada e máxima verossimilhança. Para o algoritmo estatístico, diferentes quantidades de iterações são aplicadas no teste, variando de cinco até vinte iterações. O método de simples retroprojeção foi descrito no subitem anterior e os pseudocódigos abaixo descrevem o funcionamento dos métodos de \acs{FBP} e \acs{MLEM}. A confecção do filtro para o algoritmo de retroprojeção filtrada se dá a partir dos passos ditos no item \ref{RetroprojeçãoFiltrada}.

\begin{algorithm}[H]
	\label{alg:algRetroprojecaoFiltrada}
	\caption{Retroprojeção Filtrada}
	\Entrada{$Projeções$, $Parâmetros$} 
	\Saida{$Volume3D$}
	\Inicio{
		
		\Para{cada $Projeção \in Projeções$}{
			$Projeção \leftarrow Filtrar(Projeção)$	
		}
		$Volume3D \leftarrow Retroprojeção(Projeções)$
	}
	\Retorna{$Volume3D$}
\end{algorithm}

\begin{algorithm}[H]
	\label{alg:algMlEM}
	\caption{MLEM}
	\Entrada{$Projeções$, $Parâmetros$} 
	\Saida{$Volume3D$}
	\Inicio{
	$Volume3D \leftarrow EstimativaInicial$	
		
	$FatorNorm \leftarrow Retroprojeção(1)$	
	
		\Para{cada Iteração}{
		$EstimativaProjeções \leftarrow Projeção(Volume3D)$	
		
		$RazãoProjeções \leftarrow Projeções \;/\; EstimativaProjeções$
		
		$VolumeEstimado \leftarrow Retroprojeção(RazãoProjeções)$
		
		$FatorModificação \leftarrow VolumeEstimado \;/\; FatorNorm$
		
		$Volume3D \leftarrow Volume3D \;X\; FatorModificação$
		}
	}
	\Retorna{$Volume3D$}
\end{algorithm}


\subsection{Redução de Artefatos de Alta Atenuação}\label{MetodosReduçãodeArtefatosdeAltaAtenuação} 

Para a implementação dos métodos de redução de artefatos, utilizou-se os materiais ditos na seção anterior. Em seguida aplicou-se os métodos de \citeonline[]{borges2017metal} de redução de artefatos metálicos com o intuito de avaliar os resultados apresentados. 

A constante $M$ da Equação \ref{eq:eqCap3wBPLogistic} foi ajustada de maneira a balancear o nível de ruído e a redução dos artefatos de alta atenuação na imagem.


Foram recortadas regiões de interesse (\textit{Region of Interest} - \acs{ROI}) de tamanho $90x90$, para aplicar a técnica abordada. As coordenadas da \acs{ROI} foram de 1440 até 1529 em Y e 860 até 949 em X. A Figura \ref{fig:imgCap4ROIGE} ilustra essa região referente a fatia 74 reconstruída pelo equipamento comercial assim como a Figura \ref{fig:imgCap4ROIBP} ilustra a mesma região com uma técnica de reconstrução por retroprojeção simples. 

Foram selecionadas somente as fatias de 67 a 75 para a realização dos procedimentos. Isso se deve ao fato de que o método é computacionalmente caro.

\begin{figure}[H]
	\centering
	
	\caption{\acs{ROI} reconstruídas pelo modelo (a) comercial e através da (b) retroprojeção simples a fim de testar os métodos de redução de artefatos.}
	
	\subfloat[\acs{ROI} da fatia 74 - Comercial.]{\includegraphics[scale=.5]{imgs/cap4/GEroi.png}\label{fig:imgCap4ROIGE}}
	\hfil
	\subfloat[\acs{ROI} da fatia 74 - \acs{BP}]{\includegraphics[scale=.5]{imgs/cap4/BProi.png}\label{fig:imgCap4ROIBP}}
	
	\legend{Fonte: do autor, 2018.}
	\label{fig:imgCap4ROI}
\end{figure} 

\subsection{Validação} 

Para analisar de maneira objetiva os resultados das imagens, utilizou-se métricas de avaliação da qualidade da imagem, conhecidas na literatura, como as descritas abaixo:

O Erro quadrático médio (\textit{Mean Square Error} - \acs{MSE}) é uma forma de medida pontual que analisa as diferenças de intensidade entre uma imagem referência e outra estimada. Esse método é utilizado devido a sua simplicidade de cálculo, obtenção de resultados com fácil interpretação e baixa complexidade computacional, porém críticas são feitas ao mesmo devido a sua ineficiência quando se comparado com as métricas subjetivas da percepção humana \cite{gonzalez2008digital,wang2004image}. Sua equação matemática é dada por:

\begin{equation}
MSE = \dfrac{1}{MN} \sum_{x=1}^{M} \sum_{y=1}^{N} [f(x,y) - \hat{f}(x,y)]^{2},
\label{eq:eqCap4MSE}
\end{equation} 

\noindent ou em sua versão aplicando a raiz quadrada:

\begin{equation}
RMSE = \sqrt{MSE(f,\hat{f})},
\label{eq:eqCap4RMSE}
\end{equation}
 
\noindent ou ainda em sua versão normalizada:

\begin{equation}
NRMSE = \dfrac{RMSE(f,\hat{f})}{max(\hat{f}) - min(\hat{f})},
\label{eq:eqCap4NRMSE}
\end{equation}

%\noindent onde:
%
%\begin{equation}
%S_{f} = \dfrac{1}{MN} \sum_{x=1}^{M} \sum_{y=1}^{N} f(x,y)  \;\; S_{\hat{f}} = \dfrac{1}{MN} \sum_{x=1}^{M} \sum_{y=1}^{N} \hat{f}(x,y).
%\label{eq:eqCap4NRMSESxSy}
%\end{equation}


A métrica de Índice de Similaridade Estrutural (\textit{Structural Similarity Index} - \acs{SSIM}) tem como função fazer uma avaliação que seja mais fiel a percepção humana e a formação das imagens, segundo \citeonline[]{wang2004image}. Esse método avalia componentes como a  luminância $l(\cdot)$, o contraste $c(\cdot)$ e a estrutura dos objetos $s(\cdot)$. Sua formulação matemática é dada por:

\begin{equation}
SSIM(x,y) = [l(x,y)]^{\alpha} \cdot [c(x,y)]^{\beta} \cdot [s(x,y)]^{\gamma},
\label{eq:eqCap4SSIM}
\end{equation}

\noindent onde $\alpha,\beta,\gamma$ são constantes de ajuste da métrica. É importante ressaltar que o \acs{SSIM} retorna os índices de similaridade para diversas regiões da imagem, porém quando se fala em similaridade é interessante que haja somente um índice para toda a imagem. Portanto é feito então uma média dos M índices retornados, dado pela equação:

\begin{equation}
MSSIM(x,y) = \dfrac{1}{M} \sum_{j=1}^{M} SSIM(x_{j},y_{j})
\label{eq:eqCap4MSSIM}
\end{equation} 

A métrica de \textit{Sharpness} tem como função avaliar o nível de borramento em uma imagem. Isso é importante pois os métodos de restauração ou reconstrução tendem a borrar as imagens, fazendo com que haja perda de detalhes, como por exemplo em uma microcalcificação nos exames mamografias. É importante notar que esse método não necessita de uma imagem referência. A estimativa é feita através das seguintes equações:

\begin{equation}
Sharpness = \sum_{r}^{} \sum_{c}^{}  w_{x}G^{2}_{x} + w_{y} G^{2}_{y}, 
\label{eq:eqCap4SHARPDB}
\end{equation}

\noindent onde $G_{x}$ e $G_{y}$ são gradientes direcionais, $w_{x}$ e $w_{y}$ são pesos baseados em uma vizinhança local, dados por:

\begin{equation}
w_{x} = [M(x+1,y)-M(x-1,y)]^{2},
\label{eq:eqCap4SHARPDB1}
\end{equation}

\begin{equation}
w_{y} = [M(x,y+1)-M(x,y-1)]^{2}.
\label{eq:eqCap4SHARPDB2}
\end{equation}

\colorbox{pink}{Alexandre: Arrumar essas equações de Sharpness}
	 
\chapter[Resultados Preliminares e Discussão]{Resultados Preliminares  e Discussões}\label{Capitulo5}

A partir dos métodos propostos no capítulo anterior, são apresentados nessa seção os resultados obtidos a partir da utilização da \textit{toolbox} desenvolvida nesse trabalho. São então discutidos os resultados de cada método de acordo com os gráficos, figuras e tabelas apresentadas. A \textit{toolbox} encontra-se disponível em \url{https://github.com/marcelo-vieira/DBT-Reconstruction.git}. 

\section{Reconstrução de Imagens}

A Figura \ref{fig:imgCap5Resultados1} demonstra os resultados obtidos após a aplicação de cada método de reconstrução implementado na \textit{toolbox}. Foram escolhidas 10 fatias de um total de 128 geradas pela reconstrução, sendo que essas foram selecionadas pois representam partes distintas de todo o volume. A primeira coluna apresenta as fatias do \textit{phantom} original usado como referência, a segunda o resultado da simples retroprojeção dos dados, a terceira o método de \acs{FBP} e a última o algoritmo de \acs{MLEM} com 5 iterações. As linhas de 1 a 10 representam respectivamente as fatias: 1, 15, 30, 43, 64, 78, 85, 99, 113 e 128. 

É importante destacar que houve um ajuste automático na faixa de exibição dos níveis de cinza que são mostrados, baseado nos locais onde se encontram os menores e os maiores níveis de cinza de cada imagem. 

%\colorbox{pink}{É importante destacar que para o algoritmo de retroprojeção filtrada foram feitos ajustes automáticos de brilho e de contraste em uma determinada região da imagem através do \textit{software} Fiji\footnote{\url{www.fiji.sc}} \cite{schindelin2012fiji} para uma melhor visualização.}

Levando em consideração somente o método \acs{MLEM}, a Figura \ref{fig:imgCap5Resultados3} demonstra os resultados obtidos por essa técnica e faz então uma comparação entre os diferentes números de iteração. São representas pela coluna de 1 até a 4 as iterações de número 5, 10, 15 e 20. Já para as fatias, em cada linha, são exibidas respectivamente as de número 43, 64, 78 e 85.  

Comparando os três métodos através de uma análise subjetiva é possível observar que o \acs{MLEM} com cinco iterações obteve o melhor resultado em todo o volume, levando em consideração a estrutura dos objetos, quantidade de artefatos e a nitidez de uma maneira geral. A simples retroprojeção possui uma imagem sem detalhes, como já era esperado, mas mantendo o formato exterior do crânio. Na retroprojeção filtrada é possível observar mais detalhes quando comparada com sua versão simples, porém artefatos são gerados no processo de reconstrução, devido a confecção do filtro no domínio da frequência. 

%devido ao filtro de rampa anular a frequência 0, cancelando o nível DC do sinal, o fundo não ficou com o valor 0. Devido a isso foi necessário a realização de operações de pós-processamento para a visualização da imagem. Problemas com a implementação da filtragem também foram encontrados. 

Em todas as técnicas é possível observar que as fatias iniciais e as finais possuem artefatos de reconstrução, enquanto que no meio do volume o formato da imagem reconstruída é mais fiel à original. Esse fato pode ser observado também em todos os resultados objetivos através das Figuras \ref{fig:imgCap5GraficoMSSIM}, \ref{fig:imgCap5GraficoNRMSE} e \ref{fig:imgCap5GraficoSHARPDB}, onde os melhores valores encontram-se em torno das fatias centrais. Isso decorre da utilização de uma estreita faixa de ângulo na tomossíntese. 

Analisando isoladamente os resultados da Figura \ref{fig:imgCap5Resultados3} é notável que quanto maior o número de iterações, maior é também o ruído e a quantidade de artefato nas imagens geradas. De certo modo isso era esperado como foi discutido na seção \ref{MétodoEstatístico} de métodos estatísticos.       

Todo o processamento foi feito em \acs{CPU} e as configurações do respectivo computador são: Intel Core i7-7700k 4,2GHz com 16GB de memória RAM. O tempo gasto para a execução de cada método é demonstrado pelas tabelas abaixo, onde a Tabela \ref{tab:tabCap5Tempos1} expõe os tempos de projeção, retroprojeção e filtragem para as técnicas de \acs{BP} e \acs{FBP}. Já a Tabela \ref{tab:tabCap5Tempos2} mostra os tempos para o algoritmo \acs{MLEM} com os respectivos números de iterações. Cada iteração é caraterizada por um ciclo como foi demonstrado no pseudocódigo anteriormente.

É incontestável a diferença existente entre o tempo gasto pelo método analítico e estatístico. Para cada iteração do método estatístico são necessárias uma projeção e uma retroprojeção, fazendo com que o custo computacional do método aumente muito. Já para o método analítico de \acs{FBP} é necessário somente uma retroprojeção e uma pré filtragem, tornando esse método mais eficiente quando se leva em consideração o tempo de execução. 

\begin{table}[H]
	\footnotesize
	\centering
	\caption{Tempo de execução em segundos dos algoritmos de \acs{BP} e \acs{FBP}.}
	\label{tab:tabCap5Tempos1}
	\begin{tabular}{l|c|c|c}
		\textbf{Algoritmo}	    & \textbf{Retroprojeção} 	& \textbf{Filtragem} 	& \textbf{Soma}	\\ [5pt]
		\hline
		\hline
		Retroprojeção 	 		& 0,55						& -						& 0,55			\\ 
		\hline
		Retroprojeção Filtrada	& 0,56 						& 0,03					& 0,59			\\
		\hline
	\end{tabular}
	\vspace{2ex}
	\legend{Fonte: do autor, 2018}
\end{table}

\begin{table}[H]
	\footnotesize
	\centering
	\caption{Tempos de execução em segundos do algoritmo \acs{MLEM} e seus respectivo número de iterações}
	\label{tab:tabCap5Tempos2}
	\begin{tabular}{c|c|c|c|c}
		\textbf{N\textsuperscript{\underline{o}}}&\textbf{5 Iterações}& \textbf{10 Iterações}& \textbf{15  Iterações}& \textbf{20 Iterações} 	\\ [5pt]
		\hline
		\hline
		\textbf{1 }&1,65 & 1,68 & 1,74 & 1,65 \\
		\hline
		\textbf{2 }&1,71 & 1,66 & 1,68 & 1,67 \\
		\hline
		\textbf{3 }&1,67 & 1,74 & 1,70 & 1,71 \\
		\hline
		\textbf{4 }&1,71 & 1,77 & 1,69 & 1,73 \\
		\hline
		\textbf{5 }&1,67 & 1,77 & 1,78 & 1,76 \\
		\hline
		\textbf{6 }&     & 1,74 & 1,73 & 1,75 \\
		\hline
		\textbf{7 }&     & 1,72 & 1,71 & 1,72 \\
		\hline
		\textbf{8 }&     & 1,73 & 1,73 & 1,71 \\
		\hline
		\textbf{9 }&     & 1,69 & 1,67 & 1,72 \\
		\hline
		\textbf{10}&     & 1,69 & 1,63 & 1,70 \\
		\hline
		\textbf{11}&     &      & 1,69 & 1,69 \\
		\hline
		\textbf{12}&     &      & 1,70 & 1,68 \\
		\hline
		\textbf{13}&     &      & 1,70 & 1,70 \\
		\hline
		\textbf{14}&     &      & 1,64 & 1,69 \\
		\hline
		\textbf{15}&     &      & 1,69 & 1,67 \\
		\hline
		\textbf{16}&     &      &      & 1,70 \\
		\hline
		\textbf{17}&     &      &      & 1,72 \\
		\hline
		\textbf{18}&     &      &      & 1,66 \\
		\hline
		\textbf{19}&     &      &      & 1,67 \\
		\hline
		\textbf{20}&     &      &      & 1,65 \\
		\hline
		\rowcolor{lightgray}\textbf{Soma} &8,43   	   & 17,23      & 25,55      & 34,06  \\
		\hline
	\end{tabular}
	\vspace{2ex}
	\legend{Fonte: do autor, 2018}
\end{table}  



As Figuras \ref{fig:imgCap5GraficoMSSIM}, \ref{fig:imgCap5GraficoNRMSE} e \ref{fig:imgCap5GraficoSHARPDB} ilustram os gráficos de validação dos procedimentos executados fazendo uma análise objetivas dos resultados. 

Para o \acs{SSIM}, o método iterativo obteve os melhores resultados, sendo que a maior similaridade foi obtida com 15 iterações. É interessante ressaltar que o maior índice de similaridade possível de se obter é o valor 1. 

Já para o \acs{MSE}, o método iterativo com 20 iterações obteve o menor erro, sendo observado que todos o erros tendem a diminuir nas fatias centrais. 

Por fim, para a validação através do \textit{Sharpness}, quando comparado com o \textit{phantom} virtual o método iterativo com 20 iterações obteve o melhor resultado. Novamente é possível observar uma tendência de crescimento do valor medido nas fatias centrais. Para o \textit{phantom}, a diminuição nas extremidades se deve aos valores de \textit{pixel} nulo nas imagens respectivas.   

\section{Redução de Artefatos de Alta Atenuação}

A partir da execução dos métodos ditos em \ref{MetodosReduçãodeArtefatosdeAltaAtenuação}, foram obtidos os resultados apresentados a seguir. A Figura \ref{fig:imgCap5ROIBP} demonstra a comparação entre retroprojeção simples e a técnica de redução de artefatos. Os conjuntos da esquerda significam a \acs{BP} e os da direita a \acs{BP} ponderada das fatias 1, 4, 7, 10, 13, 15 da \acs{ROI} mencionada para fins de comparação.

É possível notar que os resultados não foram satisfatórios após a aplicação dos métodos. Alguns empecilhos foram encontrados ao aplicar as técnicas. Devido ao centro de rotação do equipamento estar localizado $40mm$ acima do detector o foco de raio X muda a sua posição fazendo com que a média dos valores dos \textit{pixels} aumente onde o foco está. Isso faz com que as escolhas dos \textit{patches} aconteça de maneira errada, pois o método utiliza a técnica de \acs{MSE} para a escolha do referência. A Figura \ref{fig:imgCap5ROIPatchA} ilustra o histograma do número de contagem em que cada \textit{patch} foi escolhido, exemplificando o problema ocorrido. Já a Figura \ref{fig:imgCap5ROIPatchB} ilustra o histograma após a correção através da média local em cada patch, retirando assim o nível DC. Mesmo após a correção da média para o cálculo dos pesos, o problema ainda persistiu, devido a soma ponderada entre \textit{pixels} com diferentes valores de média.

O valor da constante M, como já foi dito, influencia na quantidade de sinal e de ruído presentes na imagem. A Figura \ref{fig:imgCap5FuncaoPesos} demonstra a relação existente entre essa constante e o formato da função. Nessa mesma figura está contido o histograma de distâncias calculadas em todos os \textit{pixels} de todas as fatias. É notável, através da Figura \ref{fig:imgCap5DiffM} que quanto menor o valor da constante M maior será o ruído na imagem. Caso contrário o resultado tende a ficar mais parecido com a simples média das imagens. 

\begin{figure}[H]
	\centering
	
	\caption{Comparação entre retroprojeção simples (Conjunto da esquerda) e a técnica de redução de artefatos (Conjunto da direita)  \colorbox{pink}{Elias: Imagem do equipamento?}.}
	
	\subfloat[1]{\includegraphics[scale=.9]{imgs/cap5/ROI/RoiBP_1.png}\label{fig:imgCap5ROIBP1}}
	\subfloat[4]{\includegraphics[scale=.9]{imgs/cap5/ROI/RoiBP_4.png}\label{fig:imgCap5ROIBP4}}
	\hfill
	\subfloat[1]{\includegraphics[scale=.9]{imgs/cap5/ROI/RoiWBP_1.png}\label{fig:imgCap5ROIWBP1}}
	\subfloat[4]{\includegraphics[scale=.9]{imgs/cap5/ROI/RoiWBP_4.png}\label{fig:imgCap5ROIWBP4}}
	
	\subfloat[7]{\includegraphics[scale=.9]{imgs/cap5/ROI/RoiBP_7.png}\label{fig:imgCap5ROIBP7}}
	\subfloat[10]{\includegraphics[scale=.9]{imgs/cap5/ROI/RoiBP_10.png}\label{fig:imgCap5ROIBP10}}
	\hfill
	\subfloat[7]{\includegraphics[scale=.9]{imgs/cap5/ROI/RoiWBP_7.png}\label{fig:imgCap5ROIWBP7}}
	\subfloat[10]{\includegraphics[scale=.9]{imgs/cap5/ROI/RoiWBP_10.png}\label{fig:imgCap5ROIWBP10}}
	
	\subfloat[13]{\includegraphics[scale=.9]{imgs/cap5/ROI/RoiBP_13.png}\label{fig:imgCap5ROIBP13}}
	\subfloat[15]{\includegraphics[scale=.9]{imgs/cap5/ROI/RoiBP_15.png}\label{fig:imgCap5ROIBP15}}
	\hfill
	\subfloat[13]{\includegraphics[scale=.9]{imgs/cap5/ROI/RoiWBP_13.png}\label{fig:imgCap5ROIWBP13}}
	\subfloat[15]{\includegraphics[scale=.9]{imgs/cap5/ROI/RoiWBP_15.png}\label{fig:imgCap5ROIWBP15}}
	
	\legend{Fonte: do autor, 2018.}
	\label{fig:imgCap5ROIBP}
\end{figure}

\begin{figure}[H]
	\centering
	
	\caption{Histograma da ocorrência de cada \textit{patch} (a) sem a técnica da média local e (b) com a técnica.}
	
	\subfloat[]{\includegraphics[scale=0.6]{imgs/cap5/ROI/EscolhaPatchSemMedia.pdf}\label{fig:imgCap5ROIPatchA}}
	\subfloat[]{\includegraphics[scale=0.6]{imgs/cap5/ROI/EscolhaPatchComMedia.pdf}\label{fig:imgCap5ROIPatchB}}
	
	\legend{Fonte: do autor, 2018.}
	\label{fig:imgCap5ROIPatch}
\end{figure}

 \begin{figure}[H]
 	\caption{Ilustração da função de pesos com diferentes valores da constante M junto com o histograma da ocorrência das distâncias.}
	\begin{center}
		\includegraphics[scale=0.6]{imgs/cap5/ROI/FuncaoPesos.pdf}
	\end{center}
	
	\legend{Fonte: do autor, 2018.}
	\label{fig:imgCap5FuncaoPesos}
\end{figure} 

\begin{figure}[H]
	\centering
	
	\caption{Influência da constante M no nível de ruído nas \acs{ROI}s. A primeira linha representa as fatias 67 e 74 para o valor de $M=0.0014$, a segunda $M=0.0033$ e a terceira $M=0.0051$.}
	
	\subfloat[67]{\includegraphics[scale=1]{imgs/cap5/ROI/M1_67.png}\label{fig:imgCap5M1_67}}
	\subfloat[74]{\includegraphics[scale=1]{imgs/cap5/ROI/M1_74.png}\label{fig:imgCap5M1_74}}
	
	\subfloat[67]{\includegraphics[scale=1]{imgs/cap5/ROI/M2_67.png}\label{fig:imgCap5M2_67}}
	\subfloat[74]{\includegraphics[scale=1]{imgs/cap5/ROI/M2_74.png}\label{fig:imgCap5M2_74}}
	
	\subfloat[67]{\includegraphics[scale=1]{imgs/cap5/ROI/M3_67.png}\label{fig:imgCap5M3_67}}
	\subfloat[74]{\includegraphics[scale=1]{imgs/cap5/ROI/M3_74.png}\label{fig:imgCap5M3_74}}
	
	\legend{Fonte: do autor, 2018.}
	\label{fig:imgCap5DiffM}
\end{figure}







\chapter[Resultados e Discussões]{Resultados e Discussões}\label{Capitulo6}

A partir dos métodos propostos no capítulo anterior, são apresentados nessa seção os resultados obtidos e suas respectivas discussões. Esse capítulo foi dividido em duas seções, respectivas aos resultados do desenvolvimento e validação do \textit{software} de reconstrução e aos resultados referentes a análise do sinal e do ruído nas imagens reconstruídas. 

\section{\textit{Toolbox} de Reconstrução}

\subsection{Disponibilização \textit{Online}}

 A \textit{toolbox} desenvolvida a partir desse trabalho, tem como objetivo contribuir para pesquisas que visam trabalhar com imagens de mamografia com a técnica de tomossíntese, mas que não tenham disponível um \textit{software} para a reconstrução. Todos os códigos desenvolvidos foram disponibilizados de maneira gratuita e contribuições advindas de outros grupos de pesquisa podem ser agregadas em novas versões. A \textit{toolbox} encontra-se disponível em \url{www.github.com/LAVI-USP/DBT-Reconstruction}. 
 
 \subsection{Validação}
 
 \subsubsection{\textit{Phantom} Virtual}
 
 Como foi dito, utilizou-se o \textit{phantom} de Shepp-Logan como ferramenta para a validação da \textit{toolbox}. A Figura \ref{fig:imgCap6SheppLogan} ilustra o resultado obtido a partir da aplicação dos quatro métodos de reconstrução. A fatia de número 64 é evidenciada para ilustrar os objetos em foco na mesma altura. Por meio de uma análise subjetiva, comparando de maneira visual, o método de \acs{BP} não possui contraste adequado, ou seja, uma imagem sem detalhes, como já era esperado, porém mantém o formato exterior do crânio. Já sua versão filtrada exibe um bom contraste, entretanto artefatos são gerados no processo de reconstrução, devido a confecção do filtro no domínio da frequência. Ambos os métodos iterativos possuem um bom contraste e evidenciam os objetos na parte inferior, ressaltando que o método algébrico contém mais artefatos de reconstrução. De maneira subjetiva,  é possível observar que o método \acs{MLEM} obteve o melhor resultado em todo o volume levando em consideração a estrutura dos objetos, quantidade de artefatos e a nitidez de uma maneira geral.
 
 A Figura \ref{fig:imgCap6SheppLogan3D_Recons} refere-se ao volume \acs{3D} reconstruído pelos métodos. É interessante notar que as fatias das extremidades possuem uma similaridade visual baixa quando comparadas com o padrão-ouro. Isso decorre da utilização de uma estreita faixa de ângulo na técnica de tomossíntese.
 
 É importante ressaltar que todas as imagens foram corrigidas através de um ajuste polinomial, sendo assim, estão na mesma faixa de escala de cinza.
    
 \begin{figure}[!h]
 	\centering	
 	\caption{Reconstrução da fatia \#64 para o método de (b) \acs{BP}, (c) \acs{FBP}, (d) \acs{MLEM} e (e) \acs{SART}, comparando-os com o (a) padrão-ouro.}
 	\subfloat[]{\includegraphics[scale=.9]{imgs/cap5/Shep_Original64.png}}
 	\hfill
 	\subfloat[]{\includegraphics[scale=.9]{imgs/cap5/Shep_BP64.png}}
 	\hfill
 	\subfloat[]{\includegraphics[scale=.9]{imgs/cap5/Shep_FBP64.png}}
 	\hfill
 	\subfloat[]{\includegraphics[scale=.9]{imgs/cap5/Shep_8MLEM64.png}}
 	\hfill
 	\subfloat[]{\includegraphics[scale=.9]{imgs/cap5/Shep_2SART64.png}}
 	\legend{Fonte: do autor, 2019.}
 	\label{fig:imgCap6SheppLogan}
 \end{figure}

\begin{figure}[!h]
	\caption{Representação do volume \acs{3D} referente aos métodos de (a) \acs{BP}, (b) \acs{FBP}, (c) \acs{MLEM} e (d) \acs{SART}.}
 	\centering	
	\subfloat[]{\includegraphics[scale=.8, clip, trim=4.8cm 4.5cm 24.5cm 4.9cm]{imgs/cap5/SheppLogan3D_Recons.pdf}}
	\hfill
	\subfloat[]{\includegraphics[scale=.8, clip, trim=11cm 4.5cm 18.5cm 4.9cm]{imgs/cap5/SheppLogan3D_Recons.pdf}}
	\hfill
	\subfloat[]{\includegraphics[scale=.8, clip, trim=17.5cm 4.5cm 12cm 4.9cm]{imgs/cap5/SheppLogan3D_Recons.pdf}}
	\hfill
	\subfloat[]{\includegraphics[scale=.8, clip, trim=23.5cm 4.43cm 6cm 4.9cm]{imgs/cap5/SheppLogan3D_Recons.pdf}}
	\legend{Fonte: do autor, 2019.}
	\label{fig:imgCap6SheppLogan3D_Recons}
\end{figure}

No âmbito das métricas objetivas, foi avaliado o \acs{SSIM} nas fatias de número 16 até a 113 referente a cada método, comparando-as com o padrão-ouro, como ilustrado pela Figura \ref{fig:imgCap6MSSIM}. É interessante destacar que o maior índice de similaridade possível de se obter é o valor 1. Através do gráfico, fica claro que o método estatístico iterativo obteve a maior similaridade no geral. É valido ressaltar que as maiores similaridades estão dispostas próximas as fatias centrais.

\begin{figure}[!h]
	\caption{Avaliação da métrica do índice de similaridade estrutural nas fatias reconstruídas de número 16 até a 113 para o \textit{phantom} virtual de Shepp-Logan, levando em consideração todos os métodos utilizados.}
	\begin{center}
		\includegraphics[scale=.8, clip, trim=3.6cm 8.4cm 4.4cm 8.9cm]{imgs/cap5/MSSIM_Shepp.pdf}
	\end{center}
	\legend{Fonte: do autor, 2019.}
	\label{fig:imgCap6MSSIM}
\end{figure}

Os tempos de processamento gastos para a execução de cada método foram medidos, como mostra a Tabela \ref{tab:tabCap6TimeSheppLogan}. Todo o processamento foi feito em \acs{CPU} e as configurações do respectivo computador são: Intel Core i7-7700k 4,2GHz com 16GB de memória RAM. Nota-se que a demanda computacional dos métodos iterativos é muito maior quando se comparada com a retroprojeção simples e a filtrada, sendo isso consequência da necessidade de uma projeção e uma retroprojeção a cada iteração. 

\begin{table}[!ht]
	\centering
	\caption{Tempo de execução em segundos para todos os quatro métodos de reconstrução referente ao \textit{phantom} de Shepp-Logan. Os números apresentados após cada método iterativo significam a quantidade de iterações executadas.}
	\label{tab:tabCap6TimeSheppLogan}
	\begin{tabular}{l|c}
		\textbf{Método}	     & \ \textbf{Tempo (s)} \\ 
		\hline
		\hline
		\rule[-0.5ex]{-3pt}{3ex}
		BP 	 		& 0,62 									\\ 
		\hline
		\rule[-0.5ex]{-3pt}{3ex}
		FBP			& 0,88 									\\
		\hline
		\rule[-0.5ex]{-3pt}{3ex}
		MLEM-8		& 13,41 								\\
		\hline
		\rule[-0.5ex]{-3pt}{3ex}
		SART-2		& 3,79								\\
		\hline
	\end{tabular}
	\vspace{2ex}
	\legend{Fonte: do autor, 2019.}
\end{table}

\subsubsection{\textit{Phantom} Antropomórfico Físico}

Para o \textit{phantom} antropomórfico físico BR3D, foram feitas análises visuais subjetivas comparando os resultados obtidos pela \textit{toolbox} com as imagens fornecidas pelo equipamento. A Figura \ref{fig:imgCap6BR3D} exemplifica os resultados atingidos, bem como a imagem fornecida pelo equipamento. Estas são referentes a fatia \#56 e estão localizadas a $50mm$ do detector. O ajuste da escala de cinza foi feito manualmente buscando o melhor contraste para cada imagem, visto que cada uma refere-se a um método de reconstrução diferente.

Observa-se novamente que o método de \acs{BP} não apresenta uma nitidez com relação as estruturas internas devido a sobreposição das baixas frequências, discutidas anteriormente. Para o método de \acs{FBP}, as estruturas internas estão claras, porém com um elevado nível de ruído. Para ambos os métodos iterativos, as imagens possuem um contraste bom e é possível visualizar as estruturas internas tal como na imagem fornecida pelo equipamento. Nota-se que todas as estruturas estão em foco na mesma fatia para todos os métodos de reconstrução, assim como na imagem fornecida pelo equipamento.

\begin{figure}[!h]
	\centering	
	\caption{Ilustração da reconstrução da fatia \#56 do \textit{phantom} BR3D referente ao (a) equipamento de \acs{DBT} da \acs{GE}, método de (b) \acs{BP}, (c) \acs{FBP}, (d) \acs{MLEM} e (e) \acs{SART} gerada pela \textit{toolbox} apresentada.}
	\subfloat[Comercial]{\includegraphics[scale=.13]{imgs/cap5/BR3D_GE56.png}}
	\hfill
	\subfloat[\acs{BP}]{\includegraphics[scale=.13]{imgs/cap5/BR3D_BP56.png}}
	\hfill
	\subfloat[\acs{FBP}]{\includegraphics[scale=.13]{imgs/cap5/BR3D_FBP56.png}}
	
	\subfloat[\acs{MLEM}]{\includegraphics[scale=.13]{imgs/cap5/BR3D_MLEM8_56.png}}	
	\hfil
	\subfloat[\acs{SART}]{\includegraphics[scale=.13]{imgs/cap5/BR3D_SART2_56_3.png}}
	\legend{Fonte: do autor, 2019.}
	\label{fig:imgCap6BR3D}
\end{figure}

Regiões de interesse (\textit{Region of Interest} - \acs{ROI}) foram extraídas na localização das microcalcificações simuladas, ilustradas pela Figura \ref{fig:imgCap6BR3D_ROI}. As mesmas características já discutidas são novamente observadas e a grande similaridade dos métodos iterativos com o padrão fornecido pelo equipamento é evidenciada.

\begin{figure}[!h]
	\centering
	\caption{Regiões de interesse extraídas de locais onde se encontram as microcalcificações simuladas, referentes aos métodos de (b) \acs{BP}, (c) \acs{FBP}, (d) \acs{MLEM}, (e) \acs{SART} e (a) do equipamento comercial.}	
	\subfloat[Comercial]{\hspace*{3ex}\includegraphics[scale=.2]{imgs/cap5/BR3D_GE56_ROI.png}\hspace*{3ex}}
	\hfill
	\subfloat[\acs{BP}]{\hspace*{3ex}\includegraphics[scale=.2]{imgs/cap5/BR3D_BP56_ROI.png}\hspace*{3ex}}
	\hfill
	\subfloat[\acs{FBP}]{\hspace*{3ex}\includegraphics[scale=.2]{imgs/cap5/BR3D_FBP56_ROI.png}\hspace*{3ex}}
	\hfill
	\subfloat[\acs{MLEM}]{\hspace*{3ex}\includegraphics[scale=.2]{imgs/cap5/BR3D_MLEM8_56_ROI.png}\hspace*{3ex}}	
	\hfill
	\subfloat[\acs{SART}]{\hspace*{3ex}\includegraphics[scale=.2]{imgs/cap5/BR3D_SART2_56_3_ROI.png}\hspace*{3ex}}
	\legend{Fonte: do autor, 2019.}
	\label{fig:imgCap6BR3D_ROI}
\end{figure}


Do mesmo modo que no \textit{phantom} virtual, os tempos de processamento também foram medidos, demonstrados pela Tabela \ref{tab:tabCap6TimeBR3D}. Novamente, fica claro o alto custo computacional relacionado com os métodos iterativos, chegando  a ser cerca de 19 vezes mais lento que uma simples retroprojeção.  É incontestável a diferença existente entre os tempos gastos pelo método analítico e estatístico iterativo. Para cada iteração é necessária uma projeção e uma retroprojeção, fazendo com que o custo computacional do método aumente muito. Já para o método analítico de \acs{FBP} é necessário somente uma retroprojeção e uma pré-filtragem, tornando-o mais eficiente quando se leva em consideração o tempo de execução.

\begin{table}[!ht]
	\centering
	\caption{Tempo de execução em segundos para todos os quatro métodos de reconstrução referente ao \textit{phantom} BR3D. Os números apresentados após cada método iterativo significam a quantidade de iterações executadas.}
	\label{tab:tabCap6TimeBR3D}
	\begin{tabular}{l|c}
		\textbf{Método}	     &   \textbf{Tempo (s)}   	\\ 
		[5pt]
		\hline
		\hline
		\rule[-0.5ex]{-3pt}{3ex}
		BP 	 		 						& 58,19					\\ 
		\hline
		\rule[-0.5ex]{-3pt}{3ex}
		FBP			 						& 58,88					\\
		\hline
		\rule[-0.5ex]{-3pt}{3ex}
		MLEM-8		 					& 1087 					\\
		\hline
		\rule[-0.5ex]{-3pt}{3ex}
		SART-2		 					& 334 					\\
		\hline
	\end{tabular}
	\vspace{2ex}
	\legend{Fonte: do autor, 2019.}
\end{table}


\subsubsection{\textit{Phantom} Antropomórfico Virtual}
 
 Já para o \textit{phantom} antropomórfico virtual gerado pelo \textit{software} OpenVCT, os resultados são apresentados pela Figura \ref{fig:imgCap6VCT}. São comparadas as reconstruções advindas tanto da \textit{toolbox} proposta nesse trabalho quanto da fornecida pelo órgão americano \acs{FDA}. Da mesma maneira avaliou-se as \acs{ROI}s provenientes de três métodos de reconstrução, comparando-as entre os dois \textit{softwares}. A \textit{toolbox} proposta por esse trabalho é denominada de ``LAVI-USP''.
 
 O agrupamento de microcalcificações no \textit{phantom} padrão-ouro está localizado na altura de  $53,7mm$ do detector. As \acs{ROI}s demonstradas na Figura \ref{fig:imgCap6VCT} foram reconstruídas na altura de $54mm$, visto que a resolução em Z escolhida foi de $0,5mm$. 
 
 Uma inspeção visual comparando as reconstruções da \textit{toolbox} LAVI-USP e do \acs{FDA} evidencia a semelhança entre ambos os resultados. Alguns artefatos de reconstrução são observados nos métodos de \acs{FBP} e \acs{SART}, denominados de \textit{undershooting}, que são causados devido a supressão de baixas frequências no sinal.
 
 \begin{figure}[!ht]
 	\caption{Ilustração do \textit{phantom} virtual antropomórfico gerado pelo software OpenVCT com a \acs{ROI} ilustrando o agrupamento de microcalcificações simuladas. As \acs{ROI}s restantes apresentam os resultados para cada método, considerando ambos \textit{softwares} de reconstrução.}
 	\begin{center}
 		\includegraphics[scale=.38, clip, trim=0.1cm 3.7cm 2cm 2.4cm]{imgs/cap5/Recons.pdf}
 	\end{center}
 	\label{fig:imgCap6VCT}
 	\legend{Fonte: do autor, 2019.}
 \end{figure} 

A avaliação objetiva feita através da similaridade entre as fatias de \acs{DBT} geradas pelos \textit{softwares} LAVI-USP e \acs{FDA} é mostrada na Figura \ref{fig:imgCap6VCTSSIMMSE}. O \acs{SSIM} médio em todas as fatias é próximo de 1 e o valor do \acs{NRMSE} é menor que 8\% para todos os métodos de reconstrução, concretizando a avaliação visual feita de que as fatias fornecidas por ambas as \textit{toolboxes} são semelhantes.

\begin{figure}[!ht]
	\centering	
	\caption{ Avaliação da similaridade estrutural comparando as fatias do \textit{phantom} virtual antropomórfico geradas pela \textit{toolbox} LAVI-USP a pelo \acs{FDA}. Métricas de (a) \acs{SSIM} e \acs{NRMSE} para cada profundidade considerando os três métodos de reconstrução.}
	\subfloat[]{\includegraphics[scale=.5, clip, trim=3.5cm 8.5cm 4.2cm 8.6cm]{imgs/cap5/MSSIM_VCT.pdf}}
	\hfill
	\subfloat[]{\includegraphics[scale=.5, clip, trim=4cm 8.5cm 4.2cm 8.6cm]{imgs/cap5/MSE.pdf}}
	\label{fig:imgCap6VCTSSIMMSE}
	\legend{Fonte: do autor, 2019.}
\end{figure}


Para a análise da dispersão do sinal no eixo Z, a métrica de \acs{ASF} foi avaliada para ambas \textit{toolboxes} e o padrão-ouro foi tomado como referência, de acordo com a Figura \ref{fig:imgCap6VCTASF}. É importante notar que a resolução em Z para o \textit{phantom} virtual é de $0,1mm$ e para as fatias reconstruídas $0,5 mm$. Analisando o padrão-ouro, é perceptível que o agrupamento de microcalcificações está principalmente situado em fatias nas altura de 53,6, 53,7 e $53,8mm$. Os maiores valores da métrica de \acs{ASF}, para ambas \textit{toolboxes}, estão localizados na altura de $54mm$, demonstrando que o agrupamento de microcalcificações está em foco na mesma altura que o \textit{phantom} padrão-ouro, considerando as duas resoluções de amostragem distintas no eixo Z.

\begin{figure}[!ht]
	
	\centering	
	\caption{ Avaliação da métrica \acs{ASF} para os algoritmos de (a) \acs{BP}, (b) \acs{FBP}, e (c) \acs{SART}. A métrica foi calculada para o \textit{phantom} padrão-ouro, para a \textit{toolbox} LAVI-USP e FDA.}
	\subfloat[]{\includegraphics[scale=.5, clip, trim=3.6cm 8.5cm 4.2cm 8.6cm ]{imgs/cap5/ASF_BP.pdf}}
	\hfill
	\subfloat[]{\includegraphics[scale=.5, clip, trim=3.6cm 8.5cm 4.2cm 8.6cm ]{imgs/cap5/ASF_FBP.pdf}}

	\subfloat[]{\includegraphics[scale=.5, clip, trim=3.6cm 8.5cm 4.2cm 8.6cm ]{imgs/cap5/ASF_SART.pdf}}

	\label{fig:imgCap6VCTASF}
	\legend{Fonte: do autor, 2019.}
\end{figure}


\subsubsection{Imagens Clínicas}

Imagens clínicas de pacientes também foram reconstruídas para a validação do \textit{software}. A Figura \ref{fig:imgCap6ClinicalRecon} exemplifica os resultados de modo a compará-los com as imagens fornecidas pelos equipamentos. Novamente é importante ressaltar que as projeções advindas do sistema \#1 foram do tipo ``não processadas'', enquanto que as do sistema \#2 do tipo ``processadas''. 

A comparação visual das estruturas demonstra que os objetos estão em foco na mesma altura para ambas as imagens de cada sistema. Nenhum tipo de pré- ou pós-processamento foi utilizado. Consequentemente, uma irregularidade na visualização da imagem reconstruída pela \textit{toolbox} pode ser observada no sistema \#1. A janela de níveis de cinza foi ajustada manualmente de modo a obter o melhor contraste para cada imagem.

As Figuras \ref{fig:imgCap6ClinicalGECC} e \ref{fig:imgCap6ClinicalGEMLO} representam um caso de uma paciente que realizou a mamografia no sistema \#1 em orientação \acs{CC} e \acs{MLO}, respectivamente. As Figuras \ref{fig:imgCap6ClinicalGECC_A} e \ref{fig:imgCap6ClinicalGEMLO_A} foram adquiridas através do equipamento e servem como  padrão para comparação. Já as Figuras \ref{fig:imgCap6ClinicalGECC_B} e \ref{fig:imgCap6ClinicalGEMLO_B} são o resultado da reconstrução feita através da \textit{toolbox} LAVI-USP sem nenhum tipo de pré- ou pós-processamento e sem ajuste da janela de nível de cinza. Por fim, as Figuras \ref{fig:imgCap6ClinicalGECC_C} e \ref{fig:imgCap6ClinicalGEMLO_C} representam as mesmas reconstruções de (b), porém com um ajuste da janela de nível de cinza para uma melhor visualização do interior da mama.  

É interessante notar, através das setas vermelhas, os objetos em foco tanto nas imagens formadas pela \textit{toolbox} quanto nas fornecidas pelos equipamentos. 

Especificamente no exame \acs{MLO}, é possível observar artefatos na parte superior e inferior das imagens, demonstrados pelas setas azuis. Estes artefatos são decorrentes da falta de informação de regiões da mama em determinadas projeções, em outras palavras, a projeção da mama foi cortada em certos ângulos, pois foram projetas para fora do detector. O equipamento comercial faz algum tipo de pós-processamento para minimizar esse efeito, mas ainda é possível ver as regiões com artefatos nas imagens. Alguns trabalhos anteriores descrevem métodos para minimizar estas irregularidades, como os demonstrados em \citeonline{zhang2009artifact} e \citeonline{lu2013diffusion}.


\begin{figure}[htb]
	\centering	
	\caption{Comparação visual entre as reconstruções de imagens de pacientes advindas do sistema (a) \#1, (b) LAVI-USP, (c) sistema \#2 e (d) LAVI-USP. As fatias estão localizadas nas alturas de $51$mm e $61$mm do detector para ambos os sistemas, respectivamente, e foram adquiridas em uma orientação \acs{CC}.}
	
	\subfloat[]{\includegraphics[scale=0.45, clip, trim=0cm 0cm 0cm 0.38cm]{imgs/cap5/Clinical/GE_Original_Slice_58.png}}
	\hfill
	\subfloat[]{\includegraphics[scale=0.45, clip, trim=0cm 0cm 0cm 0.38cm]{imgs/cap5/Clinical/GE_Lavi_Slice_58_nSART_3.png}}
	\hfill
	\subfloat[]{\includegraphics[scale=0.45]{imgs/cap5/Clinical/Hologic_Original_Slice_35.png}}
	\hfill
	\subfloat[]{\includegraphics[scale=0.45]{imgs/cap5/Clinical/Hologic_Lavi_Slice_35.png}}
	\legend{Fonte: do autor, 2019.}
	\label{fig:imgCap6ClinicalRecon}
\end{figure}

\begin{figure}[!ht]
	\centering	
	\caption{Reconstrução de uma fatia a $62mm$ do detector proveniente de um exame mamográfico realizado no equipamento do sistema \#1, adquirido em uma orientação \acs{CC}, na qual (a) foi fornecida pelo equipamento, (b) é a reconstrução através da \textit{toolbox} LAVI-USP sem ajuste da janela de nível de cinza e (c) com ajuste da janela de nível de cinza. As setas em vermelho apontam para estruturas em foco nas mesmas imagens. }
	
	
	\subfloat[]{\includegraphics[scale=0.5, clip, trim=10cm 0cm 10cm 0cm]{imgs/cap5/Clinical/Caso1/GE_Original_Slice_40_CC.pdf}\label{fig:imgCap6ClinicalGECC_A}}
	
	\subfloat[]{\includegraphics[scale=0.5, clip, trim=10cm 0cm 10cm 0cm]{imgs/cap5/Clinical/Caso1/GE_Lavi_Slice_40_CC_SemAjuste_nSART.pdf}\label{fig:imgCap6ClinicalGECC_B}}
	\hfil
	\subfloat[]{\includegraphics[scale=0.5, clip, trim=10cm 0cm 10cm 0cm]{imgs/cap5/Clinical/Caso1/GE_Lavi_Slice_40_CC_Ajuste_nSART.pdf}\label{fig:imgCap6ClinicalGECC_C}}
	\legend{Fonte: do autor, 2019.}
	\label{fig:imgCap6ClinicalGECC}
\end{figure}

\clearpage

\begin{figure}[!ht]
	\centering	
	\caption{Reconstrução de uma fatia a $81mm$ do detector proveniente de um exame mamográfico realizado no equipamento do sistema \#1, adquirido em uma orientação \acs{MLO}, na qual (a) foi fornecido pelo equipamento, (b) é a reconstrução através da \textit{toolbox} LAVI-USP sem ajuste da janela de nível de cinza e (c) com ajuste da janela de nível de cinza. As setas em vermelho apontam para estruturas em foco nas mesmas imagens e as azuis para artefatos de reconstrução. }
		
	\subfloat[]{\includegraphics[scale=0.5, clip, trim=11cm 0cm 11cm 0cm]{imgs/cap5/Clinical/Caso1/GE_Original_Slice_59_MLO.pdf}\label{fig:imgCap6ClinicalGEMLO_A}}
	
	\subfloat[]{\includegraphics[scale=0.5, clip, trim=11cm 0cm 11cm 0cm]{imgs/cap5/Clinical/Caso1/GE_Lavi_Slice_59_MLO_SemAjuste_nSART.pdf}\label{fig:imgCap6ClinicalGEMLO_B}}
	\hfil
	\subfloat[]{\includegraphics[scale=0.5, clip, trim=11cm 0cm 11cm 0cm]{imgs/cap5/Clinical/Caso1/GE_Lavi_Slice_59_MLO_Ajuste_nSART.pdf}\label{fig:imgCap6ClinicalGEMLO_C}}
	\legend{Fonte: do autor, 2019.}
	\label{fig:imgCap6ClinicalGEMLO}
\end{figure}

\clearpage


\section{Caracterização do Sinal  \& Ruído} 

A análise do sinal e do ruído referente aos dois sistemas de \acs{DBT} é apresentada nessa seção. 

\subsection{Convenção de Coordenadas}

A fim de estabelecer um padrão para a ilustração dos resultados, bem como para a discussão, foi adotado uma convenção para a nomenclatura dos eixos e para a disposição dos resultados. A Figura \ref{fig:imgCap63DVolume} exemplifica o padrão adotado, na qual a coordenada X representa a direção da parede torácica para o mamilo, denominada por perfil \ac{PA} e a coordenada Y representa a direção onde o tubo de raios X se movimenta, denominada de perfil \ac{LA}. Já a coordenada Z simboliza a profundidade onde são reconstruídas as fatias da mama.

\begin{figure}[H]
	\caption{Ilustração da convenção dos eixos adotados nesse trabalho, na qual a coordenada X representa a direção da parede torácica para o mamilo e a coordenada Y onde o tubo de raios X rotaciona. A coordenada (0,0) simboliza a posição do tubo de raios X quando no ângulo de $0^{\circ}$.}
	\begin{center}
		\includegraphics[scale=.7, clip, trim=3.6cm 8.5cm 4.2cm 8.5cm]{imgs/cap5/3D_Breast.pdf}
	\end{center}
	\legend{Fonte: do autor, 2019.}
	\label{fig:imgCap63DVolume}
\end{figure} 

\subsection{Avaliação do Sinal  \& Ruído}


A figura \ref{fig:imgCap6Means} mostra o valor médio dos \textit{pixels} na fatia reconstruída referente a cada dose para ambos os sistemas. Observa-se que há uma queda visível no valor médio dos \textit{pixels} na região posterior do detector, especialmente no sistema \#2. Isto é causado principalmente por imperfeições na calibração da uniformidade dos \textit{pixels} (\textit{flat-fielding}), que compensa fenômenos como o efeito heel, o qual leva a não uniformidade dos feixes de raios X \cite[p. 485]{marshall2017handbook}.


\begin{figure}[htb]
	\centering
	\caption{Valor médio dos \textit{pixels} na fatia reconstruída para cada respectiva dose de raios X, considerando (a) o sistema \#1 e (b) \#2.}
	\subfloat[]{\includegraphics[scale=.7, clip, trim=3.4cm 8.3cm 5cm 9cm ]{imgs/cap5/ge_Mean.pdf}}
	
	\subfloat[]{\includegraphics[scale=.7, clip, trim=3.4cm 8.3cm 5cm 9cm ]{imgs/cap5/hologic_Mean.pdf}}
	\legend{Fonte: do autor, 2019.}
	\label{fig:imgCap6Means}
\end{figure}

A característica da dependência do sinal em relação ao ruído é observada ao analisar as Figuras \ref{fig:imgCap6Means}, \ref{fig:imgCap6GE_Var} e \ref{fig:imgCap6HO_Var}. À medida que a dose aumenta, o valor esperado referente ao sinal livre do ruído também aumenta, ilustrada pela Figura \ref{fig:imgCap6Means}. Consequência deste fato e em relação direta com a Equação \ref{eq:eqCap5HeteroscedasticoVar}, a variância do ruído também aumenta, representada pelas Figuras \ref{fig:imgCap6GE_Var} e \ref{fig:imgCap6HO_Var}.

Além disso, a dependência espacial da variância do ruído é observada nas fatias reconstruídas, seguindo o mesmo comportamento demonstrado em trabalhos anteriores para o domínio das projeções. Novamente, esse fenômeno decorre devido ao \textit{flat-fielding} que evita não-uniformidades no detector causadas pelo efeito heel e pela geometria do feixe cônico no domínio das projeções \cite{borges2017pipeline,borges2017method, borges2018restoration,brito2018application,guerrero2018}.

\begin{figure}[htb]
	\centering
	\caption{Variância do ruído na fatia reconstruída (a) para cada dose respectiva, (b) a representação \acs{2D} para a maior dose e (c) o perfil na coordenada X para o sistema \#1.}
	\subfloat[]{\includegraphics[scale=0.6, clip, trim=3.6cm 8cm 4.2cm 9cm ]{imgs/cap5/ge_Variance3D.pdf}}

	\subfloat[]{\includegraphics[scale=.5, clip, trim=3.5cm 8cm 4.3cm 9cm ]{imgs/cap5/ge_Variance_2D.pdf}}
	\hfill
	\subfloat[]{\includegraphics[scale=.5, clip, trim=3.6cm 8cm 4.3cm 9cm ]{imgs/cap5/ge_Variance_PA.pdf}}
	\legend{Fonte: do autor, 2019.}
	\label{fig:imgCap6GE_Var}
\end{figure}

\begin{figure}[htb]
	\centering
	\caption{Variância do ruído na fatia reconstruída (a) para cada dose respectiva, (b) a representação \acs{2D} para a maior dose e (c) o perfil na coordenada X para o sistema \#2.}
	\subfloat[]{\includegraphics[scale=0.6, clip, trim=3.6cm 8cm 4.2cm 9cm ]{imgs/cap5/hologic_Variance3D.pdf}}
	
	\subfloat[]{\includegraphics[scale=0.5, clip, trim=3.5cm 8cm 4.3cm 9cm ]{imgs/cap5/hologic_Variance_2D.pdf}}
	\hfil
	\subfloat[]{\includegraphics[scale=0.5, clip, trim=3.6cm 8cm 4.3cm 9cm ]{imgs/cap5/hologic_Variance_PA.pdf}}
	\legend{Fonte: do autor, 2019.}
	\label{fig:imgCap6HO_Var}
\end{figure}

Em seguida, a métrica de \acs{SNR} foi calculada para cada dose, como mostra a Figura \ref{fig:imgCap6SNR}. É possível observar que as medidas da variância e também do \acs{SNR} seguem o mesmo formato de curva após a reconstrução, como relatado anteriormente para projeções de \acs{DBT} no trabalho de \citeonline{borges2017method}, evidenciando a mesma dependência do espaço.


\begin{figure}[htb]
	\centering
	\caption{Valor da métrica de \acs{SNR} na fatia reconstruída para cada dose respectiva, considerando os sistemas (a) \#1 e (b) \#2.}
	
	\subfloat[]{\includegraphics[scale=.7, clip, trim=3.5cm 8cm 3.7cm 9cm ]{imgs/cap5/ge_SNR3D.pdf}}
	
	\subfloat[]{\includegraphics[scale=.7, clip, trim=3.5cm 8cm 4.47cm 9cm ]{imgs/cap5/hologic_SNR3D.pdf}}
	\legend{Fonte: do autor, 2019.}
	\label{fig:imgCap6SNR}  
\end{figure}


Para estimar os coeficientes angulares e lineares da função afim apresentada pela Equação \ref{eq:eqCap5HeteroscedasticoVar}, um ajuste polinomial de grau 1 foi feito em relação a média do sinal e a sua respectiva variância, como mencionado na seção \ref{Metodologia}. O coeficiente de determinação $R^2$ foi calculado para verificar a exatidão da aproximação polinomial. Para ambos os sistemas, o $R^2$ médio foi de 0,99. 

Os coeficientes da função afim que modelam a variância das imagens reconstruídas do sistema \#1 são mostrados nas Figuras \ref{fig:imgCap6GECoefA} e \ref{fig:imgCap6GECoefB}, enquanto que as Figuras \ref{fig:imgCap6HOCoefA} e \ref{fig:imgCap6HOCoefB} ilustram os parâmetros estimados do sistema \#2.


\begin{figure}[htb]
	\centering
	\caption{Coeficientes (a) angular e (b) linear da função afim aproximada para a modelagem da variância das fatias reconstruídas no sistema \#1.}
	\subfloat[]{\includegraphics[scale=.54, clip, trim=3.6cm 8cm 3.6cm 8cm ]{imgs/cap5/ge_lambda_D.pdf}\label{fig:imgCap6GECoefA}}
	\hfill
	\subfloat[]{\includegraphics[scale=.54, clip, trim=3.6cm 8cm 4.3cm 8cm ]{imgs/cap5/ge_sigma_I.pdf}\label{fig:imgCap6GECoefB}}
	\legend{Fonte: do autor, 2019.}
	\label{fig:imgCap6GECoef}
\end{figure}

\begin{figure}[htb]
	\centering
	\caption{Coeficientes (a) angular e (b) linear da função afim aproximada para a modelagem da variância das fatias reconstruídas no sistema \#2.} 
	\subfloat[]{\includegraphics[scale=.54, clip, trim=3.6cm 8cm 3.6cm 8cm ]{imgs/cap5/hologic_lambda_D.pdf}\label{fig:imgCap6HOCoefA}}
	\hfill
	\subfloat[]{\includegraphics[scale=.54, clip, trim=3.6cm 8cm 4.2cm 8cm ]{imgs/cap5/hologic_sigma_I.pdf}\label{fig:imgCap6HOCoefB}}
	\legend{Fonte: do autor, 2019.}
	\label{fig:imgCap6HOCoef}
\end{figure}

A correlação gerada pelo procedimento de reconstrução foi medida por meio do \acs{NNPS}. A Figura \ref{fig:imgCap6NNPS} representa os resultados para ambos os sistemas. A frequência máxima observada pelo sistema depende do tamanho dos \textit{pixels}, dada a partir do teorema de Nyquist, mencionado na seção \ref{RetroprojeçãoFiltrada}. Portanto, ambos os sistemas têm frequência máxima distinta devido a sua diferença no tamanho do detector. As medições indicam que o processo de reconstrução correlaciona o ruído para o detector direto de \acs{a-Se}, que era virtualmente não correlacionado no domínio das projeções (ruído branco), apresentado no trabalho de \citeonline{borges2017method},  e correlaciona ainda mais o ruído no sistema \#1 de detecção indireta de  \acs{a-Si}.

\begin{figure}[htb]
	\centering
	\caption{Medidas de \acs{NNPS} para os sistemas (a) \#1 e (b) \#2.} 
	\subfloat{\includegraphics[scale=.54, clip, trim=3.6cm 8cm 4.3cm 8cm ]{imgs/cap5/ge_NNPS.pdf}\label{fig:imgGENNPS}}
	\hfill
	\subfloat{\includegraphics[scale=.54, clip, trim=3.6cm 8cm 4.3cm 8cm ]{imgs/cap5/hologic_NNPS.pdf}\label{fig:imgHONNPS}}
	\legend{Fonte: do autor, 2019.}
	\label{fig:imgCap6NNPS}
\end{figure}

Em relação a análise da normalidade, as Tabelas \ref{tab:tabCap6NormTestGE} e \ref{tab:tabCap6NormTestHologic} mostram a porcentagem de \acs{ROI}s, nas quais os testes estatísticos não rejeitaram a hipótese nula, com nível de significância de 5\%. Em outras palavras, a quantidade de \acs{ROI}s em que o teste comprova que os dados são normalmente distribuídos. É perceptível que a distribuição dos valores de cinza locais, referentes as imagens pós-reconstrução, seguem uma distribuição normal, o que suporta o \acs{CLT} discutido anteriormente.

{\begin{table}[htb]
		\caption{Porcentagem de \acs{ROI}s que não rejeitaram a hipótese nula $H_0$ referente as imagens reconstruídas do sistema \#1. Os valores representados são para três testes estatísticos diferentes que medem a normalidade dos dados.}
		\label{tab:tabCap6NormTestGE}
		\centering
		% 	\footnotesize
		\begin{tabular}{c|c|c|c|c|c}
			%			\rowcolor[HTML]{D4D4D4}
			\multirow{2}{*}{\textbf{Teste de hipótese} } &                  \multicolumn{5}{c}{\textbf{mAs} }                   \\ \cline{2-6}
			                                                 & \textbf{45} & \textbf{54} & \textbf{72} & \textbf{90} & \textbf{126} \\
			   [3pt]
			\hline
			\hline
			\rule[-0.5ex]{-3pt}{3ex}
			Shapiro-Wilk    						&    81,8     &    82,7     &    81,4     &    80,4     &     78,7     \\ \hline
			              \rule[-0.5ex]{-3pt}{3ex}
			Kolmogorov-Smirnov               &    99,8     &    99,6     &    99,2     &    98,8     &     97,6     \\ \hline
			               \rule[-0.5ex]{-3pt}{3ex}
			Anderson-Darling                 	&    85,6     &    85,1     &    84,0     &    83,3     &     82,1     \\ \hline
		\end{tabular}
		\vspace{2ex}
		\legend{Fonte: do autor, 2019.}
	\end{table}
	
	{\begin{table}[htb]
			\caption{Porcentagem de \acs{ROI}s que não rejeitaram a hipótese nula $H_0$ referente as imagens reconstruídas do sistema \#2. Os valores representados são para três testes estatísticos diferentes que medem a normalidade dos dados.}
			\label{tab:tabCap6NormTestHologic}
			\centering
			% 	\footnotesize
			\begin{tabular}{c|c|c|c|c|c}
				%				\rowcolor[HTML]{D4D4D4}
				\multirow{2}{*}{\textbf{Teste de hipótese} } &                  \multicolumn{5}{c}{\textbf{mAs} }                  \\ \cline{2-6}
				                                                    & \textbf{45} & \textbf{48} & \textbf{54} & \textbf{63} & \textbf{69} \\
				  [3pt]
				\hline
				\hline
				\rule[-0.5ex]{-3pt}{3ex}
				Shapiro-Wilk   								&    93,2     &    93,5     &    93,3     &    93,4     &    93,0     \\ \hline
				               \rule[-0.5ex]{-3pt}{3ex}
				Kolmogorov-Smirnov                	 &    99,9     &     100     &    99,9     &    99,9     &     100     \\ \hline
				                \rule[-0.5ex]{-3pt}{3ex}
				Anderson-Darling                 		&    94,1     &    93,9     &    93,4     &    93,8     &    92,4     \\ \hline
			\end{tabular}
		\vspace{2ex}
		\legend{Fonte: do autor, 2019.}
		\end{table}




\chapter[Próximas Etapas e Cronograma]{Próximas Etapas e Cronograma}\label{Capitulo7}

As próximas etapas a serem executadas levam em consideração o aprimoramento dos métodos de reconstrução já desenvolvidos, a aplicação de geometrias de equipamentos comerciais e a implementação do método iterativo estatístico junto com os conhecimentos \textit{a priori} advindos da teoria de Campo Aleatório Markoviano não-local.

A extensão da \textit{toolbox} para outras linguagens de programação, como por exemplo Python, é também necessária tendo em vista o grande aumento na sua utilização e por ser uma linguagem aberta.

Uma outra etapa estabelecida é a implementação de novas técnicas para a redução de artefatos com alta atenuação, bem como o aprimoramento da técnica já implementada no trabalho de \citeonline[]{borges2017metal}. A investigação do não funcionamento da implementação também é necessária. 

Por fim, a publicação de artigos científicos é de extrema importância, assim como a disponibilização de todos os códigos em uma plataforma aberta para a comunidade científica com o intuito de difusão do conhecimento e ampliação de pesquisas na área. 

\colorbox{pink}{Elias: Utilizar métodos específicos para ângulo limitado!}
\colorbox{pink}{Elias: FISTA para otimizar!}
\colorbox{pink}{Alexandre: Compressive sensing reconstrução de sinal usando min norma L0}
\colorbox{pink}{Alexandre: Restrição Tikonov Miller}
\colorbox{pink}{Alexandre: EM-MPM}


\includepdf{docs/Cronograma.pdf}

\chapter[Trabalhos Futuros]{Trabalhos Futuros}\label{Capitulo8}

Como trabalhos futuros, deve-se levar em consideração o aprimoramento dos algoritmos de reconstrução já desenvolvidos. Outros métodos também podem ser incorporados, tais como os algoritmos iterativos com conhecimentos \textit{a priori} para regularização. É também necessário o estudo de técnicas de reconstrução específicas para equipamentos de tomografia com ângulo limitado.

A extensão da \textit{toolbox} para outras linguagens de programação, como por exemplo Python, é também necessária tendo em vista o aumento da utilização dessa linguagem e pelo fato da mesma ser aberta.

Os métodos de projeção e retroprojeção podem ser acelerados usando técnicas computacionais paralelas em \acs{GPU}, assim como a implementação de projetores do estado da arte.  

Em relação ao modelamento do ruído pós-reconstrução, aplicações na redução do ruído no domínio da reconstrução podem ser avaliadas e comparadas com aquelas aplicadas no domínio das projeções. Além disso, abordagens híbridas podem ser empregadas para alcançar melhores resultados. A análise do ruído também pode ser avaliada em diferentes alturas do volume, para medir a influência do processo de reconstrução no ruído em diferentes fatias de imagem. Por fim, a correlação anisotrópica criada pelo processo de reconstrução pode ser investigada e aplicada nos modelos matemáticos.

\chapter[Publicações]{Publicações}\label{Capitulo9}

\section{Congresso Brasileiro de Engenharia Biomédica - CBEB}

Artigo publicado nos anais do congresso, onde é apresentado a \textit{toolbox} desenvolvida nesse trabalho de modo que seja disponibilizada de maneira aberta para toda comunidade científica.

VIMIEIRO, R. B.; BORGES, L. R.; VIEIRA, M.A.C; \textbf{Open source reconstruction toolbox for digital breast tomosynthesis}. XXVI Brazilian Congress on Biomedical Engineering, Costa-Felix, R., Machado,
J. C., and Alvarenga, A. V., eds., 70/1, IFMBE Proceedings, Springer, Singapore (2019).


\section{International Society for Optics and Photonics - SPIE}

Artigo publicado nos anais do congresso internacional, onde são feitas as medições das propriedades do ruído em imagens de tomossíntese reconstruídas. Do mesmo modo, foi apresentada a validação da \textit{toolbox} já desenvolvida pelo grupo de pesquisa.

VIMIEIRO, R. B.; BORGES, L. R.; BARUFALDI, B.; CARON, R. F.; BAKIC, P. R.; MAIDMENT, A. D.; VIEIRA, M. A. C.; \textbf{Noise Measurements from Reconstructed Digital Breast Tomosynthesis}. Medical
Imaging 2019: Physics of Medical Imaging, International Society for Optics and Photonics, San Diego
(2019).



% ---
% Finaliza a parte no bookmark do PDF, para que se inicie o bookmark na raiz
% ---
\bookmarksetup{startatroot}% 
% ---

% ---
% Conclusão
% ---

%\chapter*[Conclusão]{Conclusão}
%\addcontentsline{toc}{chapter}{Conclusão}

%\lipsum[31-33]

% ----------------------------------------------------------
% ELEMENTOS PÓS-TEXTUAIS
% ----------------------------------------------------------
\postextual

% ----------------------------------------------------------
% Referências bibliográficas
% ----------------------------------------------------------
\bibliography{bib/referencias}


% ----------------------------------------------------------
% Apêndices
% ----------------------------------------------------------
% ---
% Inicia os apêndices
% ---
%\begin{apendicesenv}
%% Imprime uma página indicando o início dos apêndices
%\partapendices
%% ----------------------------------------------------------
%% Incluir Apêndice
%% ----------------------------------------------------------
%
%\chapter{Campos Aleatórios Markovianos}\label{ApendiceA:CamposAleatóriosMarkovianos}

Nesse capítulo estão contidos os conceitos necessários para o entendimento da teoria de Campo Aleatório Markoviano (\textit{Markov Random Field} - \acs{MRF}) do Capítulo \ref{Capitulo3}. Toda essa notação foi retirada de \citeonline[p. 1-12]{li2009markov} e \citeonline[p. 11-13]{won2013stochastic}.

\section{Sites \& Rótulos}\label{ApendiceA:SitesRotulos}

Considerando $S$ um conjunto que contêm índices de um número $N$ de \textit{\textbf{sites}}. Esse conjunto é descrito da seguinte forma:  

\begin{equation}
	S \, = \, \{1,2,\dots,N\}.
	\label{eq:eqApendiceAConjuntoSites}
\end{equation}  

A palavra \textit{\textbf{site}} tem como significado representar um ponto ou região no espaço euclidiano. No caso específico de uma imagem em \acs{2D} de $NxN$, esses \textit{sites} representam a localização onde a imagem foi amostrada, ou seja, os \textit{\textbf{pixels}}. Nesse caso o conjunto $S$ contém os índices da seguinte forma:

\begin{equation}
	S \, = \, \{(i,j) \mid 1 \leq i,j \leq N\}.
	\label{eq:eqApendiceAConjuntoSites2D}
\end{equation}  

Além disso, a representação dos índices no conjunto $S$ também pode ser feita de maneira não ordenada $S \, = \, \{1,2,...,M\}$, onde $M = NxN$. Essa notação segundo o autor é muito utilizada em modelos de \acs{MRF}. 

Um evento que pode acontecer com um \textit{site} é denominado de rótulo (\textit{\textbf{label}}), e.g., um \textit{pixel} receber um certo valor de nível de cinza. Denota-se $\varGamma$ como um conjunto de rótulos, sendo esse contido por valores discretos de $r_{k}$ rótulos, como descrito a seguir:

\begin{equation}
	\varGamma \, = \, \{r_{1},r_{2},\dots,r_{k}\},
	\label{eq:eqApendiceAConjuntoRotulos}
\end{equation} 

\noindent onde em uma determinada aplicação de imagem, o conjunto de rótulos $\varGamma$ toma valores discretos que representam todos os \textbf{níveis de cinzas} possíveis que foram quantizados, por exemplo $\varGamma \, = \, \{1,2,\dots,254,255\}$.

O conjunto $f = \{f_{1},f_{2},\dots,f_{N}\},$ é denominado como ``rotulagem'' dos \textit{sites} que estão contidos no conjunto $S$, de maneira que cada rótulo do conjunto $f$ está contido no conjunto $\varGamma$ descrito acima.

O produto cartesiano é dito como o conjunto de todos os possíveis rótulos admissíveis pelos \textit{sites}, quando os mesmos tem o conjunto de rótulo $\varGamma$ em comum. Sua equação matemática se dá por:

\begin{equation}
	\varPsi \, = \, \varGamma^{N},
	\label{eq:eqApendiceAProdutoCarteziano}
\end{equation}   

\noindent onde $N$ é o tamanho do conjunto $S$. Segundo o autor, em um problema de restauração de imagem, $\varGamma$ contém todos os valores admissíveis para os \textit{pixels} (\textit{sites}) dentro do conjunto $S$ e $\varPsi$ define todas as possíveis imagens.  



\section{Vizinhança \& Cliques}\label{ApendiceA:sVizinhancaCliques}

De acordo com o autor, os \textit{sites} contidos em $S$ relacionam-se uns com os outros através de um sistema de vizinhança $\nu$:

\begin{equation}
	\nu \, = \, \{\nu_{i} \mid \forall i \in S\},
	\label{eq:eqApendiceAVizinhanca}
\end{equation}  

\noindent onde $\nu_{i}$ é um conjunto de outros \textit{sites} que fazem vizinhança com o \textit{site} $i$.

Um sistema de \textbf{vizinhança} pode ser definido de diversas formas. O de primeira ordem (vizinhança-4) é dado como mostra a Figura \ref{fig:imgApendiceAVizinhancaA}, o de segunda ordem (vizinhança-8) como a Figura \ref{fig:imgApendiceAVizinhancaB}, o de terceira ordem (vizinhança-12) como a Figura \ref{fig:imgApendiceAVizinhancaC} e assim adiante.

O par  $(S,\nu) \, \overset{\Delta}{=} \, G $, constitui um grafo, de tal maneira que $S$ contém os nós e $\nu$ determina as ligações entre os nós de acordo com a vizinhança estipulada. Um \textbf{clique} $c$ para $(S,\nu)$ é definido como um subconjunto de \textit{sites} contido no conjunto $S$, de tal maneira que o clique pode ser considerado como único $c_{1} = \{i\}$, com dois vizinhos $c_{2} = \{i,i^{'}\}$, com três vizinhos $c{3} = \{i,i^{'},i^{''}\}$ e assim sucessivamente. O conjunto de todos os cliques para $(S,\nu)$ é dado por:

\begin{equation}
	C \,=\, C_{1} \,\cup\, C_{2}\, \cup \,C_{3}\,...
	\label{eq:eqApendiceAConjuntoCliques}
\end{equation}  

O tipo de clique para o grafo $(S,\nu)$ é definido pelo seu tamanho, formato e orientação. Para a vizinhança-4 (Figura \ref{fig:imgApendiceAVizinhancaA}), seus cliques respectivos são os: \ref{fig:imgApendiceACliqueA}, \ref{fig:imgApendiceACliqueB} e \ref{fig:imgApendiceACliqueC}; já para a vizinhança-8 (Figura \ref{fig:imgApendiceAVizinhancaB}), seus cliques são os: \ref{fig:imgApendiceACliqueA}, \ref{fig:imgApendiceACliqueB}, \ref{fig:imgApendiceACliqueC}, \ref{fig:imgApendiceACliqueD}, \ref{fig:imgApendiceACliqueE}, \ref{fig:imgApendiceACliqueF}, \ref{fig:imgApendiceACliqueG}, \ref{fig:imgApendiceACliqueH}, \ref{fig:imgApendiceACliqueI} e \ref{fig:imgApendiceACliqueJ}. É possível notar, segundo o autor, que ao aumentar a ordem de vizinhança, aumenta-se também o número de cliques impactando no custo computacional do algoritmo envolvido.  

\begin{figure}[H]
	\centering
	
	\caption{Exemplo de vizinhança (a) 4, (b) 8 e (c) 12.}
	
	\subfloat[]{\includegraphics[scale=0.18]{imgs/ApenA/Vizi1.png}\label{fig:imgApendiceAVizinhancaA}}
	\subfloat[]{\includegraphics[scale=0.2]{imgs/ApenA/Vizi2.png}\label{fig:imgApendiceAVizinhancaB}}
	\subfloat[]{\includegraphics[scale=0.2]{imgs/ApenA/Vizi3.png}\label{fig:imgApendiceAVizinhancaC}}
	
	\legend{Fonte: \citeonline[p. 73]{salvadeo2013filtragem}}
	\label{fig:imgApendiceAVizinhanca}
\end{figure}


\begin{figure}[H]
	\centering
	
	\caption{Cliques possíveis para cada tipo de vizinhança.}
	
	\subfloat[]{\includegraphics[scale=1]{imgs/ApenA/Clique1.pdf}\label{fig:imgApendiceACliqueA}}
	\subfloat[]{\includegraphics[scale=1]{imgs/ApenA/Clique2.pdf}\label{fig:imgApendiceACliqueB}}
	\subfloat[]{\includegraphics[scale=1]{imgs/ApenA/Clique3.pdf}\label{fig:imgApendiceACliqueC}}
	
	
	\subfloat[]{\includegraphics[scale=1]{imgs/ApenA/Clique4.pdf}\label{fig:imgApendiceACliqueD}}
	\subfloat[]{\includegraphics[scale=1]{imgs/ApenA/Clique5.pdf}\label{fig:imgApendiceACliqueE}}
	\subfloat[]{\includegraphics[scale=1]{imgs/ApenA/Clique6.pdf}\label{fig:imgApendiceACliqueF}}
	\subfloat[]{\includegraphics[scale=1]{imgs/ApenA/Clique7.pdf}\label{fig:imgApendiceACliqueG}}
	\subfloat[]{\includegraphics[scale=1]{imgs/ApenA/Clique8.pdf}\label{fig:imgApendiceACliqueH}}
	\subfloat[]{\includegraphics[scale=1]{imgs/ApenA/Clique9.pdf}\label{fig:imgApendiceACliqueI}}
	\subfloat[]{\includegraphics[scale=1]{imgs/ApenA/Clique10.pdf}\label{fig:imgApendiceACliqueJ}}
	
	\legend{Fonte: \citeonline[p. 74]{salvadeo2013filtragem}}
	\label{fig:imgApendiceAClique}
\end{figure}

\section{Campo Aleatório}\label{ApendiceA:CampoAleatorio}

Um campo aleatório é definido por $F$, onde $F = \{F_{1},F_{2},...,F_{N}\}$ é um conjunto de variáveis aleatórias $F_{i}$ que admitem o valor de rótulo $f_{i}$ dentro do conjunto de rótulos $\varGamma$. A probabilidade de uma variável aleatória $F_{i}$ assumir um valor $f_{i}$ é dada por $P(F_{i} = f_{i} )$ ou somente $P(f_{i})$ e a probabilidade conjunta do campo aleatório é dada por:

\begin{equation}
	P(F = f) = P(F_{1}=f_{1},...,F_{N}=f_{N}),
	\label{eq:eqApendiceAProbCampoAleatorio}
\end{equation}  

\noindent ou somente por $P(f)$, dado um conjunto $\varGamma$ de valores discretos.

\chapter{Campos Aleatórios de Gibbs}\label{ApendiceB:CamposAleatóriosdeGibbs}

Nesse capítulo estão os conceitos da teoria de Campo Aleatório de Gibbs (\textit{Gibbs Random Field} - \acs{GRF}) e sua relação com Campo Aleatório Markoviano (\textit{Markov Random Field} - \acs{MRF}), referente ao Capítulo \ref{Capitulo3}. Toda esta notação foi retirada de \citeonline[p. 13-15]{li2009markov} e \citeonline[p. 14-21]{won2013stochastic}.

\section{Campo Aleatório de Gibbs}\label{ApendiceB:CampoAleatóriodeGibbs}

Dado um conjunto de variáveis aleatórias $F$, esse é dito ser um \acs{GRF} em $S$ com relação a vizinhança $\nu$ se o mesmo obedecer a distribuição de Gibbs dada a seguir:

\begin{equation}
	P(f) = \frac{1}{Z}\; e^{-U(f)},
	\label{eq:eqApendiceBDistribuicaoGibbs}
\end{equation}

\noindent onde $Z$ é uma constante de normalização denominada função de partição, dada por \eqref{eq:eqApendiceBDistribuicaoGibbsZ} e $U(f)$ é chamada de função de energia, dada por \eqref{eq:eqApendiceBDistribuicaoGibbsU1} que representa a soma dos potenciais dos cliques $V_{c}(f)$ em todos os possíveis cliques $C$. 

\begin{equation}
	Z = \sum_{f \,\in\, \varPsi}^{} \, e^{-U(f)}. 
	\label{eq:eqApendiceBDistribuicaoGibbsZ}
\end{equation}

\begin{equation}
	U(f) = \sum_{c \,\in\, C}^{} \, V_{c}(f).
	\label{eq:eqApendiceBDistribuicaoGibbsU1}
\end{equation}

A função de energia também pode ser expressa através do somatório de termos independentes, de acordo com a ordem de seu clique:

\begin{equation}
	U(f) = \sum_{\{i\} \,\in\, C_{1}}^{} \, V_{1}(f_{i}) \,+\, \sum_{\{i,i^{'}\} \,\in\, C_{2}}^{} \, V_{2}(f_{i},f_{i^{'}}) \,+\, \sum_{\{i,i^{'},i^{''}\} \,\in\, C_{3}}^{} \, V_{3}(f_{i},f_{i^{'}},f_{i^{''}}),
	\label{eq:eqApendiceBDistribuicaoGibbsU2}
\end{equation}  

\noindent ou considerando somente cliques de ordem um e dois:

\begin{equation}
	U(f) = \sum_{i \,\in\, S}^{} \, V_{1}(f_{i}) \,+\, \sum_{i \,\in\, S}^{} \, \sum_{i^{'} \,\in\, \nu_{i} }^{} \, V_{2}(f_{i},f_{i^{'}}).
	\label{eq:eqApendiceBDistribuicaoGibbsU3}
\end{equation} 

A prova da de que o \acs{GRF} é um \acs{MRF} e vice-versa, pode ser encontrada em \citeonline[p. 19]{won2013stochastic}.


\chapter{Imagens Resultados}\label{ApendiceC:ImagensResultados}

Esse apêndice contém as imagens obtidas como resultado desse trabalho. Devido a organização do espaço as mesmas foram inseridas nessa seção.

\begin{figure}[htb]
	\centering
	
	\caption{Parte 1 dos resultados obtidos após a aplicação dos métodos \acs{BP}, \acs{FBP} e \acs{MLEM}. Da coluna 1 a 5 são representados respectivamente: \textit{Phantom}, \acs{BP}, \acs{FBP} e \acs{MLEM} com 10 iterações. Capítulo \ref{Capitulo5}.}
	
	\subfloat{\includegraphics[scale=.85]{imgs/cap5/Original/1.png}}
	\hfil
	\subfloat{\includegraphics[scale=.85]{imgs/cap5/ReconBP/1.png}}
	\hfil
	\subfloat{\includegraphics[scale=.85]{imgs/cap5/ReconFBP/1.png}}
	\hfil
	\subfloat{\includegraphics[scale=.85]{imgs/cap5/ReconMLEM/5/1.png}}
	
	
	\subfloat{\includegraphics[scale=.85]{imgs/cap5/Original/15.png}}
	\hfil
	\subfloat{\includegraphics[scale=.85]{imgs/cap5/ReconBP/15.png}}
	\hfil
	\subfloat{\includegraphics[scale=.85]{imgs/cap5/ReconFBP/15.png}}
	\hfil
	\subfloat{\includegraphics[scale=.85]{imgs/cap5/ReconMLEM/5/15.png}}
	
	
	\subfloat{\includegraphics[scale=.85]{imgs/cap5/Original/30.png}}
	\hfil
	\subfloat{\includegraphics[scale=.85]{imgs/cap5/ReconBP/30.png}}
	\hfil
	\subfloat{\includegraphics[scale=.85]{imgs/cap5/ReconFBP/30.png}}
	\hfil
	\subfloat{\includegraphics[scale=.85]{imgs/cap5/ReconMLEM/5/30.png}}
	
	
	\subfloat{\includegraphics[scale=.85]{imgs/cap5/Original/43.png}}
	\hfil
	\subfloat{\includegraphics[scale=.85]{imgs/cap5/ReconBP/43.png}}
	\hfil
	\subfloat{\includegraphics[scale=.85]{imgs/cap5/ReconFBP/43.png}}
	\hfil
	\subfloat{\includegraphics[scale=.85]{imgs/cap5/ReconMLEM/5/43.png}}
	
	
	\subfloat{\includegraphics[scale=.85]{imgs/cap5/Original/64.png}}
	\hfil
	\subfloat{\includegraphics[scale=.85]{imgs/cap5/ReconBP/64.png}}
	\hfil
	\subfloat{\includegraphics[scale=.85]{imgs/cap5/ReconFBP/64.png}}
	\hfil
	\subfloat{\includegraphics[scale=.85]{imgs/cap5/ReconMLEM/5/64.png}}
	
	\label{fig:imgCap5Resultados1}
\end{figure}
\begin{figure}[!t] \ContinuedFloat
	\centering
	
	\caption{Parte 2 dos resultados obtidos após a aplicação dos métodos \acs{BP}, \acs{FBP} e \acs{MLEM}. Da coluna 1 a 5 são representados respectivamente: \textit{Phantom}, \acs{BP}, \acs{FBP} e \acs{MLEM} com 10 iterações. Capítulo \ref{Capitulo5}.}
	
	\subfloat{\includegraphics[scale=.85]{imgs/cap5/Original/78.png}}
	\hfil
	\subfloat{\includegraphics[scale=.85]{imgs/cap5/ReconBP/78.png}}
	\hfil
	\subfloat{\includegraphics[scale=.85]{imgs/cap5/ReconFBP/78.png}}
	\hfil
	\subfloat{\includegraphics[scale=.85]{imgs/cap5/ReconMLEM/5/78.png}}
	
	
	\subfloat{\includegraphics[scale=.85]{imgs/cap5/Original/85.png}}
	\hfil
	\subfloat{\includegraphics[scale=.85]{imgs/cap5/ReconBP/85.png}}
	\hfil
	\subfloat{\includegraphics[scale=.85]{imgs/cap5/ReconFBP/85.png}}
	\hfil
	\subfloat{\includegraphics[scale=.85]{imgs/cap5/ReconMLEM/5/85.png}}
	
	
	\subfloat{\includegraphics[scale=.85]{imgs/cap5/Original/99.png}}
	\hfil
	\subfloat{\includegraphics[scale=.85]{imgs/cap5/ReconBP/99.png}}
	\hfil
	\subfloat{\includegraphics[scale=.85]{imgs/cap5/ReconFBP/99.png}}
	\hfil
	\subfloat{\includegraphics[scale=.85]{imgs/cap5/ReconMLEM/5/99.png}}
	
	
	\subfloat{\includegraphics[scale=.85]{imgs/cap5/Original/113.png}}
	\hfil
	\subfloat{\includegraphics[scale=.85]{imgs/cap5/ReconBP/113.png}}
	\hfil
	\subfloat{\includegraphics[scale=.85]{imgs/cap5/ReconFBP/113.png}}
	\hfil
	\subfloat{\includegraphics[scale=.85]{imgs/cap5/ReconMLEM/5/113.png}}
	
	
	\subfloat{\includegraphics[scale=.85]{imgs/cap5/Original/128.png}}
	\hfil
	\subfloat{\includegraphics[scale=.85]{imgs/cap5/ReconBP/128.png}}
	\hfil
	\subfloat{\includegraphics[scale=.85]{imgs/cap5/ReconFBP/128.png}}
	\hfil
	\subfloat{\includegraphics[scale=.85]{imgs/cap5/ReconMLEM/5/128.png}}
	
	\legend{Fonte: do autor, 2018.}
	\label{fig:imgCap5Resultados2}
\end{figure}


\begin{figure}[!t]
	\centering
	
	\caption{Resultados obtidos após a aplicação do método \acs{MLEM} com os números de iteração: 5, 10, 15 e 20 respectivamente ao número das colunas. Capítulo \ref{Capitulo5}.}
	
	\subfloat{\includegraphics[scale=.85]{imgs/cap5/ReconMLEM/5/43.png}}
	\hfil
	\subfloat{\includegraphics[scale=.85]{imgs/cap5/ReconMLEM/10/43.png}}
	\hfil
	\subfloat{\includegraphics[scale=.85]{imgs/cap5/ReconMLEM/15/43.png}}
	\hfil
	\subfloat{\includegraphics[scale=.85]{imgs/cap5/ReconMLEM/20/43.png}}
	
	
	\subfloat{\includegraphics[scale=.85]{imgs/cap5/ReconMLEM/5/64.png}}
	\hfil
	\subfloat{\includegraphics[scale=.85]{imgs/cap5/ReconMLEM/10/64.png}}
	\hfil
	\subfloat{\includegraphics[scale=.85]{imgs/cap5/ReconMLEM/15/64.png}}
	\hfil
	\subfloat{\includegraphics[scale=.85]{imgs/cap5/ReconMLEM/20/64.png}}
	
	
	\subfloat{\includegraphics[scale=.85]{imgs/cap5/ReconMLEM/5/78.png}}
	\hfil
	\subfloat{\includegraphics[scale=.85]{imgs/cap5/ReconMLEM/10/78.png}}
	\hfil
	\subfloat{\includegraphics[scale=.85]{imgs/cap5/ReconMLEM/15/78.png}}
	\hfil
	\subfloat{\includegraphics[scale=.85]{imgs/cap5/ReconMLEM/20/78.png}}
	
	
	\subfloat{\includegraphics[scale=.85]{imgs/cap5/ReconMLEM/5/85.png}}
	\hfil
	\subfloat{\includegraphics[scale=.85]{imgs/cap5/ReconMLEM/10/85.png}}
	\hfil
	\subfloat{\includegraphics[scale=.85]{imgs/cap5/ReconMLEM/15/85.png}}
	\hfil
	\subfloat{\includegraphics[scale=.85]{imgs/cap5/ReconMLEM/20/85.png}}
	
	\legend{Fonte: do autor, 2018.}
	\label{fig:imgCap5Resultados3}
\end{figure}

 \begin{figure}[t!]
 	\caption{Gráfico de \acs{SSIM} médio para todos os métodos de reconstrução utilizados. Capítulo \ref{Capitulo5}.}
	\begin{center}
		\includegraphics[scale=0.8]{imgs/cap5/WN/MSSIM.pdf}
	\end{center}
	\legend{Fonte: do autor, 2018.}
	\label{fig:imgCap5GraficoMSSIM}
\end{figure} 

\begin{figure}[t!]
	\caption{Gráfico da raiz do \acs{MSE} normalizado para todos os métodos de reconstrução utilizados. Capítulo \ref{Capitulo5}.}
	\begin{center}
		\includegraphics[scale=0.8]{imgs/cap5/WN/NRMSE.pdf}
	\end{center}
	\legend{Fonte: do autor, 2018.}
	\label{fig:imgCap5GraficoNRMSE}
\end{figure} 

\begin{figure}[t!]
	\caption{Gráfico de \textit{Sharpness} em escala de dB para todos os métodos de reconstrução utilizados e para o \textit{Phantom} de referência. Capítulo \ref{Capitulo5}.}
	\begin{center}
		\includegraphics[scale=0.8]{imgs/cap5/WN/SHARPDB.pdf}
	\end{center}
	\legend{Fonte: do autor, 2018.}
	\label{fig:imgCap5GraficoSHARPDB}
\end{figure}

\chapter{Artigo}\label{Artigo}

\includepdf[pages=-]{docs/artigo-cbeb-2018.pdf}


%
%\end{apendicesenv}
% ---

% ----------------------------------------------------------
% Anexos
% ----------------------------------------------------------
% ---
% Inicia os anexos
% ---
%\begin{anexosenv}
% Imprime uma página indicando o início dos anexos
%\partanexos
% ----------------------------------------------------------
% Incluir Anexo
% ----------------------------------------------------------


%\end{anexosenv}


\end{document}


