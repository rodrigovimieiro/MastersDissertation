%% Qualificação de Mestrado - EESC-SEL-USP
%% 
%% Nome: Rodrigo de Barros Vimieiro
%% E-mail: rodrigo.vimieiro@gmail.com
%%
%% Universidade de São Paulo - São Carlos
%% Laboratório de Visão Computacional - LAVI
%%
%% Data: 30/11/2017


% ------------------------------------------------------------------------
% ------------------------------------------------------------------------
% eesc: Modelo de Trabalho Acadêmico (tese de doutorado, dissertação de
% mestrado e trabalhos monográficos em geral) em conformidade com 
% ABNT NBR 14724:2011. Esta classe estende as funcionalidades da classe
% abnTeX2 elaborada de forma a adequar os parâmetros exigidos pelas 
% normas USP e do departamento de elétrica da Escola de Engenharia 
% de São Carlos - USP.
% ------------------------------------------------------------------------
% ------------------------------------------------------------------------

% ------------------------------------------------------------------------
% Opções:
% 	tesedr:     Formata documento para tese de doutorado
%	qualidr:    Formata documento para qualificação de doutorado
% 	dissertmst: Formata documento para dissertação de mestrado
% 	qualimst:   Formata documento para qualificação de mestrado
% ------------------------------------------------------------------------
\documentclass[qualimst]{eesc}

% ---
% PACOTES
% ---

% ---
% Pacotes fundamentais 
% ---
\usepackage{cmap}				% Mapear caracteres especiais no PDF
\usepackage{lmodern}			% Usa a fonte Latin Modern			
\usepackage{makeidx}           	% Cria o indice
\usepackage{hyperref}  			% Controla a formação do índice
\usepackage{lastpage}			% Usado pela Ficha catalográfica
\usepackage{indentfirst}		% Indenta o primeiro parágrafo de cada seção.
\usepackage{nomencl} 			% Lista de simbolos
\usepackage{graphicx}			% Inclusão de gráficos
\usepackage{subfig}				% Inclusão de figura side-by-side (Rodrigo)
\usepackage{amsmath}			% Formulas matematicas (Rodrigo)
\usepackage{gensymb}			% Para colocar \degree (Rodrigo)
\usepackage[algo2e, portuguese, ruled]{algorithm2e}	% Para colocar pseudocódigo (Rodrigo)

% ---

% ---
% Pacotes adicionais, usados apenas no âmbito do Modelo eesc
% ---
\usepackage[printonlyused]{acronym}
\usepackage[table]{xcolor}
% ---


% ---
% Informações de dados para CAPA e FOLHA DE ROSTO
% ---
%
% Título:
%	1. Título em português
%	2. Título em inglês
\titulo{Estudo sobre métodos de reconstrução para imagens de tomossíntese digital da mama}{}
%
% Autor:
%	1. Nome completo do autor
%	2. Formato de nome para bibliografia
\autor{Rodrigo de Barros Vimieiro}{Vimieiro, Rodrigo}
%
% Cidade
\local{São Carlos}
% Ano de defesa
\data{2018}
% Área de concentração da pesquisa
\areaconcentracao{Processamento de Sinais e Instrumentação}
% Nome do orientador
\orientador{Marcelo Andrade da Costa Vieira}
% Nome do coorientador
%\coorientador{}
% ---

% ---
% compila o indice
% ---
\makeindex
% ---

% ---
% Compila a lista de abreviaturas e siglas
% ---
\makenomenclature
% ---

% ---
% Inserir ficha catalográfica
%
% Caso o comando \inserirfichacatalografica seja definido, a ficha catalográfica
% será inserida atrás da folha de rosto. Caso contrário a página será deixada em
% branco.
%
% CUIDADO: Esta opção deve ser preenchida antes do comando \maketitle
% ---
%\inserirfichacatalografica{fichaCatalografica.pdf}
% ---

% ---
% Inserir folha de aprovação
%
% Caso o comando \inserirfolhaaprovacao seja definido, a a folha de aprovação
% será inserida. Além disso, conforme Resolução CoPGr 5890, as informações 
% de rodapé são inseridas apropriadamente na folha de rosto.
%
% CUIDADO: Esta opção deve ser preenchida antes do comando \maketitle
% ---
%\inserirfolhaaprovacao{folhaAprovacao.pdf}
% ---

% ----
% Início do documento
% ----

\begin{document}

% ----------------------------------------------------------
% ELEMENTOS PRÉ-TEXTUAIS
% ----------------------------------------------------------
\pretextual

% ---
% Insere Capa, Folha de rosto, Ficha catalográfica (se inserida)
% e folha de aprovação (se inserida).
% ---
\maketitle

%%%%%%%%%%%%%%%%%%%%%%%%%%%%%%%%%%%%%%%%%%%%%%%%%%%%%%%%%%%%%%%%%%%%%%%%%%%%%%%%%%%%%%%%%%%%%%%%%%%%%%%%%%%%%%%%%%%%%%%%%%%%%%													  Dedicatória															%
%%%%%%%%%%%%%%%%%%%%%%%%%%%%%%%%%%%%%%%%%%%%%%%%%%%%%%%%%%%%%%%%%%%%%%%%%%%%%%%%%%%%%%%%%%%%%%%%%%%%%%%%%%%%%%%%%%%%%%%%%%%%%
%\imprimirdedicatoria{Este trabalho é dedicado a Deus\\
%   e aos meus pais, Erika e José Ronaldo.}

%%%%%%%%%%%%%%%%%%%%%%%%%%%%%%%%%%%%%%%%%%%%%%%%%%%%%%%%%%%%%%%%%%%%%%%%%%%%%%%%%%%%%%%%%%%%%%%%%%%%%%%%%%%%%%%%%%%%%%%%%%%%%%													 Agradecimentos															%
%%%%%%%%%%%%%%%%%%%%%%%%%%%%%%%%%%%%%%%%%%%%%%%%%%%%%%%%%%%%%%%%%%%%%%%%%%%%%%%%%%%%%%%%%%%%%%%%%%%%%%%%%%%%%%%%%%%%%%%%%%%%%
%\imprimiragradecimentos{
%
%}
%%%%%%%%%%%%%%%%%%%%%%%%%%%%%%%%%%%%%%%%%%%%%%%%%%%%%%%%%%%%%%%%%%%%%%%%%%%%%%%%%%%%%%%%%%%%%%%%%%%%%%%%%%%%%%%%%%%%%%%%%%%%%%													   Epígrafe																%
%%%%%%%%%%%%%%%%%%%%%%%%%%%%%%%%%%%%%%%%%%%%%%%%%%%%%%%%%%%%%%%%%%%%%%%%%%%%%%%%%%%%%%%%%%%%%%%%%%%%%%%%%%%%%%%%%%%%%%%%%%%%%
%\imprimirepigrafe{
%		``Somos essencialmente profissionais do sentido. Educamos, \\
%		 quando ensinamos com sentido. Educar é impregnar de sentido\\
%		 a vida. A profissão docente está centrada na vida, no bem querer.''\\
%		(Prof. Gilberto Teixeira)
%}
%%%%%%%%%%%%%%%%%%%%%%%%%%%%%%%%%%%%%%%%%%%%%%%%%%%%%%%%%%%%%%%%%%%%%%%%%%%%%%%%%%%%%%%%%%%%%%%%%%%%%%%%%%%%%%%%%%%%%%%%%%%%%%												  RESUMO e ABSTRACT															%
%%%%%%%%%%%%%%%%%%%%%%%%%%%%%%%%%%%%%%%%%%%%%%%%%%%%%%%%%%%%%%%%%%%%%%%%%%%%%%%%%%%%%%%%%%%%%%%%%%%%%%%%%%%%%%%%%%%%%%%%%%%%%
% Resumo em português
\begin{resumo}{tomossíntese digital mamária, reconstrução de imagens, processamento digital de imagens, retroprojeção filtrada, reconstrução iterativa}

	A tomossíntese digital da mama é um exame radiográfico utilizado para o rastreamento do câncer de mama, que supera a limitação de sobreposição de tecidos existente na mamografia digital \acs{2D}. Nessa técnica são adquiridas projeções radiográficas em diferentes ângulos que são processadas para a reconstrução do volume da mama. Um grande desafio é a elaboração dos algoritmos para a reconstrução tomográfica, visto que há um número limitado de projeções com uma baixa dose, abrangendo uma estreita faixa de ângulo. Em geral os algoritmos buscam reconstruir o volume da maneira mais fiel possível, reduzindo os artefatos nas imagens, adquirindo um bom contraste e tudo isso com um baixo custo computacional. Esse trabalho tem como objetivo o estudo de algoritmos de reconstrução de imagens para tomossíntese da mama, avaliando seus desempenhos em relação à qualidade da imagem formada. Os métodos analíticos de retroprojeção filtrada, bem como os interativos de máxima verossimilhança e algébricos são discutidos. A redução de artefatos de alta atenuação também é alvo de estudo e implementações práticas. Para o desenvolvimento do trabalho foram utilizados \textit{phantoms} virtuais e reais para a validação dos métodos propostos. Os algoritmos de retroprojeção, retroprojeção filtrada e máxima verossimilhança foram aplicados e suas performances comparadas. Em geral o método de máxima verossimilhança obteve o melhor resultado. Os resultados obtidos na redução de artefatos de alta atenuação ainda não foram satisfatórios e deverão ser aprimorados. Através desse trabalho foi possível implementar e testar diversas técnicas de reconstrução, analisando as vantagens e desvantagens de cada técnica.      	

\end{resumo}

% Resumo em inglês
%\begin{abstract}{Iterative. DBT. MAP-MRF}
%This is the english abstract.
	
%\end{abstract}


%%%%%%%%%%%%%%%%%%%%%%%%%%%%%%%%%%%%%%%%%%%%%%%%%%%%%%%%%%%%%%%%%%%%%%%%%%%%%%%%%%%%%%%%%%%%%%%%%%%%%%%%%%%%%%%%%%%%%%%%%%%%%%												  	DISSERTAÇÃO																%
%%%%%%%%%%%%%%%%%%%%%%%%%%%%%%%%%%%%%%%%%%%%%%%%%%%%%%%%%%%%%%%%%%%%%%%%%%%%%%%%%%%%%%%%%%%%%%%%%%%%%%%%%%%%%%%%%%%%%%%%%%%%%


% ---
% inserir lista de ilustrações
% ---
\listailustracoes
% ---

% ---
% inserir lista de tabelas
% ---
\listatabelas
% ---

% ---
% inserir lista de abreviaturas e siglas
% ---
\listasiglas{abrev/Abreviaturas}
% ---

% ---
% inserir o sumario
% ---
\sumario
% ---

% ----------------------------------------------------------
% ELEMENTOS TEXTUAIS
% ----------------------------------------------------------
\mainmatter

% ----------------------------------------------------------
% Introdução
% ----------------------------------------------------------

\chapter[Introdução]{Introdução}\label{Introdução}


%%%%%%%%%%%%%%%%%%%%%%%%%%%%%%%%%%%%%%%%%%%%%%%%%%%%%%%%%%%%%%%%%%%%%%%%%%%%%%%%%%%%%%%%%%%%%%%%%%%%%%%%%%%%%%%%%%%%%%%%%%%%%%												 	 Motivação    															%
%%%%%%%%%%%%%%%%%%%%%%%%%%%%%%%%%%%%%%%%%%%%%%%%%%%%%%%%%%%%%%%%%%%%%%%%%%%%%%%%%%%%%%%%%%%%%%%%%%%%%%%%%%%%%%%%%%%%%%%%%%%%%
\section{Motivação}
Segundo a \ac{OMS}, o câncer é uma preocupação crescente da saúde pública no âmbito mundial e requer um aumento de atenção, priorização e financiamento. Ainda, segundo a organização, o câncer é a segunda maior causa de morte em todo o mundo, com crescimento de novos casos de 14,1 milhões em 2012 para 21,6 milhões projetados em 2030. A grande preocupação está nos países em desenvolvimento, nos quais esses números crescem mais rapidamente e o gasto estimado anualmente é de 1,16 trilhão de dólares \cite{oms}.

De acordo com os dados do \ac{INCA}, no Brasil, a estimativa indica a ocorrência de 600 mil novos casos para o biênio 2018-2019. Em mulheres, o tipo mais frequente será o de mama\footnote{ Com exceção do câncer de pele não melanoma.} com previsão de 60 mil casos, sendo esse número 21\% do total para o sexo feminino \cite{inca}.

Grandes esforços são necessários para minimizar essas ocorrências. Dentre as ações recomendadas para a diminuição dos casos em geral, está o diagnóstico precoce, o qual deve ser acessível à toda população. O fornecimento de capacitação para as forças de trabalho e o aperfeiçoamento de dados para o auxílio de tomada de decisão é também de suma importância \cite{oms}. 

O câncer de mama possui uma maior ocorrência na população feminina, excetuando-se os casos de câncer de pele não melanoma, conforme as estatísticas do \ac{INCA}. No ano de 2018 foram estimados aproximadamente 627 mil óbitos no mundo, representando a mais elevada causa de morte por câncer em mulheres \cite{oms}.

Dentre as estratégias para o controle e detecção estão o rastreamento e o diagnóstico precoce. A mamografia é o método preconizado para o rastreamento, porém, outras técnicas são utilizadas como: o autoexame e o exame clínico das mamas, o uso de equipamentos de ressonância nuclear magnética, ultrassonografia, termografia e tomossíntese \cite{inca}. A utilização da mamografia como exame para o rastreamento reduz em aproximadamente 20\% a taxa de mortalidade. Essa técnica consiste em examinar mulheres em determinada faixa etária para identificar o câncer de mama antes mesmo de quaisquer sintomas. O exame com fins de rastreamento é indicado para mulheres na faixa etária de 50 a 69 anos em locais com boas condições no sistema de saúde \cite{oms}.

Apesar de ser amplamente usada para rastreamento, o exame de mamografia possui algumas limitações. A maior e mais conhecida, citada na literatura, é a sobreposição de tecidos, que obscurece possíveis lesões, dentre as quais podem haver aquelas malignas, induzindo um diagnóstico errado \cite{vedantham2015digital}. 

A tomossíntese digital da mama (\textit{Digital Breast Tomosynthesis} - \acs{DBT}) é uma técnica tomográfica de ângulo limitado, desenvolvida para minimizar os problemas relacionados à sobreposição de tecidos da mamografia \acs{2D}. Nesse método, múltiplas projeções de raios X da mama são adquiridas em diferentes ângulos, enquanto o tubo se move em uma trajetória fixa pré-definida. Ao final do exame, as imagens radiográficas são processadas para a reconstrução pseudo-\acs{3D} do volume da mama \cite{vedantham2015digital}.  

Diversos métodos de reconstrução de imagem vêm sendo estudados e comparados \cite{wu2004comparison,zhang2006comparative}. No entanto, a elaboração desses algoritmos para tomossíntese mamária é um grande desafio, uma vez que há um limitado número de projeções que são adquiridas com baixas doses de radiação \cite{wu2004comparison}. A geometria de aquisição varia para cada equipamento de \acs{DBT} \cite{vedantham2015digital}. Dessa forma, não há um consenso sobre qual é o número de projeções ideal, ou o melhor ângulo de aquisição, ou ainda qual o algoritmo de reconstrução que deve ser utilizado para o caso da \acs{DBT} \cite{sechopoulos2009optimization}. As técnicas de reconstrução são comumente divididas, de uma maneira geral, em duas categorias: os métodos analíticos e os iterativos. Dentre estes, pode-se citar os algoritmos de retroprojeção (\textit{Back-projection} - \acs{BP}), retroprojeção filtrada (\textit{Filtered Back-projection} - \acs{FBP}) e os iterativos, sendo o método de \acs{FBP} o mais comum em tomossíntese \cite{michell2018role}.  

Algoritmos de reconstrução iterativa (\textit{Iterative Reconstruction} - \acs{IR}) demandam um alto custo computacional. Devido a esse fato, historicamente, os equipamentos comerciais de tomografia por raios X não utilizam esses métodos. Todavia com o avanço do poder computacional, os métodos iterativos vêm sendo  amplamente utilizados por diversos grupos de pesquisas e fabricantes \cite{wu2003tomographic}. Sistemas comerciais de tomossíntese mamária como, por exemplo os fabricados pela Siemens\footnote{\url{www.healthcare.siemens.com}} e pela Hologic\footnote{\url{www.hologic.com}} utilizam \acs{FBP}, no entanto, algoritmos iterativos já são utilizados pelos equipamentos de \acs{DBT} da \ac{GE}\footnote{\url{www.gehealthcare.com}}. 

Dentre os objetivos dos diversos métodos de reconstrução, a redução de artefatos nas imagens reconstruídas pode ser destacada \cite{hu2008image}. Devido a utilização de um feixe cônico e uma baixa amostragem no domínio da frequência dos sistemas de tomossíntese, os algoritmos de \acs{BP} e \acs{FBP}, quando aplicados, introduzem artefatos de objetos fora do foco em imagens que estão em foco no processo de reconstrução \cite[]{levakhina2013weighted, borges2017metal}. Já os métodos iterativos são capazes de agregar no processo de reconstrução o modelamento físico do sistema em geral, além da possibilidade de incluir restrições à convergência do método a partir de conhecimentos \textit{a priori} \cite{xu2015statistical,levakhina2013weighted}.

Esses modelamentos e restrições vêm sendo incorporados nos processos iterativos de tomossíntese com as recentes pesquisas a fim de aprimorar a qualidade das imagens. Informações \textit{a priori} como a similaridade dos \textit{pixels} e a morfologia da imagem foram incorporadas na reconstrução no trabalho de \citeonline{xu2015statistical}. A redução de artefatos de alta atenuação foi incorporada em um algoritmo iterativo, demonstrada no trabalho de \citeonline[]{levakhina2013weighted}. A correlação do ruído e o respectivo borramento nos detectores indiretos foram estudados por \citeonline[]{zheng2018detector}. Foram dados então os primeiros passos para a criação das reconstruções iterativas baseadas em modelamentos (\textit{Model-Based Iterative Reconstruction }- \acs{MBIR}) aplicadas à \acs{DBT}, que já são extensivamente estudadas para o equipamento de tomografia computadorizada (\textit{Computed Tomography} - \acs{CT}).    

De fato, o ruído tem uma importância muito grande em relação a qualidade das imagens de mamografia que são fornecidas para os radiologistas, tendo em vista a detecção do câncer de mama \cite{haus2000screen,huda2003experimental,ruschin2007dose, saunders2007does, samei2007digital, mackenzie2016relationship}. É extremamente relevante conhecer as fontes dessas perturbações para então modelá-las matematicamente e incorporá-las nos métodos computacionais para os devidos propósitos. Remoção de ruído, redução da dose de radiação e função de restrição para algoritmos de reconstrução iterativa são algumas das aplicações nas quais o entendimento dessas perturbações é importante \cite{wu2012dose, romualdo2013mammographic,borges2016method,borges2017pipeline,borges2017method,mackenzie2017characterisation,zheng2018detector}.

%%%%%%%%%%%%%%%%%%%%%%%%%%%%%%%%%%%%%%%%%%%%%%%%%%%%%%%%%%%%%%%%%%%%%%%%%%%%%%%%%%%%%%%%%%%%%%%%%%%%%%%%%%%%%%%%%%%%%%%%%%%%%%												  	Objetivos	    														%
%%%%%%%%%%%%%%%%%%%%%%%%%%%%%%%%%%%%%%%%%%%%%%%%%%%%%%%%%%%%%%%%%%%%%%%%%%%%%%%%%%%%%%%%%%%%%%%%%%%%%%%%%%%%%%%%%%%%%%%%%%%%%

\section{Objetivos}

Esse trabalho tem como objetivo apresentar uma ferramenta de reconstrução de imagens para a tomossíntese da mama e fazer um estudo do comportamento do sinal e do ruído nas imagens reconstruídas.

Especificamente, o desenvolvimento dessa ferramenta visa ampliar as pesquisas em reconstrução tomográfica da mama, visto que a mesma é disponibilizada de maneira \textit{online} e gratuita. É desejável que o \textit{software} seja capaz de implementar os principais algoritmos de reconstrução de imagens para \acs{DBT}, bem como qualquer geometria de aquisição. Tanto os métodos analíticos quanto os iterativos são incluídos no \textit{software} proposto.

Em trabalhos anteriores, nosso grupo propôs um modelo de função afim para descrever a variância do ruído nas projeções de \acs{DBT}. Neste modelo, o ruído eletrônico é descrito por uma distribuição Gaussiana aditiva e o ruído quântico é descrito por um modelo Gaussiano dependente do sinal com ganho quântico dependente espacialmente \cite{borges2017method,borges2018restoration}. 

Agora, uns dos objetivos desse trabalho é realizar um estudo preliminar do ruído em fatias reconstruídas de \acs{DBT}, a fim de avaliar a adequação desse modelo para imagens pós-reconstrução. Esse modelamento tem a finalidade de fornecer uma base matemática para aplicações que envolvam a remoção de ruído em ambos os domínios, redução de dose de radiação e restrições para algoritmos iterativos de reconstrução.

%%%%%%%%%%%%%%%%%%%%%%%%%%%%%%%%%%%%%%%%%%%%%%%%%%%%%%%%%%%%%%%%%%%%%%%%%%%%%%%%%%%%%%%%%%%%%%%%%%%%%%%%%%%%%%%%%%%%%%%%%%%%%%										     Organização da Monografia														%
%%%%%%%%%%%%%%%%%%%%%%%%%%%%%%%%%%%%%%%%%%%%%%%%%%%%%%%%%%%%%%%%%%%%%%%%%%%%%%%%%%%%%%%%%%%%%%%%%%%%%%%%%%%%%%%%%%%%%%%%%%%%%

\section{Organização da Monografia}

O trabalho é dividido em seu âmbito geral em duas partes, sendo a primeira a respeito do desenvolvimento e validação do \textit{software} de reconstrução e a segunda parte relacionada com a análise e medição do sinal e do ruído em imagens reconstruídas de tomossíntese.

Mais especificamente, o documento é divido em 9 capítulos que visam demostrar todo o trabalho desenvolvido, bem como apresentar os conceitos teóricos fundamentais. 

Primeiramente, no \autoref{Capitulo2}, é feita uma revisão do conteúdo de mamografia digital \acs{2D} e da técnica de tomossíntese da mama. São abordados os fatos históricos, bem como o funcionamento geral de ambas modalidades de exame. São detalhados a física por trás dos equipamentos de \acs{DBT} assim como seus parâmetros geométricos.

Já no \autoref{Capitulo3} é exposto a teoria básica sobre reconstrução, demonstrando os princípios para \acs{2D} e para \acs{3D}. São discutidos os métodos de reconstrução analítica e iterativa. 

No \autoref{Capitulo4}, são expostos os conceitos de ruído em imagens digitais, bem como em exames de mamografia. Formulações matemáticas são feitas para a modelagem do ruído nas projeções e também pós-reconstrução.

Os materiais e métodos utilizados na confecção das duas partes mencionadas são apresentados no \autoref{Capitulo5}  do documento. 

Os resultados obtidos a partir dos experimentos e das avaliações realizadas, bem como as respectivas discussões são descritas no \autoref{Capitulo6}.

Então, no \autoref{Capitulo7}, é feita a conclusão do trabalho e no \autoref{Capitulo8} são apresentados os trabalhos futuros. 

Por fim, no \autoref{Capitulo9} são apontadas as publicações geradas a partir do desenvolvimento desse trabalho.



  

  

\chapter[Mamografia e Tomossíntese]{Mamografia e Tomossíntese}\label{Capitulo2}

%%%%%%%%%%%%%%%%%%%%%%%%%%%%%%%%%%%%%%%%%%%%%%%%%%%%%%%%%%%%%%%%%%%%%%%%%%%%%%%%%%%%%%%%%%%%%%%%%%%%%%%%%%%%%%%%%%%%%%%%%%%%%%											Câncer de Mama e Mamografia														%
%%%%%%%%%%%%%%%%%%%%%%%%%%%%%%%%%%%%%%%%%%%%%%%%%%%%%%%%%%%%%%%%%%%%%%%%%%%%%%%%%%%%%%%%%%%%%%%%%%%%%%%%%%%%%%%%%%%%%%%%%%%%%
\section{Câncer de Mama e Mamografia}

Segundo o \citeonline{inca}, o câncer de mama é o mais frequente em mulheres no mundo todo e também no Brasil, salvo os tipos de pele não melanoma. Ainda, conforme as estatísticas do instituto, para os anos de 2018 e 2019 a previsão é de 60 mil novos casos, representando para as mulheres 21\% do total de casos previstos para estas. 

É estimado que, 28\% dos casos podem ser evitados com práticas de alimentação saudável, atividade física  e adequação do peso corporal. Essas precauções podem ser tomadas a fim de diminuir os fatores de risco da doença que são: o excesso de peso corporal, particularmente após a menopausa, o consumo de bebidas alcoólicas, a terapia de reposição hormonal e a exposição à radiação ionizante \cite{inca}.

O Ministério da Saúde, em conformidade com a \ac{OMS}, recomenda que a mamografia de rastreamento seja feita periodicamente por mulheres na faixa etária entre 50 e 69 anos. Esse exame auxilia na descoberta precoce do câncer, proporcionando um começo de tratamento rápido, diminuído assim as chances de óbito. Estima-se que o uso desse equipamento para o rastreamento reduz em aproximadamente 20\% a taxa de mortalidade da doença \cite{oms}, porém, a mamografia quando aplicada a toda uma população está relacionada a riscos como: resultados incorretos e a exposição aos raios X que podem induzir o câncer na população a longo prazo \cite{yaffe2011risk,inca}. 

%As seções seguintes têm como objetivo detalhar o funcionamento dos equipamentos usados atualmente para rastreamento tanto no Brasil quanto no resto do mundo. Esse capítulo é utilizado como base para o entendimento da evolução dos equipamentos e também para a compreensão dos métodos de formação das imagens detalhadas nas próximas seções.    

%%%%%%%%%%%%%%%%%%%%%%%%%%%%%%%%%%%%%%%%%%%%%%%%%%%%%%%%%%%%%%%%%%%%%%%%%%%%%%%%%%%%%%%%%%%%%%%%%%%%%%%%%%%%%%%%%%%%%%%%%%%%%%													Mamografia																%
%%%%%%%%%%%%%%%%%%%%%%%%%%%%%%%%%%%%%%%%%%%%%%%%%%%%%%%%%%%%%%%%%%%%%%%%%%%%%%%%%%%%%%%%%%%%%%%%%%%%%%%%%%%%%%%%%%%%%%%%%%%%%
\subsection{Mamografia}

O exame de mamografia tem como finalidade proporcionar uma imagem radiográfica da mama, com o intuito de que os profissionais da saúde possam identificar possíveis lesões em meio aos tecidos e estruturas que compõe a mesma (Figura \ref{fig:imgCap2EstruturaMama}), antes mesmo de qualquer sintoma. 

A mama feminina, em sua anatomia é constituída por: lobos, que são glândulas produtoras de leite; ductos, que escoam o leite materno para o mamilo; tecido adiposo e conjuntivo, que revestem os ductos e lobos, dentre outras estruturas. O câncer se caracteriza quando células na mama começam a crescer de maneira descontrolada formando tumores malignos que se espalham por tecidos próximos ou para outras localidades do corpo. Esses tumores podem começar em diversas partes da mama, no entanto, na maioria dos casos, os tumores começam nos ductos ou nas glândulas \cite{americancancersociety2017}.      

\begin{figure}[h]
	\caption{Tecidos e estruturas que compõem uma mama normal.}
	\begin{center}
		\includegraphics[scale=0.65]{imgs/cap2/EstruturaMama.png}
	\end{center}
	\legend{Fonte: Adaptado de \citeonline{americancancersociety2017}.}
	\label{fig:imgCap2EstruturaMama}
\end{figure}


Com o auxílio do exame de mamografia, o câncer de mama pode ser detectado através de quatro indicadores, segundo \citeonline{yaffe2000handbook}:

\begin{enumerate}
	\item Morfologia característica de uma massa tumoral;
	\item Presença de depósitos minerais em forma de partículas, denominadas microcalcificações;
	\item Distorção arquitetural dos padrões teciduais normais;
	\item Assimetria entre regiões correspondentes da imagem esquerda e direita.
\end{enumerate}

A Figura \ref{fig:imgCap2ExameMamografia} ilustra dois exames de mamografia da mesma mama em orientação \ac{CC} e \ac{MLO}, onde são evidenciados os sinais de um possível câncer de mama.

\begin{figure}[htb]
	\centering
	
	\caption{Ilustração de dois exames de mamografia \acs{2D} da mama direita na orientação (a) \acs{CC} e (b) \acs{MLO}, na qual as setas em vermelho evidenciam agrupamentos de microcalcificações e nódulos .}
	
	\subfloat[]{\includegraphics[scale=0.45, clip, trim=13cm 0cm 12.2cm 0cm]{imgs/cap2/FFDM_Cli_CC} \label{fig:imgCap2ExameMamografiaCC}}	
	\hfil
	\subfloat[]{\includegraphics[scale=0.45, clip, trim=12.3cm 0cm 12.2cm 0cm]{imgs/cap2/FFDM_Cli_MLO}\label{fig:imgCap2ExameMamografiaMLO}}
	
	\legend{Fonte: do autor, 2019.}
	\label{fig:imgCap2ExameMamografia}
\end{figure}

Os primeiros registros da utilização de radiografia aplicada à mama foram feitos em 1913 com Albert Salomon. Ele demonstrou o espalhamento de tumores através de imagens radiográficas de mamas extraídas, tento suas pesquisas interrompidas em 1933 devido a segunda guerra mundial. Já em 1930, Stafford L. Warren registrou o uso  \textit{in vivo} da mamografia em 119 pacientes. O autor ressaltou em seu trabalho a exatidão das opiniões baseadas em diagnósticos através do uso da mamografia, em contrapartida com opiniões pré-operatórias sem a utilização do exame. Apesar dos registros de seu uso desde o começo do século XX, a mamografia como equipamento dedicado para o rastreamento, não desapontou até os anos de 1960, quando em 1965, Charles Gros em parceria com a empresa Compagnie Générale de Radiologie desenvolveu, na França, o primeiro equipamento restrito ao exame mamografia, ilustrado pela Figura \ref{fig:imgCap2MamografiaPrimeiroEq} \cite{bassett1988evolution,gold1990highlights}. 

\begin{figure}[htb]
	\caption{Primeiro equipamento dedicado à mamografia.}
	\begin{center}
		\includegraphics[scale=0.75]{imgs/cap2/MamografiaPrimeiroEq}
	\end{center}
	\legend{Fonte: \citeonline[p. 12]{gold1990highlights}.}
	\label{fig:imgCap2MamografiaPrimeiroEq}
\end{figure}

A mamografia por filme fotográfico foi amplamente utilizada por inúmeros anos e considerada por muito tempo um equipamento preconizado para o rastreamento do câncer de mama, porém foi sendo substituída devido as suas limitações técnicas como: a curva característica de contraste limitada, a presença de granularidade no filme, a impossibilidade de pós-processamento e a difícil detecção de lesões em tecidos moles na presença de tecido glandular denso \cite{karellas2008breast,lewin2001comparison}. 

Para superar essas limitações existentes, foi desenvolvida a mamografia digital de campo total (\textit{Full Field Digital Mammography} - \acs{FFDM})  \cite{nishikawa1987scanned,yaffe1988development}. Esse equipamento possibilita a captura de uma imagem digital através de múltiplos sensores, além de poder armazenar, transmitir e mostrar os exames através da tela de um computador. Com essa evolução houve uma maior precisão em diagnósticos para mulheres na faixa etária de 49 anos ou menos e para outras com mamas densas \cite{vedantham2015digital}. A Figura \ref{fig:imgCap2EsquematicoMamografia} ilustra o esquemático de um mamógrafo e detalha suas partes específicas. 

\begin{figure}[htb]
	\caption{Esquemático geral de um equipamento de mamografia atual.}
	\begin{center}
		\includegraphics[scale=0.85]{imgs/cap2/FFDMGeometry}
	\end{center}
	\legend{Fonte: \citeonline[]{guerrero2018}.}
	\label{fig:imgCap2EsquematicoMamografia}
\end{figure}

Apesar de sua grande evolução e ampla utilização, o equipamento de mamografia digital impõe limitações físicas para o diagnóstico dos exames, podendo ser destacada a sobreposição de tecidos \cite{vedantham2015digital}. A mama por si só é constituída de uma estrutura volumétrica em \acs{3D}, mas o equipamento proporciona uma imagem bidimensional, como ilustrado pela Figura \ref{fig:imgCap2ExameMamografia}. Por esse motivo, há a sobreposição de estruturas (Figura \ref{fig:imgCap2MamografiaSobreposicao}), fazendo com que tecidos normais (quadrado vermelho) venham a obscurecer lesões malignas (círculo verde), levando o exame à uma menor sensibilidade na detecção de estruturas que podem ser um indicativo de câncer, altas taxas de falsos-positivos e por fim uma alta taxa de \textit{recall}\footnote{Palavra da língua inglesa que tem como significado, no contexto, de chamar a paciente para realizar um novo exame devido a alguma suspeita ao realizar o diagnóstico.} \cite{roth2014digital}. 

Com o propósito de contornar esses problemas foi desenvolvido o equipamento de \acs{DBT} que é abordado com mais detalhes na próxima seção. 

\begin{figure}[htb]
	\caption{Ilustração da sobreposição de tecidos.}
	\begin{center}
		\includegraphics[scale=0.7, clip, trim=13cm 1.5cm 10cm 2.5cm]{imgs/cap2/FFDM}
	\end{center}
	\legend{Fonte: do autor, 2019.}
	\label{fig:imgCap2MamografiaSobreposicao}
\end{figure}    

%%%%%%%%%%%%%%%%%%%%%%%%%%%%%%%%%%%%%%%%%%%%%%%%%%%%%%%%%%%%%%%%%%%%%%%%%%%%%%%%%%%%%%%%%%%%%%%%%%%%%%%%%%%%%%%%%%%%%%%%%%%%%%											Tomossíntese Digital Mamária													%
%%%%%%%%%%%%%%%%%%%%%%%%%%%%%%%%%%%%%%%%%%%%%%%%%%%%%%%%%%%%%%%%%%%%%%%%%%%%%%%%%%%%%%%%%%%%%%%%%%%%%%%%%%%%%%%%%%%%%%%%%%%%%
\section{Tomossíntese Digital Mamária}     

\subsection{Histórico e Atualidade}

Uma radiografia analógica ou até mesmo a digital, é a representação de um corpo anatômico em \acs{3D} por uma simples projeção \acs{2D}. Ela traz consigo a sobreposição de órgãos, tecidos e outras formações do corpo humano, tornando impossível a localização precisa das estruturas internas \cite{levakhina2014three}. 

Essa preocupação era eminente logo após a descoberta dos raios X por Wilhelm Röntgen, em 1895, onde pesquisadores da época já buscavam soluções para a reconstrução volumétrica dos objetos. O conceito de impressão tridimensional de uma cena, todavia, foi introduzido pelo princípio da estereoscopia por Charles Wheatstone, em 1838, muito antes do descobrimento dos raios X. Já no começo do século XX, inúmeros cientistas ao redor do mundo relataram trabalhos e pesquisas com o intuito da reconstrução de estruturas internas através dos raios X \cite{dobbins2003digital,levakhina2014three}. 

\begin{figure}[htb]
	\caption{Princípio da tomografia convencional.}
	\begin{center}
		\includegraphics[scale=0.6, clip, trim=12cm 6.2cm 12cm 1.8cm]{imgs/cap2/TomographyPriciple}
	\end{center}
	\legend{Fonte: do autor, 2019.}
	\label{fig:imgCap2TomographyPriciple}
\end{figure}

No ano de 1921, A. E. M. Bocage descreveu o primeiro sistema tomográfico convencional, onde o tubo de raios X e o filme moviam-se linearmente em direções opostas ao longo do paciente (Figura \ref{fig:imgCap2TomographyPriciple}) com o propósito de gerar um plano em foco do objeto e desfocar as estruturas fora desse plano \cite{hsieh2009computed}. Contudo, um dos grandes problemas relacionados a esta descoberta, é o fato de ser preciso a realização de diversos exames para a obtenção de diferentes fatias do corpo. 

Já no ano de 1932, o pesquisador holandês B. G. Ziedses des Plantes publicou um trabalho \cite{des1932neue} que descrevia a possibilidade da formação de diversos planos do objeto a partir de um número definido de projeções do mesmo. As implementações mais marcantes da teoria de Ziedses des Plantes foram feitas por \citeonline []{garrison1969three} e mais tarde por \citeonline []{miller1971infinite}. Os autores demonstraram o princípio da tomografia convencional discreta por meio da utilização de luz, lentes e espelhos para a reconstrução de um número arbitrário de planos tomográficos do objeto examinado. O termo tomossíntese, entretanto, só ficou conhecido um ano mais tarde com o trabalho de \citeonline []{grant1972tomosynthesis}. Uma descrição mais detalhada sobre a história do desenvolvimento da tomossíntese pode ser encontrada em  \citeonline []{dobbins2003digital,levakhina2014three} e \citeonline []{goodsitt2014history}. 

Embora a descoberta da técnica de tomossíntese tenha sido feita no início do século passado, seu real interesse e sua ampla utilização para exames da mama aconteceram recentemente com o trabalho de  \citeonline []{niklason1997digital} no Massachusetts General Hospital na década de 90. Isso se deve ao fato da sua difícil implementação e da escassez de recursos tecnológicos naquela época. 

O avanço da computação, a introdução de detectores digitais de tela plana e a possibilidade da sua rápida leitura foram peças essenciais para a volta da tomossíntese ao foco das pesquisas junto a sua vasta utilização, principalmente no âmbito do rastreamento do câncer de mama \cite{Niklason20185}. 

Hoje essa técnica está se espalhando rapidamente e substituindo os equipamento de mamografia digital, de acordo com \citeonline [p. 6]{Niklason20185}. O gráfico da Figura \ref{fig:imgCap2TimelineDBT} ilustra o número de publicações científicas nos últimos 10 anos com o termo ``\textit{Digital Breast Tomosynthesis}'' na base de dados do site PubMed\footnote{\url{www.ncbi.nlm.nih.gov/pubmed/}}, demonstrando um grande aumento de interesse na área ao longo dos anos.
 
\begin{figure}[H]
	\caption{Número de publicações científicas nos últimos 10 anos com o termo ``\textit{Digital Breast Tomosynthesis}''.}
	\begin{center}
		\includegraphics[scale=0.55, clip, trim=1.7cm 9.4cm 1.9cm 10.8cm]{imgs/cap2/TimelineDBT}
	\end{center}
	\legend{Fonte: PubMed, 2018.}
	\label{fig:imgCap2TimelineDBT}
\end{figure}

No ano de 2011, o equipamento da empresa Hologic foi aprovado pelo órgão regulamentador dos Estados Unidos (\textit{Food and Drug Administration} - \acs{FDA})\footnote{\url{www.fda.gov}} para ser comercializado e utilizado na prática clínica. 

A \acs{DBT} é caracterizada por ser uma técnica tomográfica de ângulo limitado. Nesse exame, múltiplas radiografias, denominadas de projeções, são adquiridas em diferentes ângulos, demonstrado na Figura \ref{fig:imgCap2DBTEstrutura}, enquanto o tubo se move em uma trajetória fixa pré-definida, amenizando assim o problema de sobreposição de tecidos existente no equipamento de \acs{FFDM}. Ao final do conjunto de exposições, as projeções são processadas por um algoritmo e, por fim, é reconstruído o volume pseudo-\acs{3D} da mama, como ilustra a Figura \ref{fig:imgCap2ExameDBTRecon} \cite{vedantham2015digital,michell2018role}. Essas fatias são então apresentadas ao radiologista para seu devido laudo médico, como demonstra a Figura \ref{fig:imgCap2ExameDBT}. 

Resumidamente, a técnica de \acs{DBT} é muito semelhante a mamografia digital comum,  diferenciando-se somente na rotação do tubo de raios X e na utilização de um algoritmo para a reconstrução do volume da mama \cite{michell2018role}.

\begin{figure}[htb]
	\caption{Geometria básica de aquisição de um equipamento de \acs{DBT}.}
	\begin{center}
		\includegraphics[scale=0.7, clip, trim=11cm 1.48cm 9.5cm 2.5cm]{imgs/cap2/DBT}
	\end{center}
	\legend{Fonte: do autor, 2019.}
	\label{fig:imgCap2DBTEstrutura}
\end{figure}


\begin{figure}[htb]
	\caption{Esquemático geral do procedimento de reconstrução do volume \acs{3D} da mama a partir das projeções de raios X.}
	\begin{center}
		\includegraphics[scale=0.7, clip, trim=3.5cm 1.2cm 8.4cm 1.9cm]{imgs/cap2/Recon}
	\end{center}
	\legend{Fonte: do autor, 2019.}
	\label{fig:imgCap2ExameDBTRecon}
\end{figure}

\begin{figure}[htb]
	\caption{Fatias do volume \acs{3D} reconstruídas referente a mama apresentada na Figura \ref{fig:imgCap2ExameMamografia} na sua versão \acs{2D}.}
	\begin{center}
		\includegraphics[scale=0.45, clip, trim=10cm 1cm 10cm 0cm]{imgs/cap2/DBT_Cli}
	\end{center}
	\legend{Fonte: do autor, 2019.}
	\label{fig:imgCap2ExameDBT}
\end{figure}
%são formadas as imagens de recortes da mama paralelas ao detector. 

As variações no processo de aquisição das imagens dessa técnica ocorrem de acordo com cada fabricante, contudo, de uma maneira geral e simplificada, a Figura \ref{fig:imgCap2DBTEstrutura} exemplifica a geometria de aquisição de um equipamento de \acs{DBT}. Nessa figura, o tubo de raios X se move na trajetória de um arco emitindo radiação nas posições de A até C, com angulação de $\theta_{1}$ até $\theta_{3}$ respectivamente. Cada exposição gera uma projeção dos objetos no detector plano, representada na parte inferior da figura.

\subsection{Parâmetros Físicos e Geométricos}\label{ParâmetrosFísicoseGeométricos}

Em geral, os equipamentos de tomossíntese possuem sua estrutura física muito semelhante aos de  mamografia digital. Durante o exame clínico a paciente mantêm-se posicionada de pé, junto ao tubo e ao detector posicionados em uma orientação \ac{CC} ou \ac{MLO} \cite{Niklason20185}. Então, o prato de compressão comprime a mama para realização do exame, com o propósito de deixar os tecidos o mais distribuídos possível e reduzir o movimento da paciente durante o exame. Após esse procedimento são realizadas as exposições de raios X em uma determinada faixa de ângulo com um tipo de movimento do tubo preestabelecido, como ilustra a Figura \ref{fig:imgCap2DBTEstrutura1} \cite{baker2011breast}.  

\begin{figure}[htb]
	\caption{Geometria de um equipamento de \acs{DBT}.}
	\begin{center}
		\includegraphics[scale=0.75, clip, trim=13cm 4cm 9.5cm 5cm]{imgs/cap2/DBT1}
	\end{center}
	\legend{Fonte: do autor, 2019.}
	\label{fig:imgCap2DBTEstrutura1}
\end{figure}

A movimentação pode ser dada de duas maneiras: \textit{Step-and-shoot} ou contínuo (\textit{continuous tube motion}). No segundo modo, o tubo se move de maneira ininterrupta disparando a radiação em tempos espaçados igualmente ao longo de todo período. Esse método gera uma redução no tempo do exame e na movimentação do paciente, em contrapartida, há um aumento no borramento da imagem devido ao movimento do ponto focal durante os disparos \cite{glick2014system}. 

Já no modo \textit{Step-and-shoot}, o tubo para o movimento totalmente antes de cada exposição e logo em seguida retorna ao seu movimento. Esse processo elimina quase totalmente o borramento relacionado à movimentação, porém ainda existente em pequenas vibrações do tubo devido a sua inércia e seu peso \cite{glick2014system}.

No que se refere à rotação dos detectores, alguns fabricantes optam por sua movimentação ou não. Mantê-lo estacionário reduz sua complexidade mecânica e consequentemente o borramento das imagens. Já sua angulação possibilita a confecção de detectores menores e também uma redução no borramento dos raios X, que são incididos com uma maior angulação nos detectores fixos \cite{glick2014system}. 

Para a angulação do tubo, os equipamentos comercias variam entre 15$\degree$ e 50$\degree$, com um número de 9 a 25 exposições, porém de acordo com \citeonline [p. 25]{glick2014system} não existe ao certo um valor ótimo para um número de projeções nem para a faixa de ângulo. Um estudo de \citeonline []{sechopoulos2009optimization} com 63 combinações de ângulos e projeções, concluiu que os parâmetros que obtiveram melhores resultados no experimento foram com uma faixa total de ângulo de 60$\degree$ com 13 projeções. Ainda segundo os autores, com o intuito de aumentar a resolução na direção Z é necessário maximizar a extensão do ângulo de aquisição, entretanto o aumento no número de projeções não está relacionado com essa melhoria. Segundo \citeonline []{hu2008image}, a diminuição do ângulo além de causar a redução de resolução em Z, produz também um borramento entre estruturas em profundidades diferentes. Todavia \citeonline [p. 7]{Niklason20185} afirma que a utilização de ângulos menores está associada a uma melhor visualização de calcificações na mama. 

Devido a substituição da mamografia digital pela tomossíntese, a preocupação com a manutenção da dose de radiação é eminente. Conforme \citeonline [p. 9]{Hooley20189} a dose cumulativa é aproximadamente a mesma para ambos os exames ou um pouco maior para a \acs{DBT}. Ainda de acordo com os autores, a dose de radiação da combinação de ambos exames é próxima a $2,65 mGy$ (\textit{miligray})\footnote{Quantidade de energia de radiação absorvida em um quilograma de matéria \cite{Hooley20189}.}, onde $1,2 mGy$ e $1,45 mGy$ são oriundos da mamografia e tomossíntese respectivamente. É importante ressaltar que esses valores estão abaixo dos limites estabelecidos pela norma norte-americana \ac{MQSA} e que o valor da dose é dependente do fabricante e do seu respectivo modelo. 

O número de projeções está intimamente ligado com a dose total e com os artefatos na imagem reconstruída. Ainda, em conformidade com \citeonline [p. 7]{Niklason20185}, um baixo número de projeções está associado com a geração de artefatos em planos fora de foco decorrentes de objetos com alta atenuação, como as microcalcificações. Em contrapartida, um elevado número de projeções está limitado ao valor da dose total de radiação associado ao exame. 

Dado uma dose total de $X\,mGy$ com um número $N$ de projeções, a dose individual $X_{i}$ de cada projeção é dada por: $X_{i} = \frac{X}{N}$, ou seja, a dose total é dividida igualmente para cada projeção. Quanto menor a dose por imagem $X_{i}$, menor será sua relação sinal-ruído (\textit{Signal-to-Noise Ratio} - \acs{SNR}), devido ao ruído ser modelado através de uma distribuição \textit{Poisson}, como detalhado no \autoref{Capitulo4}. Sendo assim, a \acs{SNR} de cada projeção é inversamente proporcional ao número total de projeções \cite{sechopoulos2009optimization}. Isso resulta ao final em uma degradação na qualidade da imagem reconstruída caso o número de exposições seja elevado.  No entanto, existe um compromisso entre a dose de radiação e a qualidade da imagem exposta ao radiologista, de modo que, do ponto de vista da dosagem de radiação deseja-se minimizar o número de projeções, mas para a qualidade da imagem reconstruída ser boa é preciso aumentar o número de projeções.

Outra consideração a ser feita para os equipamentos de \acs{DBT} é a sua geometria de emissão dos feixes. Diferente de \acs{CT}, a emissão dos feixes de raios X é feita através de uma geometria de meio cone (\textit{half-cone-beam}). Esse fato deve ser levado em consideração em ambos procedimentos de projeção e retroprojeção nos algoritmos de reconstrução \cite{wu2004comparison}. 
        
Tomando como base os sistemas atuais, eles se diferem em diversos aspectos como: a geometria física, a angulação do tubo, a movimentação do detector, o número de projeções, o tamanho do \textit{pixel}, o método de aquisição direto ou indireto, os algoritmos de reconstrução, o tempo do exame etc. 

\begin{table}[h]
	\centering
	\caption{Caraterísticas dos sistemas  de \acs{DBT}.}
	\label{tab:tabCap2SistemasDBT}
	\footnotesize
	\begin{tabular}{l|c|c|c}
		\textbf{Fabricantes}                                       &        \textbf{\acs{GE}}        &                 \textbf{Hologic}                 &   \textbf{Siemens}   \\
		[5pt]
		\hline
		\hline
		\rule[-0.5ex]{-3pt}{3ex}
		Modelo &    SenoClaire{\footnotesize\texttrademark}    & Selenia\textsuperscript{\textregistered} Dimensions\textsuperscript{\textregistered} & Mammomat Inspiration \\ \hline
		\rule[-0.5ex]{-3pt}{3ex}
		Número de projeções              &                9                &                        15                        &          25          \\ \hline
		\rule[-0.5ex]{-3pt}{3ex}
		Angulação do tubo                &           25$\degree$           &                   15$\degree$                    &     50$\degree$      \\ \hline
		\rule[-0.5ex]{-3pt}{3ex}
		Angulação do detector            &          Estacionário           &                   4,2$\degree$                   &     Estacionário     \\ \hline
		\rule[-0.5ex]{-3pt}{3ex}
		Movimento do tubo                &     \textit{Step-and-shoot}     &                     Contínuo                     &       Contínuo       \\ \hline
		\rule[-0.5ex]{-3pt}{3ex}
		Tempo do exame                   &               7s                &                       3,7s                       &         25s          \\ \hline
		\rule[-0.5ex]{-3pt}{3ex}
		Tamanho do detector              &             24x30cm             &                     24x29cm                      &       24x30cm        \\ \hline
		\rule[-0.5ex]{-3pt}{3ex}
		Tamanho do \textit{pixel}        &            100$\mu$m            &         70$\mu$m (2x2 \textit{Binning})          &       85$\mu$m       \\ \hline
		\rule[-0.5ex]{-3pt}{3ex}
		Tipo do detector                 &         \acs{a-Si} (Indireto)         &                  \acs{a-Se} (Direto)                   &    \acs{a-Se} (Direto)     \\ \hline
		\rule[-0.5ex]{-3pt}{3ex}
		Método de reconstrução           & Iterativo (ASiR\textsuperscript{\textregistered}) &                    \acs{FBP}                     &      \acs{FBP}       \\ \hline
	\end{tabular}
	\vspace{2ex}
	\legend{Fonte: \citeonline{michell2018role,vedantham2015digital,sechopoulos2013review,baker2011breast}.}
\end{table}

A Tabela \ref{tab:tabCap2SistemasDBT} mostra a relação entre as principais diferenças encontradas nos equipamentos comerciais de \acs{DBT} aprovados pelo \acs{FDA}. É possível observar uma grande variação entre as propriedades de cada modelo especificado. Isso se deve ao fato de que cada fabricante utiliza de suas ferramentas de pesquisa e desenvolvimento para solucionar um determinado problema e cada sistema possui os seus benefícios e suas limitações. 

Os equipamentos em sua essência buscam a melhor forma para obtenção das imagens, sendo assim, visam um bom contraste, menor dose, maior resolução  e uma boa relação sinal-ruído. Para isso, são utilizadas diferentes abordagens na aquisição dessas imagens \cite{vedantham2015digital}. 




 
    


\chapter[Reconstrução]{Reconstrução de imagens}\label{Capitulo3}

%%%%%%%%%%%%%%%%%%%%%%%%%%%%%%%%%%%%%%%%%%%%%%%%%%%%%%%%%%%%%%%%%%%%%%%%%%%%%%%%%%%%%%%%%%%%%%%%%%%%%%%%%%%%%%%%%%%%%%%%%%%%%%												  Introdução    															%
%%%%%%%%%%%%%%%%%%%%%%%%%%%%%%%%%%%%%%%%%%%%%%%%%%%%%%%%%%%%%%%%%%%%%%%%%%%%%%%%%%%%%%%%%%%%%%%%%%%%%%%%%%%%%%%%%%%%%%%%%%%%%%

Com o objetivo de obter imagens radiográficas feitas através de um exame de tomografia, seja esse advindo de qualquer equipamento, é necessária a aplicação de procedimentos para a reconstrução dessas imagens, dado as informações que o equipamento gera. Tais procedimentos envolvem aplicações de técnicas matemáticas, físicas e computacionais. As informações geradas pelo equipamento são comumente denominadas de projeções, pois representam ``sombras'' que o objeto produz ao ser irradiado por feixes de raios X. As sombras, ou mais precisamente, projeções, fornecem informações das atenuações sofridas pelos raios X quando estes interagem com o tecido biológico ao longo de seu caminho \cite{buzug2008computed,avinash1988principles}.
 
Os métodos de reconstrução atuais destinados à tomossíntese foram primeiramente desenvolvidos para \acs{CT}. Posteriormente, estes foram adaptados as necessidades específicas de cada equipamento e em geral, os métodos utilizados para \acs{CT} são aplicáveis à tomossíntese, respeitando as devidas particularidades. Contudo o desenvolvimento de algoritmos para tomossíntese é considerado um desafio pelo fato dessa técnica possuir poucas projeções em uma estreita extensão angular e baixa dose de radiação \cite{levakhina2014three,yang2012numerical}.   

%%%%%%%%%%%%%%%%%%%%%%%%%%%%%%%%%%%%%%%%%%%%%%%%%%%%%%%%%%%%%%%%%%%%%%%%%%%%%%%%%%%%%%%%%%%%%%%%%%%%%%%%%%%%%%%%%%%%%%%%%%%%%%											Problemas Inversos	    														%
%%%%%%%%%%%%%%%%%%%%%%%%%%%%%%%%%%%%%%%%%%%%%%%%%%%%%%%%%%%%%%%%%%%%%%%%%%%%%%%%%%%%%%%%%%%%%%%%%%%%%%%%%%%%%%%%%%%%%%%%%%%%%%
\section{Problemas Inversos}

Em geral, reconstruções de imagens tomográficas são consideradas, do ponto de vista matemático, como um problema inverso, ou seja, busca-se a formação da estrutura espacial do objeto a partir das projeções do mesmo em diferentes ângulos \cite{buzug2008computed}. A Figura \ref{fig:imgCap3ProbInver} ilustra o procedimento tomando projeções do objeto por meio de ângulos distintos. A Figura \ref{fig:imgCap3ProbInverA} exemplifica o imageamento do objeto e a formação de suas respectivas projeções, já a Figura \ref{fig:imgCap3ProbInverB} exemplifica o problema inverso para a reconstrução da estrutura espacial do objeto. 

%Forward problem and Inverse problem 

\begin{figure}[H]
	\centering
	
	\caption{Processo de reconstrução de objetos \acs{2D} através de: (a) suas projeções em diferentes ângulos e a (b) formulação do problema inverso.}
	
	\subfloat[]{\includegraphics[scale=0.8, clip, trim=12.6cm 7.5cm 12.3cm 3.2cm]{imgs/cap3/ProbInverA.pdf}	\label{fig:imgCap3ProbInverA}}
	\hfill
	\subfloat[]{\includegraphics[scale=0.8, clip, trim=12.6cm 7.5cm 12.3cm 3.2cm]{imgs/cap3/ProbInverB.pdf} \label{fig:imgCap3ProbInverB}}
	
	\legend{Fonte: do autor, 2019.}
	\label{fig:imgCap3ProbInver}
\end{figure}

Os equipamentos de raios X por transmissão têm como modelo físico básico a formulação de \textit{Beer-Lambert} \cite{zeng2010medical}. Essa fórmula relaciona a função de atenuação $\mu(\sigma)$ do feixe de radiação quando este interage com um certo material por um determinado caminho. Considerando um feixe de raio X ideal, infinitamente pequeno, monoenergético e sem espalhamento, sua intensidade de energia inicial $I_{0}$ interage com a matéria por um caminho $L$ e sua intensidade de energia final $I$ é dada por \eqref{eq:eqCap3BeerLambert1}, segundo \citeonline[p. 13]{levakhina2014three}:

\begin{equation}
 I = I_{0} \, e \, ^{-\int_{L}^{} \,\mu(\sigma) \, d\sigma}.
\label{eq:eqCap3BeerLambert1}
\end{equation}

Considerando a Equação \ref{eq:eqCap3BeerLambert1}, se tomarmos o logaritmo da razão entre a intensidade final e a inicial com seu sinal negativo e denotarmos essa formulação como $p_{L}$, ou seja, $p_{L} = -\ln \, \left(\frac{I}{I_{0}}\right)$, tem-se o seguinte resultado:

\begin{equation}
p_{L} = {\int_{L}^{} \,\mu(\sigma) \, d\sigma},
\label{eq:eqCap3BeerLambert2}
\end{equation}

\noindent onde $\sigma$ é o incremento ao longo de $L$ e o problema inverso, ou de reconstrução, passa a ser a recuperação da função  $\mu(\sigma)$, tendo disponível um conjunto de integrais de linha $p_{L}$, calculadas através da transformada de Radon detalhada na seção seguinte \cite[p. 33]{levakhina2014three}.  
 
 
%%%%%%%%%%%%%%%%%%%%%%%%%%%%%%%%%%%%%%%%%%%%%%%%%%%%%%%%%%%%%%%%%%%%%%%%%%%%%%%%%%%%%%%%%%%%%%%%%%%%%%%%%%%%%%%%%%%%%%%%%%%%%%											Transformada de Radon    														%
%%%%%%%%%%%%%%%%%%%%%%%%%%%%%%%%%%%%%%%%%%%%%%%%%%%%%%%%%%%%%%%%%%%%%%%%%%%%%%%%%%%%%%%%%%%%%%%%%%%%%%%%%%%%%%%%%%%%%%%%%%%%%%
\section{Transformada de Radon}\label{TransformadaRadon2D}

No ano de 1917, o matemático Johann Radon publicou um estudo detalhando a solução para os ditos problemas inversos em \acs{CT}, tornando assim a transformada mais importante no aspecto teórico matemático da área atualmente \cite{radon1917uber}. Esse trabalho foi posteriormente traduzido para a língua inglesa \cite{radon1986determination}.

Essencialmente, a transformada de Radon direta visa calcular as projeções $P(\rho,\theta)$ de um objeto com coordenadas $f(x,y)$ a partir de determinados ângulos $\theta$ e feixes $L_{n}$, como ilustra a Figura \ref{fig:imgCap3Radon}. Essas projeções são obtidas através de um conjunto de integrais de linha de $f(x,y)$ ao longo de feixes paralelos  $L_{n}$, como demonstra a Equação \ref{eq:eqCap3Radon1}, onde $d\sigma$ é o incremento ao longo dos feixes. Já a sua transformada inversa é a simples tomada dos valores da projeção e a distribuição destes ao longo do caminho percorrido pelo feixe paralelo. Se tomarmos suficientes projeções do objeto, podemos ao final reconstruir sua estrutura espacial. 

\begin{equation} 
P(\rho,\theta) = {\int_{L}^{} f(x,y) \, d\sigma}.
\label{eq:eqCap3Radon1}
\end{equation}


\begin{figure}[H]
	\caption{Ilustração da transformada de Radon direta.}
	\begin{center}
		\includegraphics[scale=0.8, clip, trim=12cm 4.5cm 12.8cm 5.2cm]{imgs/cap3/Radon.pdf}
	\end{center}
	\legend{Fonte: do autor, 2019.}
	\label{fig:imgCap3Radon}
\end{figure}

Como pode ser visto na Equação \ref{eq:eqCap3Radon1} e na Figura \ref{fig:imgCap3Radon} a projeção está em função das duas variáveis: $\rho$ e $\theta$. Isso pode ser feito, uma vez que, as retas $L$ que representam os feixes são parametrizadas na forma polar, ou seja, $L$ é perpendicular ao segmento de reta que liga  o ponto $(\rho, \theta)$ a origem do plano,  \cite{yang2012numerical}. Isto é, sua forma é dada pela seguinte equação:

\begin{equation} 
L_{n} = 
\begin{cases}
x = \rho \, cos(\theta) - \sigma \, sin(\theta)\\ 
y = \rho \, sin(\theta) + \sigma \, cos(\theta)
\end{cases},
\label{eq:eqCap3Radon2}
\end{equation}

\noindent na qual $\sigma$ é a distância entre $\rho$ e qualquer ponto na reta $L$. A Figura \ref{fig:imgCap3RadonRhoTheta} ilustra a relação entre essas variáveis.

\begin{figure}[H]
	\caption{Ilustração da relação entre $x$, $y$, $\rho$, $\theta$ e $\sigma$.}
	\begin{center}
		\includegraphics[scale=1.2]{imgs/cap3/RadonRhoTheta.pdf}
	\end{center}
	\legend{Fonte: \citeonline[p. 74]{yang2012numerical}.}
	\label{fig:imgCap3RadonRhoTheta}
\end{figure}

Após a parametrização das retas, podemos reescrever a Equação \ref{eq:eqCap3Radon1} em função da Equação \ref{eq:eqCap3Radon2} e então teremos o seguinte resultado:

\begin{equation} 
P(\rho,\theta) = \int_{L}^{} f(x,y) \, d\sigma = {\int_{-\infty}^{+\infty} f\left((\rho \, cos(\theta) - \sigma \, sin(\theta)),(\rho \, sin(\theta) + \sigma \, cos(\theta)\right) \, d\sigma},
\label{eq:eqCap3Radon3}
\end{equation}

\noindent ou fazendo o uso da função delta de Dirac $\delta(\rho)$ teremos:

\begin{equation} 
P(\rho,\theta) = {\int_{-\infty}^{+\infty} \, \int_{-\infty}^{+\infty} f(x,y) \, \delta(\rho - (x\,cos(\theta)+y\,sin(\theta)))  \, dxdy},
\label{eq:eqCap3Radon4}
\end{equation} 

\noindent na qual a Equação \ref{eq:eqCap3Radon4} é conhecida como a transformada direta de Radon da função $f(x,y)$. Um maior detalhamento dos equacionamentos pode ser encontrado em \citeonline[p. 73-74]{yang2012numerical}.

%%%%%%%%%%%%%%%%%%%%%%%%%%%%%%%%%%%%%%%%%%%%%%%%%%%%%%%%%%%%%%%%%%%%%%%%%%%%%%%%%%%%%%%%%%%%%%%%%%%%%%%%%%%%%%%%%%%%%%%%%%%%%%											Métodos de Reconstrução    														%
%%%%%%%%%%%%%%%%%%%%%%%%%%%%%%%%%%%%%%%%%%%%%%%%%%%%%%%%%%%%%%%%%%%%%%%%%%%%%%%%%%%%%%%%%%%%%%%%%%%%%%%%%%%%%%%%%%%%%%%%%%%%%%
\section{Métodos de Reconstrução}

Após a formulação da transformada de Radon é possível obter a estrutura espacial do objeto a partir de suficientes projeções. Todo equacionamento, no entanto, é proposto em modo contínuo fazendo o uso de integrais, por exemplo, e sendo necessário o uso de uma geometria com feixes de raios paralelos. Tendo dito isso, é eminente a necessidade da discretização dessa transformada e adequação ao sistema físico de cada equipamento para implementações práticas \cite{levakhina2014three}. 

As técnicas de projeção e retroprojeção são essenciais nos métodos de reconstrução, pois representam através de um modelo matemático o processo físico de aquisição da imagem. Para os métodos iterativos, a importância é ainda maior pelo fato da utilização de ambas a cada iteração \cite{levakhina2014three}.   

\subsection{Discretização da Imagem}\label{DiscretizaçãodaImagem}

Para a implementação prática da transformada é necessária a representação de um objeto contínuo de uma forma discreta, ou seja, transformá-lo em uma matriz de valores finitos em duas ou três dimensões, dependendo da aplicação. Isso porque os coeficientes de atenuação ao longo de um objeto, por exemplo uma mama, são dados em uma forma contínua e são dependentes de fatores como: número atômico, densidade e espessura do meio \cite{yang2012numerical}.  

Portanto é possível fazer o uso de \textit{pixels} ou \textit{voxels} para representações em \acs{2D} ou \acs{3D} respectivamente. Além dessas duas representações também é possível a utilização de bolhas, ou \textit{blobs} do inglês. Um maior detalhamento sobre o modelamento dessas técnicas pode ser encontrado em  \citeonline[p. 45-69]{levakhina2014three}. A Figura \ref{fig:imgCap3DiscretizacaoImagem1} demonstra a discretização dos coeficientes de atenuações da mama representados por \textit{voxels} cúbicos, dado uma matriz de $m$ linhas por $n$ colunas e $l$ níveis.   

\begin{figure}[H]
	\caption{Discretização dos coeficientes de atenuações da mama.}
	\begin{center}
		\includegraphics[scale=1.1]{imgs/cap3/DiscretizacaoImg1.pdf}
	\end{center}
	\legend{Fonte: \citeonline[p. 72]{yang2012numerical}.}
	\label{fig:imgCap3DiscretizacaoImagem1}
\end{figure}

Se tomarmos qualquer linha dessa matriz, temos a Figura \ref{fig:imgCap3DiscretizacaoImagem2}, ilustrando os coeficientes discretos e os níveis de energia inicial e final da respectiva linha, considerando um feixe ideal. Dado isto, pode-se adaptar a Equação \ref{eq:eqCap3BeerLambert1} para a sua forma discreta, demonstrada a seguir:

\begin{equation}
I = I_{0} \, e \, ^{-\sum_{i=1}^{n} \mu_{i}}.
\label{eq:eqCap3BeerLambertDiscreto1}
\end{equation} 

\begin{figure}[H]
	\caption{Coeficientes de atenuação discretos de qualquer linha da matriz mostrada na Figura \ref{fig:imgCap3DiscretizacaoImagem1}.}
	\begin{center}
		\includegraphics[scale=1.3]{imgs/cap3/DiscretizacaoImg2.pdf}
	\end{center}
	\legend{Fonte: \citeonline[p. 72]{yang2012numerical}.}
	\label{fig:imgCap3DiscretizacaoImagem2}
\end{figure}

\subsection{Projeção e Retroprojeção 2D}

 Os operadores de projeção, essencialmente, visam projetar um objeto em (\acs{3D} ou \acs{2D}) em um anteparo (\acs{2D} ou \acs{1D}) respectivamente, ou seja, somam as contribuições de cada objeto ao longo do eixo de projeção. A Figura \ref{fig:imgCap3ProjeçãoDireta} ilustra a projeção de objetos em \acs{2D} para sensores em \acs{1D}, tomando ângulos de $0\degree$, $45\degree$, $90\degree$ e $135\degree$. 

\begin{figure}[H]
	\caption{Exemplo de projeção \acs{2D} para \acs{1D}.}
	\begin{center}
		\includegraphics[scale=0.55, clip, trim=8cm 1cm 7cm 3.7cm]{imgs/cap3/FP.pdf}
	\end{center}
	\legend{Fonte: do autor, 2019.}
	\label{fig:imgCap3ProjeçãoDireta}
\end{figure}


Já os operadores de retroprojeção atuam no processo inverso, de reprojetar os dados (\acs{2D} ou \acs{1D}) para um espaço (\acs{3D} ou \acs{2D}) nesta ordem, isto é, distribuir os dados de projeção ao longo dos eixos que foram adquiridos. Assim como na seção anterior, a Figura \ref{fig:imgCap3BPMosaico} ilustra o processo da retroprojeção dos dados no espaço da imagem, adquiridos pelos sensores da Figura \ref{fig:imgCap3ProjeçãoDireta} nos respectivos ângulos que foram gerados. 

\begin{figure}[H]
	\centering
	
	\caption{Exemplo de retroprojeção \acs{1D} para \acs{2D}, onde as figuras de (a) até (d) representam os ângulos de $0\degree$, $45\degree$, $90\degree$ e $135\degree$ respectivamente.}
	
	\subfloat[]{\includegraphics[scale=0.2]{imgs/cap3/iRadon_0deg.png}\label{fig:imgCap3BPMosaico0}}
	\hfil
	\subfloat[]{\includegraphics[scale=0.2]{imgs/cap3/iRadon_45deg.png}\label{fig:imgCap3BPMosaico45}}
	\hfil
	\subfloat[]{\includegraphics[scale=0.2]{imgs/cap3/iRadon_90deg.png}\label{fig:imgCap3BPMosaico90}}
	\hfil
	\subfloat[]{\includegraphics[scale=0.2]{imgs/cap3/iRadon_135deg.png}\label{fig:imgCap3BPMosaico135}}
	\hfil
	
	\legend{Fonte: do autor, 2019.}
	\label{fig:imgCap3BPMosaico}
\end{figure}


Ao somar as contribuições de cada ângulo da Figura \ref{fig:imgCap3BPMosaico} a fim de reconstruir o objeto, a Figura \ref{fig:imgCap3BPParcial} é obtida como resultado. Nota-se que a partir dessa operação ainda não é possível identificar as estruturas espaciais com clareza. Já se forem consideradas mais projeções com uma maior faixa de ângulo, como por exemplo de $0\degree$ \text{a} $180\degree$, tornam-se visíveis os objetos, porém ainda de maneira borrada como demostra a Figura \ref{fig:imgCap3BPFinal}. Não há a necessidade de se utilizar $360\degree$, já que as projeções do primeiro e do segundo quadrante são espelhadas em relação ao terceiro e quarto quadrante.   

\begin{figure}[H]
	\centering
	
	\caption{Recuperação dos objetos \acs{2D} através da soma de suas retroprojeções em diferentes ângulos. Onde (a) representa a técnica considerando apenas 4 projeções, ilustrada pela Figura \ref{fig:imgCap3BPMosaico}, e (b) representa a utilização de 180 projeções na faixa de $180\degree$.}
	
	\subfloat[]{\includegraphics[scale=0.3]{imgs/cap3/Backprojection_Parcial.png}\label{fig:imgCap3BPParcial}}
	\hfil
	\subfloat[]{\includegraphics[scale=0.3]{imgs/cap3/Backprojection_Final.png}\label{fig:imgCap3BPFinal}}
	\hfil
	
	\legend{Fonte: do autor, 2019.}
	\label{fig:imgCap3BPSoma}
\end{figure}

O método descrito acima é o mais simples encontrado na literatura e é conhecido como \textit{Pixel Driven} (Figura \ref{fig:imgCap3Projetores1}) \cite[p. 47]{levakhina2014three}. Em sua projeção direta, o método consiste em projetar o centro de cada \textit{pixel} no detector e o seu valor de atenuação é repartido entre os detectores vizinhos por meio da interpolação linear ou por outros métodos mais complexos. O mesmo princípio é aplicado para o operador de retroprojeção, no qual os \textit{pixels} da imagem recebem os valores obtidos por sensores vizinhos, também através de métodos de interpolação.

Outras técnicas mais sofisticadas também são aplicadas como: \textit{Ray Casting} ou \textit{Ray Driven}, \textit{Distance-Driven} \cite{de2002distance,de2004distance} e \textit{Trapezoid footprint}, como demonstra a Figura \ref{fig:imgCap3Projetores}. Um maior detalhamento das técnicas mencionadas pode ser encontrado em \citeonline[p. 47-49]{levakhina2014three} ou ainda, métodos do estado da arte como demostrado em \citeonline[]{liu2017gpu} e \citeonline[]{zheng2017segmented}.

\begin{figure}[H]
	\centering
	
	\caption{Projetores utilizados para técnicas de projeção e retroprojeção para modelos de representação utilizando \textit{Pixels}, onde os diferentes métodos são ilustrados por: (a) \textit{Pixel Driven}, (b) \textit{Ray Casting}, (c) \textit{Distance-Driven} e (d) \textit{Trapezoid footprint}.}
	
	\subfloat[]{\includegraphics[scale=1.1]{imgs/cap3/Projetor1.pdf}\label{fig:imgCap3Projetores1}}
	\hfil
	\subfloat[]{\includegraphics[scale=1.1]{imgs/cap3/Projetor2.pdf}\label{fig:imgCap3Projetores2}}
	\hfil
	\subfloat[]{\includegraphics[scale=1.1]{imgs/cap3/Projetor3.pdf}\label{fig:imgCap3Projetores3}}
	\hfil
	\subfloat[]{\includegraphics[scale=1.1]{imgs/cap3/Projetor4.pdf}\label{fig:imgCap3Projetores4}}
	\hfil
	
	\legend{Fonte: \citeonline[p. 47]{levakhina2014three}.}
	\label{fig:imgCap3Projetores}
\end{figure}  
  
\subsection{Projeção e Retroprojeção 3D}\label{ProjeçãoeRetroprojeção3D}

Os problemas para a reconstrução de imagens em tomossíntese estão relacionados à estruturas tridimensionais, sendo assim é importante discorrer sobre os operadores de projeção e retroprojeção aplicados a essa técnica. A Figura \ref{fig:imgCap33DGeometry} ilustra como um feixe de raio X é atenuando e quais \textit{voxels} são responsáveis por esta interação. Em geral, os operadores utilizados para \acs{2D} são também atribuídos as geometrias \acs{3D}, porém sendo necessários ajustes para cada técnica e geometria. O processamento é feito então plano por plano, passando por todos os \textit{voxels} daquele nível. 

\begin{figure}[htb]
	\caption{Ilustração da geometria de meio cone referente a tomossíntese e a atenuação do feixe de raio X pelos respectivos \textit{voxels}.}
	\begin{center}
		\includegraphics[scale=0.9]{imgs/cap3/3DGeometry.pdf}
	\end{center}
	\legend{Fonte: \citeonline[p. 71]{Duarte2009}.}
	\label{fig:imgCap33DGeometry}
\end{figure}

Para a implementação do operador \textit{Pixel Driven} é necessário encontrar a posição específica onde cada \textit{voxel} será projetado. Segundo \citeonline[p. 77]{dobbins2003digital}, os equipamentos de tomossíntese, em geral, são caracterizados por um movimento isocêntrico. Isso porque tubo e detector ou apenas o tubo se move em um arco com o mesmo centro de movimento. 

Uma das geometrias de movimento isocêntrico foi primeiramente apresentada por \citeonline[]{niklason1997digital}, como ilustra a Figura \ref{fig:imgCap3ProjectionGeometry}, na qual o detector se mantém estacionário e o tubo se move em um arco com um determinado centro de rotação. 

\begin{figure}[htb]
	\caption{Geometria de movimento isocêntrico apresentado por \citeonline[]{niklason1997digital}, em que o tubo se move em um determinado arco e o detector se mantêm estacionário.}
	\begin{center}
		\includegraphics[scale=0.4]{imgs/cap3/ProjectionGeometry.png}
	\end{center}
	\legend{Fonte: Adaptado de \citeonline[p. 76]{dobbins2003digital}.}
	\label{fig:imgCap3ProjectionGeometry}
\end{figure}

Os passos necessários para a formação da imagem com este tipo de movimento foram apresentados no trabalho do autor, onde o mesmo descreve as equações para a projeção de um objeto no plano da imagem. Dado um ângulo de projeção $\theta$ e um ponto qualquer $(X,Y,Z)$ no espaço \acs{3D}, a coordenada \acs{2D} $(X_{i},Y_{i})$ projetada no plano da imagem é dada pelas seguintes equações:

\begin{equation}
Y_{i}(\theta,Y,Z) = Y \, + \dfrac{ Z(L \, sin(\theta) \, + Y)}{L \, cos(\theta) \, + D - Z},
\label{eq:eqCap3ProjectionY}
\end{equation} 

\begin{equation}
X_{i}(\theta,X,Z) = \dfrac{X \, (L \, cos(\theta)\,+\, D)}{L \, cos(\theta) \, + D - Z}.
\label{eq:eqCap3ProjectionX}
\end{equation} 

Já a retroprojeção dos dados é dada a partir da inversão dessas equações a fim de calcular os valores da coordenada $(X,Y)$ para cada fatia em $Z$. O procedimento é executado a partir de cada projeção obtida em um determinado ângulo $\theta$ e por fim são somadas a contribuições de cada projeção para cada fatia em $Z$. O fluxograma da Figura \ref{fig:imgCap3FluxogramaBP} ilustra o procedimento que deve ser feito para a reconstrução do volume. Em geral a combinação matemática utilizada para os planos retroprojetados é uma simples soma ou um média ponderada. Técnicas mais sofisticadas para a redução de artefatos de alta atenuação são apresentadas em \citeonline{levakhina2013weighted} e \citeonline{borges2017metal}.

\begin{figure}[H]
	\caption{Fluxograma representando o processo de simples retroprojeção a fim de reconstruir o volume \acs{3D}.}
	\begin{center}
		\includegraphics[scale=0.5, clip, trim= 5.5cm 0cm 6cm 0cm]{imgs/cap3/FluxogramaBP.pdf}
	\end{center}
	\legend{Fonte: do autor, 2019.}
	\label{fig:imgCap3FluxogramaBP}
\end{figure}

%%%%%%%%%%%%%%%%%%%%%%%%%%%%%%%%%%%%%%%%%%%%%%%%%%%%%%%%%%%%%%%%%%%%%%%%%%%%%%%%%%%%%%%%%%%%%%%%%%%%%%%%%%%%%%%%%%%%%%%%%%%%%%											Reconstrução Analítica    														%
%%%%%%%%%%%%%%%%%%%%%%%%%%%%%%%%%%%%%%%%%%%%%%%%%%%%%%%%%%%%%%%%%%%%%%%%%%%%%%%%%%%%%%%%%%%%%%%%%%%%%%%%%%%%%%%%%%%%%%%%%%%%%%

\subsection{Reconstrução Analítica}

As técnicas utilizadas para a reconstrução em equipamentos de tomossíntese são provenientes de \acs{CT}, pelo fato de já estarem consolidadas e serem vastamente conhecidas na literatura. Isso não é diferente quando o assunto é reconstrução de forma analítica. Todo processo vem da teoria da transformada inversa de Radon, que foi discutida com mais detalhes no item \ref{TransformadaRadon2D}. É importante notar que devido ao problema de reconstrução ser mal condicionado, tal como uma aquisição incompleta de dados, a resolução de forma analítica é uma solução aproximada e não leva em conta diversos fatores, como o ruído quântico presente nas projeções. A melhor forma de explicar esse método é partindo de duas dimensões e posteriormente estendê-lo para a terceira dimensão, como é demonstrado nos itens a seguir \cite{mertelmeier2014filtered,xu2014tomographic}.  


\subsubsection{Retroprojeção Filtrada}\label{RetroprojeçãoFiltrada}

Um grande problema relacionado com a simples retroprojeção dos dados do detector é a soma das baixas frequências que ocorre a partir dos diferentes ângulos, como foi demosntrado pela Figura \ref{fig:imgCap3BPMosaico} e \ref{fig:imgCap3BPParcial}. Isso é facilmente observado quando faz-se o uso do \textit{Fourier slice theorem}, também conhecido como \textit{central slice theorem}, ou traduzindo para o português como teorema do corte de Fourier. 

O problema dito acima pode ser observado na Figura \ref{fig:imgCap3FourierSliceTheorem}, caso sejam tomados diferentes ângulos de projeção. A formulação matemática desse teorema é encontrada em \citeonline[p. 61-65]{hsieh2009computed}. Aplicando a transformada de Fourier inversa no domínio da frequência, ao serem adquiridas diversas projeções em toda a faixa de ângulo, é possível recuperar a estrutura espacial do objeto, porém de forma ``borrada'' devido a sobreposição de baixas frequências.     

\begin{figure}[htb]
	\caption{Ilustração do teorema do corte de Fourier.}
	\begin{center}
		\includegraphics[scale=0.75, clip, trim=8.5cm 4cm 10.4cm 5.6cm]{imgs/cap3/FourierSliceTheorem.pdf}
	\end{center}
	\legend{Fonte: do autor, 2019.}
	\label{fig:imgCap3FourierSliceTheorem}
\end{figure}

Com o propósito de contornar esse problema, é aplicado em cada projeção um filtro de rampa \acs{1D}, conhecido como \textit{ramp filter} (\ref{fig:imgCap3FBPFiltersA}), até a frequência de Nyquist $\frac{1}{2 \varDelta x}$, no qual $\varDelta x$ é o tamanho espacial do \textit{pixel}. Esse filtro faz com que as baixas frequências sejam atenuadas mantendo somente as altas, porém devido ao seu formato, o ruído que predomina na alta frequência é amplificado. 

\begin{figure}[htb]
	\centering
	
	\caption{Ilustração (a) do filtro de rampa no domínio da frequência (passa alta) e (b) o mesmo ``janelado'' por uma função Hanning para redução de ruídos (passa faixa).}
	
	\subfloat[]{\includegraphics[scale=0.5, clip, trim=3.5cm 8.5cm 4.1cm 8.6cm]{imgs/cap3/FiltroRampa.pdf}\label{fig:imgCap3FBPFiltersA}}
	\hfil
	\subfloat[]{\includegraphics[scale=0.5, clip, trim=3.5cm 8.5cm 4.1cm 8.6cm]{imgs/cap3/FiltroRampaHanning.pdf}\label{fig:imgCap3FBPFiltersB}}
	\hfil
	
	\legend{Fonte: do autor, 2019.}
	\label{fig:imgCap3FBPFilters}
\end{figure}

Devido a esse impasse é aplicado um ``janelamento'' no filtro, como por exemplo: Hamming, Hanning, Shepp-Logan ou cosseno. A Figura \ref{fig:imgCap3FBPFiltersA} ilustra o filtro rampa no domínio da frequência e a Figura \ref{fig:imgCap3FBPFiltersB} demonstra o mesmo após um ``janelamento'' pela função Hanning. Seguido do procedimento de filtragem, os dados são retomados para o domínio do espaço e então são retroprojetados para a formação da estrutura espacial do objeto \cite{xu2014tomographic}. 

Já para a aplicação em tomossíntese, diversas aproximações devem ser feitas, pois esta técnica abrange somente uma pequena extensão de ângulo com poucas projeções, quando comparada aos equipamentos de \acs{CT}. Outro problema é com a geometria de emissão dos feixes, que são dispostos em formato de meio cone (\textit{half-cone-beam}) \cite{mertelmeier2014filtered}. 

Tendo em vista esses fatores, com uma aproximação da geometria dos feixes, dispondo-os paralelamente, pode-se utilizar o teorema do corte de Fourier para analisar o comportamento das projeções no domínio da frequência. Isso pode ser feito para grandes distâncias entre o tubo e o detector. A Figura \ref{fig:imgCap3FourierSliceTheorem3D} ilustra a utilização do teorema na técnica de tomossíntese para a análise das projeções no domínio da frequência \cite[p. 101-106]{mertelmeier2014filtered}.

\begin{figure}[H]
	\caption{Ilustração do teorema do corte de Fourier aproximado para a técnica de tomossíntese.}
	\begin{center}
		\includegraphics[scale=1.3]{imgs/cap3/FourierSliceTheorem3D.pdf}
	\end{center}
	\legend{Fonte: Adaptado de \citeonline[p. 102]{mertelmeier2014filtered}.}
	\label{fig:imgCap3FourierSliceTheorem3D}
\end{figure} 

O algoritmo de \citeonline[]{feldkamp1984practical} é amplamente utilizado para \acs{CT} e \acs{DBT}. De maneira geral, o método apresenta uma aproximação do algoritmo de \acs{FBP} utilizado em \acs{CT} proveniente da geometria de feixe em formato de leque (\textit{fan-beam}) para uma geometria de feixe cônico (\textit{cone-beam}) \cite{fessler2014fundamentals}. Este algoritmo foi implementado em tomossíntese por \citeonline[]{wu2004comparison}. Basicamente, a implementação da técnica para tomossíntese ocorre em 5 passos \cite[p. 16]{xu2014tomographic}, como descrito a seguir:   

\begin{enumerate}
	\item Transformada de Fourier da projeção em cada linha paralela a trajetória do tubo;
	\item Aplicar o filtro de rampa no domínio da frequência;
	\item Aplicar o filtro de ``janelamento'' para redução de ruído e artefatos;
	\item Transformada inversa de Fourier das projeções filtradas;
	\item Retroprojetar os dados no domínio espacial.   
\end{enumerate} 

%%%%%%%%%%%%%%%%%%%%%%%%%%%%%%%%%%%%%%%%%%%%%%%%%%%%%%%%%%%%%%%%%%%%%%%%%%%%%%%%%%%%%%%%%%%%%%%%%%%%%%%%%%%%%%%%%%%%%%%%%%%%%%											Reconstrução Iterativa    														%
%%%%%%%%%%%%%%%%%%%%%%%%%%%%%%%%%%%%%%%%%%%%%%%%%%%%%%%%%%%%%%%%%%%%%%%%%%%%%%%%%%%%%%%%%%%%%%%%%%%%%%%%%%%%%%%%%%%%%%%%%%%%%%

\subsection{Reconstrução Iterativa}

Como mencionado, os algoritmos de \acs{FBP} têm sido os mais utilizados atualmente em tomossíntese da mama \cite{michell2018role}, pelo fato de serem rápidos e obterem resultados aproximados que são considerados bons \cite{das2011penalized}. Entretanto devido a baixa amostragem no domínio da frequência, esses algoritmos introduzem erros na reconstrução e sofrem dificuldades de confecção dos filtros no domínio de Fourier \cite{xu2015statistical}.

Os métodos iterativos demandam um alto custo computacional para a resolução dos problemas de reconstrução. Devido a isto, os mesmos não eram utilizados por equipamentos comerciais no passado, sendo somente alvos de pesquisas científicas. Com o avanço do poder computacional essas técnicas têm chamado a atenção para aplicações em reconstrução de imagens médicas e demonstram ser promissoras para \acs{DBT} \cite{zeng2010medical,zheng2018detector}.

Para solucionar o problema de maneira iterativa é necessário primeiramente a discretização do modelo físico do sistema, como foi dito na seção \ref{DiscretizaçãodaImagem}. Após essa etapa, faz-se o uso de equações lineares para o modelamento e resolução do problema \cite[p. 125]{zeng2010medical}. Cada \textit{voxel} é denotado por $f_{j} \mid (j=1,2,...,N)$ e cada raio da projeção por $p_{i} \mid (i=1,2,...,M)$. A Figura \ref{fig:imgCap3SistemaMatriz} ilustra um exemplo da retirada de um plano vertical da Figura \ref{fig:imgCap33DGeometry}, onde cada elemento do detector recebe somente um feixe de raio X  e cada \textit{voxel} possui tamanho unitário.  

\begin{figure}[H]
	\caption{Ilustração do princípio de reconstrução iterativa através de equações lineares simples.}
	\begin{center}
		\includegraphics[scale=0.55, clip, trim=10cm 3cm 10cm 2.3cm]{imgs/cap3/SistemaMatriz.pdf}
	\end{center}
	\legend{Fonte: Adaptado de \citeonline[p. 202]{buzug2008computed}.}
	\label{fig:imgCap3SistemaMatriz}
\end{figure} 

Pode-se então relacionar o valor de cada projeção com os valores dos \textit{voxels}, nos quais cada feixe de raio interage, através das seguintes equações lineares:

\begin{equation}
\begin{cases}
f_{1} \, + f_{3} \, = p_{1}, \\ 
f_{2} \, + f_{4} \, = p_{2}, \\ 
\sqrt{2} \, f_{1} \, + \sqrt{2} \, f_{4} \, = p_{3}, \\ 
f_{1} \, + f_{2} \, = p_{4}, \\ 
\end{cases}
\label{eq:eqCap3EquacoesLineares}
\end{equation} 

\noindent sendo também possível reescrever o sistema acima na sua forma matricial, representado pela equação abaixo:

\begin{equation}
P \, = A \, f,
\label{eq:eqCap3MatrizEquacoesLineares1}
\end{equation}

\noindent na qual $P = [p_{1},p_{2},p_{3},...,p_{M}]^{T}$ é um vetor coluna que representa os valores do sinograma no espaço de Radon, ou seja as projeções, {$f = [f_{1},f_{2},f_{3},...,f_{N}]^{T}$ é um vetor coluna que representa os coeficientes de atenuação dos \textit{voxels} no espaço \acs{3D} e $A$ é uma matriz $MxN$, onde cada elemento $a_{ij}$ representa a contribuição que o ``j-ésimo'' \textit{voxel} $f_{j}$ apresenta para a atenuação do feixe de raio X que forma a ``i-ésima'' projeção $p_{i}$, por exemplo, o valor $\sqrt{2}$ em $f_{1}$ e $f_{4}$ para $p_{3}$. Tratando-se da matriz $A$, sua dimensão $M$ significa o número de elementos detectores vezes o número de projeções e $N$ representa o número total de \textit{voxels} do objeto \cite{zeng2010medical,levakhina2014three}.

Como o objetivo da reconstrução é encontrar os coeficientes de atenuação do objeto, deve-se então solucionar $f$. Para isso, se a matriz $A$ é inversível, a imagem reconstruída pode ser facilmente encontrada algebricamente através da equação abaixo:

 \begin{equation}
 f \, = A^{-1} \, P,
 \label{eq:eqCap3MatrizEquacoesLineares2}
 \end{equation}      

\noindent porém na prática esse processo é inviável devido a reconstrução tomográfica ser um problema mal condicionado e ao grande número de equações no sistema, em outras palavras, a matriz $A$ é muito grande e esparsa \cite{levakhina2014three}. Considerando o que foi dito, a melhor alternativa é encontrar uma \textbf{solução aproximada} através de métodos matemáticos de otimização de problemas de uma forma iterativa \cite{buzug2008computed}.  

Para isso, deve-se otimizar uma função $E(x)$, normalmente chamada de função de custo, objetivo ou de energia, com o intuito de encontrar a melhor solução $\hat{x}$, ou a mais aproximada. Assim, é necessário minimizar ou maximizar a função $E(x)$, de acordo com as equações a seguir:
 
\begin{equation}
\hat{x} = \underset{}{\arg\min} \; E(x) \;\;\;\; \text{ou} \;\;\;\; \hat{x} = \underset{}{\arg\max} \; E(x).
\label{eq:eqCap3Minimizacao}
\end{equation} 

Segundo \citeonline[p. 206]{buzug2008computed}, uma solução generalizada para a Equação \ref{eq:eqCap3MatrizEquacoesLineares1} é dada pela minimização de \eqref{eq:eqCap3MinimosQuadrados} dita como norma mínima dos mínimos quadrados, caracterizando-se por um problema de otimização:

\begin{equation}
\chi^{2} \, = \left\| A \, f \, - \,P \right\|^{2},
\label{eq:eqCap3MinimosQuadrados}
\end{equation} 

\noindent onde $E(x) = \chi^{2}$, ou seja, \eqref{eq:eqCap3MinimosQuadrados} é a função de energia. Outras técnicas são encontradas na literatura para a otimização das funções de custo, tais como métodos de gradientes descendentes, iterativos algébricos e estatísticos, os quais são abordados nos próximos itens.

%%%%%%%%%%%%%%%%%%%%%%%%%%%%%%%%%%%%%%%%%%%%%%%%%%%%%%%%%%%%%%%%%%%%%%%%%%%%%%%%%%%%%%%%%%%%%%%%%%%%%%%%%%%%%%%%%%%%%%%%%%%%%%						               Reconstrução Iterativa (Método Algébrico)    										%
%%%%%%%%%%%%%%%%%%%%%%%%%%%%%%%%%%%%%%%%%%%%%%%%%%%%%%%%%%%%%%%%%%%%%%%%%%%%%%%%%%%%%%%%%%%%%%%%%%%%%%%%%%%%%%%%%%%%%%%%%%%%%%

\subsubsection{Método Algébrico}

Como já foi dito, para a resolver o problema de reconstrução é necessário encontrar a solução para as equações demonstradas em \eqref{eq:eqCap3EquacoesLineares}, porém as aplicações práticas tendem a ter uma enorme quantidade de equações, tornando inviável a resolução por procedimentos usuais \cite[p. 210]{buzug2008computed}. 

Baseado nisso, um método comumente encontrado na literatura é a técnica de reconstrução algébrica (\textit{Algebraic Reconstruction Technique} - \acs{ART}). Esse método visa resolver iterativamente um sistema enorme de equações lineares satisfazendo-as uma a uma, em outras palavras, estima-se uma imagem (coeficientes de atenuação) que solucione uma equação por vez \cite{rangayyan2004biomedical}. 

Segundo \citeonline[p. 211]{buzug2008computed}, esse método idealmente estima que a solução da imagem, i.e. o vetor {$f = [f_{1},f_{2},f_{3},...,f_{N}]^{T}$, seja um ponto no espaço $\mathcal{N}$ dimensional, onde as $M$ equações \eqref{eq:eqCap3EquacoesLineares} sejam hiperplanos que cruzam esse ponto, dado como solução.

Acompanhando a linha do autor, o exemplo abaixo ilustra como é dada a solução do problema através do método especificado. Para o exemplo, o mesmo é simplificado para $N = 2$, ou seja, são considerados somente dois \textit{pixels} e cada um é interceptado por somente um feixe de raio, tendo como resultado duas projeções $M=2$. Através disso, as equações podem ser modeladas seguindo o mesmo raciocínio de \eqref{eq:eqCap3EquacoesLineares}, demonstradas a seguir:

\begin{equation}
\begin{cases}
a_{11}f_{1} \, + a_{12}f_{2} \, = p_{1} \\ 
a_{21}f_{1} \, + a_{22}f_{2} \, = p_{2} \\  
\end{cases}.
\label{eq:eqCap3ARTEquacoesLineares}
\end{equation}

Cada equação, nesse caso, simboliza uma reta no espaço bidimensional, como demonstra a Figura \ref{fig:imgCap3ART}. Uma estimativa inicial $f^{(0)}$ deve ser feita e então esse ponto deve ser projetado perpendicularmente na reta, ou hiperplano, que representa $p_{1}$ para obter a nova imagem $f^{(1)}$. O processo volta a se repetir, projetando $f^{(1)}$ no hiperplano definido pela equação de $p_{2}$ e assim até convergir para a solução desejada $f = (f_{1},f_{2})^{T}$.


\begin{figure}[h]
	\caption{Ilustração geométrica do processo de reconstrução através do método \acs{ART}.}
	\begin{center}
		\includegraphics[scale=1]{imgs/cap3/ART1.pdf}
	\end{center}
	\legend{Fonte: \citeonline[p. 212]{buzug2008computed}.}
	\label{fig:imgCap3ART}
\end{figure} 

A equação abaixo demonstra a fórmula para o cálculo do método \acs{ART}, na qual os \textit{voxels} referentes ao raio $i$ são atualizados ao final do processamento desse mesmo raio, ou seja, a atualização de $\hat{f}$ é feita raio por raio \cite{buzug2008computed, levakhina2014three}:


\begin{equation}
f^{n} = f^{n-1} - \dfrac{(a_{i} \, f^{n-1}- p_{i})}{a_{i}\,(a_{i})^{T}}   (a_{i})^{T}   .  
\label{eq:eqCap3ARTEquacao}
\end{equation}

%\begin{equation}
%f^{Próximo} = f^{Atual} - Retroprojeção_{raio}\{\dfrac{Projeção_{raio}(f^{Atual})- Medição_{raio}}{Fator\,Normalização}\}.  
%\label{eq:eqCap3ARTEquacoesSimbolica}
%\end{equation}

Os métodos ditados acima assumem um caso ideal. As projeções dos feixes em sistemas reais sofrem de inconsistências de ruídos e artefatos, bem como na prática o problema é constituído de inúmeras equações. Dado isso, a solução para as equações lineares não é encontrada em somente um ponto, mas sim em uma região que possui múltiplas soluções. Nesse caso uma solução aproximada $\hat{f}$ deve ser estimada no sistema caracterizado como possível e indeterminado \cite[p. 218]{buzug2008computed}. A Figura \ref{fig:imgCap3ARTIndeterminado} exemplifica o caso acima. 

\begin{figure}[h]
	\caption{Ilustração geométrica de como ocorre a reconstrução de um sistema possível indeterminado por meio do método \acs{ART}.}
	\begin{center}
		\includegraphics[scale=1]{imgs/cap3/ART2.pdf}
	\end{center}
	\legend{Fonte: \citeonline[p. 218]{buzug2008computed}.}
	\label{fig:imgCap3ARTIndeterminado}
\end{figure}

Variações do método algébrico, dito acima, são encontradas na literatura. O algoritmo que utiliza a técnica de reconstrução iterativa simultânea (\textit{Simultaneous Iterative Reconstruction Technique} - \acs{SIRT}) é uma modificação do método convencional algébrico que atualiza a estimativa da imagem $f$ somente depois do processamento de todas as projeções, ao contrário do \acs{ART} que atualiza a cada raio \cite{zhang2006comparative,yang2012numerical,zeng2010medical}. Já o método que utiliza a técnica de reconstrução algébrica simultânea (\textit{Simultaneous Algebraic Reconstruction Technique} - \acs{SART}), atualiza $f$ ao final de cada projeção geométrica \cite{zhang2006comparative,levakhina2014three,yang2012numerical}.   

%%%%%%%%%%%%%%%%%%%%%%%%%%%%%%%%%%%%%%%%%%%%%%%%%%%%%%%%%%%%%%%%%%%%%%%%%%%%%%%%%%%%%%%%%%%%%%%%%%%%%%%%%%%%%%%%%%%%%%%%%%%%%%						               Reconstrução Iterativa (Método Estatístico)    										%
%%%%%%%%%%%%%%%%%%%%%%%%%%%%%%%%%%%%%%%%%%%%%%%%%%%%%%%%%%%%%%%%%%%%%%%%%%%%%%%%%%%%%%%%%%%%%%%%%%%%%%%%%%%%%%%%%%%%%%%%%%%%%%

\subsubsection{Método Estatístico}\label{MétodoEstatístico}

Os problemas de reconstrução também podem ser resolvidos por meio de métodos estatísticos que são equivalentes aos problemas de otimização \cite[p. 79]{levakhina2014three}. O algoritmo da máxima verossimilhança (\textit{Maximum Likelihood} - \acs{ML}) é um exemplo de modelamento estatístico. Nesse caso, a função de custo torna-se a função de verossimilhança \cite[p. 77]{levakhina2014three}.

Segundo \citeonline[p. 10]{Fessler2000handbook}, esse método busca estimar um parâmetro, por exemplo os coeficientes de atenuação $(ImgEstimada)$, que maximizam a verossimilhança dado um conjunto de observações $(Medidas)$. Essa estimativa pode ser modelada através da função de verossimilhança representada por meio da Equação \ref{eq:eqCap3ModeloVerossimilhança} e explicada com detalhes a seguir:

\begin{equation}
ImgReconstruida = \underset{}{\arg\max} \;\; l(ImgEstimada \mid Medidas),
\label{eq:eqCap3ModeloVerossimilhança}
\end{equation} 

\noindent onde $l(ImgEstimada)$ é a função de verossimilhança a ser otimizada tendo as medidas das projeções.


De acordo com \citeonline[p. 230]{buzug2008computed}, em um equipamento que utiliza raios X de transmissão, o número de \textit{quanta} emitido pelo tubo e recebido pelo detector segue a distribuição de Poisson, descrita matematicamente por:

\begin{equation}
P(Y_{i} = y_{i} \mid \hat{y_{i}}(f) ) = \dfrac{e^{-\hat{y_{i}}(f) } \, [\hat{y_{i}}(f)] ^{y_{i}}}  {y_{i}!},
\label{eq:eqCap3DistribuicaoPoisson}
\end{equation}

\noindent tal que $Y_{i}$ é uma variável aleatória contando o número de \textit{quanta} em cada observação da projeção $y_{i}$. O valor observado $y_{i}$ difere do valor esperado $\hat{y_{i}}$ pelo fato do detector ser um contador de fótons que obedece a distribuição descrita acima, por radiações espalhadas de Comptom, \textit{crosstalk}\footnote{Interferência entre \textit{pixels} vizinhos que ocorrem em detectores indiretos de \acs{DBT}, causados pela difusão da luz no fósforo ou através dos cintiladores \cite{zheng2018detector}.} entre os detectores e também pelo tamanho finito dos mesmos, contrariando as situações ideais da Equação \ref{eq:eqCap3BeerLambert1} \cite[p. 6]{Fessler2000handbook}. Esse valor esperado pode ser calculado segundo uma aproximação da lei de Beer-Lambert, demonstrada abaixo \cite[p. 9]{Fessler2000handbook}:

\begin{equation}
\hat{y_{i}}(f) = b_{i} \, e \, ^{-\hat{p_{i}}(f)} \, + r_{i},
\label{eq:eqCap3BeerLambertDiscreto2}
\end{equation}

\noindent em que $\hat{p_{i}}(f)$ é a função de atenuação que cada raio sofre, $b_{i}$ é o número médio de \textit{quanta} recebido pelo detector caso não exista nenhum objeto entre o tubo e o detector e $r_{i}$ modela os fatores externos mencionados acima. 

A função $\hat{p_{i}}(f)$ é calculada pelo somatório dos produtos da contribuição $a_{ij}$ para a atenuação $f_{j}$ de todos os \textit{voxels} $j$ no percurso do feixe de raio $i$, dado por: 


\begin{equation}
\hat{p_{i}}(f) = \sum_{j=1}^{N} a_{ij} \, f_{j}.
\label{eq:eqCap3BeerEsperancaAtenuacao}
\end{equation}

A probabilidade conjunta $P \mid (P = {y_{1},y_{2},y_{3},...,y_{M}})$ das projeções independentes $y_{i}$, dado o valor esperado $\hat{y_{i}}(f)$, é dada pela seguinte equação:   

\begin{equation}
P(P = p\mid f ) = \prod_{i=1}^{M}  \dfrac{e^{-\hat{y_{i}}(f)} \, [\hat{y_{i}}(f)]^{y_{i}}}  {y_{i}!},
\label{eq:eqCap3DistribuicaoPoissonConjunta}
\end{equation}

\noindent porém segundo \citeonline[p. 231]{buzug2008computed}, não faz sentido analisar a probabilidade de uma projeção $P$, ou seja uma observação, dado o valor esperado $\hat{y_{i}}(f)$ calculado através dos coeficientes de atenuação $f$, já que o intuito da resolução do problema é estimar os valores desses coeficientes $\hat{f}$. Nesse sentido, pode-se fazer o uso da \textbf{função de verossimilhança} para a resolução do problema.

A \textbf{verossimilhança} diz que uma observação qualquer $Y = y$ é uma evidência que favorece a hipótese A sobre a hipótese B, dada as duas hipóteses inicialmente, se a condição abaixo é satisfeita \cite{morettin2010}:

\begin{equation}
p_{A}(y) >p_{B}(y).
\label{eq:eqCap3LeiVerossimilhanca}
\end{equation}

Existe uma diferença entre probabilidade e verossimilhança. Quando em uma função de distribuição de probabilidade, e.g. a Equação \ref{eq:eqCap3DistribuicaoPoisson}, o parâmetro fixo é o valor esperado $\hat{y_{i}}(f)$ e o variável é a observação $y_{i}$, esta é denominada de função de probabilidade. Já se a observação é fixa e o parâmetro esperado é variável denomina-se \textbf{função de verossimilhança} \cite{morettin2010}.

Dito esses conceitos junto ao problema encontrado com a Equação \ref{eq:eqCap3DistribuicaoPoissonConjunta}, é plausível a utilização da função de verossimilhança para a resolução do mesmo, tendo como resultado a equação abaixo: 

\begin{equation}
l(f \mid P = p) = \prod_{i=1}^{M}  \dfrac{e^{-\hat{y_{i}}(f)} \, [\hat{y_{i}}(f)]^{y_{i}}}  {y_{i}!},
\label{eq:eqCap3DistribuicaoPoissonConjuntaVerossimilhanca1}
\end{equation} 
  
Sendo assim, a finalidade da função acima é de variar os valores de $f$ buscando o valor ótimo de verossimilhança $l$, dado um conjunto de observações $P$ \cite[p. 231]{buzug2008computed}. A busca pelo valor máximo da equação está intimamente ligada com o conceito de verossimilhança ditado acima, que busca a melhor hipótese $\hat{f}$ para os valores de $f$ dado o conjunto de observações $P$, como ilustra a figura abaixo. 


\begin{figure}[H]
	\caption{Ilustração de busca pela máxima verossimilhança.}
	\begin{center}
		\includegraphics[scale=0.4]{imgs/cap3/Verossimilhanca.pdf}
	\end{center}
	\legend{Fonte: do autor, 2019.}
	\label{fig:imgCap3Verossimilhanca}
\end{figure}   

Substituindo a Equação \ref{eq:eqCap3BeerLambertDiscreto2} na Equação \ref{eq:eqCap3DistribuicaoPoissonConjuntaVerossimilhanca1} e fazendo o uso da função logaritmo a fim de simplificar os cálculos e por conveniência, tem-se a Equação \ref{eq:eqCap3DistribuicaoPoissonConjuntaVerossimilhanca2}, desprezando constantes independentes de $f$ \cite[p. 10]{Fessler2000handbook}. 

\begin{equation}
L(f \mid P = p) = \log (l(f \mid P = p)) = \sum_{i=1}^{M} \, y_{i} \log (b_{i} \, e \, ^{-\hat{p_{i}}(f)} \, + r_{i}) \,-\,(b_{i} \, e \, ^{-\hat{p_{i}}(f)} \, + r_{i}),
\label{eq:eqCap3DistribuicaoPoissonConjuntaVerossimilhanca2}
\end{equation} 

\noindent onde a equação acima é a log-verossimilhança para \acs{CT} de transmissão e reescrevendo-a de maneira mais simples abaixo, podendo fazer uma analogia com a Equação \ref{eq:eqCap3ModeloVerossimilhança}.

\begin{equation}
\hat{f} = \underset{f\geq 0}{\arg\max} \; L(f \mid P),
\label{eq:eqCap3VerossimilhançaFinal}
\end{equation}

Ainda de acordo com \citeonline[p. 11]{Fessler2000handbook}, devido ao problema de reconstrução tomográfica ser mal condicionado, a maximização de $L(f)$ por si só leva a resultados de imagens ruidosas. Isso porque existe um grande conjunto de mapas de atenuação que se encaixam bem, como o resultado da equação. Então, segundo o autor, a função de verossimilhança por si só não consegue encontrar o melhor resultado. Uma das soluções encontradas para esse problema, que é vastamente conhecida na literatura, é a introdução de um termo de regularização na função objetivo, ou seja, incorporar informações já conhecidas para limitar o espaço de soluções de um vasto conjunto para um subconjunto menor. Então a maximização da sua nova forma ``penalizada'' é feita para encontrar a melhor solução contida nesse subconjunto, como é demonstrado abaixo:     

\begin{equation}
\hat{f} = \underset{f\geq 0}{\arg\max} \; \Phi(f), \;\; \text{onde} \;\; \Phi(f) = L(f) - \beta R(f),
\label{eq:eqCap3VerossimilhançaPenalizada}
\end{equation}

\noindent ou simbolicamente:

\begin{equation}
(\text{Função \textit{a posteriori}}) = (\text{Função de Verossimilhança}) - \beta (\text{Função \textit{a priori}}).
\label{eq:eqCap3VerossimilhançaPenalizadaSimbolica}
\end{equation}

Nessa notação, $R(f)$ modela a função de regularização do sistema, o fator $\beta$ controla a força de influência dessa função e $L(f)$ é a função objetivo formulada anteriormente. O algoritmos que introduz essa técnica é conhecido como \ac{PML}. 

Da mesma maneira, a formulação de \eqref{eq:eqCap3VerossimilhançaPenalizada} pode ser dada também a partir do modelo Bayesiano, ou teorema de Bayes, demonstrado abaixo, que utiliza de conhecimentos de probabilidade \textit{a priori} para a estimativa de parâmetros \textit{a posteriori}, dado as probabilidades condicionais \cite[p. 144]{zeng2010medical}. Esses algoritmos são conhecidos como métodos \textit{Bayesianos} ou \ac{MAP}. O cálculo da estimativa \acs{MAP} ocorre através do teorema de Bayes:

\begin{equation}
Prob(f \mid P) = \dfrac{Prob(P \mid f) Prob(f)}{Prob(P)},
\label{eq:eqCap3TeoremaBayes}
\end{equation}  

\noindent tal qual,  aplicando o logaritmo na equação acima, tem-se \eqref{eq:eqCap3TeoremaBayesLog}, onde o elemento que não depende do coeficiente de atenuação $f$ é desprezado, gerando \eqref{eq:eqCap3TeoremaBayesLogSimplificada}:

\begin{equation}
\log(Prob(f \mid P)) = \log(Prob(P \mid f)) + \log(Prob(f)) - \log(Prob(P)),
\label{eq:eqCap3TeoremaBayesLog}
\end{equation}

\begin{equation}
\log(Prob(f \mid P)) = \log(Prob(P \mid f)) + \log(Prob(f)),
\label{eq:eqCap3TeoremaBayesLogSimplificada}
\end{equation}

\noindent onde o primeiro termo é o modelo estatístico da aquisição física do sistema, i.e. função de máxima verossimilhança \eqref{eq:eqCap3VerossimilhançaFinal}, e o segundo termo é o modelo estatístico que faz a restrição do sistema para que este possa convergir para um solução apropriada. Simplificadamente tem-se a equação abaixo, sendo possível sua assimilação com a Equação \ref{eq:eqCap3VerossimilhançaPenalizadaSimbolica}.

\begin{equation}
\Phi(f) = l(f) - \beta R(f).
\label{eq:eqCap3TeoremaBayesLogSimplificadaFinal}
\end{equation}

É conhecido da literatura que os mapas de atenuações não variam bruscamente em uma pequena faixa de extensão, ou seja, os níveis de cinza entre os \textit{pixels} vizinhos não alteram significantemente em relação à média local \cite{Fessler2000handbook,buzug2008computed}. Essa informação pode ser considerada como um conhecimento \textit{a priori} sobre a imagem desejada e a partir desse fato, pode ser incorporado uma equação que desestimule a ``rispidez'' sobre a imagem processada, impondo restrições de suavidade. Sendo assim, essa equação pode ser modelada através de funções matemáticas e incorporada como uma restrição para o conjunto de soluções do modelo estatístico.

Um das maneiras mais simples de modelar uma função de restrição de suavidade na imagem é através da discrepância existente entre os valores dos \textit{pixels} vizinhos \cite[p. 12]{Fessler2000handbook}. Esse modelamento pode ser feito de acordo com \citeonline[p. 4]{das2011penalized}:

\begin{equation}
R(f) = \sum_{j=1}^{N}  \sum_{k=1}^{N_{i}} \omega_{jk} \, \psi(f_{j}-f_{k}),
\label{eq:eqCap3DiscrepanciaPixels}
\end{equation}  

\noindent no qual $f$ é o coeficiente de atenuação, $\psi$ mede a discrepância entre $f_{j}$ e $f_{k}$, $N_{i}$ é o número de \textit{voxels} vizinhos, $\omega_{jk} = \omega_{kj}$ define a influência do ``k-ésimo'' \textit{voxel} no ``j-ésimo'' \textit{voxel}. A influência de cada \textit{voxel} em seu vizinho é definida por: $\omega_{jk} = 1$ para os seis \textit{voxels} vizinhos, sendo dois em cada direção do eixo, $\omega_{jk} = 1/\sqrt{2}$ para os diagonais e $\omega_{jk} = 0$ para qualquer outro. A Figura \ref{fig:imgCap3PesosVizinhanca} exemplifica os pesos dados a cada vizinho, em um espaço \acs{2D}, sendo que esses pesos são normalizados.


\begin{figure}[H]
	\centering
	
	\caption{Pesos normalizados atribuídos para a vizinhança (a) quatro e (b) oito.}
	
	\subfloat[]{\includegraphics[scale=0.8]{imgs/cap3/PesoVizinhancaB.pdf}	\label{fig:imgCap3PesosVizinhancaA}}
	
	\subfloat[]{\includegraphics[scale=0.8]{imgs/cap3/PesoVizinhancaA.pdf} \label{fig:imgCap3PesosVizinhancaB}.}
	
	\legend{Fonte: \citeonline[p. 2]{chen2009bayesian}.}
	\label{fig:imgCap3PesosVizinhanca}
\end{figure}   

%\rv{Corta daqui pra baixo}
%
%Ainda no contexto da restrição de suavidade, outra técnica utilizada é o modelamento da função de restrição por meio do Campo Aleatório Markoviano (Markov Random Field - \acs{MRF}), através do uso de distribuição de probabilidades que levam em conta o contexto espacial dos \textit{pixels} \cite[p. 71]{salvadeo2013filtragem}.  
%
%Segundo a definição de \citeonline[p. 8]{li2009markov}, a teoria do Campo Aleatório Markoviano é um ramo da teoria de probabilidade que tem como função analisar as \textbf{dependências espaciais} ou contextuais de dado \textbf{fenômeno físico}. 
%
%Com a finalidade de explicar todo o conceito por trás dessa teoria, é necessário a utilização de notações formais que auxiliarão todo o processo de formulação. Todas essas notações estão contidas no Apêndice \ref{ApendiceA:CamposAleatóriosMarkovianos}, junto com conceitos teóricos importantes a serem definidos. Recomenda-se ao leitor investir um tempo analisando esse capítulo a fim de compreender melhor a teoria descrita com detalhes a seguir.    
%
%Dado todos os conceitos apresentados, um campo aleatório $F$ (\ref{ApendiceA:CampoAleatorio}) em $S$ (\ref{ApendiceA:SitesRotulos}) em relação a sua vizinhança $\nu$ (\ref{ApendiceA:sVizinhancaCliques}) é dito ser de Markov quando o campo atende as duas condições ditadas abaixo:  
%
%
%\begin{enumerate}
%	\item \textbf{Positividade}: $P(f) > 0, \;\; \forall\, f \,\in\, \varPsi$,
%	\item \textbf{Markovianidade}: $P(f_{i} \mid f_{j},\;\; \forall \; j \in S,\; j \neq i) = P(f_{i}\mid f_{\nu_{i}})$,
%\end{enumerate}
%
%\noindent onde $f_{\nu_{i}} = \{f_{i^{'}} \mid i^{'} \in \nu_{i} \} $ é o conjunto de rótulos admitidos para os \textit{sites} na vizinhança de $i$. Em outras palavras, a \textbf{Markovianidade} dita as características locais do campo, onde somente a vizinhança tem influência direta no valor do \textit{site}, ou \textit{pixel} nesse caso. Já a \textbf{positividade}, quando satisfeita, diz que a probabilidade conjunta do campo $P(f)$ é determinada pelas probabilidades condicionais locais do campo \cite{li2009markov}.  
%
%De acordo com \citeonline[p. 4]{won2013stochastic}, os \acs{MRF} em \acs{2D} são uma generalização não-causal das cadeias de Markov \acs{1D} proveniente das análises de sequências. Devido a representação de uma dependência de características locais do campo com a sua vizinhança de um modo não-causal, a probabilidade conjunta não pode ser fatorada\footnote{A fatoração é a conexão entre a probabilidade conjunta e a probabilidade condicional local, segundo \citeonline[p. 18]{won2013stochastic}} em probabilidades das características locais como é feito nas cadeias de Markov \acs{1D}. Tendo isso em mente é necessário a busca de um método que especifique a probabilidade conjunta de um \acs{MRF} por meio de suas características locais, ou seja, probabilidades condicionais locais. 
%
%Isso só foi possível devido a equivalência entre o \acs{MRF} e o Campo Aleatório de Gibbs (\textit{Gibbs Random Field} - \acs{GRF}) dada pelo teorema de \citeonline[]{hammersley1971markov}. Com isso foi possível estabelecer uma relação entre a \textbf{probabilidade conjunta global} (\acs{GRF}) e a \textbf{probabilidade condicional local} (\acs{MRF}) e vice-versa.  
%
%O teorema mencionado acima diz que: \textit{$F$ é um \acs{MRF} em $S$, dado uma vizinhança $\nu$, se e somente se $F$ é um \acs{GRF} em $S$, dado uma vizinhança $\nu$}. Em outras palavras, $F$ é dito ser de Markov somente se a sua probabilidade conjunta $P(f)$ seguir a distribuição de Gibbs (Apêndice \ref{ApendiceB:CamposAleatóriosdeGibbs}).
%
%Diversos modelos distintos de \acs{MRF} podem ser encontrados na literatura \cite{li2009markov,won2013stochastic}, sendo que cada um é utilizado para uma determinada aplicação de acordo com suas funções potenciais. O modelo Gaussiano, denominado \ac{GMRF}, é bastante utilizado em problemas de restauração e reconstrução de imagens \cite{salvadeo2016nonlocal,xu2015statistical,jeng1991compound}. Sua equação de probabilidade condicional local é definida segundo \citeonline[p. 18]{li2009markov}:
%
%\begin{equation}
%P(f_{i} \mid f_{\nu_{i}}) = \dfrac{1}{\sqrt{2 \pi \sigma^{2}_{i}}}  e^{-\frac{1}{2\sigma^{2}_{i}} \, [f_{i}-\mu_{i}-\sum_{i^{'}\in \,\nu_{i}}^{} \beta_{i,i^{'}}(f_{i^{'}} - \mu_{i^{'}}) ]^{2} },
%\label{eq:eqCap3GMRF1}
%\end{equation} 
%
%\noindent onde $\sigma^{2}_{i}$ e $\mu_{i}$ correspondem a variância e a média local respectivamente e $\beta_{i,i^{'}}$ corresponde aos coeficientes de interação dos cliques. 
%
%A probabilidade conjunta é dada de maneira generalizada segundo \citeonline[p. 4]{xu2015statistical}:
%
%\begin{equation}
%P(f) \sim \prod_{i}^{N} \prod_{i^{'}}^{\nu_{i}} e^{\rho(\varDelta_{ii^{'}})},
%\label{eq:eqCap3GMRF2}
%\end{equation} 
%
%\noindent onde: 
%
%\begin{equation}
%\rho(\varDelta_{ii^{'}}) = - \omega_{ii^{'}} \dfrac{\varDelta_{ii^{'}}^{2}}{2\sigma^{2}},
%\label{eq:eqCap3GMRF3}
%\end{equation}
%
%\noindent $\varDelta_{ii^{'}}^{2} = (f_{i}-f_{i^{'}})^{2}$ é a função potencial quadrática e $\omega_{ii^{'}}$ são os pesos da vizinhança, calculados pelo inverso da distância euclidiana entre o \textit{pixel} central e seus vizinhos, dado por:
%
%\begin{equation}
%\omega_{ii^{'}} = \dfrac{1}{((x_{i}-x_{i^{'}})^{2}+((y_{i}-y_{i^{'}}))^{2})^{1/2}}.
%\label{eq:eqCap3GMRF4}
%\end{equation}
%
%No ano de 2005, o trabalho de \citeonline[]{buades2005non} propôs a filtragem de imagens corrompidas por ruído Gaussiano por meio de um filtro da média não-local. Esse filtro explora o fato de que imagens em sua grande maioria possuem um alto grau de redundância. Basicamente, a filtragem é feita através da similaridade entre conjuntos de \textit{pixels}, denominados \textit{patches}, ao invés de ser feita somente com a vizinhança local como era anteriormente, em outras palavras, a remoção do ruído da imagem $I$ se dá através de uma média ponderada $w_{st}$ entre os \textit{patches} similares $P_{s}\, \text{e}\, P_{t} \, \forall \, t \in SW_{s}$. O \textit{patch} $P_{s}$ de um determinado \textit{pixel} $i_{s}$ pode ser caracterizado por uma região quadrada ao entorno daquele \textit{pixel}, e.g. $7x7$. Buscando a redução do custo computacional do método, uma janela de busca $SW$, e.g. $17x17$, é estipulada para a comparação dos \textit{patches}. O valor estimado $\hat{i}_{s}$ do \textit{pixel} livre de ruído é dado por meio de \eqref{eq:eqCap3NLM}, para todo $s$ pertencente a imagem $I$. A Figura \ref{fig:imgCap3Patches} ilustra como é feito o procedimento não-local. 
%
%\begin{equation}
%NLM(\hat{i}_{s}) = \sum_{t \,\in\, SW_{s}}^{} w_{st} \cdot i_{t},
%\label{eq:eqCap3NLM}
%\end{equation}
%
%
%%\begin{figure}[H]
%%	\caption{Ilustração do procedimento não-local baseado na similaridade entre os \textit{patches}.}
%%	\begin{center}
%%		\includegraphics[scale=1.3]{imgs/cap3/Patches.pdf}
%%	\end{center}
%%	\legend{Fonte: \citeonline[p. 2]{salvadeo2016nonlocal}}
%%	\label{fig:imgCap3Patches}
%%\end{figure}
%
%\noindent onde o peso da similaridade $w_{st}$ entre os \textit{patches} é obtido por meio de \eqref{eq:eqCap3NLMw}, calculado pela distância Euclidiana entre os vetores $P_{s}\, \text{e}\, P_{t}$ de forma normalizada. É importante ressaltar que a soma dos pesos dentro da janela deve somar 1, em outras palavras, $\sum_{t\,\in \,SW_{s}}^{}w_{st} = 1$, o que leva a normalização dada pelo denominador da equação abaixo.
%
%\begin{equation}
%w(s,t) = \dfrac{e^{-\|P_{s} - P_{t}\|^{2}/h^{2}}}{\sum_{t\,\in \,SW_{s}}^{}e^{-\|P_{s} - P_{t}\|^{2}/h^{2}}},
%\label{eq:eqCap3NLMw}
%\end{equation}
%
%\noindent onde $h$ é um parâmetro que controla o nível de borramento do filtro. 
%
%Baseado no conceito não-local de similaridade, o trabalho de \citeonline[]{zhang2017applications} propõe a expansão dessa técnica para problemas inversos tais como: remoção de ruído, redução de borramento, reconstrução e reconstrução com alta resolução. O trabalho de \citeonline[]{salvadeo2016nonlocal} explorou os conceitos de \acs{MRF} não-local para remover ruídos de imagens.
%
%Adicionando o conceito não-local no cálculo dos coeficientes de interação dos cliques da equação de \acs{GMRF}, temos a equação \eqref{eq:eqCap3GMRF1} modificada para:
%
%\begin{equation}
%P(f_{i} \mid f_{\nu_{i}}) = \dfrac{1}{\sqrt{2 \pi \sigma^{2}_{i}}}  e^{-\frac{1}{2\sigma^{2}_{i}} \, [f_{i}-\mu_{i}-\sum_{i^{'}\in \,\nu_{i}}^{} \omega(i,i^{'})\beta_{i,i^{'}}(f_{i^{'}} - \mu_{i^{'}}) ]^{2} },
%\label{eq:eqCap3GMRFNL}
%\end{equation}     
%
%
%\rv{Voltar a partir daqui}


Após a formulação da função objetivo e da restrição do sistema é necessário resolver essa equação para a obtenção da imagem reconstruída. Segundo \citeonline[p. 12]{Fessler2000handbook} não existe uma fórmula fechada para a resolução da Equação \ref{eq:eqCap3TeoremaBayesLogSimplificadaFinal} de forma analítica. Desse modo, o uso de algoritmos iterativos é uma alternativa para a solução do problema. Ainda segundo o autor, um algoritmo iterativo primeiro estima uma imagem inicial $f^{(0)}$ como hipótese e então recursivamente busca uma sequência $f^{(1)}, f^{(2)}, \dots , f^{(n)}$ até convergir para o máximo ou mínimo da função. A Figura \ref{fig:imgCap3ProcedimentoIterativo} ilustra o esquemático de um algoritmo iterativo.


\begin{figure}[H]
	\caption{Esquemático do procedimento geral realizado pelos algoritmos de reconstrução iterativa.}
	\begin{center}
		\includegraphics[scale=0.5, clip, trim= 7.5cm 2.8cm 5.3cm 3.5cm]{imgs/cap3/ProcedimentoIterativo.pdf}
	\end{center}
	\legend{Fonte: Adaptado de \citeonline[p. 129]{zeng2010medical}.}
	\label{fig:imgCap3ProcedimentoIterativo}
\end{figure}

Diversos algoritmos para a otimização das funções objetivo são encontrados na literatura \cite{Fessler2000handbook,das2011penalized,zeng2010medical,sidky2014iterative,xu2015statistical,zheng2018detector}. No entanto nesse trabalho somente o algoritmo de Máxima Expectativa (\textit{Expectativa Maximization} - \acs{EM}) é discutido.

Considerando a constante $\beta = 0$ da Equação \ref{eq:eqCap3TeoremaBayesLogSimplificadaFinal} elimina-se a informação \textit{a priori} do sistema e então o problema volta a lidar somente com a equação de máxima verossimilhança \eqref{eq:eqCap3DistribuicaoPoissonConjuntaVerossimilhanca2}, ou do inglês \acs{ML}.  

Basicamente, seguindo o trabalho de \citeonline[p. 111]{sidky2014iterative}, a equação do método de \acs{MLEM} é dada por:

\begin{equation}
f^{n} = f^{n-1}  \left(A^{T} \, \dfrac{P}{A \, f^{n-1}}\right)   /    (A^{T}\,1),
\label{eq:eqCap3MLEM}
\end{equation}

\noindent na qual $A$ é a matriz da projeção, a sua transposta $A^{T}$ representa a retroprojeção e o fator de normalização no denominador é calculado fazendo a retroprojeção com valores de P igual a 1.

%Basicamente, o algoritmo de \acs{EM} é constituído por duas etapas \cite[p. 149]{zeng2010medical}. A primeira é a substituição da variável aleatória $y_{i}$ de \eqref{eq:eqCap3DistribuicaoPoissonConjuntaVerossimilhanca2} por sua expectativa, caracterizando o passo ``E''. O próximo passo é maximizar a nova equação, igualando a derivada de $f_{j}$ a zero, sendo essa etapa caracterizada por ``M''. Daí então que é derivado o algoritmo de \acs{MLEM}. A derivação matemática dos passos apresentados pode ser encontrada em \citeonline[p. 148-150]{zeng2010medical}. A equação matemática do algoritmo de \acs{MLEM} é dada por:  \rv{Elias: EM para emissão?}
%
%\begin{equation}
%f_{j}^{Próximo} = \dfrac{f_{j}^{Atual}}{\sum_{i}^{} a_{ij}} \sum_{i}^{} a_{ij} \, \dfrac{y_{i}}{\sum_{j}^{} a_{ij}f_{j}^{Atual}},
%\label{eq:eqCap3MLEM}
%\end{equation}    
%
%\noindent ou de forma simbólicas:  
%
%\begin{equation}
%f^{Próximo} = f^{Atual} \, \dfrac{Retroprojeção\left\{\dfrac{Medição}{Projeção(f^{Atual})}\right\}}{Retroprojeção\{1\}},
%\label{eq:eqCap3MLEMSimbolico}
%\end{equation} 
%
%\noindent onde a retroprojeção do vetor 1 no denominador significa uma constante de normalização do volume.

%Para incorporar a restrição \acs{MAP} no sistema, o algoritmo exposto acima pode ser modificado para sua forma denominada \textit{Green’s one-step late} \cite[p. 151]{zeng2010medical}, adicionando o termo de penalidade no denominador, demonstrado por: 
%
% \begin{equation}
% f_{j}^{Próximo} = \dfrac{f_{j}^{Atual}}{\sum_{i}^{} a_{ij} \, + \beta \dfrac{\partial U(F^{Atual})}{ \partial f_{j}^{Atual}}} \sum_{i}^{} a_{ij} \, \dfrac{y_{i}}{\sum_{j}^{} a_{ij}f_{j}^{Atual}},
% \label{eq:eqCap3MLEMMAP}
% \end{equation}
% 
% \noindent onde $U(F)$ é a função de energia do modelo \acs{MAP}, dado nesse trabalho por \eqref{eq:eqApendiceBDistribuicaoGibbsU1} ou mais precisamente em \eqref{eq:eqCap3GMRFNL} no modelo não-local \acs{GMRF}.


%%%%%%%%%%%%%%%%%%%%%%%%%%%%%%%%%%%%%%%%%%%%%%%%%%%%%%%%%%%%%%%%%%%%%%%%%%%%%%%%%%%%%%%%%%%%%%%%%%%%%%%%%%%%%%%%%%%%%%%%%%%%%%											Redução de Artefatos Metálicos	  												 %
%%%%%%%%%%%%%%%%%%%%%%%%%%%%%%%%%%%%%%%%%%%%%%%%%%%%%%%%%%%%%%%%%%%%%%%%%%%%%%%%%%%%%%%%%%%%%%%%%%%%%%%%%%%%%%%%%%%%%%%%%%%%%%
%\section{Redução de Artefatos de Alta Atenuação} \label{ReduçãodeArtefatosdeAltaAtenuação}
%
%Artefatos inter-planos de objetos com alta atenuação são motivos de pesquisa em \acs{DBT} \cite{hu2008image,levakhina2013weighted,borges2017metal}. Como foi abordado na seção \ref{ProjeçãoeRetroprojeção3D}, a simples retroprojeção calcula as coordenadas de $(X,Y)$ para cada nível das fatias em $Z$ para todas as projeções (\ref{fig:imgCap3wBP2}). 	
%
%\begin{figure}[H]
%	\centering
%	
%	\caption{Ilustração da (a) projeção de objetos em diferentes níveis, seguido da (b) retroprojeção dos mesmos, (c) mostrando como os objetos se dispõem nos planos retroprojetados em um mesmo nível.}
%	
%	\subfloat[Projeção]{\includegraphics[scale=0.6]{imgs/cap3/BpPonderado/wBP1.png}\label{fig:imgCap3wBP1}}
%	\hfil
%	\subfloat[Retroprojeção]{\includegraphics[scale=0.6]{imgs/cap3/BpPonderado/wBP2.png}\label{fig:imgCap3wBP2}}
%	\hfil
%	\subfloat[Foco no processo de retroprojeção]{\includegraphics[scale=0.6]{imgs/cap3/BpPonderado/wBP3.png}\label{fig:imgCap3wBP3}}
%		
%	\legend{Fonte: \citeonline[p. 3]{levakhina2013weighted}.}
%	\label{fig:imgCap3wBPIlustracao}
%\end{figure}
%
%A ocorrência desse tipo de artefato é inevitável devido ao fato da técnica de tomossíntese abranger uma estreita faixa de ângulo junto a um número limitado de projeções \cite{hu2008image}. 
%
%O entendimento desse acontecimento fica mais claro a partir da Figura \ref{fig:imgCap3wBP3}, onde no plano em evidência o círculo vermelho aparece três vezes de forma borrada devido a esse objeto não estar localizado naquele plano. Já o triângulo fica em evidência pois o mesmo está localizado no plano de interesse. É interessante notar que a interferência do círculo se dá pelo aparecimento de cópias de acordo com o número de projeções realizadas e essas cópias se localizam na direção em que cada projeção foi adquirida.
%
%De acordo com \citeonline[p. 4]{levakhina2013weighted}, o princípio do borramento se dá pelo fato de que na retroprojeção dos dados, as estruturas que estão contidas no plano de interesse se coincidem e mutuamente contribuem para aparecerem em foco, e.g. voxel $x_{1}$ na Figura \ref{fig:imgCap3wBP3}, já as estruturas que não pertencem àquele plano não se coincidem e então geram cópias borradas de suas estruturas, e.g. voxel $x_{2}$ na Figura \ref{fig:imgCap3wBP3}.
%
%Como mencionado anteriormente, normalmente a combinação matemática (Figura \ref{fig:imgCap3FluxogramaBP}) utilizada para os planos retroprojetados é uma simples soma ou média. Segundo \citeonline[]{borges2017metal}, uma solução para a diminuição dos artefatos de alta atenuação é a realização de uma média ponderada entre os planos retroprojetados.   
%
%No trabalho de \citeonline[]{borges2017metal} foi encontrado um \textit{patch} como sendo o referência entre os planos e a distância euclidiana entre o escolhido e os outros foi então calculada. Baseado nessa distância, foi estipulado um peso para os \textit{patches} de acordo com a função logística \eqref{eq:eqCap3wBPLogistic}. Após a estipulação dos pesos, foi realizado a média ponderada entre os \textit{pixels} centrais daqueles \textit{patches}. A Figura \ref{fig:imgCap3wBPPesos} ilustra o gráfico resultado da aplicação da função logística para o cálculo dos pesos em função da distância, variando o valor da constante M de inclinação da curva.
%
%\begin{equation}
%w_{x}(d_{x,y}) = 1 - \dfrac{1}{1 + e^{\frac{4,6}{M}(M-2d_{x,y})}} 
%\label{eq:eqCap3wBPLogistic}
%\end{equation}
%
%\begin{figure}[H]
%	\caption{Gráfico da atribuição dos pesos em relação a distância.}
%	\begin{center}		
%	\includegraphics[scale=0.6]{imgs/cap3/wBPPesos.pdf}
%	\end{center}
%	\legend{Fonte: do autor, 2019.}
%	\label{fig:imgCap3wBPPesos}
%\end{figure} 
%



	

\chapter[Ruído em Imagens]{Ruído em Imagens}\label{Capitulo4}

%%%%%%%%%%%%%%%%%%%%%%%%%%%%%%%%%%%%%%%%%%%%%%%%%%%%%%%%%%%%%%%%%%%%%%%%%%%%%%%%%%%%%%%%%%%%%%%%%%%%%%%%%%%%%%%%%%%%%%%%%%%%%%												  Introdução    															%
%%%%%%%%%%%%%%%%%%%%%%%%%%%%%%%%%%%%%%%%%%%%%%%%%%%%%%%%%%%%%%%%%%%%%%%%%%%%%%%%%%%%%%%%%%%%%%%%%%%%%%%%%%%%%%%%%%%%%%%%%%%%%%

Uma imagem é a representação da captura das intensidades da luz advindas de uma cena natural. Na forma digital, a conversão dessas intensidades é feita e as mesmas são representadas na forma numérica através de elementos de imagem, denominados \textit{pixels}. A captura das ondas eletromagnéticas normalmente é feita por sensores que idealmente convertem de maneira proporcional a quantidade de luz recebida em valores de intensidade de \textit{pixel}  \cite{bertalmiodenoising2018}. Segundo \citeonline{prince2006medical}, ruído em uma imagem se refere a um termo genérico utilizado para qualquer variação aleatória em um sinal que pode degradar drasticamente a qualidade do mesmo. As fontes dessas variações e as suas respectivas intensidades variam de acordo com cada sistema de aquisição de imagem. O conhecimento dessas fontes de interferências e seu modelamento é importante na área médica para aplicações na remoção de ruído em imagens, redução da dose de radiação em mamografias e como função de restrição para algoritmos de reconstrução iterativa de tomossíntese \cite{wu2012dose, romualdo2013mammographic,borges2016method,borges2017pipeline,borges2017method,mackenzie2017characterisation,zheng2018detector}. De fato o impacto do ruído no diagnóstico do câncer de mama tem sido reportado em trabalhos anteriores \cite{haus2000screen,huda2003experimental,ruschin2007dose,saunders2007does, samei2007digital, mackenzie2016relationship}.


%%%%%%%%%%%%%%%%%%%%%%%%%%%%%%%%%%%%%%%%%%%%%%%%%%%%%%%%%%%%%%%%%%%%%%%%%%%%%%%%%%%%%%%%%%%%%%%%%%%%%%%%%%%%%%%%%%%%%%%%%%%%%%											Ruído em Imagens	    														%
%%%%%%%%%%%%%%%%%%%%%%%%%%%%%%%%%%%%%%%%%%%%%%%%%%%%%%%%%%%%%%%%%%%%%%%%%%%%%%%%%%%%%%%%%%%%%%%%%%%%%%%%%%%%%%%%%%%%%%%%%%%%%%
\section{Fontes de Ruído em Imagens}

De acordo com \citeonline{bertalmiodenoising2018}, em uma imagem de longa exposição de uma cena qualquer, por exemplo, onde aparentemente não existem variações na intensidade da luz, a quantidade de fótons incidentes no detector varia no decorrer do tempo e a conversão dos fótons em cargas elétricas não se dá de forma ideal. Essas variações são uma das formas mais comuns de ruído em imagens e é conhecida como ruído de aquisição ou ruído quântico. O termo quântico, do latim \textit{quantum} (plural: \textit{quanta}), significa quantidade, e na física moderna, a menor quantidade discreta de uma propriedade física, e.g., um fóton é um \textit{quantum} da luz \cite{bushberg2011essential}. 

Outra fonte de flutuação é causada por variações na temperatura do sensor que causam agitações térmicas nas cargas elétricas dos condutores. Tal ruído é denominado térmico ou eletrônico, devido a este ser proporcional a temperatura em que o equipamento opera, estando presente nos condutores mesmo na ausência de um potencial elétrico. A movimentação dessas cargas gera uma certa corrente elétrica, denominada corrente de fundo. Essa fonte de variação é importante, especialmente em imagens médicas geradas por raios X com baixa dose de radiação \cite{bertalmiodenoising2018}.

Ruído estrutural é também uma causa de perturbações nos sinais adquiridos. Ele está relacionado com características eletrônicas na leitura dos sensores como por exemplo, a variação no ganho dos amplificadores ao longo das regiões do detector. Tais variações normalmente são fixas espacialmente, ou seja, dependem de cada região do detector \cite{bushberg2011essential,marshall2017handbook}.

Especificamente em imagens médicas, estruturas anatômicas que não são importantes para o exame, mas estão presentes assim mesmo, causam ruídos denominados anatômicos. Segundo \citeonline{bushberg2011essential}, em radiografias torácicas, quando se deseja avaliar a estrutura anatômica do pulmão, as costelas são causadoras do ruído anatômico em potencial.

 
\section{Representação do Ruído}

As diversas formas de flutuações referentes aos eventos aleatórios do ruído podem ser representadas por variáveis aleatórias que são matematicamente descritas através de funções de distribuição de probabilidade \cite{prince2006medical}. Ou seja, essas funções descrevem a probabilidade da ocorrência de um certo valor, ou de um dado intervalo de valores. 

Dado que $P_{N}(\eta)$ é a \ac{FDP} da variável aleatória $N$ e $\eta$ são valores contínuos, temos que seu valor esperado ou também chamado de valor médio é calculado por:


\begin{equation}
\mu_{N} = E\{N\}= {\int_{-\infty}^{\infty} \,\eta \, P_{N}(\eta) \, d\eta}.
\label{eq:eqCapRuidoValorEsperado}
\end{equation}

De fato, quando analisamos a Figura \ref{fig:imgCapRuidoPhantom}, a qual ilustra imagens que foram corrompidas por ruído com diferentes características e os seus respectivos perfis radiográficos, a média do sinal se mantém a mesma para todos os casos. Isso torna difícil a especificação do sinal somente através de sua média. Sendo assim, é necessário a caracterização dessas interferências de outra maneira \cite{dougherty2009digital}.

\begin{figure}[H]
	\centering
	
	\caption{Figura ilustrando (a) o \textit{phantom} de \citeonline{shepp1974fourier} e seu respectivo perfil radiométrico, junto as imagens (b) e (c) corrompidas por um ruído Gaussiano independente do sinal com diferentes variâncias.}
	
	\subfloat[]{\includegraphics[scale=0.41, clip, trim=4.3cm 8.5cm 5cm 9cm]{imgs/capRuido/PhantomProf.pdf}	
	\label{fig:imgCapRuidoPhantomOriProf}}
	\hfil
	\subfloat[]{\includegraphics[scale=0.41, clip, trim=4.3cm 8.5cm 5cm 9cm]{imgs/capRuido/PhantomStd1Prof.pdf} \label{fig:imgCapRuidoPhantomStd1Prof}}
	\hfil
	\subfloat[]{\includegraphics[scale=0.41, clip, trim=4.3cm 8.5cm 5cm 9cm]{imgs/capRuido/PhantomStd2Prof.pdf} \label{fig:imgCapRuidoPhantomStd2Prof}}
	\hfil
	
	\subfloat{\includegraphics[scale=0.4, clip, trim=5cm 8cm 5cm 7cm]{imgs/capRuido/Phantom_RedLine.pdf}	\label{fig:imgCapRuidoPhantomOri}}
	\hfil
	\subfloat{\includegraphics[scale=0.4, clip, trim=5cm 8cm 5cm 7cm]{imgs/capRuido/PhantomStd1.pdf} \label{fig:imgCapRuidoPhantomStd1}}
	\hfil
	\subfloat{\includegraphics[scale=0.4, clip, trim=5cm 8cm 5cm 7cm]{imgs/capRuido/PhantomStd2.pdf} \label{fig:imgCapRuidoPhantomStd2}}
	
	\legend{Fonte: do autor, 2019.}
	\label{fig:imgCapRuidoPhantom}
\end{figure}

Uma forma diferente é a caracterização através da média quadrática, segundo:
   
\begin{equation}
\mu_{N^{2}} = E\{N^{2}\}= {\int_{-\infty}^{\infty} \,\eta^{2} \, P_{N}(\eta) \, d\eta},
\label{eq:eqCapRuidoEsperacaQuadratica}
\end{equation}

\noindent tal que, a variância do sinal é dada pela relação entre a média e a média quadrática:

\begin{equation}
\sigma^{2}_{N} = Var\{N\}= {\int_{-\infty}^{\infty} \,(\eta -\mu_{N} )^{2} \, P_{N}(\eta) \, d\eta}, 
\label{eq:eqCapRuidoVariancia1}
\end{equation}

\begin{equation}
\sigma^{2}_{N}  = \mu_{N^{2}} - (\mu_{N} )^{2} .
\label{eq:eqCapRuidoVariancia2}
\end{equation}


Diferentes distribuições de probabilidades são utilizadas para caracterizar cada tipo de ruído mencionado anteriormente, quando se tem o conhecimento detalhado do sistema de aquisição do sinal. Em imagens digitais, em geral, o ruído pode ser caraterizado por uma combinação dessas distribuições.

O teorema do limite central (\textit{Central Limit Theorem} - \acs{CLT}) é comumente usado para modelar sistemas físicos que são complexos do ponto de vista onde inúmeras variáveis aleatórias com diferentes distribuições são utilizadas para modelos físicos distintos. Ele afirma que a soma dessas variáveis aleatórias converge para uma distribuição normal $\mathcal{N}(\mu,\,\sigma^{2})$ com uma respectiva média $\mu$ e variância $\sigma^{2}$. No entanto, na prática, a média e a variância dos sinais podem mudar de acordo com a posição espacial, alterando assim as características dessas distribuições nas diversas posições do detector, mesmo que ainda sejam normais. Sendo assim, são necessários modelos matemáticos que levem em consideração esses fatores, como demonstrado nas seções abaixo \cite{bertalmiodenoising2018}.


\subsection{Ruído Independente do Sinal}

Agitações térmicas podem ser representadas por uma distribuição Gaussiana, também chamada de normal. Isso se deve ao fato dessas variações serem independentes da quantidade de sinal, ou seja, a variação de temperatura em média é a mesma em todo o detector, como ilustra a Figura \ref{fig:imgCapRuidoPhantom}. Esse tipo de flutuação é descrita matematicamente através do modelo de ruído aditivo Gaussiano branco  (\textit{Additive White Gaussian Noise} - \acs{AWGN}), independente e identicamente distribuído (\textit{Independent and Identically Distributed} - \acs{IID}). Sendo assim, a equação que descreve a contaminação da imagem por esse ruído é dada por:


\begin{equation}
z(x) = y(x) + N(x),
\label{eq:eqCapRuidoTermico}
\end{equation}

\noindent na qual, $y$ é a imagem livre de ruído, $N$ é uma variável aleatória com uma distribuição normal $\mathcal{N}(0,\,\sigma^{2})$ de média zero e variância $\sigma^{2}$, $z$ é a imagem observada e $x$ descreve as coordenadas espaciais ao longo da imagem. O termo branco de \acs{AWGN} está relacionado com  o fato da variância do espectro de Fourier do ruído ser constante. A palavra branco em si deriva da espectroscopia, onde a luz branca possui todos os componentes de frequência do espectro visível. Esses conceitos são detalhados nas próximas seções.


\subsection{Ruído Dependente do Sinal}

Já o ruído quântico está relacionado com o fluxo de fótons em cada sensor, de modo que a variância das cargas elétricas é proporcional a esse fluxo. Isso significa que em locais onde mais fótons são detectados, maior é a variação nas cargas elétricas e consequentemente os valores dos \textit{pixels} serão mais contaminados pelo ruído quântico. Sendo assim, a dependência do sinal é uma característica desse tipo de ruído, que é matematicamente modelado através de uma distribuição Poisson \cite{bertalmiodenoising2018}. De fato, a equação simplificada que descreve a captura de uma imagem por um foto-sensor é dada por:

\begin{equation}
z(x) =  \mathcal{P}(y(x)),
\label{eq:eqCapRuidoQuantico}
\end{equation}

\noindent em que $\mathcal{P}$ significa a família de distribuições de Poisson. Uma das características marcantes dessa distribuição é o fato da variância ter o mesmo valor da média:

\begin{equation}
 E\{z(x)\mid y(x)\} =  Var\{z(x)\mid y(x)\} = y(x).
\label{eq:eqCapRuidoQuanticoMediaVar}
\end{equation}

Essa propriedade pode ser observada através do exemplo na Figura \ref{fig:imgCapRuidoPoisson}, onde uma imagem sintética representa a aquisição de fótons com intensidade variada e a outra simula a inserção do ruído com uma distribuição Poisson. O perfil dessa perturbação é evidenciando na Figura \ref{fig:imgCapRuidoNoise3D}, exemplificando essa dependência do sinal. 

\begin{figure}[H]
	\centering
	
	\caption{Ilustração de (a) uma imagem sintética, (b) da inserção de ruído Poisson na mesma, (c) da subtração de (a) por (b) para extração apenas do ruído e (d) do seu respectivo perfil em \acs{3D}.}
	
	\subfloat[]{\includegraphics[scale=0.4, clip, trim=5cm 8cm 5cm 7cm]{imgs/capRuido/PhantomPoisson.pdf}	
	\label{fig:imgCapRuidoPhantomPoisson}}
	\hfill
	\subfloat[]{\includegraphics[scale=0.4, clip, trim=5cm 8cm 5cm 7cm]{imgs/capRuido/PhantomWithNoise.pdf} \label{fig:imgCapRuidoPhantomWithNoise}}
	\hfill
	\subfloat[]{\includegraphics[scale=0.4, clip, trim=5cm 8cm 5cm 7cm]{imgs/capRuido/PoissonNoise.pdf} \label{fig:imgCapRuidoPoissonNoise}}
	\hfill
	\subfloat[]{\includegraphics[scale=0.55, clip, trim=3.1cm 8.7cm 3.7cm 8.2cm]{imgs/capRuido/Noise3D.pdf} \label{fig:imgCapRuidoNoise3D}}
	
	\legend{Fonte: do autor, 2019.}
	\label{fig:imgCapRuidoPoisson}
\end{figure}

Consequentemente a essa propriedade, o \acs{SNR} é diretamente proporcional a média do sinal:

\begin{equation}
SNR =  \dfrac{\mu}{\sigma} =  \dfrac{y(x)}{\sqrt{y(x)}}  = \sqrt{y(x)},
\label{eq:eqCapRuidoQuanticoSNR}
\end{equation}

\noindent na qual, $\mu$ representa a média e $\sigma$ o desvio padrão.

\section{Ruído em Mamografia}\label{Capitulo4:RuidoemMamografia}

Para a realização dos exames de mamografia, são utilizadas ondas eletromagnéticas na faixa dos raios X. Apesar da maioria dos detectores serem integradores de energia e não de fato contadores de fótons, a flutuação da intensidade dos raios X, ou seja, o ruído quântico é modelado através da distribuição de Poisson \cite{boone2000handbook,haus2000screen}. De fato, o ruído quântico é a maior fonte de flutuações nos exames de mamografia quando há uma alta intensidade de radiação, porém outras fontes de ruído também são observadas \cite{huda2003experimental,marshall2017handbook}. Especialmente em casos onde a dose de radiação é baixa, o ruído eletrônico se torna muito importante. Tais interferências são causadas por variações térmicas e pelo circuito eletrônico dos detectores em geral. Um exemplo de exame que utiliza uma baixa dose de radiação é a \acs{DBT}. Nessa técnica, o número de \textit{quanta} por projeção é menor, quando comparado com a mamografia digital convencional, sendo assim o ruído dependente do sinal é menor, tornando a parcela do ruído independente significativa \cite{sechopoulos2013review,vedantham2015digital}. 

A calibração é um fator importante na rotina dos equipamentos de mamografia. A uniformidade das intensidades dos \textit{pixels} ao longo do detector e a não negatividade são fatores levados em conta. A uniformidade do campo é feita através de um procedimento conhecido como correção de \textit{flat-field} \cite{marshall2017handbook}. Nesse procedimento, é dado um ganho nos detectores em cada posição a fim de corrigir as imperfeições de iluminação dos raios X causadas pelo efeito heel e pela geometria do cone. Já a não negatividade é garantida por meio de um \textit{offset} no valor lido pelos detectores. Todos esses fatores alteram significativamente as características do ruído, tornando-o  dependente do espaço \cite{borges2017pipeline,borges2017method, borges2018restoration,brito2018application,guerrero2018}. Esse procedimento de certa forma corrige também os ruídos estruturais presentes devido as variações nos ganhos do detector, fixas espacialmente \cite{VanEngen2013,marshall2017handbook}.

De fato, levando em consideração o ruído quântico, térmico e os fatores de calibração, um modelo matemático que se aproxima desses efeitos físicos é o Poisson-Gaussiano:

\begin{equation}
z(x) = ap(x) + N(x),
\label{eq:eqCapRuidoPossionGaussiano}
\end{equation}

\noindent no qual N é uma função que tem a distribuição aproximada por uma Gaussiana de média 0 e variância b e p uma função com distribuição Poisson com um fator de escala a, segundo:

\begin{equation}
p(x) \sim \mathcal{P}(a^{-1}\,y(x)) \qquad \text{e}  \qquad  N(x) \sim \mathcal{N}(0,\,b),
\label{eq:eqCapRuidoPossionGaussiano1}
\end{equation}

\noindent onde sua média e variância são dadas por:

\begin{equation}
E\{z(x)\mid y(x)\} =  y(x),
\label{eq:eqCapRuidoPossionGaussianoMedia}
\end{equation}

\begin{equation}
Var\{z(x)\mid y(x)\} = ay(x) + b.
\label{eq:eqCapRuidoPossionGaussianoVar}
\end{equation}

De acordo com \citeonline{bertalmiodenoising2018}, para contagens que possuem uma média alta, a distribuição de Poisson se aproxima de uma normal, segundo:

\begin{equation}
\mathcal{P}(y(x)) \approx \mathcal{N}(y(x),\,y(x)),
\label{eq:eqCapRuidoPossionGaussianoAproximacao}
\end{equation}

\noindent do mesmo modo, é possível aproximar  a Equação \ref{eq:eqCapRuidoPossionGaussiano} para o dito modelo Gaussiano heteroscedástico dependente do sinal:

\begin{equation}
z(x) = y(x) + \sigma(y(x)) \, \xi(x),
\label{eq:eqCapRuidoHeteroscedastico}
\end{equation}

\noindent onde a função  $\xi(x)$ é \acs{IID} e tem uma distribuição normal com média zero e variância um, segundo:

\begin{equation}
\xi(x) \sim  \mathcal{N}(0,\,1),
\label{eq:eqCapRuidoHeteroscedastico1}
\end{equation}

\noindent e a função $\sigma(y(x)) $, nomeada como função desvio padrão, modela o desvio padrão da parte dependente do sinal de acordo com a imagem livre de ruído, conforme:

\begin{equation}
\sigma(y(x)) = \sqrt{ay(x)+b}.
\label{eq:eqCapRuidoHeteroscedasticoStd}
\end{equation} 

A Equação \ref{eq:eqCapRuidoHeteroscedastico} pode ser escrita ainda de maneira mais simplificada:

\begin{equation}
z(x) \sim  \mathcal{N}(\,y(x),\,\sigma(\,y(x)) \,).
\label{eq:eqCapRuidoHeteroscedasticoSimplificado}
\end{equation}

\subsection{Correlação Espacial}\label{Correlação}

Até então, todo o modelamento do ruído foi feito considerando cada unidade do detector como uma variável aleatória independente. Na prática, isso não é a realidade observada nos equipamentos. A correlação, ou a associação existente entre duas variáveis, ocorre devido a fenômenos físicos, tais como: o processo de aquisição do sistema, o procedimento de leitura dos sensores, o \textit{cross-talk} existente entre \textit{pixels} vizinhos, dentre outros \cite{morettin2010,bertalmiodenoising2018}. 

Esses fenômenos podem ser quantificados matematicamente no domínio do espaço através de uma função de autocovariância, porém são mais comumente descritos no domínio da frequência através do espectro de potência do ruído (\textit{Noise Power Spectrum} - \acs{NPS}), também conhecido como espectro de Wiener \cite[p. 485]{marshall2017handbook}. Segundo \citeonline{marshall2017handbook}, o \acs{NPS} é o modulo da transformada de Fourier da função de autocovariância, ou segundo \citeonline{bertalmiodenoising2018} corresponde a variância da transformada de Fourier do ruído. Essa medida é dada pela potência do ruído por unidade de frequência \cite{marshall2017handbook}.

A correlação do ruído é comumente associada com as cores do espectro de luz visível. Isso se deve ao fato de sua medida objetiva ser feita mais frequentemente no domínio da frequência. Por isso é feita essa associação. 

O ruído é dito colorido quando seu espectro de potência não é constante, ou seja, quando existe uma preponderância da variância em certa faixa de frequência. O ruído que afeta as baixas frequências é denominado de ``ruído vermelho'', enquanto o que afeta as altas é chamado de ``ruído azul'', dentre outras denominações. Ao caracterizar o ruído como ``branco'', significa que o seu espectro de frequência é constante \cite{marshall2017handbook,bertalmiodenoising2018}. A Figura \ref{fig:imgCapRuidoColoredNoise} exemplifica os tipos de ruídos correlacionados mencionados.


\begin{figure}[H]
	\centering
	
	\caption{Ilustração de alguns tipos de ruídos correlacionados denominados coloridos, especificamente (a) branco, (b) vermelho, (c) azul e (d) horizontal.}
	
	\subfloat[]{\includegraphics[scale=0.7, clip, trim=3.2cm 6.7cm 14.3cm 16cm]{imgs/capRuido/ColoredNoise.pdf}}
	\hfil
	\subfloat[]{\includegraphics[scale=0.7, clip, trim=6.75cm 6.7cm 10.4cm 16cm]{imgs/capRuido/ColoredNoise.pdf}}
	\hfil
	\subfloat[]{\includegraphics[scale=0.7, clip, trim=10.35cm 6.7cm 6.75cm 16cm]{imgs/capRuido/ColoredNoise.pdf}}
	\hfil
	\subfloat[]{\includegraphics[scale=0.7, clip, trim=14.2cm 6.7cm 3.4cm 16cm]{imgs/capRuido/ColoredNoise.pdf}}
	
	\legend{Fonte: \citeonline{bertalmiodenoising2018}.}
	\label{fig:imgCapRuidoColoredNoise}
\end{figure}


O modelo mais simples para a formulação do ruído correlacionado é dado pelo modelo dito estacionário, matematicamente formulado por:

\begin{equation}
z(x) =  y(x) + N(x), \qquad N(x) = (\nu\circledast g)(x),
\label{eq:eqCapRuidoCorrelatedNoiseStationary}
\end{equation}

\noindent no qual, N é a função que modela o ruído aditivo correlacionado estacionário de média zero, $\nu$ representa o ruído independente de média zero e variância unitária e g é a máscara de convolução que modela a correlação do ruído e sua variância. Os modelos mais complexos são encontrados em \citeonline[p. 27-30]{bertalmiodenoising2018}.





      


 
\chapter[Materiais e Métodos]{Materiais e Métodos}\label{Capitulo5}
 

A seção de materiais e métodos é dividida em duas partes. A primeira está relacionada com a elaboração e validação dos algoritmos de reconstrução em forma de um \textit{software} de código aberto. Já a segunda parte é referente à análise das propriedades do sinal e do ruído nos equipamentos de \acs{DBT} em imagens reconstruídas. 

%%%%%%%%%%%%%%%%%%%%%%%%%%%%%%%%%%%%%%%%%%%%%%%%%%%%%%%%%%%%%%%%%%%%%%%%%%%%%%%%%%%%%%%%%%%%%%%%%%%%%%%%%%%%%%%%%%%%%%%%%%%%%%														Elaboração e Validação da \textit{Toolbox} de Reconstrução																%
%%%%%%%%%%%%%%%%%%%%%%%%%%%%%%%%%%%%%%%%%%%%%%%%%%%%%%%%%%%%%%%%%%%%%%%%%%%%%%%%%%%%%%%%%%%%%%%%%%%%%%%%%%%%%%%%%%%%%%%%%%%% 
\section{Elaboração e Validação da \textit{Toolbox} de Reconstrução}

De modo geral, a Figura \ref{fig:imgCap4Toolbox} ilustra como foi desenvolvido o \textit{software} de reconstrução tomográfica para imagens de \acs{DBT}, nesse trabalho denominado de \textit{toolbox}. O desenvolvimento dessa ferramenta se deu pelo fato da indisponibilidade de \textit{softwares} abertos no início deste trabalho que exerciam a  função de reconstrução de imagens especificamente para \acs{DBT} . Nesse meio tempo, foi disponibilizado através de um repositório \textit{online} do setor de divisão de imagem, diagnóstico e confiabilidade de \textit{software} do órgão americano \acs{FDA} uma \textit{toolbox}, também de código aberto, para reconstrução de imagens \cite{Zengtoolbox}.  

Para a elaboração da ferramenta de reconstrução, primeiramente foi utilizado como base um \textit{software} de reconstrução para \acs{CT} com geometria de feixe cônico. A partir daí, foi feito o desenvolvimento da geometria de meio feixe cônico, ou \textit{half-cone beam}, específico para a técnica de tomossíntese. Para avaliações através de métricas objetivas, a fim de  validar os métodos de reconstrução implementados, é necessário ter um padrão-ouro (\textit{ground-truth}), sendo assim, utilizou-se de \textit{phantoms} virtuais sintéticos. Imagens de pacientes e \textit{phantoms} físicos também foram utilizados para a validação. Diferentes técnicas de projeção e retroprojeção geométricas foram então aplicadas para a obtenção das projeções e retroprojeções dos \textit{phantoms} virtuais, para então ser possível a aplicação dos métodos de reconstrução. Métricas objetivas, bem como as subjetivas para casos clínicos foram aplicadas. Então, consolidou-se o desenvolvimento da ferramenta, possibilitando diversas aplicações, como a da análise de ruído proposta por esse trabalho.

\begin{figure}[H]
	\caption{Fluxograma simplificado ilustrando o desenvolvimento do \textit{software} de reconstrução tomográfica para imagens de \acs{DBT}.}
	\begin{center}
		\includegraphics[scale=0.65, clip, trim=7.4cm 3.3cm 5.9cm 3.2cm]{imgs/cap4/Toolbox.pdf}
	\end{center}
	\legend{Fonte: do autor, 2019.}
	\label{fig:imgCap4Toolbox}
\end{figure}

Como mencionado, utilizou-se \textit{phantoms} e imagens de pacientes para validar os métodos de reconstrução. Um \textit{phantom} pode ser físico ou virtual e tem como objetivo simular condições reais de exame para que os métodos alvos de pesquisas possam ser estudados e testados de maneira sistemática. Isso se deve ao fato de que exames clínicos, ou seja, com pacientes, não podem ser realizados toda vez que um novo método necessitar ser testado.

\subsection{\textit{Phantom} Virtual}

Dois modelos virtuais diferentes foram aplicados para a validação.  O primeiro, uma versão modificada de \citeonline[]{shepp1974fourier}. O \textit{phantom} originalmente é constituído por uma seção da cabeça, que foi desenvolvido para testar o algoritmo proposto pelos autores naquela época. Esse \textit{phantom} simula diversas estruturas anatômicas com diferentes densidades $D$ como: a água ($D=1$) nos ventrículos, massa cinzenta ($D=1,2$), tumores ($D=1,03;1,04$) e para o crânio ($D=2$). A Figura \ref{fig:imgCap4SheppLogan2D} ilustra o \textit{phantom} de Shepp-Logan em sua versão modificada.

Como esse trabalho visa a reconstrução de estruturas anatômicas em \acs{3D} pela técnica de tomossíntese, é necessário a utilização de um objeto tri-dimensional para validar as técnicas aplicadas. Para isso, foi aplicado um algoritmo que gera uma extensão em \acs{3D} do \textit{phantom} Shepp-Logan \cite{Schabel2006}. Esse código permite a criação do simuluador com um número de fatias qualquer. A Figura \ref{fig:imgCap4SheppLogan3D} ilustra o objeto \acs{3D} criado pelo respectivo algoritmo.

\begin{figure}[H]
	\centering
	
	\caption{Ilustração do \textit{phantom} de Shepp-Logan na sua versão modificada em (a) \acs{2D} e (b) \acs{3D}.}
	
	\subfloat[\acs{2D}.]{\includegraphics[scale=0.4, clip, trim=5cm 8cm 5cm 7cm]{imgs/capRuido/Phantom.pdf}\label{fig:imgCap4SheppLogan2D}}
	\hfil
	\subfloat[\acs{3D}.]{\includegraphics[scale=0.6, clip, trim=14cm 4.6cm 15cm 5.6cm]{imgs/cap4/SheppLogan3D.pdf}\label{fig:imgCap4SheppLogan3D}}
	
	\legend{Fonte: do autor, 2019.}
	\label{fig:imgCap4SheppLogan}
\end{figure}

Para uma avaliação mais de acordo com a realidade dos exames de mamografia, utilizou-se de forma semelhante um \textit{phantom} antropomórfico virtual da mama \cite{bakic2002mammogram1,bakic2002mammogram2,bakic2003mammogram}, gerado pelo \textit{software} gratuito de ensaios clínicos virtuais (\textit{Virtual Clinical Trials} - \acs{VCT}), denominado OpenVCT \cite{BakicVCT}, desenvolvido na Universidade da Pensilvânia nos Estados Unidos. Esse programa possibilita a geração de modelos de mama customizados, deformações como a compressão exercida pelo prato, inserção de lesões como microcalcificações e também proporciona a simulação das projeções de raios X geradas pelo mamógrafo \cite{barufaldi2018openvct}. A Figura \ref{fig:imgCap4PhantomVCT} ilustra uma fatia retirada do volume do \textit{phantom} antropomórfico simulado, assim como uma ilustração de seu volume \acs{3D}. Foi simulado uma mama de espessura $63.3mm$ advinda de uma paciente com idade de 48 anos, $1.65m$ de altura e $60Kg$. Lesões que mimetizam microcalcificações foram inseridas na altura de $54mm$ do volume aproximadamente. 

\begin{figure}[H]
	\centering
	
	\caption{Ilustração de uma (a) fatia retirada do volume do \textit{phantom} antropomórfico gerada pelo \acs{VCT} e (b) seu respectivo volume \acs{3D} .}
	
	\subfloat[\acs{2D}.]{\includegraphics[scale=0.5, clip, trim=0cm 0cm 0cm 0cm]{imgs/cap4/PhantomVCT.png}}
	\hfil
	\subfloat[\acs{3D}.]{\includegraphics[scale=0.5, clip, trim=11.2cm 0cm 10.5cm 0cm]{imgs/cap4/PhantomVCT3D.pdf}}
	\hfil

	\legend{Fonte: do autor, 2019.}
	\label{fig:imgCap4PhantomVCT}
\end{figure}

\subsection{\textit{Phantom} Físico}

 O modelo antropomórfico físico da mama BR3D \cite{PhantomBR3D} foi empregado da mesma forma. Esse simulador de tecido é constituído de 6 fatias, cada uma com $10mm$ de espessura, com um material heterogêneo que representa uma mistura de tecido $100\%$ adiposo e $100\%$ glandular de uma mama com razão de 50/50. Junto a uma fatia específica, estão contidas microcalcificações, fibras e massas que simulam os indicadores do câncer de mama, na qual a Tabela \ref{tab:tabCap4BR3D} especifica suas respectivas dimensões. Todas essas estruturas possuem diferentes tamanhos e buscam mimetizar os aspectos reais encontrados em uma mama com possíveis lesões, como ilustra a Figura \ref{fig:imgCap4BR3D}.
 
 \begin{table}[H]
 	\centering
 	\caption{Dimensões dos indicadores de câncer simulados.}
 	\label{tab:tabCap4BR3D}
 	\begin{tabular}{c|c|c}
 		\textbf{Fibras {\O}mm}                                              & \textbf{Massas esferoidais {\O}mm} & \textbf{Microcalcificações {\O}mm} \\
 		[4pt]
 		\hline
 		\hline
 		\rule[0ex]{0pt}{1ex}
 		0,6 							 &             6,3             &            0,400            \\ \hline
 		\rule[0ex]{0pt}{1ex}
 		0,41                             &             4,7             &            0,290            \\ \hline
 		\rule[0ex]{0pt}{1ex}
 		0,38                             &             3,9             &            0,230            \\ \hline
 		\rule[0ex]{0pt}{1ex}
 		0,28                             &             3,1             &            0,196            \\ \hline
 		\rule[0ex]{0pt}{1ex}
 		0,23                             &             2,3             &            0,165            \\ \hline
 		\rule[0ex]{0pt}{1ex}
 		0,18                             &             1,8             &            0,130            \\ \hline
 		\rule[0ex]{0pt}{1ex}
 		0,15                             &                             &            	             \\ \hline
 	\end{tabular}
 	\vspace{2ex}
 	\legend{Fonte: \citeonline{PhantomBR3D}.}
 \end{table}

\begin{figure}[H]
	\caption{Ilustração das fatias do \textit{phantom} BR3D.}
	\begin{center}
		\includegraphics[scale=0.5, clip, trim=5.6cm 12.8cm 5.4cm 5cm]{imgs/cap4/BR3D.pdf}
	\end{center}
	\legend{Fonte: \citeonline{PhantomBR3D}.}
	\label{fig:imgCap4BR3D}
\end{figure} 

\subsection{Imagens Clínicas}

Através de uma parceria com o Hospital de Amor da cidade de Barretos, São Paulo, foram adquiridas imagens clínicas tanto de equipamento da \acs{GE}, denominado nesse trabalho por sistema \#1, quanto de equipamento da Hologic, denominados por sistema \#2. Junto às projeções adquiridas nos equipamentos, as respectivas reconstruções também foram capturadas permitindo uma comparação visual direta com as reconstruções feitas pela \textit{toolbox}, avaliando a similaridade entre as mesmas. Esse projeto foi aprovado segundo o comitê de ética número CAAE: 78625417.1.1001.5437.

%%%%%%%%%%%%%%%%%%%%%%%%%%%%%%%%%%%%%%%%%%%%%%%%%%%%%%%%%%%%%%%%%%%%%%%%%%%%%%%%%%%%%%%%%%%%%%%%%%%%%%%%%%%%%%%%%%%%%%%%%%%%%%													Geometria																%
%%%%%%%%%%%%%%%%%%%%%%%%%%%%%%%%%%%%%%%%%%%%%%%%%%%%%%%%%%%%%%%%%%%%%%%%%%%%%%%%%%%%%%%%%%%%%%%%%%%%%%%%%%%%%%%%%%%%%%%%%%%% 
\subsection{Geometria de Aquisição} 


Como mencionado na seção \ref{ParâmetrosFísicoseGeométricos}, a geometria de cada equipamento comercial muda de acordo com o fabricante. As projeções de raios X dos \textit{phantoms} e das imagens clínicas foram adquiridas pelo sistema 1 ou 2. 

Para ambos os \textit{phantoms} antropomórficos, a geometria de aquisição do sistema 1 foi utilizada. 

No caso do simulador de tecidos virtual, as projeções de raios X foram adquiridas através do OpenVCT, ilustrado pela projeção central na Figura \ref{fig:imgCap4PhantomVCT_CentralProj}. Uma exposição de raios X, com a tensão de pico do tubo de $35kVp$ e o produto da corrente elétrica vezes o tempo de $70mAs$, foi simulada. Do mesmo modo, utilizou-se o anodo de tungstênio com o filtro de alumínio. As projeções simuladas não levaram em consideração o ruído e foram quantizadas em 14\textit{bits}.

\begin{figure}[H]
	\caption{Projeção central do \textit{phantom} antropomórfico virtual gerada através do OpenVCT.}
	\begin{center}
		\includegraphics[scale=0.18, clip, trim=9cm 4cm 0cm 4cm]{imgs/cap4/PhantomVCT_CentralProj.png}
	\end{center}
	\legend{Fonte: do autor, 2019.}
	\label{fig:imgCap4PhantomVCT_CentralProj}
\end{figure} 

No caso do \textit{phantom} físico, utilizou-se  o equipamento comercial de \acs{DBT} modelo Senographe Essential{\footnotesize \texttrademark} \space da empresa \acs{GE}, como ilustra a Figura \ref{fig:imgCap4BR3D}. Foi feita uma exposição de raios X com $29kVp$ e $63mAs$. O equipamento foi configurado com o ânodo de ródio e com o filtro também de ródio. Foram geradas projeções com resolução em nível de cinza de 14 \textit{bits}.

\begin{figure}[H]
	\centering
	
	\caption{(a) Equipamento de tomossíntese modelo Senographe Essential{\footnotesize \texttrademark} \ utilizado para realizar as projeções e (b) o mesmo com o \textit{phantom} BR3D.}
	
	\subfloat[]{\includegraphics[scale=0.08]{imgs/cap4/EquipamentoGE.jpg}\label{fig:imgCap4EquipamentoGEA}}
	\hfil
	\subfloat[]{\includegraphics[scale=.49]{imgs/cap4/BR3Deq.jpeg}\label{fig:imgCap4EquipamentoGEB}}
	
	\legend{Fonte: do autor, 2019.}
	\label{fig:imgCap4EquipamentoGE}
\end{figure} 

Já para as imagens clínicas, ambos sistemas foram empregados. A Tabela \ref{tab:tabCap4ParametrosEquipamentos} expõe os parâmetros físicos dos equipamentos comerciais que foram utilizados.

\begin{table}[H]
	\centering
	\caption{Parâmetros físicos dos sistemas comerciais.}
	\label{tab:tabCap4ParametrosEquipamentos}
	\begin{tabular}{l|c|c}
		\textbf{Sistema}                            &   \textbf{\#1}   &   \textbf{\#2}\\
		[5pt]
		\hline
		\hline
		\rule[-0.5ex]{-3pt}{1ex}
		Fabricante												& \acs{GE} & Hologic \\ \hline
		\rule[-0.5ex]{-3pt}{1ex}
		Modelo													&  Senographe Essential &  Selenia Dimensions\\ \hline
		\rule[-0.5ex]{-3pt}{1ex}
		Tamanho do detector 						  &      24x30cm  &      24x29cm      \\ \hline
		\rule[-0.5ex]{-3pt}{1ex}
		Tamanho do \textit{pixel}                     &     100$\mu$m  &     140$\mu$m     \\ \hline
		\rule[-0.5ex]{-3pt}{1ex}
		Número de projeções                           &         9 &         15         \\ \hline
		\rule[-0.5ex]{-3pt}{1ex}
		Angulação do tubo                             &    25$\degree$   &    15$\degree$    \\ \hline
		\rule[-0.5ex]{-3pt}{1ex}
		Angulação do detector                         &    Estacionário  &    4,2$\degree$  \\ \hline
		\rule[-0.5ex]{-3pt}{1ex}
		Tamanho do \textit{Voxel}                     &   0,1x0,1x0,5mm   &   0,112x0,112x1mm \\ \hline
		\rule[-0.5ex]{-3pt}{1ex}
		Distância fonte para detector                 &       660mm  &       700mm       \\ \hline
		\rule[-0.5ex]{-3pt}{1ex}
		Distância detector para centro de rotação     &        40mm     &        0mm   \\ \hline
		\rule[-0.5ex]{-3pt}{1ex}
		Tamanho lacuna de ar                          &        22mm   &        25mm     \\ \hline
	\end{tabular}
	\vspace{2ex}
	\legend{Fonte: \citeonline{michell2018role,vedantham2015digital,sechopoulos2013review,baker2011breast}.}
\end{table}

A \textit{toolbox} desenvolvida nesse trabalho permite simular as projeções de raios X , quando o volume do objeto está expresso em valores de coeficientes de atenuação, ou seja, somente a projeção geométrica é simulada, não levando em conta as interações da radiação com a matéria como é feito no OpenVCT. Esse procedimento foi feito com o \textit{phantom} de Shepp-Logan com uma geometria semelhante ao do sistema 1 e a Figura \ref{fig:imgCap4Projecoes} ilustra as projeções geradas.

\begin{figure}[H]
	\centering
	
	\caption{Projeções geradas a partir do volume ilustrado na Figura \ref{fig:imgCap4SheppLogan3D}, onde de (a) até (i) representam as projeções de 1 a 9 respectivamente.}
	
	\subfloat[]{\includegraphics[scale=.45]{imgs/cap4/Proj/1.png}\label{fig:imgCap4Projecao1}}
	\subfloat[]{\includegraphics[scale=.45]{imgs/cap4/Proj/2.png}\label{fig:imgCap4Projecao2}}
	\subfloat[]{\includegraphics[scale=.45]{imgs/cap4/Proj/3.png}\label{fig:imgCap4Projecao3}}
	\subfloat[]{\includegraphics[scale=.45]{imgs/cap4/Proj/4.png}\label{fig:imgCap4Projecao4}}
	\subfloat[]{\includegraphics[scale=.45]{imgs/cap4/Proj/5.png}\label{fig:imgCap4Projecao5}}
	\subfloat[]{\includegraphics[scale=.45]{imgs/cap4/Proj/6.png}\label{fig:imgCap4Projecao6}}
	\subfloat[]{\includegraphics[scale=.45]{imgs/cap4/Proj/7.png}\label{fig:imgCap4Projecao7}}
	\subfloat[]{\includegraphics[scale=.45]{imgs/cap4/Proj/8.png}\label{fig:imgCap4Projecao8}}
	\subfloat[]{\includegraphics[scale=.45]{imgs/cap4/Proj/9.png}\label{fig:imgCap4Projecao9}}
	
	\legend{Fonte: do autor, 2019.}
	\label{fig:imgCap4Projecoes}
\end{figure}

 
A Figura \ref{fig:imgCap4GeometriaProjecao} ilustra parte da animação gerada pela \textit{toolbox} quando é feita a projeção geométrica  disponível no programa, sendo que a Figura \ref{fig:imgCap4GeometriaProjecao2} representa a projeção central, \ref{fig:imgCap4GeometriaProjecao1} a primeira e \ref{fig:imgCap4GeometriaProjecao3} a última. O cubo azul ilustra espacialmente o \textit{phantom} virtual da Figura \ref{fig:imgCap4SheppLogan3D}, o retângulo em amarelo a projeção do objeto no plano da imagem, as linhas pontilhadas de cor preta as extremidades do detector, as linhas coloridas os feixes de raio X  e o ponto vermelho o tubo de raios X. 

\begin{figure}[h]
	\centering
	
	\caption{Parte da animação gerada pela \textit{toolbox} no procedimento de projeção geométrica, na qual (a) representa a projeção central, (b) a primeira e (c) a última.}
	
	\subfloat[]{\includegraphics[scale=.3, clip, trim=1.8cm 3.1cm 1.5cm 4cm]{imgs/cap4/Proj5.pdf}\label{fig:imgCap4GeometriaProjecao2}}
	
	\subfloat[]{\includegraphics[scale=.3, clip, trim=1.8cm 3.1cm 1.5cm 4cm]{imgs/cap4/Proj1.pdf}\label{fig:imgCap4GeometriaProjecao1}}
	\subfloat[]{\includegraphics[scale=.3, clip, trim=1.8cm 3.1cm 1.5cm 4cm]{imgs/cap4/Proj9.pdf}\label{fig:imgCap4GeometriaProjecao3}}
	
	\legend{Fonte: do autor, 2019.}
	\label{fig:imgCap4GeometriaProjecao}
\end{figure}

\subsection{Projeção e Retroprojeção Simples} 

Para a implementação da projeção e retroprojeção simples foi utilizado o método de \textit{Pixel Driven} (Figura \ref{fig:imgCap3Projetores1}) descrito nos capítulos anteriores. Para isso, foram aplicadas as Equações \ref{eq:eqCap3ProjectionY} e \ref{eq:eqCap3ProjectionX}.

 Os cálculos das projeções e retroprojeções são feitos com o objetivo de converter as coordenadas do mundo, e.g. $(X,Y,Z)$, para as coordenadas da imagem, e.g. $(X_{i},Y_{i})$, porém após isso ainda são necessárias as conversões das coordenadas da imagem para as coordenadas em \textit{pixel}, e.g. $(i,j)$. Isso pode ser feito segundo as Equações \ref{eq:eqCap4ConversaoCoordImgparaPixelY} e \ref{eq:eqCap4ConversaoCoordImgparaPixelX}, seguidas por uma interpolação entre as coordenadas em \textit{pixel} da projeção e as coordenadas em \textit{pixel} do detector:
 
 \begin{equation}
 i = \dfrac{Y_{i}}{dy} + y_{0}  
 \label{eq:eqCap4ConversaoCoordImgparaPixelY}
 \end{equation} 
 
 \begin{equation}
 j = -\dfrac{X_{i}}{dx} + x_{0},
 \label{eq:eqCap4ConversaoCoordImgparaPixelX}
 \end{equation} 
 
 \noindent nas quais $dx\,\text{e}\,dy$ são os tamanhos dos \textit{pixels} do detector em (mm) para as coordenadas $x\,\text{e}\,y$ respectivamente e $x_{0}$ e $y_{0}$ são as coordenadas em \textit{pixel} do tubo de raios X. A Figura \ref{fig:imgCap4ConversaoCoord} ilustra a relação entre as coordenadas e os pseudocódigos \ref{alg:algProjecao} e \ref{alg:algRetroprojecao} resumem todo o processo.  
 
 \begin{figure}[H]
 	\caption{Ilustração da relação entre as coordenadas da imagem e as coordenadas do detector em \textit{pixel}.}
 	\begin{center}
 		\includegraphics[scale=0.8]{imgs/cap4/ConversaoCoord.pdf}
 	\end{center}
 	\legend{Fonte: do autor, 2019.}
 	\label{fig:imgCap4ConversaoCoord}
\end{figure} 

\begin{algorithm}[H]
	\SetAlgoLined
	\Entrada{Volume3D, Parâmetros} 
	\Saida{Projeções}
	\Inicio{
		
		\Para{cada Projeção $\in$ Projeções}{
			$\theta \leftarrow$ Ângulo(Projeção)
			
			\Para{cada Fatia $\in$ Volume3D}{
				
				Calcular $Y \;\text{e}\; X \; \forall \; (y_{i}, x_{i}) \; \in \text{Projeção}$
				
				Calcular $i \;\text{e}\; j \; \forall \; (Y, X)$
				
				Projeção $\leftarrow$ Projeção $+$ Interpolação(Fatia$\,,\,(i,j)\,)$			
			}
		}
	}
	\Retorna{Projeções}
	\LinesNumberedHidden
	\caption{Projeção}
	\label{alg:algProjecao}	
\end{algorithm}


\begin{algorithm}[H]
%	\SetAlgoLined
	\Entrada{Projeções, Parâmetros} 
	\Saida{Volume3D}
	\Inicio{
		
		\Para{cada Projeção $\in$ Projeções}{
			$\theta \leftarrow$ Ângulo(Projeção)
			
			\Para{cada Fatia $\in$ Volume3D}{
				
				Calcular $y_{i} \;\text{e}\; x_{i} \; \forall \; (X,Y,Z) \; \in$ Fatia
				
				Calcular $i \;\text{e}\; j \; \forall \; (y_{i}, x_{i})$
				
				Fatia $\leftarrow$ Fatia + Interpolação(Projeção$\,,\,(i,j)\,)$			
			}
		}
	}
	\Retorna{Volume3D}
	\caption{Retroprojeção}
	\label{alg:algRetroprojecao}
\end{algorithm} 

\subsection{Projeção e Retroprojeção \textit{Distance Driven}} 

O método denominado \textit{Distance Driven} (Figura \ref{fig:imgCap3Projetores3}) é uma combinação dos métodos \textit{Pixel Driven} e \textit{Ray Driven}. Nessa técnica, as coordenadas do plano do detector e as do objeto são projetadas em um plano comum, e então a sobreposição entre as coordenadas de ambos são calculadas, tal como o comprimento do raio X que intercede cada \textit{voxel}. Tais valores são utilizados como pesos para os cálculos de projeção e retroprojeção. Esse método foi implementado nesse trabalho segundo o passo a passo descrito em \citeonline{levakhina2014three}. O código implementado está disponível de maneira aberta em sua versão \acs{2D}\footnote{\url{https://github.com/rodrigovimieiro/OpenCodes}} e \acs{3D} no diretório da \textit{toolbox} \cite{de2002distance,de2004distance}.

%%%%%%%%%%%%%%%%%%%%%%%%%%%%%%%%%%%%%%%%%%%%%%%%%%%%%%%%%%%%%%%%%%%%%%%%%%%%%%%%%%%%%%%%%%%%%%%%%%%%%%%%%%%%%%%%%%%%%%%%%%%%%%												Reconstrução																%
%%%%%%%%%%%%%%%%%%%%%%%%%%%%%%%%%%%%%%%%%%%%%%%%%%%%%%%%%%%%%%%%%%%%%%%%%%%%%%%%%%%%%%%%%%%%%%%%%%%%%%%%%%%%%%%%%%%%%%%%%%%%
\subsection{Métodos de Reconstrução de Imagens} 

Após o detalhamento das técnicas de projeção e retroprojeção, os métodos de reconstrução implementados na \textit{toolbox} são detalhados. As quatro técnicas apresentadas nos capítulos anteriores foram aplicadas através dos algoritmos de \acs{BP}, \acs{FBP}, \acs{MLEM} e \acs{SART}. 

Os pseudocódigos \ref{alg:algFBP}, \ref{alg:algMlEM} e \ref{alg:algSART} resumem o procedimento para os métodos de \acs{FBP}, \acs{MLEM} e \acs{SART} respectivamente, enquanto que o método de \acs{BP} foi ilustrado no pseudocódigo \ref{alg:algRetroprojecao}.  

\begin{algorithm}[ht]
%	\SetAlgoLined
	\Entrada{Projeções, Parâmetros} 
	\Saida{Volume3D}
	\Inicio{
		
		\Para{cada Projeção $\in$ Projeções}{
			Projeção $ \leftarrow$ \textbf{Filtrar} (Projeção)	
		}
		Volume3D $\leftarrow$ \textbf{Retroprojeção} (Projeções)
	}
	\Retorna{Volume3D}
	\caption{\acs{FBP}}
	\label{alg:algFBP}
\end{algorithm}

\begin{algorithm}[ht]
%	\SetAlgoLined
	\Entrada{Projeções, Parâmetros} 
	\Saida{Volume3D}
	\Inicio{
	Volume3D $\leftarrow$ EstimativaInicial	
		
	FatorNormVolume $\leftarrow$ \textbf{Retroprojeção} (1)	
	
		\Para{cada Iteração}{
		EstimativaProjeções $\leftarrow$ \textbf{Projeção} (Volume3D)
		
		RazãoProjeções $\leftarrow$ Projeções $\;/\;$ EstimativaProjeções
		
		VolumeEstimado $\leftarrow$ \textbf{Retroprojeção} (RazãoProjeções)
		
		FatorModificação $\leftarrow$ VolumeEstimado $\;/\;$ FatorNormVolume
		
		Volume3D $\leftarrow$ Volume3D $\;\times\;$ FatorModificação
		}
	}
	\Retorna{$Volume3D$}
	\caption{\acs{MLEM}}
	\label{alg:algMlEM}
\end{algorithm}

\begin{algorithm}[ht]
	%	\SetAlgoLined
	\Entrada{Projeções, Parâmetros} 
	\Saida{Volume3D}
	\Inicio{
		Volume3D $\leftarrow$ EstimativaInicial	
		
		FatorNormVolume $\leftarrow$ \textbf{Retroprojeção} (1)	
		
		FatorNormProjeção $\leftarrow$ \textbf{Projeção} (1)
		
		\Para{cada Iteração}{
			\Para{cada Projeção $\in$ Projeções}{
			EstimativaProjeção $\leftarrow$ \textbf{Projeção} (Volume3D)
			
			DiferençaProjeção $\leftarrow$ Projeção $\;-\;$ EstimativaProjeção
			
			DiferençaProjeçãoNorm $\leftarrow$ DiferençaProjeção $\;/\;$ FatorNormProjeção
			
			VolumeEstimado $\leftarrow$ \textbf{Retroprojeção} (DiferençaProjeçãoNorm)
			
			FatorModificação $\leftarrow$ VolumeEstimado $\;/\;$ FatorNormVolume
			
			Volume3D $\leftarrow$ Volume3D $\;+\;$ FatorModificação
		}
		}
	}
	\Retorna{$Volume3D$}
	\caption{\acs{SART}}
	\label{alg:algSART}
\end{algorithm}

O número de iterações dos métodos iterativos foi escolhido através do estudo de trabalhos da literatura e em alguns casos através de análises visuais \cite{wu2004comparison,zhang2006comparative}. Isso se deve ao fato de que os métodos não convergem para uma solução, devido ao mal condicionamento da técnica de tomossíntese como discutido anteriormente, sendo necessário parar antecipadamente para evitar a amplificação do ruído. Foram utilizadas 8 iterações para todas as reconstruções iterativas, exceto para o \textit{phantom} antropomórfico virtual, o qual utilizou-se somente 2.

A confecção do filtro para o algoritmo de retroprojeção filtrada se dá a partir dos passos ditos na seção \ref{RetroprojeçãoFiltrada}. No caso das imagens reais, foi utilizado o janelamento de Hanning, com frequência de corte em 75\% da máxima frequência.

\subsection{Validação da \textit{Toolbox}} 

Para a validação da \textit{toolbox} foram utilizados os \textit{phantoms} virtuais, reais e as imagens clínicas. Para cada tipo de validação foi feita uma análise de modo quantitativo ou qualitativo.

Para o simulador de Shepp-Logan, tanto a análise visual quanto a quantitativa foram feitas. As projeções foram reconstruídas através dos quatro métodos de reconstrução e a qualidade de cada imagem foi comparada com o padrão-ouro através da métrica de índice de similaridade estrutural (\textit{Structural Similarity Index} - \acs{SSIM}). Uma equação polinomial foi calculada para converter os níveis de cinza das imagens reconstruídas para os níveis de cinza do \textit{phantom} padrão-ouro.

Já para o BR3D, uma simples comparação visual foi feita, dado que encontrava-se disponível a imagem reconstruída pelo equipamento comercial. Foram analisados os resultados dos quatro métodos, comparando-os em aspectos visuais como ruído, contraste e objetos em foco na mesma altura. 

Especificamente, no \textit{phantom} gerado pelo OpenVCT, foi feita uma comparação entre a \textit{toolbox} desenvolvida nesse trabalho e outra apresentada pelo \acs{FDA}. Para isso, foram reconstruídas as projeções em ambos \textit{softwares} pelos métodos de \acs{BP}, \acs{FBP} e \acs{SART}. O método de \acs{MLEM} não foi comparado, visto que a implementação dessa técnica se difere entre ambos os \textit{softwares}. Então, as fatias reconstruídas por cada \textit{toolbox} foram comparadas entre si quantitativamente através das métricas de \acs{SSIM} e do erro quadrático médio (\textit{Mean Square Error} - \acs{MSE}). Em outras palavras, o \textit{software} do órgão \acs{FDA} foi tido como o padrão para as reconstruções.  Um ajuste polinomial de primeira ordem foi feito para combinar os níveis de cinza das fatias de ambas \textit{toolboxes} entre métodos de reconstrução iguais, como apresentado no trabalho de \citeonline {borges2018restoration}. A métrica de medida de artefatos fora do foco, denominada de função de espalhamento de artefatos (\textit{Artifact Spread Function} - \acs{ASF}) foi aplicada nas reconstruções de ambas \textit{toolboxes} e também no \textit{phantom} original.

Por fim, para as imagens clínicas, as mesmas foram reconstruídas seguindo o método que mais se aproxima do equipamento comercial. Ou seja, para o sistema \#1 e \#2 foram utilizados os métodos \acs{SART} e \acs{FBP} respectivamente. As imagens reconstruídas cedidas pelo equipamento foram tomadas como padrão e então uma comparação visual foi feita em termos de objetos em foco na mesma altura e também de similaridade. As projeções utilizadas referentes ao sistema \#1 foram do tipo ``não processadas'', enquanto as do sistema \#2 do tipo ``processadas''. Isso se deve ao fato de não termos acesso a ambos os tipos nos dois sistemas.  


\subsubsection{Métricas Objetivas} 

Para analisar de maneira objetiva os resultados das imagens, utilizou-se métricas de avaliação da qualidade da imagem, conhecidas na literatura, como as descritas abaixo.

O  \acs{MSE} é uma forma de medida pontual que analisa as diferenças de intensidade entre uma imagem referência e outra estimada. Esse método é utilizado devido a sua simplicidade de cálculo, obtenção de resultados com fácil interpretação e baixa complexidade computacional. Porém críticas são feitas a este devido a sua ineficiência quando comparado com as métricas subjetivas da percepção humana \cite{gonzalez2008digital,wang2004image}. Sua equação matemática é dada por:

\begin{equation}
MSE = \dfrac{1}{MN} \sum_{x=1}^{M} \sum_{y=1}^{N} [f(x,y) - \hat{f}(x,y)]^{2},
\label{eq:eqCap4MSE}
\end{equation} 

\noindent ou em sua versão aplicando a raiz quadrada:

\begin{equation}
RMSE = \sqrt{MSE(f,\hat{f})},
\label{eq:eqCap4RMSE}
\end{equation}
 
\noindent ou ainda em sua versão normalizada, como demonstrado em \citeonline{brito2018application}:

\begin{equation}
NRMSE = \dfrac{RMSE(f,\hat{f})}{max(\hat{f}) - min(\hat{f})},
\label{eq:eqCap4NRMSE}
\end{equation}

%\noindent onde:
%
%\begin{equation}
%S_{f} = \dfrac{1}{MN} \sum_{x=1}^{M} \sum_{y=1}^{N} f(x,y)  \;\; S_{\hat{f}} = \dfrac{1}{MN} \sum_{x=1}^{M} \sum_{y=1}^{N} \hat{f}(x,y).
%\label{eq:eqCap4NRMSESxSy}
%\end{equation}


A métrica de \acs{SSIM} tem como função fazer uma avaliação que seja mais fiel a percepção humana e a formação das imagens \cite[]{wang2004image}. Esse método avalia componentes como a  luminância $l(\cdot)$, o contraste $c(\cdot)$ e a estrutura dos objetos $s(\cdot)$. Sua formulação matemática é dada por:

\begin{equation}
SSIM(x,y) = [l(x,y)]^{\alpha} \cdot [c(x,y)]^{\beta} \cdot [s(x,y)]^{\gamma},
\label{eq:eqCap4SSIM}
\end{equation}

\noindent na qual $\alpha,\beta \text{ e } \gamma$ são constantes de ajuste da métrica. É importante ressaltar que o \acs{SSIM} retorna os índices de similaridade para diversas regiões da imagem, porém quando se fala em similaridade é interessante que exista somente um índice para toda a imagem. Portanto é feito então uma média dos M índices retornados, dado pela equação:

\begin{equation}
MSSIM(x,y) = \dfrac{1}{M} \sum_{j=1}^{M} SSIM(x_{j},y_{j})
\label{eq:eqCap4MSSIM}
\end{equation} 

Por fim, a avaliação do \acs{ASF} tem a finalidade de medir a quantidade de artefatos fora de foco presentes na reconstrução \cite{zhang2006comparative,borges2017metal}. Essa métrica calcula basicamente a diferença no valor médio dos \textit{pixels} entre o objeto e o fundo na altura em foco, e em diferentes alturas, avaliando o quanto o objeto se espalha para outras fatias, como mostra sua equação matemática:

\begin{equation}
ASF(z) = \dfrac{\overbar{\mu}_{A}(z) - \overbar{\mu}_{BG}(z)}{\overbar{\mu}_{F}(z_{0}) - \overbar{\mu}_{BG}(z_{0})}
\label{eq:eqCap4ASF}
\end{equation} 

\noindent na qual, $\overbar{\mu}_{F}$ é o valor médio dos \textit{pixels} na área do objeto em foco na altura $z_{0}$, $\overbar{\mu}_{A}$ o valor médio dos \textit{pixels} na área do objeto fora de foco e $\overbar{\mu}_{BG}$ o valor médio dos \textit{pixels} na área do fundo em diferentes alturas.

%A métrica de \textit{Sharpness} tem como função avaliar o nível de borramento em uma imagem. Isso é importante pois os métodos de restauração ou reconstrução tendem a borrar as imagens, fazendo com que haja perda de detalhes, como por exemplo em uma microcalcificação nos exames mamografias. É importante notar que esse método não necessita de uma imagem referência. A estimativa é feita através das seguintes equações:
%
%\begin{equation}
%Sharpness = \sum_{r}^{} \sum_{c}^{}  w_{x}G^{2}_{x} + w_{y} G^{2}_{y}, 
%\label{eq:eqCap4SHARPDB}
%\end{equation}
%
%\noindent onde $G_{x}$ e $G_{y}$ são gradientes direcionais, $w_{x}$ e $w_{y}$ são pesos baseados em uma vizinhança local, dados por:
%
%\begin{equation}
%w_{x} = [M(x+1,y)-M(x-1,y)]^{2},
%\label{eq:eqCap4SHARPDB1}
%\end{equation}
%
%\begin{equation}
%w_{y} = [M(x,y+1)-M(x,y-1)]^{2}.
%\label{eq:eqCap4SHARPDB2}
%\end{equation}
%
%\colorbox{pink}{Alexandre: Arrumar essas equações de Sharpness}

%%%%%%%%%%%%%%%%%%%%%%%%%%%%%%%%%%%%%%%%%%%%%%%%%%%%%%%%%%%%%%%%%%%%%%%%%%%%%%%%%%%%%%%%%%%%%%%%%%%%%%%%%%%%%%%%%%%%%%%%%%%%%%														Caracterização do sinal  \& ruído																%
%%%%%%%%%%%%%%%%%%%%%%%%%%%%%%%%%%%%%%%%%%%%%%%%%%%%%%%%%%%%%%%%%%%%%%%%%%%%%%%%%%%%%%%%%%%%%%%%%%%%%%%%%%%%%%%%%%%%%%%%%%%% 
\section{Caracterização do Sinal  \& Ruído}

Com o desenvolvimento da \textit{toolbox} na seção anterior, a possibilidade de aplicações em \acs{DBT} é ampla. Nessa seção foram investigados as propriedades do sinal e do ruído em imagens uniformes reconstruídas de \acs{DBT}. Essa investigação é necessária tendo em vista a utilização desses modelos por métodos de redução de ruído tanto nas imagens de projeção de raios X quanto em outras já reconstruídas \cite{wu2012dose, borges2017pipeline}. Além disso, como já foi mencionado, esse modelamento pode ser incorporado como um modelo de restrição para métodos de reconstrução iterativa \cite{zheng2018detector}. 

Em trabalhos anteriores do nosso grupo, o modelamento do ruído foi feito para os equipamentos de mamografia convencional e também para os de \acs{DBT}. Como proposto no \autoref{Capitulo4}, um modelo  Gaussiano heteroscedástico dependente do sinal foi aplicado no equacionamento das propriedades do ruído quântico e térmico, de acordo com a Equação \ref{eq:eqCapRuidoHeteroscedastico}.  Através desses trabalhos prévios, os autores concluíram que existe uma dependência do sinal e principalmente do espaço em relação ao ruído da mamografia, além da correlação devido as propriedades do detector \cite{borges2017pipeline,borges2017method, borges2018restoration,brito2018application,guerrero2018}. 

Dado isso, o foco dessa seção é realizar um estudo preliminar do ruído em imagens pós-reconstrução de \acs{DBT}, para avaliar se a adequação do modelo aplicado para as projeções é também aplicável nessas imagens já reconstruídas. Da mesma forma, analisamos se o processo de reconstrução, que envolve a combinação de imagens de projeção, correlaciona o ruído tanto para equipamentos que possuíam o espectro de frequência do ruído constante quanto para os que já eram correlacionados.   

\subsection{Sistemas Utilizados}

Os procedimentos envolveram a utilização de dois sistemas comerciais de \acs{DBT}. O primeiro possui um sistema de detecção indireta com detector plano formado por um fotodiodo de silício amorfo (\textit{Amorphous Silicon} - \acs{a-Si}) e tecnologia de cintilador formado por uma camada de fósforo com iodeto de césio dopado com tálio (\textit{Thallium-Doped Cesium Iodide} - \acs{CsI:Tl}). Já o segundo sistema é caracterizado pela detecção direta de raios X através da tecnologia de um detector de selênio amorfo (\textit{Amorphous Selenium} - \acs{a-Se}) \cite[p. 531]{mcEntee2017handbook}. Como já foi dito, denotamos de sistema \#1 e \#2 os equipamentos com detecção indireta e direta respectivamente.

\subsection{\textit{Phantom} Uniforme}

Para a execução dos procedimentos, foi utilizado um \textit{phantom} de \ac{PMMA}. Esse simulador é empregado na rotina clínica com o propósito de controle de qualidade, pois o mesmo simula a absorção da radiação semelhante a uma mama considerada padrão. Testes como a calibração do controle automático de exposição (\textit{Automatic Exposure Control} - \acs{AEC}), análises de ruído, \textit{kerma} incidente, correções de \textit{flat-field} etc, são exemplos nos quais são aplicados o simulador \cite{caron2017,vanegen2018protocol}. Nesse trabalho, sua utilização foi para a estimativa e a medida dos parâmetros do sinal e do ruído. Isso se deve ao fato do bloco de acrílico, ou \acs{PMMA}, ser uniforme, proporcionando teoricamente um sinal também uniforme no detector. A espessura de cada bloco de \acs{PMMA} mede 3 e 4cm, respectivamente, para cada sistema.

 \begin{figure}[H]
	\caption{Ilustração do \textit{phantom} uniforme de \acs{PMMA}.}
	\begin{center}
		\includegraphics[scale=0.8, clip, trim=2cm 9cm 2cm 12.6cm]{imgs/cap4/PMMA.pdf}
	\end{center}
	\legend{Fonte: Adaptado de \citeonline{vanegen2018protocol}.}
	\label{fig:imgCap4PMMA}
\end{figure} 


\subsection{Métricas Objetivas} 

Para a análise do sinal e do ruído nas imagens reconstruídas, foram feitas medidas objetivas como as descritas abaixo.

%Valor médio  e a variância do sinal foram calculados para análise da dependência espacial.

A métrica de \acs{SNR}, já mencionada, avalia a relação sinal-ruído em uma determinada imagem através da média $\mu$ pelo desvio padrão $\sigma$, matematicamente definido por: 

 \begin{equation}
 SNR = \dfrac{\mu}{\sqrt[]{\sigma^{2}}}.    	\qquad 
 \label{eq:eqSNR}
 \end{equation}
 
 Já a medida de \acs{NPS} avalia a variância do ruído no domínio da frequência. Esta medida demonstra em que banda o ruído predomina, ou se seu espectro é constante, denominado ``branco'' \cite{bertalmiodenoising2018}. Sua fórmula matemática é dada por:
 
 \begin{equation}
 NPS_{v} = \dfrac{Ns^{2}}{P} \sum_{p=1}^{P}  \vert \mathcal{F}\{I_{p}-S_{p}\}\vert^{2},    	\qquad 
 NNPS_{v} = \dfrac{NPS_{v}}{L^{2}},	 
 \label{eq:eqNNPS}
 \end{equation}

\noindent na qual, $N$ é o número de \textit{pixels} do \textit{patch}\footnote{ Termo usado em processamento de imagens para nomear regiões de \textit{pixels} (normalmente retangulares) de uma imagem para a execução de cálculos locais.Podem ter tamanhos variados.}, $s$ é o tamanho do detector em milímetros, $P$ é a quantidade de \textit{patches} utilizados na imagem, $\mathcal{F}$  denota a transformada de Fourier,$I_{p}$, $S_{p}$ denotam o sinal adquirido e o mesmo livre de ruído, respectivamente, e $L$ é o maior valor de \textit{pixel} da imagem \cite{dobbins2000handbook}.

\subsection{Metodologia} \label{Metodologia}

Projeções uniforme do \textit{phantom} de \acs{PMMA} foram adquiridas nos sistemas \#1 e \#2. As imagens provenientes do sistema de detecção indireta foram feitas por meio da parceria com o Hospital de Amor da cidade de Barretos. Para o sistema com detecção direta, as imagens foram adquiridas mediante colaboração com o Hospital da Universidade da Pensilvânia nos Estados Unidos. 

Devido a possíveis irregularidades nas bordas do detector causadas, por exemplo, por linhas de \textit{pixels} mortos e também pela presença da placa de compressão, foram retiradas 50/300 linhas de \textit{pixels} na direção da parede torácica para o mamilo (\textit{chest wall-nipple}) e na direção ortogonal, respectivamente, para o sistema \#1; e 100 linhas de \textit{pixels} em ambas as direções no sistema \#2.

Todas as imagens foram reconstruídas por meio do \textit{software} de reconstrução desenvolvido em nosso laboratório, como descrito nas seções anteriores. O algoritmo de \acs{BP} foi utilizado, visto que este é baseado em uma simples retroprojeção geométrica, não alterando as propriedades do sinal das imagens de projeção. A técnica de \textit{Distance Driven} foi aplicada \cite{de2004distance}. 

Para a análise do ruído, as fatias também foram cortadas devido a inconsistência dos \textit{voxels} reconstruídos nas bordas. Tais inconsistências estão presentes porque os \textit{voxels} próximos as bordas do detector são projetados do lado de fora da extensão da placa detectora em certos ângulos de aquisição. Um total de 648/339 linhas de \textit{pixels} foram removidas na direção X/Y respectivamente para o sistema \#1 e 378/319 na direção X/Y para o sistema \#2 na orientação do \textit{chest wall-nipple} e na direção ortogonal respectivamente. A Figura \ref{fig:imgCap4ReconUniform} representa uma fatia reconstruída do \textit{phantom} uniforme antes e depois do corte de algumas linhas de \textit{pixels}.

\begin{figure}[H]
	\centering	
	\caption{Representação de uma fatia reconstruída do \textit{phantom} uniforme (a) antes e (b) depois do corte de algumas linhas de \textit{pixels}.}
	
	\subfloat[]{\includegraphics[scale=0.4,clip, trim=0cm 0cm 0cm 0cm]{imgs/cap4/ReconUniform_Hologic_31_45_NoCut.png}}	
	\hfill
	\subfloat[]{\includegraphics[scale=0.4,clip, trim=0cm 0cm 0cm -1.7cm]{imgs/cap4/ReconUniform_Hologic_31_45_Cut.png}}
	
	\legend{Fonte: do autor, 2019.}
	\label{fig:imgCap4ReconUniform}
\end{figure}

A linearização das imagens foi realizada antes do processo de reconstrução, subtraindo o \textit{offset} de cada \textit{pixel} das projeções. O valor de \textit{offset} foi obtido através de trabalhos anteriores \cite{young2008technical,borges2017pipeline, borges2017method}.

A análise do sinal e do ruído foi realizada considerando a fatia reconstruída na profundidade central de cada \textit{phantom}. O tamanho dos \textit{voxels} nas direções de X e Y foi escolhido de tal modo que seja o mesmo dos elementos detectores, para que a imagem reconstruída possa cobrir toda a área do detector. 

Neste estudo, os valores de intensidade dos \textit{pixels} não foram convertidos em coeficientes de atenuação, ou seja, não foi realizado a transformação logarítmica nas projeções, como ilustrado na Equação \ref{eq:eqCap3BeerLambert2}. Essa estratégia foi adotada para facilitar as comparações entre os resultados deste trabalho e os reportados anteriormente para o domínio da projeção \cite{borges2017pipeline,borges2017method, borges2018restoration,brito2018application,guerrero2018}.

Como foi mencionado na seção anterior, o valor médio do sinal , a variância do ruído, a \acs{SNR} e o \acs{NNPS} foram medidos após a reconstrução considerando ambas as unidades de \acs{DBT}. 

Seguindo o mesmo conceito matemático adotado para as imagens no domínio da projeção, discutido na seção \ref{Capitulo4:RuidoemMamografia}, optou-se do mesmo modo o modelo Gaussiano heteroscedástico dependente do sinal, como demostrado na Equação \ref{eq:eqCapRuidoHeteroscedastico}, para o equacionamento  do ruído nas imagens reconstruídas. 

Se elevarmos ao quadrado ambos os lados da Equação \ref{eq:eqCapRuidoHeteroscedasticoStd}, temos uma nova equação que modela a variância da imagem em função do sinal livre de ruído:

\begin{equation}
\sigma^{2}(y(x)) = ay(x)+b.
\label{eq:eqCap5HeteroscedasticoVar}
\end{equation} 

Dado isso, um equacionamento polinomial de primeira ordem foi feito para estimar os coeficientes angulares e lineares da Equação \ref{eq:eqCap5HeteroscedasticoVar}, a partir dos cálculos de média e variância.

Para o procedimento da estimativa da média e variância, foram adotados \textit{patches} de tamanho 64$\times$64. Para melhorar a precisão da estimativa dos coeficientes da Equação \ref{eq:eqCap5HeteroscedasticoVar}, é desejável que se tenha muitos pontos com diferentes valores em x e y. Para isso, foram adquiridas imagens de raios X do \textit{phantom} uniforme com diferentes doses para obter esse pontos distintos. Para o sistema \#1, foram utilizadas as configurações de $29kVp$ e 45, 54, 72, 90 e $126mAs$ e $31kVp$, 45, 48, 54, 63 e $69mAs$ para o sistema \#2.


Por fim, avaliou-se a distribuição de probabilidade que o sinal segue após o processo de reconstrução. Foi mencionado nas seções anteriores que as projeções de \acs{DBT} seguem a distribuição Poisson-Gaussiana escalada. O processo de reconstrução combina essas projeções para criar as fatias do volume. Portanto, com base no \acs{CLT}, realizou-se testes estatísticos de normalidade para avaliar a distribuição das variáveis aleatórias nas fatias reconstruídas. Foram utilizados\textit{patches} de tamanho $32\times32$ com três métodos diferentes para avaliar a normalidade: Shapiro-Wilk, Kolmogorov-Smirnov e Anderson-Darling \cite{razali2011power}. A hipótese nula $H_{0}$ desses testes afirma que os dados vêm de uma função normalmente distribuída com 5\% de nível de significância.






 
\chapter[Conclusões Preliminares]{Conclusões Preliminares}\label{Capitulo6}

Com o desenvolvimento do presente trabalho até o momento é possível concluir que a técnica de tomossíntese digital da mama possui um enorme campo para o desenvolvimento de pesquisa científica especialmente no ramo da reconstrução de imagens. 

Dentre os tipos de algoritmos estudados, fica evidente que a retroprojeção filtrada (\acs{FBP}) ainda é a mais utilizada por equipamentos comerciais, porém os métodos iterativos são promissores devido a possibilidade de agregação de modelamentos matemáticos de ruído, conhecimentos \textit{a priori}, fácil adaptação a geometria e com os avanços do poder computacional em geral. Grandes empresas já vêm apostando em métodos iterativos para seus respectivos equipamentos.

A aplicação desses métodos foi realizada nesse trabalho, porém alguns algoritmos não obtiveram resultados satisfatórios como era esperado, talvez por falta de pré ou pós processamento. A aplicação do método iterativo estatístico com restrição através da utilização de Campo Aleatório Markoviano não-local não foi possível ainda devido a complexidade teórica envolvida no contexto e problemas na implementação prática das técnicas, mas deve ser objeto de estudo no futuro.

Ainda não estão definidos com exatidão os parâmetros utilizados na geometria dos equipamentos. Faixa de angulação, número de projeções, tipo de movimentação do tubo e angulação do detector, são campos de pesquisas que devem ser explorado a fim de buscar os benefícios da escolha de cada configuração. Para isso é necessário uma plataforma aberta com implementações de algoritmos de reconstrução que possibilite a pesquisa para toda a comunidade acadêmica, que é o objetivo principal desse trabalho.  

Um grande empecilho encontrado no desenvolvimento da pesquisa nessa área é a escassez de recursos de propriedade intelectual como implementações de técnicas e algoritmos. Isso se deve ao fato do ramo ser novo e da proteção existente dentre as grandes empresas envolvidas.       



\chapter[Conclusões]{Conclusões}\label{Capitulo7}

Como conclusão, foi desenvolvida uma ferramenta de reconstrução de imagens específica para a técnica de tomossíntese mamária. Esse \textit{software} está disponibilizado \textit{online} para toda a comunidade acadêmica. Da mesma forma, foi apresentada a validação da ferramenta de reconstrução \acs{DBT} através dos \textit{phantoms} virtuais, físicos e casos reais de pacientes.  

Utilizou-se um simulador de mama antropomórfico, gerado por um sistema \acs{VCT}. As fatias reconstruídas pela \textit{toolbox} foram comparadas com o \textit{software} disponibilizado pelo \acs{FDA}. As reconstruções do \textit{phantom} físico BR3D e das imagens clínicas também foram comparadas, mas estas com as imagens fornecidas pelo equipamento comercial, demonstrando assim a eficácia da ferramenta desenvolvida. 

Ainda não estão definidos com exatidão os parâmetros utilizados na geometria dos equipamentos. Faixa de angulação, número de projeções, tipo de movimentação do tubo e angulação do detector, são campos de pesquisas que devem ser explorados a fim de buscar os benefícios da escolha de cada configuração. Para isso é necessária uma plataforma aberta com implementações de algoritmos de reconstrução que possibilite a pesquisa para toda a comunidade acadêmica, que é o objetivo principal deste trabalho.     

Dentre os tipos de algoritmos estudados, fica evidente que a retroprojeção filtrada (\acs{FBP}) ainda é a mais utilizada por equipamentos comerciais. Porém os métodos iterativos são promissores devido a possibilidade de agregação de modelamentos matemáticos do ruído, conhecimentos \textit{a priori} e fácil adaptação à geometria. No entanto os mesmos demandam um alto custo computacional, mas tornam-se atrativos com o avanço da computação paralela em unidades de processamento gráfico (\textit{Graphics Processing Unit} - \acs{GPU}). Grandes empresas já vêm apostando em métodos iterativos para seus respectivos equipamentos.

Foram analisados também as imagens \acs{DBT} após a reconstrução, em termos do valor médio dos \textit{pixels}, variância do ruído, \acs{SNR} e \acs{NPS}. Além disso, foi observada a mesma dependência espacial e do sinal, referente ao ruído nas fatias de reconstrução, como apresentado em trabalhos anteriores no domínio das projeções.

Este trabalho focou nas propriedades espaciais da variância do ruído, e nenhuma suposição foi feita em relação às propriedades espectrais, como apresentadas na seção \ref{Correlação}. Devido a geometria do procedimento de reconstrução, é criada uma correlação anisotrópica nos dados reconstruídos, e isso deve ser levado em conta para gerar modelos de ruído mais precisos. No entanto esse trabalho limitou-se a análise espacial da variância do ruído, bem como na medida preliminar da correlação do processo de reconstrução.

Por fim, de uma maneira ampla, a disponibilização de códigos em uma plataforma aberta para a comunidade científica com o intuito de difusão do conhecimento e ampliação das pesquisas é de extrema importância e deve ser levado em consideração pelos diversos pesquisadores. 










\chapter[Publicações]{Publicações}\label{Capitulo8}

\section{Congresso Brasileiro de Engenharia Biomédica}

Artigo submetido para o congresso, onde é apresentado a \textit{toolbox} desenvolvida nesse trabalho de modo que seja disponibilizada de maneira aberta para toda comunidade científica. O artigo se encontra disponível no Apêndice \ref{Artigo}.

VIMIEIRO, R. B.; BORGES, L. R.; VIEIRA, M.A.C; \textbf{Open source reconstruction toolbox for digital breast tomosynthesis}. XXVI Brazilian Congress on Biomedical Engineering, Búzios, October 21-25 2018.



% ---
% Finaliza a parte no bookmark do PDF, para que se inicie o bookmark na raiz
% ---
\bookmarksetup{startatroot}% 
% ---

% ---
% Conclusão
% ---

%\chapter*[Conclusão]{Conclusão}
%\addcontentsline{toc}{chapter}{Conclusão}

%\lipsum[31-33]

% ----------------------------------------------------------
% ELEMENTOS PÓS-TEXTUAIS
% ----------------------------------------------------------
\postextual

% ----------------------------------------------------------
% Referências bibliográficas
% ----------------------------------------------------------
\bibliography{bib/referencias}


% ----------------------------------------------------------
% Apêndices
% ----------------------------------------------------------
% ---
% Inicia os apêndices
% ---
\begin{apendicesenv}
% Imprime uma página indicando o início dos apêndices
\partapendices
% ----------------------------------------------------------
% Incluir Apêndice
% ----------------------------------------------------------

\chapter{Campos Aleatórios Markovianos}\label{ApendiceA:CamposAleatóriosMarkovianos}

Nesse capítulo estão contidos os conceitos necessários para o entendimento da teoria de Campo Aleatório Markoviano (\textit{Markov Random Field} - \acs{MRF}) do Capítulo \ref{Capitulo3}. Toda essa notação foi retirada de \citeonline[p. 1-12]{li2009markov} e \citeonline[p. 11-13]{won2013stochastic}.

\section{Sites \& Rótulos}\label{ApendiceA:SitesRotulos}

Considerando $S$ um conjunto que contêm índices de um número $N$ de \textit{\textbf{sites}}. Esse conjunto é descrito da seguinte forma:  

\begin{equation}
	S \, = \, \{1,2,\dots,N\}.
	\label{eq:eqApendiceAConjuntoSites}
\end{equation}  

A palavra \textit{\textbf{site}} tem como significado representar um ponto ou região no espaço euclidiano. No caso específico de uma imagem em \acs{2D} de $NxN$, esses \textit{sites} representam a localização onde a imagem foi amostrada, ou seja, os \textit{\textbf{pixels}}. Nesse caso o conjunto $S$ contém os índices da seguinte forma:

\begin{equation}
	S \, = \, \{(i,j) \mid 1 \leq i,j \leq N\}.
	\label{eq:eqApendiceAConjuntoSites2D}
\end{equation}  

Além disso, a representação dos índices no conjunto $S$ também pode ser feita de maneira não ordenada $S \, = \, \{1,2,...,M\}$, onde $M = NxN$. Essa notação segundo o autor é muito utilizada em modelos de \acs{MRF}. 

Um evento que pode acontecer com um \textit{site} é denominado de rótulo (\textit{\textbf{label}}), e.g., um \textit{pixel} receber um certo valor de nível de cinza. Denota-se $\varGamma$ como um conjunto de rótulos, sendo esse contido por valores discretos de $r_{k}$ rótulos, como descrito a seguir:

\begin{equation}
	\varGamma \, = \, \{r_{1},r_{2},\dots,r_{k}\},
	\label{eq:eqApendiceAConjuntoRotulos}
\end{equation} 

\noindent onde em uma determinada aplicação de imagem, o conjunto de rótulos $\varGamma$ toma valores discretos que representam todos os \textbf{níveis de cinzas} possíveis que foram quantizados, por exemplo $\varGamma \, = \, \{1,2,\dots,254,255\}$.

O conjunto $f = \{f_{1},f_{2},\dots,f_{N}\},$ é denominado como ``rotulagem'' dos \textit{sites} que estão contidos no conjunto $S$, de maneira que cada rótulo do conjunto $f$ está contido no conjunto $\varGamma$ descrito acima.

O produto cartesiano é dito como o conjunto de todos os possíveis rótulos admissíveis pelos \textit{sites}, quando os mesmos tem o conjunto de rótulo $\varGamma$ em comum. Sua equação matemática se dá por:

\begin{equation}
	\varPsi \, = \, \varGamma^{N},
	\label{eq:eqApendiceAProdutoCarteziano}
\end{equation}   

\noindent onde $N$ é o tamanho do conjunto $S$. Segundo o autor, em um problema de restauração de imagem, $\varGamma$ contém todos os valores admissíveis para os \textit{pixels} (\textit{sites}) dentro do conjunto $S$ e $\varPsi$ define todas as possíveis imagens.  



\section{Vizinhança \& Cliques}\label{ApendiceA:sVizinhancaCliques}

De acordo com o autor, os \textit{sites} contidos em $S$ relacionam-se uns com os outros através de um sistema de vizinhança $\nu$:

\begin{equation}
	\nu \, = \, \{\nu_{i} \mid \forall i \in S\},
	\label{eq:eqApendiceAVizinhanca}
\end{equation}  

\noindent onde $\nu_{i}$ é um conjunto de outros \textit{sites} que fazem vizinhança com o \textit{site} $i$.

Um sistema de \textbf{vizinhança} pode ser definido de diversas formas. O de primeira ordem (vizinhança-4) é dado como mostra a Figura \ref{fig:imgApendiceAVizinhancaA}, o de segunda ordem (vizinhança-8) como a Figura \ref{fig:imgApendiceAVizinhancaB}, o de terceira ordem (vizinhança-12) como a Figura \ref{fig:imgApendiceAVizinhancaC} e assim adiante.

O par  $(S,\nu) \, \overset{\Delta}{=} \, G $, constitui um grafo, de tal maneira que $S$ contém os nós e $\nu$ determina as ligações entre os nós de acordo com a vizinhança estipulada. Um \textbf{clique} $c$ para $(S,\nu)$ é definido como um subconjunto de \textit{sites} contido no conjunto $S$, de tal maneira que o clique pode ser considerado como único $c_{1} = \{i\}$, com dois vizinhos $c_{2} = \{i,i^{'}\}$, com três vizinhos $c{3} = \{i,i^{'},i^{''}\}$ e assim sucessivamente. O conjunto de todos os cliques para $(S,\nu)$ é dado por:

\begin{equation}
	C \,=\, C_{1} \,\cup\, C_{2}\, \cup \,C_{3}\,...
	\label{eq:eqApendiceAConjuntoCliques}
\end{equation}  

O tipo de clique para o grafo $(S,\nu)$ é definido pelo seu tamanho, formato e orientação. Para a vizinhança-4 (Figura \ref{fig:imgApendiceAVizinhancaA}), seus cliques respectivos são os: \ref{fig:imgApendiceACliqueA}, \ref{fig:imgApendiceACliqueB} e \ref{fig:imgApendiceACliqueC}; já para a vizinhança-8 (Figura \ref{fig:imgApendiceAVizinhancaB}), seus cliques são os: \ref{fig:imgApendiceACliqueA}, \ref{fig:imgApendiceACliqueB}, \ref{fig:imgApendiceACliqueC}, \ref{fig:imgApendiceACliqueD}, \ref{fig:imgApendiceACliqueE}, \ref{fig:imgApendiceACliqueF}, \ref{fig:imgApendiceACliqueG}, \ref{fig:imgApendiceACliqueH}, \ref{fig:imgApendiceACliqueI} e \ref{fig:imgApendiceACliqueJ}. É possível notar, segundo o autor, que ao aumentar a ordem de vizinhança, aumenta-se também o número de cliques impactando no custo computacional do algoritmo envolvido.  

\begin{figure}[H]
	\centering
	
	\caption{Exemplo de vizinhança (a) 4, (b) 8 e (c) 12.}
	
	\subfloat[]{\includegraphics[scale=0.18]{imgs/ApenA/Vizi1.png}\label{fig:imgApendiceAVizinhancaA}}
	\subfloat[]{\includegraphics[scale=0.2]{imgs/ApenA/Vizi2.png}\label{fig:imgApendiceAVizinhancaB}}
	\subfloat[]{\includegraphics[scale=0.2]{imgs/ApenA/Vizi3.png}\label{fig:imgApendiceAVizinhancaC}}
	
	\legend{Fonte: \citeonline[p. 73]{salvadeo2013filtragem}}
	\label{fig:imgApendiceAVizinhanca}
\end{figure}


\begin{figure}[H]
	\centering
	
	\caption{Cliques possíveis para cada tipo de vizinhança.}
	
	\subfloat[]{\includegraphics[scale=1]{imgs/ApenA/Clique1.pdf}\label{fig:imgApendiceACliqueA}}
	\subfloat[]{\includegraphics[scale=1]{imgs/ApenA/Clique2.pdf}\label{fig:imgApendiceACliqueB}}
	\subfloat[]{\includegraphics[scale=1]{imgs/ApenA/Clique3.pdf}\label{fig:imgApendiceACliqueC}}
	
	
	\subfloat[]{\includegraphics[scale=1]{imgs/ApenA/Clique4.pdf}\label{fig:imgApendiceACliqueD}}
	\subfloat[]{\includegraphics[scale=1]{imgs/ApenA/Clique5.pdf}\label{fig:imgApendiceACliqueE}}
	\subfloat[]{\includegraphics[scale=1]{imgs/ApenA/Clique6.pdf}\label{fig:imgApendiceACliqueF}}
	\subfloat[]{\includegraphics[scale=1]{imgs/ApenA/Clique7.pdf}\label{fig:imgApendiceACliqueG}}
	\subfloat[]{\includegraphics[scale=1]{imgs/ApenA/Clique8.pdf}\label{fig:imgApendiceACliqueH}}
	\subfloat[]{\includegraphics[scale=1]{imgs/ApenA/Clique9.pdf}\label{fig:imgApendiceACliqueI}}
	\subfloat[]{\includegraphics[scale=1]{imgs/ApenA/Clique10.pdf}\label{fig:imgApendiceACliqueJ}}
	
	\legend{Fonte: \citeonline[p. 74]{salvadeo2013filtragem}}
	\label{fig:imgApendiceAClique}
\end{figure}

\section{Campo Aleatório}\label{ApendiceA:CampoAleatorio}

Um campo aleatório é definido por $F$, onde $F = \{F_{1},F_{2},...,F_{N}\}$ é um conjunto de variáveis aleatórias $F_{i}$ que admitem o valor de rótulo $f_{i}$ dentro do conjunto de rótulos $\varGamma$. A probabilidade de uma variável aleatória $F_{i}$ assumir um valor $f_{i}$ é dada por $P(F_{i} = f_{i} )$ ou somente $P(f_{i})$ e a probabilidade conjunta do campo aleatório é dada por:

\begin{equation}
	P(F = f) = P(F_{1}=f_{1},...,F_{N}=f_{N}),
	\label{eq:eqApendiceAProbCampoAleatorio}
\end{equation}  

\noindent ou somente por $P(f)$, dado um conjunto $\varGamma$ de valores discretos.

\chapter{Campos Aleatórios de Gibbs}\label{ApendiceB:CamposAleatóriosdeGibbs}

Nesse capítulo estão os conceitos da teoria de Campo Aleatório de Gibbs (\textit{Gibbs Random Field} - \acs{GRF}) e sua relação com Campo Aleatório Markoviano (\textit{Markov Random Field} - \acs{MRF}), referente ao Capítulo \ref{Capitulo3}. Toda esta notação foi retirada de \citeonline[p. 13-15]{li2009markov} e \citeonline[p. 14-21]{won2013stochastic}.

\section{Campo Aleatório de Gibbs}\label{ApendiceB:CampoAleatóriodeGibbs}

Dado um conjunto de variáveis aleatórias $F$, esse é dito ser um \acs{GRF} em $S$ com relação a vizinhança $\nu$ se o mesmo obedecer a distribuição de Gibbs dada a seguir:

\begin{equation}
	P(f) = \frac{1}{Z}\; e^{-U(f)},
	\label{eq:eqApendiceBDistribuicaoGibbs}
\end{equation}

\noindent onde $Z$ é uma constante de normalização denominada função de partição, dada por \eqref{eq:eqApendiceBDistribuicaoGibbsZ} e $U(f)$ é chamada de função de energia, dada por \eqref{eq:eqApendiceBDistribuicaoGibbsU1} que representa a soma dos potenciais dos cliques $V_{c}(f)$ em todos os possíveis cliques $C$. 

\begin{equation}
	Z = \sum_{f \,\in\, \varPsi}^{} \, e^{-U(f)}. 
	\label{eq:eqApendiceBDistribuicaoGibbsZ}
\end{equation}

\begin{equation}
	U(f) = \sum_{c \,\in\, C}^{} \, V_{c}(f).
	\label{eq:eqApendiceBDistribuicaoGibbsU1}
\end{equation}

A função de energia também pode ser expressa através do somatório de termos independentes, de acordo com a ordem de seu clique:

\begin{equation}
	U(f) = \sum_{\{i\} \,\in\, C_{1}}^{} \, V_{1}(f_{i}) \,+\, \sum_{\{i,i^{'}\} \,\in\, C_{2}}^{} \, V_{2}(f_{i},f_{i^{'}}) \,+\, \sum_{\{i,i^{'},i^{''}\} \,\in\, C_{3}}^{} \, V_{3}(f_{i},f_{i^{'}},f_{i^{''}}),
	\label{eq:eqApendiceBDistribuicaoGibbsU2}
\end{equation}  

\noindent ou considerando somente cliques de ordem um e dois:

\begin{equation}
	U(f) = \sum_{i \,\in\, S}^{} \, V_{1}(f_{i}) \,+\, \sum_{i \,\in\, S}^{} \, \sum_{i^{'} \,\in\, \nu_{i} }^{} \, V_{2}(f_{i},f_{i^{'}}).
	\label{eq:eqApendiceBDistribuicaoGibbsU3}
\end{equation} 

A prova da de que o \acs{GRF} é um \acs{MRF} e vice-versa, pode ser encontrada em \citeonline[p. 19]{won2013stochastic}.


\chapter{Imagens Resultados}\label{ApendiceC:ImagensResultados}

Esse apêndice contém as imagens obtidas como resultado desse trabalho. Devido a organização do espaço as mesmas foram inseridas nessa seção.

\begin{figure}[htb]
	\centering
	
	\caption{Parte 1 dos resultados obtidos após a aplicação dos métodos \acs{BP}, \acs{FBP} e \acs{MLEM}. Da coluna 1 a 5 são representados respectivamente: \textit{Phantom}, \acs{BP}, \acs{FBP} e \acs{MLEM} com 10 iterações. Capítulo \ref{Capitulo5}.}
	
	\subfloat{\includegraphics[scale=.85]{imgs/cap5/Original/1.png}}
	\hfil
	\subfloat{\includegraphics[scale=.85]{imgs/cap5/ReconBP/1.png}}
	\hfil
	\subfloat{\includegraphics[scale=.85]{imgs/cap5/ReconFBP/1.png}}
	\hfil
	\subfloat{\includegraphics[scale=.85]{imgs/cap5/ReconMLEM/5/1.png}}
	
	
	\subfloat{\includegraphics[scale=.85]{imgs/cap5/Original/15.png}}
	\hfil
	\subfloat{\includegraphics[scale=.85]{imgs/cap5/ReconBP/15.png}}
	\hfil
	\subfloat{\includegraphics[scale=.85]{imgs/cap5/ReconFBP/15.png}}
	\hfil
	\subfloat{\includegraphics[scale=.85]{imgs/cap5/ReconMLEM/5/15.png}}
	
	
	\subfloat{\includegraphics[scale=.85]{imgs/cap5/Original/30.png}}
	\hfil
	\subfloat{\includegraphics[scale=.85]{imgs/cap5/ReconBP/30.png}}
	\hfil
	\subfloat{\includegraphics[scale=.85]{imgs/cap5/ReconFBP/30.png}}
	\hfil
	\subfloat{\includegraphics[scale=.85]{imgs/cap5/ReconMLEM/5/30.png}}
	
	
	\subfloat{\includegraphics[scale=.85]{imgs/cap5/Original/43.png}}
	\hfil
	\subfloat{\includegraphics[scale=.85]{imgs/cap5/ReconBP/43.png}}
	\hfil
	\subfloat{\includegraphics[scale=.85]{imgs/cap5/ReconFBP/43.png}}
	\hfil
	\subfloat{\includegraphics[scale=.85]{imgs/cap5/ReconMLEM/5/43.png}}
	
	
	\subfloat{\includegraphics[scale=.85]{imgs/cap5/Original/64.png}}
	\hfil
	\subfloat{\includegraphics[scale=.85]{imgs/cap5/ReconBP/64.png}}
	\hfil
	\subfloat{\includegraphics[scale=.85]{imgs/cap5/ReconFBP/64.png}}
	\hfil
	\subfloat{\includegraphics[scale=.85]{imgs/cap5/ReconMLEM/5/64.png}}
	
	\label{fig:imgCap5Resultados1}
\end{figure}
\begin{figure}[!t] \ContinuedFloat
	\centering
	
	\caption{Parte 2 dos resultados obtidos após a aplicação dos métodos \acs{BP}, \acs{FBP} e \acs{MLEM}. Da coluna 1 a 5 são representados respectivamente: \textit{Phantom}, \acs{BP}, \acs{FBP} e \acs{MLEM} com 10 iterações. Capítulo \ref{Capitulo5}.}
	
	\subfloat{\includegraphics[scale=.85]{imgs/cap5/Original/78.png}}
	\hfil
	\subfloat{\includegraphics[scale=.85]{imgs/cap5/ReconBP/78.png}}
	\hfil
	\subfloat{\includegraphics[scale=.85]{imgs/cap5/ReconFBP/78.png}}
	\hfil
	\subfloat{\includegraphics[scale=.85]{imgs/cap5/ReconMLEM/5/78.png}}
	
	
	\subfloat{\includegraphics[scale=.85]{imgs/cap5/Original/85.png}}
	\hfil
	\subfloat{\includegraphics[scale=.85]{imgs/cap5/ReconBP/85.png}}
	\hfil
	\subfloat{\includegraphics[scale=.85]{imgs/cap5/ReconFBP/85.png}}
	\hfil
	\subfloat{\includegraphics[scale=.85]{imgs/cap5/ReconMLEM/5/85.png}}
	
	
	\subfloat{\includegraphics[scale=.85]{imgs/cap5/Original/99.png}}
	\hfil
	\subfloat{\includegraphics[scale=.85]{imgs/cap5/ReconBP/99.png}}
	\hfil
	\subfloat{\includegraphics[scale=.85]{imgs/cap5/ReconFBP/99.png}}
	\hfil
	\subfloat{\includegraphics[scale=.85]{imgs/cap5/ReconMLEM/5/99.png}}
	
	
	\subfloat{\includegraphics[scale=.85]{imgs/cap5/Original/113.png}}
	\hfil
	\subfloat{\includegraphics[scale=.85]{imgs/cap5/ReconBP/113.png}}
	\hfil
	\subfloat{\includegraphics[scale=.85]{imgs/cap5/ReconFBP/113.png}}
	\hfil
	\subfloat{\includegraphics[scale=.85]{imgs/cap5/ReconMLEM/5/113.png}}
	
	
	\subfloat{\includegraphics[scale=.85]{imgs/cap5/Original/128.png}}
	\hfil
	\subfloat{\includegraphics[scale=.85]{imgs/cap5/ReconBP/128.png}}
	\hfil
	\subfloat{\includegraphics[scale=.85]{imgs/cap5/ReconFBP/128.png}}
	\hfil
	\subfloat{\includegraphics[scale=.85]{imgs/cap5/ReconMLEM/5/128.png}}
	
	\legend{Fonte: do autor, 2018.}
	\label{fig:imgCap5Resultados2}
\end{figure}


\begin{figure}[!t]
	\centering
	
	\caption{Resultados obtidos após a aplicação do método \acs{MLEM} com os números de iteração: 5, 10, 15 e 20 respectivamente ao número das colunas. Capítulo \ref{Capitulo5}.}
	
	\subfloat{\includegraphics[scale=.85]{imgs/cap5/ReconMLEM/5/43.png}}
	\hfil
	\subfloat{\includegraphics[scale=.85]{imgs/cap5/ReconMLEM/10/43.png}}
	\hfil
	\subfloat{\includegraphics[scale=.85]{imgs/cap5/ReconMLEM/15/43.png}}
	\hfil
	\subfloat{\includegraphics[scale=.85]{imgs/cap5/ReconMLEM/20/43.png}}
	
	
	\subfloat{\includegraphics[scale=.85]{imgs/cap5/ReconMLEM/5/64.png}}
	\hfil
	\subfloat{\includegraphics[scale=.85]{imgs/cap5/ReconMLEM/10/64.png}}
	\hfil
	\subfloat{\includegraphics[scale=.85]{imgs/cap5/ReconMLEM/15/64.png}}
	\hfil
	\subfloat{\includegraphics[scale=.85]{imgs/cap5/ReconMLEM/20/64.png}}
	
	
	\subfloat{\includegraphics[scale=.85]{imgs/cap5/ReconMLEM/5/78.png}}
	\hfil
	\subfloat{\includegraphics[scale=.85]{imgs/cap5/ReconMLEM/10/78.png}}
	\hfil
	\subfloat{\includegraphics[scale=.85]{imgs/cap5/ReconMLEM/15/78.png}}
	\hfil
	\subfloat{\includegraphics[scale=.85]{imgs/cap5/ReconMLEM/20/78.png}}
	
	
	\subfloat{\includegraphics[scale=.85]{imgs/cap5/ReconMLEM/5/85.png}}
	\hfil
	\subfloat{\includegraphics[scale=.85]{imgs/cap5/ReconMLEM/10/85.png}}
	\hfil
	\subfloat{\includegraphics[scale=.85]{imgs/cap5/ReconMLEM/15/85.png}}
	\hfil
	\subfloat{\includegraphics[scale=.85]{imgs/cap5/ReconMLEM/20/85.png}}
	
	\legend{Fonte: do autor, 2018.}
	\label{fig:imgCap5Resultados3}
\end{figure}

 \begin{figure}[t!]
 	\caption{Gráfico de \acs{SSIM} médio para todos os métodos de reconstrução utilizados. Capítulo \ref{Capitulo5}.}
	\begin{center}
		\includegraphics[scale=0.8]{imgs/cap5/WN/MSSIM.pdf}
	\end{center}
	\legend{Fonte: do autor, 2018.}
	\label{fig:imgCap5GraficoMSSIM}
\end{figure} 

\begin{figure}[t!]
	\caption{Gráfico da raiz do \acs{MSE} normalizado para todos os métodos de reconstrução utilizados. Capítulo \ref{Capitulo5}.}
	\begin{center}
		\includegraphics[scale=0.8]{imgs/cap5/WN/NRMSE.pdf}
	\end{center}
	\legend{Fonte: do autor, 2018.}
	\label{fig:imgCap5GraficoNRMSE}
\end{figure} 

\begin{figure}[t!]
	\caption{Gráfico de \textit{Sharpness} em escala de dB para todos os métodos de reconstrução utilizados e para o \textit{Phantom} de referência. Capítulo \ref{Capitulo5}.}
	\begin{center}
		\includegraphics[scale=0.8]{imgs/cap5/WN/SHARPDB.pdf}
	\end{center}
	\legend{Fonte: do autor, 2018.}
	\label{fig:imgCap5GraficoSHARPDB}
\end{figure}

\chapter{Artigo}\label{Artigo}

\includepdf[pages=-]{docs/artigo-cbeb-2018.pdf}



\end{apendicesenv}
% ---

% ----------------------------------------------------------
% Anexos
% ----------------------------------------------------------
% ---
% Inicia os anexos
% ---
%\begin{anexosenv}
% Imprime uma página indicando o início dos anexos
%\partanexos
% ----------------------------------------------------------
% Incluir Anexo
% ----------------------------------------------------------


%\end{anexosenv}


\end{document}


