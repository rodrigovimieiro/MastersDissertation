\chapter[Mamografia e Tomossíntese]{Mamografia e Tomossíntese}\label{Capitulo2}

%%%%%%%%%%%%%%%%%%%%%%%%%%%%%%%%%%%%%%%%%%%%%%%%%%%%%%%%%%%%%%%%%%%%%%%%%%%%%%%%%%%%%%%%%%%%%%%%%%%%%%%%%%%%%%%%%%%%%%%%%%%%%%											Câncer de Mama e Mamografia														%
%%%%%%%%%%%%%%%%%%%%%%%%%%%%%%%%%%%%%%%%%%%%%%%%%%%%%%%%%%%%%%%%%%%%%%%%%%%%%%%%%%%%%%%%%%%%%%%%%%%%%%%%%%%%%%%%%%%%%%%%%%%%%
\section{Câncer de Mama e Mamografia}

Segundo o \citeonline{inca}, o câncer de mama é o mais frequente em mulheres no mundo todo e também no Brasil, salvo os tipos de pele não melanoma. Ainda, conforme as estatísticas do instituto, para os anos de 2018 e 2019 a previsão é de 60 mil novos casos, representando para as mulheres 21\% do total de casos previstos para estas. 

É estimado que, 28\% dos casos podem ser evitados com práticas de alimentação saudável, atividade física  e adequação do peso corporal. Essas precauções podem ser tomadas a fim de diminuir os fatores de risco da doença que são: o excesso de peso corporal, particularmente após a menopausa, o consumo de bebidas alcoólicas, a terapia de reposição hormonal e a exposição à radiação ionizante \cite{inca}.

O Ministério da Saúde, em conformidade com a \ac{OMS}, recomenda que a mamografia de rastreamento seja feita periódicamente por mulheres na faixa etária entre 50 e 69 anos. Esse exame auxilia na descoberta precoce do câncer, proporcionando um começo de tratamento rápido, diminuído assim as chances de óbito. Estima-se que o uso desse equipamento para o rastreamento reduz em aproximadamente 20\% a taxa de mortalidade da doença \cite{oms}, porém, a mamografia quando aplicada a toda uma população está relacionada a riscos como: resultados incorretos e a exposição aos raios X que podem induzir o câncer na população a longo prazo \cite{yaffe2011risk,inca}. 

%As seções seguintes têm como objetivo detalhar o funcionamento dos equipamentos usados atualmente para rastreamento tanto no Brasil quanto no resto do mundo. Esse capítulo é utilizado como base para o entendimento da evolução dos equipamentos e também para a compreensão dos métodos de formação das imagens detalhadas nas próximas seções.    

%%%%%%%%%%%%%%%%%%%%%%%%%%%%%%%%%%%%%%%%%%%%%%%%%%%%%%%%%%%%%%%%%%%%%%%%%%%%%%%%%%%%%%%%%%%%%%%%%%%%%%%%%%%%%%%%%%%%%%%%%%%%%%													Mamografia																%
%%%%%%%%%%%%%%%%%%%%%%%%%%%%%%%%%%%%%%%%%%%%%%%%%%%%%%%%%%%%%%%%%%%%%%%%%%%%%%%%%%%%%%%%%%%%%%%%%%%%%%%%%%%%%%%%%%%%%%%%%%%%%
\subsection{Mamografia}

O exame de mamografia tem como finalidade proporcionar uma imagem radiográfica da mama, com o intuito de que os profissionais da saúde possam identificar possíveis lesões em meio aos tecidos e estruturas que compõe a mesma (Figura \ref{fig:imgCap2EstruturaMama}), antes mesmo de qualquer sintoma. 

A mama feminina, em sua anatomia é constituída por: lobos, que são glândulas produtoras de leite; ductos, que escoam o leite materno para o mamilo; tecido adiposo e conjuntivo, que revestem os ductos e lobos, dentre outras estruturas. O câncer se caracteriza quando células na mama começam a crescer de maneira descontrolada formando tumores malignos que se espalham por tecidos próximos ou para outras localidades do corpo. Esses tumores podem começar em diversas partes da mama, no entanto, na maioria dos casos, os tumores começam nos ductos ou nas glândulas \cite{americancancersociety2017}.      

\begin{figure}[H]
	\caption{Tecidos e estruturas que compõem uma mama normal.}
	\begin{center}
		\includegraphics[scale=0.7]{imgs/cap2/EstruturaMama.png}
	\end{center}
	\legend{Fonte: Adaptado de \citeonline{americancancersociety2017}.}
	\label{fig:imgCap2EstruturaMama}
\end{figure}


Com o auxílio do exame de mamografia, o câncer de mama pode ser detectado através de quatro indicadores, segundo \citeonline{boone2000handbook}:

\begin{enumerate}
	\item morfologia característica de uma massa tumoral,
	\item presença de depósitos minerais em forma de partículas, denominadas microcalcificações,
	\item distorção arquitetural dos padrões teciduais normais,
	\item assimetria entre regiões correspondentes da imagem esquerda e direita.
\end{enumerate}

A Figura \ref{fig:imgCap2ExameMamografia} ilustra dois exames de mamografia da mesma mama em orientação \ac{CC} e \ac{MLO}, onde são evidenciados os sinais de um possível câncer de mama.

\begin{figure}[H]
	\centering
	
	\caption{Ilustração de dois exames de mamografia \acs{2D} da mama direita na orientação (a) \acs{CC} e (b) \acs{MLO}, na qual as setas em vermelho evidenciam agrupamentos de microcalcificações e nódulos .}
	
	\subfloat[]{\includegraphics[scale=0.5, clip, trim=13cm 0cm 12.2cm 0cm]{imgs/cap2/FFDM_Cli_CC} \label{fig:imgCap2ExameMamografiaCC}}	
	\hfil
	\subfloat[]{\includegraphics[scale=0.5, clip, trim=12.3cm 0cm 12.2cm 0cm]{imgs/cap2/FFDM_Cli_MLO}\label{fig:imgCap2ExameMamografiaMLO}}
	
	\legend{Fonte: do autor, 2019.}
	\label{fig:imgCap2ExameMamografia}
\end{figure}

Os primeiros registros da utilização de radiografia aplicada à mama foram feitos em 1913 com Albert Salomon. Ele demonstrou o espalhamento de tumores através de imagens radiográficas de mamas extraídas, tento suas pesquisas interrompidas em 1933 devido a segunda guerra mundial. Já em 1930, Stafford L. Warren registrou o uso  \textit{in vivo} da mamografia em 119 pacientes. O autor ressaltou em seu trabalho a exatidão das opiniões baseadas em diagnósticos através do uso da mamografia, em contrapartida com opiniões pré-operatórias sem a utilização do exame. Apesar dos registros de seu uso desde o começo do século XX, a mamografia como equipamento dedicado para o rastreamento, não desapontou até os anos de 1960, quando em 1965, Charles Gros em parceria com a empresa Compagnie Générale de Radiologie desenvolveu, na França, o primeiro equipamento restrito ao exame mamografia, ilustrado pelo Figura \ref{fig:imgCap2MamografiaPrimeiroEq} \cite{bassett1988evolution,gold1990highlights}. 

\begin{figure}[H]
	\caption{Primeiro equipamento dedicado à mamografia.}
	\begin{center}
		\includegraphics[scale=0.95]{imgs/cap2/MamografiaPrimeiroEq}
	\end{center}
	\legend{Fonte: \citeonline[p. 12]{gold1990highlights}.}
	\label{fig:imgCap2MamografiaPrimeiroEq}
\end{figure}

A mamografia por filme fotográfico foi amplamente utilizada por inúmeros anos e considerada por muito tempo um equipamento preconizado para o rastreamento do câncer de mama, porém foi sendo substituída devido as suas limitações técnicas como: a curva característica de contraste limitada, a presença de granularidade no filme, a impossibilidade de pós processamento e a difícil detecção de lesões em tecidos moles na presença de tecido glandular denso \cite{karellas2008breast,lewin2001comparison}. 

Para superar essas limitações existentes, foi desenvolvida a mamografia digital de campo total (\textit{Full Field Digital Mammography} - \acs{FFDM})  \cite{nishikawa1987scanned,yaffe1988development}. Esse equipamento possibilita a captura de uma imagem digital através de múltiplos sensores, além de poder armazenar, transmitir e mostrar os exames através da tela de um computador. Com essa evolução houve uma maior precisão em diagnósticos para mulheres na faixa etária de 49 anos ou menos e para outras com mamas densas \cite{vedantham2015digital}. A Figura \ref{fig:imgCap2EsquematicoMamografia} ilustra o esquemático de um mamógrafo e detalha suas partes específicas. 

\begin{figure}[H]
	\caption{Esquemático geral de um equipamento de mamografia atual.}
	\begin{center}
		\includegraphics[scale=0.8]{imgs/cap2/FFDMGeometry}
	\end{center}
	\legend{Fonte: \citeonline[]{guerrero2018}.}
	\label{fig:imgCap2EsquematicoMamografia}
\end{figure}

Apesar de sua grande evolução e ampla utilização, o equipamento de mamografia digital impõe limitações físicas para o diagnóstico dos exames, podendo ser destacada a sobreposição de tecidos \cite{vedantham2015digital}. A mama por si só é constituída de uma estrutura volumétrica em \acs{3D}, mas o equipamento proporciona uma imagem bidimensional, como ilustrado pela Figura \ref{fig:imgCap2ExameMamografia}. Por esse motivo, há a sobreposição de estruturas (Figura \ref{fig:imgCap2MamografiaSobreposicao}), fazendo com que tecidos normais (quadrado vermelho) venham a obscurecer lesões malignas (círculo verde), levando o exame à uma menor sensibilidade na detecção de estruturas que podem ser um indicativo de câncer, altas taxas de falsos-positivos e por fim uma alta taxa de \textit{recall} \footnote{Palavra da língua inglesa que tem como significado, no contexto, de chamar a paciente para realizar um novo exame devido a alguma suspeita ao realizar o diagnóstico.} \cite{roth2014digital}. 

Com o propósito de contornar esses problemas foi desenvolvido o equipamento de \acs{DBT} que é abordado com mais detalhes na próxima seção. 

\begin{figure}[H]
	\caption{Ilustração da sobreposição de tecidos.}
	\begin{center}
		\includegraphics[scale=0.8, clip, trim=13cm 1.5cm 10cm 2.5cm]{imgs/cap2/FFDM}
	\end{center}
	\legend{Fonte: do autor, 2019.}
	\label{fig:imgCap2MamografiaSobreposicao}
\end{figure}    

%%%%%%%%%%%%%%%%%%%%%%%%%%%%%%%%%%%%%%%%%%%%%%%%%%%%%%%%%%%%%%%%%%%%%%%%%%%%%%%%%%%%%%%%%%%%%%%%%%%%%%%%%%%%%%%%%%%%%%%%%%%%%%											Tomossíntese Digital Mamária													%
%%%%%%%%%%%%%%%%%%%%%%%%%%%%%%%%%%%%%%%%%%%%%%%%%%%%%%%%%%%%%%%%%%%%%%%%%%%%%%%%%%%%%%%%%%%%%%%%%%%%%%%%%%%%%%%%%%%%%%%%%%%%%
\section{Tomossíntese Digital Mamária}     

\subsection{Histórico e Atualidade}

Uma radiografia analógica ou até mesmo a digital, é a representação de um corpo anatômico em \acs{3D} por uma simples projeção \acs{2D}. Ela traz consigo a sobreposição de órgãos, tecidos e outras formações do corpo humano, tornando impossível a localização precisa das estruturas internas \cite{levakhina2014three}. 

Essa preocupação era eminente logo após a descoberta dos raios X por Wilhelm Röntgen, em 1895, onde pesquisadores da época já buscavam soluções para a reconstrução volumétrica dos objetos. O conceito de impressão tridimensional de uma cena, todavia, foi introduzido pelo princípio da estereoscopia por Charles Wheatstone, em 1838, muito antes do descobrimento do raio X. Já no começo do século XX, inúmeros cientistas ao redor do mundo relataram trabalhos e pesquisas com o intuito da reconstrução de estruturas internas através dos raios X \cite{dobbins2003digital,levakhina2014three}. 

\begin{figure}[H]
	\caption{Princípio da Tomografia Convencional.}
	\begin{center}
		\includegraphics[scale=0.8, clip, trim=12cm 6.2cm 12cm 1.8cm]{imgs/cap2/TomographyPriciple}
	\end{center}
	\legend{Fonte: do autor, 2019.}
	\label{fig:imgCap2TomographyPriciple}
\end{figure}

No ano de 1921, A. E. M. Bocage descreveu o primeiro sistema tomográfico convencional, onde o tubo de raios X e o filme moviam-se linearmente em direções opostas ao longo do paciente (Figura \ref{fig:imgCap2TomographyPriciple}) com o propósito de gerar um plano em foco do objeto e desfocar as estruturas fora desse plano \cite{hsieh2009computed}. Contudo, um dos grandes problemas relacionados a esta descoberta, é o fato de ser preciso a realização de diversos exames para a obtenção de diferentes fatias do corpo. 

Já no ano de 1932, o pesquisador holandês B. G. Ziedses des Plantes publicou um trabalho \cite{des1932neue} que descrevia a possibilidade da formação de diversos planos do objeto a partir de um número definido de projeções do mesmo. As implementações mais marcantes da teoria de Ziedses des Plantes foram feitas por \citeonline []{garrison1969three} e mais tarde por \citeonline []{miller1971infinite}. Os autores demonstraram o princípio da tomografia convencional discreta por meio da utilização de luz, lentes e espelhos para a reconstrução de um número arbitrário de planos tomográficos do objeto examinado. O termo tomossíntese, entretanto, só ficou conhecido um ano mais tarde com o trabalho de \citeonline []{grant1972tomosynthesis}. Uma descrição mais detalhada sobre a história do desenvolvimento da tomossíntese pode ser encontrada em  \citeonline []{dobbins2003digital,levakhina2014three} e \citeonline []{goodsitt2014history}. 

Embora a descoberta da técnica de tomossíntese tenha sido feita no início do século passado, seu real interesse e sua ampla utilização para exames da mama aconteceram recentemente com o trabalho de  \citeonline []{niklason1997digital} no \textit{Massachusetts General Hospital} na década de 90. Isso se deve ao fato da sua difícil implementação e da escassez de recursos tecnológicos naquela época. 

O avanço da computação, a introdução de detectores digitais de tela plana e a possibilidade da sua rápida leitura foram peças essenciais para a volta da tomossíntese ao foco das pesquisas junto a sua vasta utilização, principalmente no âmbito do rastreamento do câncer de mama \cite{Niklason20185}. 

Hoje essa técnica está se espalhando rapidamente e substituindo os equipamento de mamografia digital, de acordo com \citeonline [p. 6]{Niklason20185}. O gráfico da Figura \ref{fig:imgCap2TimelineDBT} ilustra o número de publicações científicas nos últimos 10 anos com o termo ``\textit{Digital Breast Tomosynthesis}'' na base de dados do site \textit{PubMed}\footnote{\url{www.ncbi.nlm.nih.gov/pubmed/}}, demonstrando um grande aumento de interesse na área ao longo dos anos.

No ano de 2011, o equipamento da empresa \textit{Hologic}\footnote{\url{www.hologic.com/}} foi aprovado pelo órgão regulamentador dos Estados Unidos (\textit{Food and Drug Administration} - \acs{FDA})\footnote{\url{www.fda.gov}} para ser comercializado e utilizado na prática clínica.  
    
 
\begin{figure}[H]
	\caption{Número de publicações científicas nos últimos 10 anos com o termo ``\textit{Digital Breast Tomosynthesis}''.}
	\begin{center}
		\includegraphics[scale=0.6, clip, trim=1.7cm 9.4cm 1.9cm 10.8cm]{imgs/cap2/TimelineDBT}
	\end{center}
	\legend{Fonte: PubMed, 2018.}
	\label{fig:imgCap2TimelineDBT}
\end{figure}

A \acs{DBT} é caracterizada por ser uma técnica tomográfica de ângulo limitado. Nesse exame, múltiplas radiografias, denominadas de projeções, são adquiridas em diferentes ângulos, demonstrado na Figura \ref{fig:imgCap2DBTEstrutura}, enquanto o tubo se move em uma trajetória fixa pré-definida, amenizando assim o problema de sobreposição de tecidos existente no equipamento de \acs{FFDM}. Ao final do conjunto de exposições, as projeções são processadas por um algoritmo e, por fim, é reconstruído o volume pseudo-\acs{3D} da mama, como ilustra a Figura \ref{fig:imgCap2ExameDBTRecon} \cite{vedantham2015digital,michell2018role}. Essas fatias são então apresentadas ao radiologista para serem laudadas, como demonstra a Figura \ref{fig:imgCap2ExameDBT}. 

Resumidamente, a técnica de \acs{DBT} é muito semelhante a mamografia digital comum,  diferenciando-se somente na rotação do tubo de raios X e na utilização de um algoritmo para a reconstrução do volume da mama \cite{michell2018role}.

\begin{figure}[H]
	\caption{Geometria básica de aquisição de um equipamento de \acs{DBT}.}
	\begin{center}
		\includegraphics[scale=0.9, clip, trim=11cm 1.5cm 9.5cm 2.5cm]{imgs/cap2/DBT}
	\end{center}
	\legend{Fonte: do autor, 2019.}
	\label{fig:imgCap2DBTEstrutura}
\end{figure}


\begin{figure}[H]
	\caption{Esquemático geral do procedimento de reconstrução do volume \acs{3D} da mama a partir das projeções de raios X.}
	\begin{center}
		\includegraphics[scale=0.5, clip, trim=3.5cm 1.2cm 8.4cm 1.9cm]{imgs/cap2/Recon}
	\end{center}
	\legend{Fonte: do autor, 2019.}
	\label{fig:imgCap2ExameDBTRecon}
\end{figure}

\begin{figure}[H]
	\caption{Fatias do volume \acs{3D} reconstruídas referente a mama apresentada na Figura \ref{fig:imgCap2ExameMamografia} na sua versão \acs{2D}.}
	\begin{center}
		\includegraphics[scale=0.5, clip, trim=10cm 1cm 10cm 0cm]{imgs/cap2/DBT_Cli}
	\end{center}
	\legend{Fonte: do autor, 2019.}
	\label{fig:imgCap2ExameDBT}
\end{figure}
%são formadas as imagens de recortes da mama paralelas ao detector. 

As variações no processo de aquisição das imagens dessa técnica ocorrem de acordo com cada fabricante, contudo, de uma maneira geral e simplificada, a Figura \ref{fig:imgCap2DBTEstrutura} exemplifica a geometria de aquisição de um equipamento de \acs{DBT}. Nessa figura, o tubo de raios X se move na trajetória de um arco emitindo radiação nas posições de A até C, com angulação de $\theta_{1}$ até $\theta_{3}$ respectivamente. Cada exposição gera uma projeção dos objetos no detector plano, representada na parte inferior da figura.

\subsection{Parâmetros Físicos e Geométricos}\label{ParâmetrosFísicoseGeométricos}

Em geral, os equipamentos de tomossíntese possuem sua estrutura física muito semelhante aos de  mamografia digital. Durante o exame clínico a paciente mantêm-se posicionada de pé, junto ao tubo e ao detector posicionados em uma orientação \ac{CC} ou \ac{MLO} \cite{Niklason20185}. Então, o prato de compressão comprime a mama para realização do exame, com o propósito de deixar os tecidos o mais distribuído possível e reduzir o movimento da paciente durante o exame. Após esse procedimento são realizadas as exposições de raios X em uma determinada faixa de ângulo com um tipo de movimento do tubo preestabelecido, como ilustra a Figura \ref{fig:imgCap2DBTEstrutura1} \cite{baker2011breast}.  

\begin{figure}[H]
	\caption{Geometria de um equipamento de \acs{DBT}.}
	\begin{center}
		\includegraphics[scale=0.75, clip, trim=13cm 4cm 9.5cm 5cm]{imgs/cap2/DBT1}
	\end{center}
	\legend{Fonte: do autor, 2019.}
	\label{fig:imgCap2DBTEstrutura1}
\end{figure}

A movimentação pode ser dada de duas maneiras: \textit{Step-and-shoot} ou contínuo (\textit{continuous tube motion}). No segundo modo, o tubo se move de maneira ininterrupta disparando a radiação em tempos espaçados igualmente ao longo de todo período. Esse método gera uma redução no tempo do exame e na movimentação do paciente, em contrapartida, há um aumento no borramento da imagem devido ao movimento do ponto focal durante os disparos \cite{glick2014system}. 

Já no modo \textit{Step-and-shoot}, o tubo para o movimento totalmente antes de cada exposição e logo em seguida retorna ao seu movimento. Esse processo elimina quase totalmente o borramento relacionado à movimentação, porém ainda existente em pequenas vibrações do tubo devido a sua inércia e seu peso \cite{glick2014system}.

No que se refere à rotação dos detectores, alguns fabricantes optam por sua movimentação ou não. Mantê-lo estacionário reduz sua complexidade mecânica e consequentemente o borramento das imagens. Já sua angulação possibilita a confecção de detectores menores e também uma redução no borramento dos raios X, que são incididos com uma maior angulação nos detectores fixos \cite{glick2014system}. 

Para a angulação do tubo, os equipamentos comercias variam entre 15$\degree$ a 50$\degree$, com um número de 9 a 25 exposições, porém de acordo com \citeonline [p. 25]{glick2014system} não existe ao certo um valor ótimo para um número de projeção nem para a faixa de ângulo. Um estudo de \citeonline []{sechopoulos2009optimization} com 63 combinações de ângulos e projeções, concluiu que os parâmetros que obtiveram melhores resultados no experimento foram com uma faixa total de ângulo de 60$\degree$ com 13 projeções. Ainda segundo os autores, com o intuito de aumentar a resolução na direção Z é necessário maximizar a extensão do ângulo de aquisição, entretanto o aumento no número de projeções não está relacionado com essa melhoria. Segundo \citeonline []{hu2008image}, a diminuição do ângulo além de causar a redução de resolução em Z, produz também um borramento entre estruturas em profundidades diferentes. Todavia \citeonline [p. 7]{Niklason20185} afirma que a utilização de ângulos menores está associada a uma melhor visualização de calcificações na mama. 

Devido a substituição da mamografia digital pela tomossíntese, a preocupação com a manutenção da dose de radiação é eminente. Conforme \citeonline [p. 9]{Hooley20189} a dose cumulativa é aproximadamente a mesma para ambos os exames ou um pouco maior para a \acs{DBT}. Ainda de acordo com os autores, a dose de radiação da combinação de ambos exames é próxima a $2.65 mGy$ (\textit{miligray})\footnote{Quantidade de energia de radiação absorvida em um quilograma de matéria \ref{Hooley20189}.}, onde $1.2 mGy$ e $1.45 mGy$ são oriundos da mamografia e tomossíntese respectivamente. É importante ressaltar que esses valores estão abaixo dos limites estabelecidos pela norma norte-americana \ac{MQSA} e que o valor da dose é dependente do fabricante e do seu respectivo modelo. 

O número de projeções está intimamente ligado com a dose total e com os artefatos na imagem reconstruída. Ainda, em conformidade com \citeonline [p. 7]{Niklason20185}, um baixo número de projeções está associado com a geração de artefatos em planos fora de foco decorrentes de objetos com alta atenuação, como as microcalcificações. Em contrapartida, um elevado número de projeções está limitado ao valor da dose total de radiação associado ao exame. 

Dado uma dose total de $X\,mGy$ com um número $N$ de projeções, a dose individual $X_{i}$ de cada projeção é dada por: $X_{i} = \frac{X}{N}$, ou seja, a dose total é dividida igualmente para cada projeção. Quanto menor a dose por imagem $X_{i}$, menor será sua relação sinal-ruído (\textit{Signal to Noise Ratio} - \acs{SNR}), devido ao ruído ser modelado através de uma distribuição \textit{Poisson}, como detalhado no \autoref{CapituloRuido}. Sendo assim, a \acs{SNR} de cada projeção é inversamente proporcional ao número total de projeções \cite{sechopoulos2009optimization}. Isso resulta ao final em uma degradação na qualidade da imagem reconstruída caso o número de exposições seja elevado.  No entanto, existe um compromisso entre a dose de radiação e a qualidade da imagem exposta ao radiologista, de modo que, do ponto de vista da dosagem de radiação deseja-se minimizar o número de projeções, mas para a qualidade da imagem reconstruída ser boa é preciso aumentar o número de projeções.

Outra consideração a ser feita para os equipamentos de \acs{DBT} é a sua geometria de emissão dos feixes. Diferente de \acs{CT}, a emissão dos feixes de raios X é feita através de uma geometria de meio cone (\textit{half cone-beam}). Esse fato deve ser levado em consideração em ambos procedimentos de projeção e retroprojeção nos algoritmos de reconstrução \cite{wu2004comparison}. 
    
Tomando como base os sistemas atuais, eles se diferem em diversos aspectos como: a geometria física, a angulação do tubo, a movimentação do detector, o número de projeções, o tamanho do \textit{pixel}, o método de aquisição direto ou indireto, os algoritmos de reconstrução, o tempo do exame etc. 

A Tabela \ref{tab:tabCap2SistemasDBT} mostra a relação entre as principais diferenças encontradas nos equipamentos comerciais de \acs{DBT} aprovados pelo \acs{FDA}. É possível observar uma grande variação entre as propriedades de cada modelo especificado. Isso se deve ao fato de que cada fabricante utiliza de suas ferramentas de pesquisa e desenvolvimento para solucionar um determinado problema e cada sistema possui os seus benefícios e suas limitações. 

Os equipamentos em sua essência buscam a melhor forma para obtenção das imagens, sendo assim, visam um bom contraste, menor dose, maior resolução  e uma boa relação sinal-ruído. Para isso, são utilizadas diferentes abordagens na aquisição dessas imagens \cite{vedantham2015digital}. 

\begin{table}[H]
	\centering
	\caption{Caraterísticas dos sistemas  de \acs{DBT}.}
	\label{tab:tabCap2SistemasDBT}
	\begin{tabular}{l|c|c|c}
		\textbf{Fabricantes}                                       &        \textbf{\acs{GE}}        &                 \textbf{Hologic}                 &   \textbf{Siemens}   \\
		[5pt]
		\hline
		\hline
		\rule[-0.5ex]{-3pt}{1ex}
		Modelo &    SenoClaire{\footnotesize\texttrademark}    & Selenia\textsuperscript{\textregistered} Dimensions\textsuperscript{\textregistered} & Mammomat Inspiration \\ \hline
		\rule[-0.5ex]{-3pt}{1ex}
		Número de projeções              &                9                &                        15                        &          25          \\ \hline
		\rule[-0.5ex]{-3pt}{1ex}
		Angulação do tubo                &           25$\degree$           &                   15$\degree$                    &     50$\degree$      \\ \hline
		\rule[-0.5ex]{-3pt}{1ex}
		Angulação do detector            &          Estacionário           &                   4,2$\degree$                   &     Estacionário     \\ \hline
		\rule[-0.5ex]{-3pt}{1ex}
		Movimento do tubo                &     \textit{Step-and-shoot}     &                     Contínuo                     &       Contínuo       \\ \hline
		\rule[-0.5ex]{-3pt}{1ex}
		Tempo do exame                   &               7s                &                       3,7s                       &         25s          \\ \hline
		\rule[-0.5ex]{-3pt}{1ex}
		Tamanho do detector              &             24x30cm             &                     24x29cm                      &       24x30cm        \\ \hline
		\rule[-0.5ex]{-3pt}{1ex}
		Tamanho do \textit{pixel}        &            100$\mu$m            &         70$\mu$m (2x2 \textit{Binning})          &       85$\mu$m       \\ \hline
		\rule[-0.5ex]{-3pt}{1ex}
		Tipo do detector                 &         \acs{a-Si} (Indireto)         &                  \acs{a-Se} (Direto)                   &    \acs{a-Se} (Direto)     \\ \hline
		\rule[-0.5ex]{-3pt}{1ex}
		Método de reconstrução           & Iterativo (ASiR\textsuperscript{\textregistered}) &                    \acs{FBP}                     &      \acs{FBP}       \\ \hline
	\end{tabular}
	\vspace{2ex}
	\legend{Fonte: \citeonline{michell2018role,vedantham2015digital,sechopoulos2013review,baker2011breast}.}
\end{table}


 
    

