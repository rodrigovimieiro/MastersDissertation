\chapter[Materiais e Métodos]{Materiais e Métodos}\label{Capitulo4}
 

%%%%%%%%%%%%%%%%%%%%%%%%%%%%%%%%%%%%%%%%%%%%%%%%%%%%%%%%%%%%%%%%%%%%%%%%%%%%%%%%%%%%%%%%%%%%%%%%%%%%%%%%%%%%%%%%%%%%%%%%%%%%%%														Phantom																%
%%%%%%%%%%%%%%%%%%%%%%%%%%%%%%%%%%%%%%%%%%%%%%%%%%%%%%%%%%%%%%%%%%%%%%%%%%%%%%%%%%%%%%%%%%%%%%%%%%%%%%%%%%%%%%%%%%%%%%%%%%%% 
\section{Materiais}

No presente trabalho foram utilizados \textit{phantoms} a fim de simular todos os métodos de reconstrução. Um \textit{phantom} pode ser físico ou virtual e tem como objetivo simular condições reais de exame para que os métodos alvos de pesquisas possam ser estudados e testados de maneira sistemática. Isso se deve ao fato de que exames clínicos, ou seja, com pacientes, não podem ser realizados toda vez que um novo método necessitar ser testado.

\subsection{\textit{Phantom} Virtual}

Uma versão modificada de \citeonline[]{shepp1974fourier} foi usada. O \textit{phantom} originalmente é constituído por uma seção da cabeça que foi desenvolvido para testar o algoritmo proposto pelos autores naquela época. Esse \textit{phantom} simula diversas estruturas anatômicas com diferentes densidades $D$ como: a água ($D=1$) nos ventrículos, massa cinzenta ($D=1,2$), tumores ($D=1,03;1,04$) e para o crânio ($D=2$). A Figura \ref{fig:imgCap4SheppLogan} ilustra o \textit{phantom} de Shepp-Logan em sua versão modificada.

\begin{figure}[H]
	\caption{Ilustração do \textit{phantom} de Shepp-Logan em sua versão modificada.}
	\begin{center}
		\includegraphics[scale=1]{imgs/cap4/SheppLogan.png}
	\end{center}
	\legend{Fonte: do autor, 2018.}
	\label{fig:imgCap4SheppLogan}
\end{figure}   

Como esse trabalho visa a reconstrução de estruturas anatômicas em \acs{3D} pela técnica de tomossíntese, deve-se usar um objeto tri-dimensional para validar as técnicas aplicadas. Tendo isso em mente, foi utilizado um algoritmo\footnote{\url{www.mathworks.com/matlabcentral/fileexchange/9416-3d-shepp-logan-phantom}}, retirado da comunidade \textit{MathWorks}\footnote{\url{www.mathworks.com}}, que gera uma extensão do \textit{phantom} Shepp-Logan em \acs{3D}. A Figura \ref{fig:imgCap4SheppLogan3D} ilustra o \textit{phantom} criado pelo respectivo algoritmo e suas medidas físicas são descritas na seção seguinte.

\begin{figure}[H]
	\caption{Ilustração do \textit{phantom} de Shepp-Logan \acs{3D}.}
	\begin{center}
		\includegraphics[scale=0.3]{imgs/cap4/SheppLogan3D.pdf}
	\end{center}
	\legend{Fonte: do autor, 2018.}
	\label{fig:imgCap4SheppLogan3D}
\end{figure} 

\subsection{\textit{Phantom} Físico}

Já para os testes envolvendo a redução de artefatos de alta atenuação foi usado o \textit{phantom} físico BR3D\footnote{\url{www.cirsinc.com/products/all/51/br3d-breast-imaging-phantom/}} \cite{PhantomBR3D}, como ilustra a Figura \ref{fig:imgCap4BR3D}. Esse \textit{phantom} é constituído de 6 fatias com um material que simula o tecido $100\%$ adiposo e $100\%$ glandular de uma mama. Junto ao tecido estão contidas microcalcificações, fibrilas e massas em formato de esfera. Todas essas estruturas possuem diferentes tamanhos e buscam mimetizar os aspectos reais encontrados em uma mama com possíveis lesões.   

\begin{figure}[H]
	\caption{Ilustração do \textit{phantom} BR3D.}
	\begin{center}
		\includegraphics[scale=0.34]{imgs/cap4/BR3D.jpeg}
	\end{center}
	\legend{Fonte: do autor, 2018.}
	\label{fig:imgCap4BR3D}
\end{figure} 
   

%%%%%%%%%%%%%%%%%%%%%%%%%%%%%%%%%%%%%%%%%%%%%%%%%%%%%%%%%%%%%%%%%%%%%%%%%%%%%%%%%%%%%%%%%%%%%%%%%%%%%%%%%%%%%%%%%%%%%%%%%%%%%%													Geometria																%
%%%%%%%%%%%%%%%%%%%%%%%%%%%%%%%%%%%%%%%%%%%%%%%%%%%%%%%%%%%%%%%%%%%%%%%%%%%%%%%%%%%%%%%%%%%%%%%%%%%%%%%%%%%%%%%%%%%%%%%%%%%% 
\section{Geometria} 

%\subsection{\textit{Phantom} Virtual}
 
Para a aplicação dos métodos no \textit{phantom} virtual foi proposto uma geometria física do sistema que se aproxima de um equipamento de \acs{DBT} comercial. A Tabela \ref{tab:tabCap4ParametrosPhantoms} expõe os parâmetros físicos do sistema que foram utilizados, seguindo as propriedades geométricas de um sistema hipotético de \acs{DBT}. %O leitor pode fazer uma comparação com os valores dos equipamentos comerciais demonstrados na Tabela \ref{tab:tabCap2SistemasDBT}.

A Figura \ref{fig:imgCap4GeometriaProjecao} ilustra a geometria que foi usada para a projeção e retroprojeção, sendo que a Figura \ref{fig:imgCap4GeometriaProjecao1} representa a projeção de número 1, \ref{fig:imgCap4GeometriaProjecao2} de número 9 e \ref{fig:imgCap4GeometriaProjecao3} de número 5, onde os ângulos são: $-1,25\degree$, $1,25\degree$ e $0\degree$ respectivamente. O cubo azul representa espacialmente o \textit{phantom} virtual da Figura \ref{fig:imgCap4SheppLogan3D}, o plano em amarelo a posição da projeção do objeto no detector, o plano pontilhado as extremidades do detector e o ponto vermelho o tubo de raios X. 


\begin{table}[H]
	\footnotesize
	\centering
	\caption{Parâmetros físicos do sistema implementado com \textit{phantom} virtual e do sistema comercial utilizado pelo \textit{phantom} BR3D.}
	\label{tab:tabCap4ParametrosPhantoms}
	\begin{tabular}{l|c|c}
		\textbf{Parâmetro}                            &  \textbf{Shepp-Logan} &  \textbf{BR3D}  \\
		[5pt]
		\hline
		\hline
		\rule[-0.5ex]{0pt}{3ex}
		Tamanho do detector 						  &     28x35cm       &      24x30cm       \\ \hline
		\rule[-0.5ex]{0pt}{3ex}
		Tamanho do \textit{pixel}                     &       1mm        &     100$\mu$m      \\ \hline
		\rule[-0.5ex]{0pt}{3ex}
		Número de projeções                           &        9         &         9          \\ \hline
		\rule[-0.5ex]{0pt}{3ex}
		Angulação do tubo                             &   2,5$\degree$   &    25$\degree$     \\ \hline
		\rule[-0.5ex]{0pt}{3ex}
		Angulação do detector                         &   Estacionário   &    Estacionário    \\ \hline
		\rule[-0.5ex]{0pt}{3ex}
		Tamanho do objeto                             & 128x128x128 u.v. & 2004x1092x129 u.v. \\ \hline
		\rule[-0.5ex]{0pt}{3ex}
		Tamanho do \textit{Voxel}                     &     1x1x10mm     &   0.1x0.1x0.5mm    \\ \hline
		\rule[-0.5ex]{0pt}{3ex}
		Distância fonte para detector                 &      660cm       &       660mm        \\ \hline
		\rule[-0.5ex]{0pt}{3ex}
		Distância detector para centro de rotação     &       40cm       &        40mm        \\ \hline
		\rule[-0.5ex]{0pt}{3ex}
		Tamanho lacuna de ar                          &       22cm       &        22mm        \\ \hline
	\end{tabular}
	\vspace{2ex}
	\legend{Fonte: \citeonline{michell2018role,vedantham2015digital,sechopoulos2013review,baker2011breast}}
\end{table}


\begin{figure}[t!]
	\centering
	
	\caption{Geometria utilizada na aplicação dos métodos, onde (a) representa a projeção de número 1, (b) de número 9 e (c) de número 5, onde os ângulos são: $-1,25\degree$, $1,25\degree$ e $0\degree$ respectivamente.}
	
	\subfloat[]{\includegraphics[scale=.5]{imgs/cap4/Geometria/Proj1.png}\label{fig:imgCap4GeometriaProjecao1}}
	\subfloat[]{\includegraphics[scale=.5]{imgs/cap4/Geometria/Proj9.png}\label{fig:imgCap4GeometriaProjecao2}}
	
	\subfloat[]{\includegraphics[scale=.55]{imgs/cap4/Geometria/Proj5.png}\label{fig:imgCap4GeometriaProjecao3}}
	
	\legend{Fonte: do autor, 2018.}
	\label{fig:imgCap4GeometriaProjecao}
\end{figure}

 
%\subsection{\textit{Phantom} Físico}\label{PhantomReal}

Já para o \textit{phantom} BR3D, as projeções foram realizadas em um equipamento da empresa \acs{GE} modelo Senographe Essential\texttrademark, ilustrado pela Figura \ref{fig:imgCap4EquipamentoGE}. A geometria utilizada pelo equipamento é detalhada também pela Tabela \ref{tab:tabCap4ParametrosPhantoms}.

%\begin{table}[H]
%	\footnotesize
%	\centering
%	\caption{Parâmetros físicos do sistema comercial.}
%	\label{tab:tabCap4ParametrosGESenographeEssential}
%	\begin{tabular}{l|c}
%		\textbf{Parâmetro}                            &   \textbf{Valor}   \\
%		[5pt]
%		\hline
%		\hline
%		\rule[-0.5ex]{0pt}{3ex}
%		Tamanho do detector 						  &      24x30cm       \\ \hline
%		\rule[-0.5ex]{0pt}{3ex}
%		Tamanho do \textit{pixel}                     &     100$\mu$m      \\ \hline
%		\rule[-0.5ex]{0pt}{3ex}
%		Número de projeções                           &         9          \\ \hline
%		\rule[-0.5ex]{0pt}{3ex}
%		Angulação do tubo                             &    25$\degree$     \\ \hline
%		\rule[-0.5ex]{0pt}{3ex}
%		Angulação do detector                         &    Estacionário    \\ \hline
%		\rule[-0.5ex]{0pt}{3ex}
%		Tamanho do objeto                             & 2004x1092x129 u.v. \\ \hline
%		\rule[-0.5ex]{0pt}{3ex}
%		Tamanho do \textit{Voxel}                     &   0.1x0.1x0.5mm    \\ \hline
%		\rule[-0.5ex]{0pt}{3ex}
%		Distância fonte para detector                 &       660mm        \\ \hline
%		\rule[-0.5ex]{0pt}{3ex}
%		Distância detector para centro de rotação     &        40mm        \\ \hline
%		\rule[-0.5ex]{0pt}{3ex}
%		Tamanho lacuna de ar                          &        22mm        \\ \hline
%	\end{tabular}
%	\vspace{2ex}
%	\legend{Fonte: \citeonline{michell2018role,vedantham2015digital,sechopoulos2013review,baker2011breast}}
%\end{table}

\begin{figure}[H]
	\centering
	
	\caption{(a) Equipamento de tomossíntese modelo Senographe Essential\texttrademark \ utilizado para realizar as projeções e (b) o mesmo com o \textit{phantom} BR3D.}
	
	\subfloat[]{\includegraphics[scale=0.08]{imgs/cap4/EquipamentoGE.jpg}\label{fig:imgCap4EquipamentoGEA}}
	\hfil
	\subfloat[]{\includegraphics[scale=.49]{imgs/cap4/BR3Deq.jpeg}\label{fig:imgCap4EquipamentoGEB}}
	
	\legend{Fonte: do autor, 2018.}
	\label{fig:imgCap4EquipamentoGE}
\end{figure} 


%%%%%%%%%%%%%%%%%%%%%%%%%%%%%%%%%%%%%%%%%%%%%%%%%%%%%%%%%%%%%%%%%%%%%%%%%%%%%%%%%%%%%%%%%%%%%%%%%%%%%%%%%%%%%%%%%%%%%%%%%%%%%%														Métodos																%
%%%%%%%%%%%%%%%%%%%%%%%%%%%%%%%%%%%%%%%%%%%%%%%%%%%%%%%%%%%%%%%%%%%%%%%%%%%%%%%%%%%%%%%%%%%%%%%%%%%%%%%%%%%%%%%%%%%%%%%%%%%% 
\section{Métodos} 

Para os métodos de reconstrução, uma \textit{toolbox}\footnote{www.mathworks.com/matlabcentral/fileexchange/35548-3d-cone-beam-ct--cbct--projection-backprojection-fdk--iterative-reconstruction-matlab-examples.} foi utilizada para auxílio da implementação dos métodos propostos. Foram aplicadas as técnicas de simples retroprojeção, retroprojeção filtrada e o método estatístico de máxima verossimilhança. Entretanto antes de implementar essas técnicas é necessário a obtenção das projeções dado o volume do objeto. Em função disso, são discutidos os métodos utilizados para a obtenção delas.  

\subsection{Projeção e Retroprojeção} 

Para a implementação da projeção direta foi utilizado o método de \textit{Pixel Driven} (Figura \ref{fig:imgCap3Projetores1}) descrito nos capítulos anteriores. Para isso então são utilizadas as Equações \ref{eq:eqCap3ProjectionY} e \ref{eq:eqCap3ProjectionX}, considerando a distância entre o detector e o eixo de rotação igual a zero, ou seja, $D=0$. 

 \colorbox{pink}{Reescrever esse parágrafo} De modo que nenhum \textit{pixel} no detector fique sem valor, um processo inverso é feito, ou seja, para cada \textit{pixel} no detector é calculado o respectivo \textit{voxel} do qual se atribui o seu valor. O mesmo é feito para a retroprojeção, onde são calculados para todos os \textit{voxels} qual o respectivo \textit{pixel} que contribui para o seu valor. 


 Os cálculos das projeções e retroprojeções são feitos com o objetivo de converter as coordenadas do mundo, e.g. $(X,Y,Z)$, para as coordenadas da imagem, e.g. $(x_{i},y_{i})$, porém após isso ainda são necessárias as conversões das coordenadas da imagem para as coordenadas em \textit{pixel}, e.g. $(i,j)$. Isso pode ser feito segundo as Equações \ref{eq:eqCap4ConversaoCoordImgparaPixelY} e \ref{eq:eqCap4ConversaoCoordImgparaPixelX}, seguido por uma interpolação linear entre as coordenadas em \textit{pixel} e a imagem a ser projetada ou reprojetada, onde $dx\,\text{e}\,dy$ são os tamanhos dos \textit{pixels} em (mm) para as coordenadas $x\,\text{e}\,y$ respectivamente. A Figura \ref{fig:imgCap4ConversaoCoord} ilustra a relação entre as coordenadas e os pseudocódigos abaixo resumem todo o processo. 
 
\begin{equation}
i = \dfrac{y_{i}}{dy} + y_{0}  
\label{eq:eqCap4ConversaoCoordImgparaPixelY}
\end{equation} 
 
\begin{equation}
 j = -\dfrac{x_{i}}{dx} + x_{0}.  
\label{eq:eqCap4ConversaoCoordImgparaPixelX}
\end{equation} 
 
 
 \begin{figure}[H]
 	\caption{Ilustração da relação entre as coordenadas.}
 	\begin{center}
 		\includegraphics[scale=1]{imgs/cap4/ConversaoCoord.pdf}
 	\end{center}
 	\legend{Fonte: do autor, 2018.}
 	\label{fig:imgCap4ConversaoCoord}
\end{figure} 

\begin{algorithm}[H]
	\label{alg:algProjecao}
	\caption{Projeção}
	\Entrada{$Volume3D$, $Parâmetros$} 
	\Saida{$Projeções$}
	\Inicio{
		
		\Para{cada $Projeção \in Projeções$}{
			$\theta \leftarrow Ângulo(Projeção)$
			
			\Para{cada $Fatia \in Volume3D$}{
				
				Calcular $Y \;\text{e}\; X \; \forall \; (y_{i}, x_{i}) \; \in Projeção$
				
				Calcular $i \;\text{e}\; j \; \forall \; (Y, X)$
				
				$Projeção \leftarrow Projeção + \text{Interpolação}(Fatia\,,\,(i,j)\,)$			
			}
		}
	}
	\Retorna{$Projeções$}
\end{algorithm}


\begin{algorithm}[H]
	\label{alg:algRetroprojecao}
	\caption{Retroprojeção}
	\Entrada{$Projeções$, $Parâmetros$} 
	\Saida{$Volume3D$}
	\Inicio{
		
		\Para{cada $Projeção \in Projeções$}{
			$\theta \leftarrow Ângulo(Projeção)$
			
			\Para{cada $Fatia \in Volume3D$}{
				
				Calcular $y_{i} \;\text{e}\; x_{i} \; \forall \; (X,Y,Z) \; \in Fatia$
				
				Calcular $i \;\text{e}\; j \; \forall \; (y_{i}, x_{i})$
				
				$Fatia \leftarrow Fatia + \text{Interpolação}(Projeção\,,\,(i,j)\,)$			
			}
		}
	}
	\Retorna{$Volume3D$}
\end{algorithm}


Após a aplicação dos passos mencionados acima em todo o volume \acs{3D}, são geradas as projeções da Figura \ref{fig:imgCap4Projecoes}, onde as imagens de (a) até (i) representam as projeções de 1 a 9 respectivamente. A partir dessas projeções é possível fazer a reconstrução do volume aplicando os métodos propostos.
 
\begin{figure}[H]
	\centering
	
	\caption{Projeções geradas a partir da Figura \ref{fig:imgCap4GeometriaProjecao}, onde de (a) até (i) representam as projeções de 1 a 9.}
	
	\subfloat[]{\includegraphics[scale=.45]{imgs/cap4/Proj/1.png}\label{fig:imgCap4Projecao1}}
	\subfloat[]{\includegraphics[scale=.45]{imgs/cap4/Proj/2.png}\label{fig:imgCap4Projecao2}}
	\subfloat[]{\includegraphics[scale=.45]{imgs/cap4/Proj/3.png}\label{fig:imgCap4Projecao3}}
	\subfloat[]{\includegraphics[scale=.45]{imgs/cap4/Proj/4.png}\label{fig:imgCap4Projecao4}}
	\subfloat[]{\includegraphics[scale=.45]{imgs/cap4/Proj/5.png}\label{fig:imgCap4Projecao5}}
	\subfloat[]{\includegraphics[scale=.45]{imgs/cap4/Proj/6.png}\label{fig:imgCap4Projecao6}}
	\subfloat[]{\includegraphics[scale=.45]{imgs/cap4/Proj/7.png}\label{fig:imgCap4Projecao7}}
	\subfloat[]{\includegraphics[scale=.45]{imgs/cap4/Proj/8.png}\label{fig:imgCap4Projecao8}}
	\subfloat[]{\includegraphics[scale=.45]{imgs/cap4/Proj/9.png}\label{fig:imgCap4Projecao9}}
	
	\legend{Fonte: do autor, 2018.}
	\label{fig:imgCap4Projecoes}
\end{figure}

%%%%%%%%%%%%%%%%%%%%%%%%%%%%%%%%%%%%%%%%%%%%%%%%%%%%%%%%%%%%%%%%%%%%%%%%%%%%%%%%%%%%%%%%%%%%%%%%%%%%%%%%%%%%%%%%%%%%%%%%%%%%%%												Reconstrução																%
%%%%%%%%%%%%%%%%%%%%%%%%%%%%%%%%%%%%%%%%%%%%%%%%%%%%%%%%%%%%%%%%%%%%%%%%%%%%%%%%%%%%%%%%%%%%%%%%%%%%%%%%%%%%%%%%%%%%%%%%%%%%
\subsection{Reconstrução de Imagens} 

Após a execução do passo a passo exemplificado anteriormente é possível a aplicação dos métodos de reconstrução. São utilizadas três técnicas para a comparação do melhor resultado: retroprojeção, retroprojeção filtrada e máxima verossimilhança. Para o algoritmo estatístico, diferentes quantidades de iterações são aplicadas no teste, variando de cinco até vinte iterações. O método de simples retroprojeção foi descrito no subitem anterior e os pseudocódigos abaixo descrevem o funcionamento dos métodos de \acs{FBP} e \acs{MLEM}. A confecção do filtro para o algoritmo de retroprojeção filtrada se dá a partir dos passos ditos no item \ref{RetroprojeçãoFiltrada}.

\begin{algorithm}[H]
	\label{alg:algRetroprojecaoFiltrada}
	\caption{Retroprojeção Filtrada}
	\Entrada{$Projeções$, $Parâmetros$} 
	\Saida{$Volume3D$}
	\Inicio{
		
		\Para{cada $Projeção \in Projeções$}{
			$Projeção \leftarrow Filtrar(Projeção)$	
		}
		$Volume3D \leftarrow Retroprojeção(Projeções)$
	}
	\Retorna{$Volume3D$}
\end{algorithm}

\begin{algorithm}[H]
	\label{alg:algMlEM}
	\caption{MLEM}
	\Entrada{$Projeções$, $Parâmetros$} 
	\Saida{$Volume3D$}
	\Inicio{
	$Volume3D \leftarrow EstimativaInicial$	
		
	$FatorNorm \leftarrow Retroprojeção(1)$	
	
		\Para{cada Iteração}{
		$EstimativaProjeções \leftarrow Projeção(Volume3D)$	
		
		$RazãoProjeções \leftarrow Projeções \;/\; EstimativaProjeções$
		
		$VolumeEstimado \leftarrow Retroprojeção(RazãoProjeções)$
		
		$FatorModificação \leftarrow VolumeEstimado \;/\; FatorNorm$
		
		$Volume3D \leftarrow Volume3D \;X\; FatorModificação$
		}
	}
	\Retorna{$Volume3D$}
\end{algorithm}


\subsection{Redução de Artefatos de Alta Atenuação}\label{MetodosReduçãodeArtefatosdeAltaAtenuação} 

Para a implementação dos métodos de redução de artefatos, utilizou-se os materiais ditos na seção anterior. Em seguida aplicou-se os métodos de \citeonline[]{borges2017metal} de redução de artefatos metálicos com o intuito de avaliar os resultados apresentados. 

A constante $M$ da Equação \ref{eq:eqCap3wBPLogistic} foi ajustada de maneira a balancear o nível de ruído e a redução dos artefatos de alta atenuação na imagem.


Foram recortadas regiões de interesse (\textit{Region of Interest} - \acs{ROI}) de tamanho $90x90$, para aplicar a técnica abordada. As coordenadas da \acs{ROI} foram de 1440 até 1529 em Y e 860 até 949 em X. A Figura \ref{fig:imgCap4ROIGE} ilustra essa região referente a fatia 74 reconstruída pelo equipamento comercial assim como a Figura \ref{fig:imgCap4ROIBP} ilustra a mesma região com uma técnica de reconstrução por retroprojeção simples. 

Foram selecionadas somente as fatias de 67 a 75 para a realização dos procedimentos. Isso se deve ao fato de que o método é computacionalmente caro.

\begin{figure}[H]
	\centering
	
	\caption{\acs{ROI} reconstruídas pelo modelo (a) comercial e através da (b) retroprojeção simples a fim de testar os métodos de redução de artefatos.}
	
	\subfloat[\acs{ROI} da fatia 74 - Comercial.]{\includegraphics[scale=.5]{imgs/cap4/GEroi.png}\label{fig:imgCap4ROIGE}}
	\hfil
	\subfloat[\acs{ROI} da fatia 74 - \acs{BP}]{\includegraphics[scale=.5]{imgs/cap4/BProi.png}\label{fig:imgCap4ROIBP}}
	
	\legend{Fonte: do autor, 2018.}
	\label{fig:imgCap4ROI}
\end{figure} 

\subsection{Validação} 

Para analisar de maneira objetiva os resultados das imagens, utilizou-se métricas de avaliação da qualidade da imagem, conhecidas na literatura, como as descritas abaixo:

O Erro quadrático médio (\textit{Mean Square Error} - \acs{MSE}) é uma forma de medida pontual que analisa as diferenças de intensidade entre uma imagem referência e outra estimada. Esse método é utilizado devido a sua simplicidade de cálculo, obtenção de resultados com fácil interpretação e baixa complexidade computacional, porém críticas são feitas ao mesmo devido a sua ineficiência quando se comparado com as métricas subjetivas da percepção humana \cite{gonzalez2008digital,wang2004image}. Sua equação matemática é dada por:

\begin{equation}
MSE = \dfrac{1}{MN} \sum_{x=1}^{M} \sum_{y=1}^{N} [f(x,y) - \hat{f}(x,y)]^{2},
\label{eq:eqCap4MSE}
\end{equation} 

\noindent ou em sua versão aplicando a raiz quadrada:

\begin{equation}
RMSE = \sqrt{MSE(f,\hat{f})},
\label{eq:eqCap4RMSE}
\end{equation}
 
\noindent ou ainda em sua versão normalizada:

\begin{equation}
NRMSE = \dfrac{RMSE(f,\hat{f})}{max(\hat{f}) - min(\hat{f})},
\label{eq:eqCap4NRMSE}
\end{equation}

%\noindent onde:
%
%\begin{equation}
%S_{f} = \dfrac{1}{MN} \sum_{x=1}^{M} \sum_{y=1}^{N} f(x,y)  \;\; S_{\hat{f}} = \dfrac{1}{MN} \sum_{x=1}^{M} \sum_{y=1}^{N} \hat{f}(x,y).
%\label{eq:eqCap4NRMSESxSy}
%\end{equation}


A métrica de Índice de Similaridade Estrutural (\textit{Structural Similarity Index} - \acs{SSIM}) tem como função fazer uma avaliação que seja mais fiel a percepção humana e a formação das imagens, segundo \citeonline[]{wang2004image}. Esse método avalia componentes como a  luminância $l(\cdot)$, o contraste $c(\cdot)$ e a estrutura dos objetos $s(\cdot)$. Sua formulação matemática é dada por:

\begin{equation}
SSIM(x,y) = [l(x,y)]^{\alpha} \cdot [c(x,y)]^{\beta} \cdot [s(x,y)]^{\gamma},
\label{eq:eqCap4SSIM}
\end{equation}

\noindent onde $\alpha,\beta,\gamma$ são constantes de ajuste da métrica. É importante ressaltar que o \acs{SSIM} retorna os índices de similaridade para diversas regiões da imagem, porém quando se fala em similaridade é interessante que haja somente um índice para toda a imagem. Portanto é feito então uma média dos M índices retornados, dado pela equação:

\begin{equation}
MSSIM(x,y) = \dfrac{1}{M} \sum_{j=1}^{M} SSIM(x_{j},y_{j})
\label{eq:eqCap4MSSIM}
\end{equation} 

A métrica de \textit{Sharpness} tem como função avaliar o nível de borramento em uma imagem. Isso é importante pois os métodos de restauração ou reconstrução tendem a borrar as imagens, fazendo com que haja perda de detalhes, como por exemplo em uma microcalcificação nos exames mamografias. É importante notar que esse método não necessita de uma imagem referência. A estimativa é feita através das seguintes equações:

\begin{equation}
Sharpness = \sum_{r}^{} \sum_{c}^{}  w_{x}G^{2}_{x} + w_{y} G^{2}_{y}, 
\label{eq:eqCap4SHARPDB}
\end{equation}

\noindent onde $G_{x}$ e $G_{y}$ são gradientes direcionais, $w_{x}$ e $w_{y}$ são pesos baseados em uma vizinhança local, dados por:

\begin{equation}
w_{x} = [M(x+1,y)-M(x-1,y)]^{2},
\label{eq:eqCap4SHARPDB1}
\end{equation}

\begin{equation}
w_{y} = [M(x,y+1)-M(x,y-1)]^{2}.
\label{eq:eqCap4SHARPDB2}
\end{equation}

\colorbox{pink}{Alexandre: Arrumar essas equações de Sharpness}
	