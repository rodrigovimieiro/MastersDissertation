\chapter[Reconstrução]{Reconstrução de imagens}\label{Capitulo3}

%%%%%%%%%%%%%%%%%%%%%%%%%%%%%%%%%%%%%%%%%%%%%%%%%%%%%%%%%%%%%%%%%%%%%%%%%%%%%%%%%%%%%%%%%%%%%%%%%%%%%%%%%%%%%%%%%%%%%%%%%%%%%%												  Introdução    															%
%%%%%%%%%%%%%%%%%%%%%%%%%%%%%%%%%%%%%%%%%%%%%%%%%%%%%%%%%%%%%%%%%%%%%%%%%%%%%%%%%%%%%%%%%%%%%%%%%%%%%%%%%%%%%%%%%%%%%%%%%%%%%%

Com o objetivo de obter imagens radiográficas feitas através de um exame de tomografia, seja esse advindo de qualquer equipamento, é necessária a aplicação de procedimentos para a reconstrução dessas imagens, dado as informações que o equipamento gera. Tais procedimentos envolvem aplicações de técnicas matemáticas, físicas e computacionais. As informações geradas pelo equipamento são comumente denominadas de projeções, pois representam ``sombras'' que o objeto produz ao ser irradiado por feixes de raios X. As sombras, ou mais precisamente, projeções, fornecem informações das atenuações sofridas pelos raios X quando estes interagem com o tecido biológico ao longo de seu caminho \cite{buzug2008computed,avinash1988principles}.
 
Os métodos de reconstrução atuais destinados à tomossíntese foram primeiramente desenvolvidos para \acs{CT}. Posteriormente, estes foram adaptados as necessidades específicas de cada equipamento e em geral, os métodos utilizados para \acs{CT} são aplicáveis à tomossíntese, respeitando as devidas particularidades. Contudo o desenvolvimento de algoritmos para tomossíntese é considerado um desafio pelo fato dessa técnica possuir poucas projeções em uma estreita extensão angular e baixa dose de radiação \cite{levakhina2014three,yang2012numerical}.   

%%%%%%%%%%%%%%%%%%%%%%%%%%%%%%%%%%%%%%%%%%%%%%%%%%%%%%%%%%%%%%%%%%%%%%%%%%%%%%%%%%%%%%%%%%%%%%%%%%%%%%%%%%%%%%%%%%%%%%%%%%%%%%											Problemas Inversos	    														%
%%%%%%%%%%%%%%%%%%%%%%%%%%%%%%%%%%%%%%%%%%%%%%%%%%%%%%%%%%%%%%%%%%%%%%%%%%%%%%%%%%%%%%%%%%%%%%%%%%%%%%%%%%%%%%%%%%%%%%%%%%%%%%
\section{Problemas Inversos}

Em geral, reconstruções de imagens tomográficas são consideradas, do ponto de vista matemático, como um problema inverso, ou seja, busca-se a formação da estrutura espacial do objeto a partir das projeções do mesmo em diferentes ângulos \cite{buzug2008computed}. A Figura \ref{fig:imgCap3ProbInver} ilustra o procedimento tomando projeções do objeto por meio de ângulos distintos. A Figura \ref{fig:imgCap3ProbInverA} exemplifica o imageamento do objeto e a formação de suas respectivas projeções, já a Figura \ref{fig:imgCap3ProbInverB} exemplifica o problema inverso para a reconstrução da estrutura espacial do objeto. 

%Forward problem and Inverse problem 

\begin{figure}[H]
	\centering
	
	\caption{Processo de reconstrução de objetos \acs{2D} através de: (a) suas projeções em diferentes ângulos e a (b) formulação do problema inverso.}
	
	\subfloat[]{\includegraphics[scale=0.8, clip, trim=12.6cm 7.5cm 12.3cm 3.2cm]{imgs/cap3/ProbInverA.pdf}	\label{fig:imgCap3ProbInverA}}
	\hfill
	\subfloat[]{\includegraphics[scale=0.8, clip, trim=12.6cm 7.5cm 12.3cm 3.2cm]{imgs/cap3/ProbInverB.pdf} \label{fig:imgCap3ProbInverB}}
	
	\legend{Fonte: do autor, 2019.}
	\label{fig:imgCap3ProbInver}
\end{figure}

Os equipamentos de raios X por transmissão têm como modelo físico básico a formulação de \textit{Beer-Lambert} \cite{zeng2010medical}. Essa fórmula relaciona a função de atenuação $\mu(\sigma)$ do feixe de radiação quando este interage com um certo material por um determinado caminho. Considerando um feixe de raio X ideal, infinitamente pequeno, monoenergético e sem espalhamento, sua intensidade de energia inicial $I_{0}$ interage com a matéria por um caminho $L$ e sua intensidade de energia final $I$ é dada por \eqref{eq:eqCap3BeerLambert1}, segundo \citeonline[p. 13]{levakhina2014three}:

\begin{equation}
 I = I_{0} \, e \, ^{-\int_{L}^{} \,\mu(\sigma) \, d\sigma}.
\label{eq:eqCap3BeerLambert1}
\end{equation}

Considerando a Equação \ref{eq:eqCap3BeerLambert1}, se tomarmos o logaritmo da razão entre a intensidade final e a inicial com seu sinal negativo e denotarmos essa formulação como $p_{L}$, ou seja, $p_{L} = -\ln \, \left(\frac{I}{I_{0}}\right)$, tem-se o seguinte resultado:

\begin{equation}
p_{L} = {\int_{L}^{} \,\mu(\sigma) \, d\sigma},
\label{eq:eqCap3BeerLambert2}
\end{equation}

\noindent onde $\sigma$ é o incremento ao longo de $L$ e o problema inverso, ou de reconstrução, passa a ser a recuperação da função  $\mu(\sigma)$, tendo disponível um conjunto de integrais de linha $p_{L}$, calculadas através da transformada de Radon detalhada na seção seguinte \cite[p. 33]{levakhina2014three}.  
 
 
%%%%%%%%%%%%%%%%%%%%%%%%%%%%%%%%%%%%%%%%%%%%%%%%%%%%%%%%%%%%%%%%%%%%%%%%%%%%%%%%%%%%%%%%%%%%%%%%%%%%%%%%%%%%%%%%%%%%%%%%%%%%%%											Transformada de Radon    														%
%%%%%%%%%%%%%%%%%%%%%%%%%%%%%%%%%%%%%%%%%%%%%%%%%%%%%%%%%%%%%%%%%%%%%%%%%%%%%%%%%%%%%%%%%%%%%%%%%%%%%%%%%%%%%%%%%%%%%%%%%%%%%%
\section{Transformada de Radon}\label{TransformadaRadon2D}

No ano de 1917, o matemático Johann Radon publicou um estudo detalhando a solução para os ditos problemas inversos em \acs{CT}, tornando assim a transformada mais importante no aspecto teórico matemático da área atualmente \cite{radon1917uber}. Esse trabalho foi posteriormente traduzido para a língua inglesa \cite{radon1986determination}.

Essencialmente, a transformada de Radon direta visa calcular as projeções $P(\rho,\theta)$ de um objeto com coordenadas $f(x,y)$ a partir de determinados ângulos $\theta$ e feixes $L_{n}$, como ilustra a Figura \ref{fig:imgCap3Radon}. Essas projeções são obtidas através de um conjunto de integrais de linha de $f(x,y)$ ao longo de feixes paralelos  $L_{n}$, como demonstra a Equação \ref{eq:eqCap3Radon1}, onde $d\sigma$ é o incremento ao longo dos feixes. Já a sua transformada inversa é a simples tomada dos valores da projeção e a distribuição destes ao longo do caminho percorrido pelo feixe paralelo. Se tomarmos suficientes projeções do objeto, podemos ao final reconstruir sua estrutura espacial. 

\begin{equation} 
P(\rho,\theta) = {\int_{L}^{} f(x,y) \, d\sigma}.
\label{eq:eqCap3Radon1}
\end{equation}


\begin{figure}[H]
	\caption{Ilustração da transformada de Radon direta.}
	\begin{center}
		\includegraphics[scale=0.8, clip, trim=12cm 4.5cm 12.8cm 5.2cm]{imgs/cap3/Radon.pdf}
	\end{center}
	\legend{Fonte: do autor, 2019.}
	\label{fig:imgCap3Radon}
\end{figure}

Como pode ser visto na Equação \ref{eq:eqCap3Radon1} e na Figura \ref{fig:imgCap3Radon} a projeção está em função das duas variáveis: $\rho$ e $\theta$. Isso pode ser feito, uma vez que, as retas $L$ que representam os feixes são parametrizadas na forma polar, ou seja, $L$ é perpendicular ao segmento de reta que liga  o ponto $(\rho, \theta)$ a origem do plano,  \cite{yang2012numerical}. Isto é, sua forma é dada pela seguinte equação:

\begin{equation} 
L_{n} = 
\begin{cases}
x = \rho \, cos(\theta) - \sigma \, sin(\theta)\\ 
y = \rho \, sin(\theta) + \sigma \, cos(\theta)
\end{cases},
\label{eq:eqCap3Radon2}
\end{equation}

\noindent na qual $\sigma$ é a distância entre $\rho$ e qualquer ponto na reta $L$. A Figura \ref{fig:imgCap3RadonRhoTheta} ilustra a relação entre essas variáveis.

\begin{figure}[H]
	\caption{Ilustração da relação entre $x$, $y$, $\rho$, $\theta$ e $\sigma$.}
	\begin{center}
		\includegraphics[scale=1.2]{imgs/cap3/RadonRhoTheta.pdf}
	\end{center}
	\legend{Fonte: \citeonline[p. 74]{yang2012numerical}.}
	\label{fig:imgCap3RadonRhoTheta}
\end{figure}

Após a parametrização das retas, podemos reescrever a Equação \ref{eq:eqCap3Radon1} em função da Equação \ref{eq:eqCap3Radon2} e então teremos o seguinte resultado:

\begin{equation} 
P(\rho,\theta) = \int_{L}^{} f(x,y) \, d\sigma = {\int_{-\infty}^{+\infty} f\left((\rho \, cos(\theta) - \sigma \, sin(\theta)),(\rho \, sin(\theta) + \sigma \, cos(\theta)\right) \, d\sigma},
\label{eq:eqCap3Radon3}
\end{equation}

\noindent ou fazendo o uso da função delta de Dirac $\delta(\rho)$ teremos:

\begin{equation} 
P(\rho,\theta) = {\int_{-\infty}^{+\infty} \, \int_{-\infty}^{+\infty} f(x,y) \, \delta(\rho - (x\,cos(\theta)+y\,sin(\theta)))  \, dxdy},
\label{eq:eqCap3Radon4}
\end{equation} 

\noindent na qual a Equação \ref{eq:eqCap3Radon4} é conhecida como a transformada direta de Radon da função $f(x,y)$. Um maior detalhamento dos equacionamentos pode ser encontrado em \citeonline[p. 73-74]{yang2012numerical}.

%%%%%%%%%%%%%%%%%%%%%%%%%%%%%%%%%%%%%%%%%%%%%%%%%%%%%%%%%%%%%%%%%%%%%%%%%%%%%%%%%%%%%%%%%%%%%%%%%%%%%%%%%%%%%%%%%%%%%%%%%%%%%%											Métodos de Reconstrução    														%
%%%%%%%%%%%%%%%%%%%%%%%%%%%%%%%%%%%%%%%%%%%%%%%%%%%%%%%%%%%%%%%%%%%%%%%%%%%%%%%%%%%%%%%%%%%%%%%%%%%%%%%%%%%%%%%%%%%%%%%%%%%%%%
\section{Métodos de Reconstrução}

Após a formulação da transformada de Radon é possível obter a estrutura espacial do objeto a partir de suficientes projeções. Todo equacionamento, no entanto, é proposto em modo contínuo fazendo o uso de integrais, por exemplo, e sendo necessário o uso de uma geometria com feixes de raios paralelos. Tendo dito isso, é eminente a necessidade da discretização dessa transformada e adequação ao sistema físico de cada equipamento para implementações práticas \cite{levakhina2014three}. 

As técnicas de projeção e retroprojeção são essenciais nos métodos de reconstrução, pois representam através de um modelo matemático o processo físico de aquisição da imagem. Para os métodos iterativos, a importância é ainda maior pelo fato da utilização de ambas a cada iteração \cite{levakhina2014three}.   

\subsection{Discretização da Imagem}\label{DiscretizaçãodaImagem}

Para a implementação prática da transformada é necessária a representação de um objeto contínuo de uma forma discreta, ou seja, transformá-lo em uma matriz de valores finitos em duas ou três dimensões, dependendo da aplicação. Isso porque os coeficientes de atenuação ao longo de um objeto, por exemplo uma mama, são dados em uma forma contínua e são dependentes de fatores como: número atômico, densidade e espessura do meio \cite{yang2012numerical}.  

Portanto é possível fazer o uso de \textit{pixels} ou \textit{voxels} para representações em \acs{2D} ou \acs{3D} respectivamente. Além dessas duas representações também é possível a utilização de bolhas, ou \textit{blobs} do inglês. Um maior detalhamento sobre o modelamento dessas técnicas pode ser encontrado em  \citeonline[p. 45-69]{levakhina2014three}. A Figura \ref{fig:imgCap3DiscretizacaoImagem1} demonstra a discretização dos coeficientes de atenuações da mama representados por \textit{voxels} cúbicos, dado uma matriz de $m$ linhas por $n$ colunas e $l$ níveis.   

\begin{figure}[H]
	\caption{Discretização dos coeficientes de atenuações da mama.}
	\begin{center}
		\includegraphics[scale=1.1]{imgs/cap3/DiscretizacaoImg1.pdf}
	\end{center}
	\legend{Fonte: \citeonline[p. 72]{yang2012numerical}.}
	\label{fig:imgCap3DiscretizacaoImagem1}
\end{figure}

Se tomarmos qualquer linha dessa matriz, temos a Figura \ref{fig:imgCap3DiscretizacaoImagem2}, ilustrando os coeficientes discretos e os níveis de energia inicial e final da respectiva linha, considerando um feixe ideal. Dado isto, pode-se adaptar a Equação \ref{eq:eqCap3BeerLambert1} para a sua forma discreta, demonstrada a seguir:

\begin{equation}
I = I_{0} \, e \, ^{-\sum_{i=1}^{n} \mu_{i}}.
\label{eq:eqCap3BeerLambertDiscreto1}
\end{equation} 

\begin{figure}[H]
	\caption{Coeficientes de atenuação discretos de qualquer linha da matriz mostrada na Figura \ref{fig:imgCap3DiscretizacaoImagem1}.}
	\begin{center}
		\includegraphics[scale=1.3]{imgs/cap3/DiscretizacaoImg2.pdf}
	\end{center}
	\legend{Fonte: \citeonline[p. 72]{yang2012numerical}.}
	\label{fig:imgCap3DiscretizacaoImagem2}
\end{figure}

\subsection{Projeção e Retroprojeção 2D}

 Os operadores de projeção, essencialmente, visam projetar um objeto em (\acs{3D} ou \acs{2D}) em um anteparo (\acs{2D} ou \acs{1D}) respectivamente, ou seja, somam as contribuições de cada objeto ao longo do eixo de projeção. A Figura \ref{fig:imgCap3ProjeçãoDireta} ilustra a projeção de objetos em \acs{2D} para sensores em \acs{1D}, tomando ângulos de $0\degree$, $45\degree$, $90\degree$ e $135\degree$. 

\begin{figure}[H]
	\caption{Exemplo de projeção \acs{2D} para \acs{1D}.}
	\begin{center}
		\includegraphics[scale=0.55, clip, trim=8cm 1cm 7cm 3.7cm]{imgs/cap3/FP.pdf}
	\end{center}
	\legend{Fonte: do autor, 2019.}
	\label{fig:imgCap3ProjeçãoDireta}
\end{figure}


Já os operadores de retroprojeção atuam no processo inverso, de reprojetar os dados (\acs{2D} ou \acs{1D}) para um espaço (\acs{3D} ou \acs{2D}) nesta ordem, isto é, distribuir os dados de projeção ao longo dos eixos que foram adquiridos. Assim como na seção anterior, a Figura \ref{fig:imgCap3BPMosaico} ilustra o processo da retroprojeção dos dados no espaço da imagem, adquiridos pelos sensores da Figura \ref{fig:imgCap3ProjeçãoDireta} nos respectivos ângulos que foram gerados. 

\begin{figure}[H]
	\centering
	
	\caption{Exemplo de retroprojeção \acs{1D} para \acs{2D}, onde as figuras de (a) até (d) representam os ângulos de $0\degree$, $45\degree$, $90\degree$ e $135\degree$ respectivamente.}
	
	\subfloat[]{\includegraphics[scale=0.2]{imgs/cap3/iRadon_0deg.png}\label{fig:imgCap3BPMosaico0}}
	\hfil
	\subfloat[]{\includegraphics[scale=0.2]{imgs/cap3/iRadon_45deg.png}\label{fig:imgCap3BPMosaico45}}
	\hfil
	\subfloat[]{\includegraphics[scale=0.2]{imgs/cap3/iRadon_90deg.png}\label{fig:imgCap3BPMosaico90}}
	\hfil
	\subfloat[]{\includegraphics[scale=0.2]{imgs/cap3/iRadon_135deg.png}\label{fig:imgCap3BPMosaico135}}
	\hfil
	
	\legend{Fonte: do autor, 2019.}
	\label{fig:imgCap3BPMosaico}
\end{figure}


Ao somar as contribuições de cada ângulo da Figura \ref{fig:imgCap3BPMosaico} a fim de reconstruir o objeto, a Figura \ref{fig:imgCap3BPParcial} é obtida como resultado. Nota-se que a partir dessa operação ainda não é possível identificar as estruturas espaciais com clareza. Já se forem consideradas mais projeções com uma maior faixa de ângulo, como por exemplo de $0\degree$ \text{a} $180\degree$, tornam-se visíveis os objetos, porém ainda de maneira borrada como demostra a Figura \ref{fig:imgCap3BPFinal}. Não há a necessidade de se utilizar $360\degree$, já que as projeções do primeiro e do segundo quadrante são espelhadas em relação ao terceiro e quarto quadrante.   

\begin{figure}[H]
	\centering
	
	\caption{Recuperação dos objetos \acs{2D} através da soma de suas retroprojeções em diferentes ângulos. Onde (a) representa a técnica considerando apenas 4 projeções, ilustrada pela Figura \ref{fig:imgCap3BPMosaico}, e (b) representa a utilização de 180 projeções na faixa de $180\degree$.}
	
	\subfloat[]{\includegraphics[scale=0.3]{imgs/cap3/Backprojection_Parcial.png}\label{fig:imgCap3BPParcial}}
	\hfil
	\subfloat[]{\includegraphics[scale=0.3]{imgs/cap3/Backprojection_Final.png}\label{fig:imgCap3BPFinal}}
	\hfil
	
	\legend{Fonte: do autor, 2019.}
	\label{fig:imgCap3BPSoma}
\end{figure}

O método descrito acima é o mais simples encontrado na literatura e é conhecido como \textit{Pixel Driven} (Figura \ref{fig:imgCap3Projetores1}) \cite[p. 47]{levakhina2014three}. Em sua projeção direta, o método consiste em projetar o centro de cada \textit{pixel} no detector e o seu valor de atenuação é repartido entre os detectores vizinhos por meio da interpolação linear ou por outros métodos mais complexos. O mesmo princípio é aplicado para o operador de retroprojeção, no qual os \textit{pixels} da imagem recebem os valores obtidos por sensores vizinhos, também através de métodos de interpolação.

Outras técnicas mais sofisticadas também são aplicadas como: \textit{Ray Casting} ou \textit{Ray Driven}, \textit{Distance-Driven} \cite{de2002distance,de2004distance} e \textit{Trapezoid footprint}, como demonstra a Figura \ref{fig:imgCap3Projetores}. Um maior detalhamento das técnicas mencionadas pode ser encontrado em \citeonline[p. 47-49]{levakhina2014three} ou ainda, métodos do estado da arte como demostrado em \citeonline[]{liu2017gpu} e \citeonline[]{zheng2017segmented}.

\begin{figure}[H]
	\centering
	
	\caption{Projetores utilizados para técnicas de projeção e retroprojeção para modelos de representação utilizando \textit{Pixels}, onde os diferentes métodos são ilustrados por: (a) \textit{Pixel Driven}, (b) \textit{Ray Casting}, (c) \textit{Distance-Driven} e (d) \textit{Trapezoid footprint}.}
	
	\subfloat[]{\includegraphics[scale=1.1]{imgs/cap3/Projetor1.pdf}\label{fig:imgCap3Projetores1}}
	\hfil
	\subfloat[]{\includegraphics[scale=1.1]{imgs/cap3/Projetor2.pdf}\label{fig:imgCap3Projetores2}}
	\hfil
	\subfloat[]{\includegraphics[scale=1.1]{imgs/cap3/Projetor3.pdf}\label{fig:imgCap3Projetores3}}
	\hfil
	\subfloat[]{\includegraphics[scale=1.1]{imgs/cap3/Projetor4.pdf}\label{fig:imgCap3Projetores4}}
	\hfil
	
	\legend{Fonte: \citeonline[p. 47]{levakhina2014three}.}
	\label{fig:imgCap3Projetores}
\end{figure}  
  
\subsection{Projeção e Retroprojeção 3D}\label{ProjeçãoeRetroprojeção3D}

Os problemas para a reconstrução de imagens em tomossíntese estão relacionados à estruturas tridimensionais, sendo assim é importante discorrer sobre os operadores de projeção e retroprojeção aplicados a essa técnica. A Figura \ref{fig:imgCap33DGeometry} ilustra como um feixe de raio X é atenuando e quais \textit{voxels} são responsáveis por esta interação. Em geral, os operadores utilizados para \acs{2D} são também atribuídos as geometrias \acs{3D}, porém sendo necessários ajustes para cada técnica e geometria. O processamento é feito então plano por plano, passando por todos os \textit{voxels} daquele nível. 

\begin{figure}[htb]
	\caption{Ilustração da geometria de meio cone referente a tomossíntese e a atenuação do feixe de raio X pelos respectivos \textit{voxels}.}
	\begin{center}
		\includegraphics[scale=0.9]{imgs/cap3/3DGeometry.pdf}
	\end{center}
	\legend{Fonte: \citeonline[p. 71]{Duarte2009}.}
	\label{fig:imgCap33DGeometry}
\end{figure}

Para a implementação do operador \textit{Pixel Driven} é necessário encontrar a posição específica onde cada \textit{voxel} será projetado. Segundo \citeonline[p. 77]{dobbins2003digital}, os equipamentos de tomossíntese, em geral, são caracterizados por um movimento isocêntrico. Isso porque tubo e detector ou apenas o tubo se move em um arco com o mesmo centro de movimento. 

Uma das geometrias de movimento isocêntrico foi primeiramente apresentada por \citeonline[]{niklason1997digital}, como ilustra a Figura \ref{fig:imgCap3ProjectionGeometry}, na qual o detector se mantém estacionário e o tubo se move em um arco com um determinado centro de rotação. 

\begin{figure}[htb]
	\caption{Geometria de movimento isocêntrico apresentado por \citeonline[]{niklason1997digital}, em que o tubo se move em um determinado arco e o detector se mantêm estacionário.}
	\begin{center}
		\includegraphics[scale=0.4]{imgs/cap3/ProjectionGeometry.png}
	\end{center}
	\legend{Fonte: Adaptado de \citeonline[p. 76]{dobbins2003digital}.}
	\label{fig:imgCap3ProjectionGeometry}
\end{figure}

Os passos necessários para a formação da imagem com este tipo de movimento foram apresentados no trabalho do autor, onde o mesmo descreve as equações para a projeção de um objeto no plano da imagem. Dado um ângulo de projeção $\theta$ e um ponto qualquer $(X,Y,Z)$ no espaço \acs{3D}, a coordenada \acs{2D} $(X_{i},Y_{i})$ projetada no plano da imagem é dada pelas seguintes equações:

\begin{equation}
Y_{i}(\theta,Y,Z) = Y \, + \dfrac{ Z(L \, sin(\theta) \, + Y)}{L \, cos(\theta) \, + D - Z},
\label{eq:eqCap3ProjectionY}
\end{equation} 

\begin{equation}
X_{i}(\theta,X,Z) = \dfrac{X \, (L \, cos(\theta)\,+\, D)}{L \, cos(\theta) \, + D - Z},
\label{eq:eqCap3ProjectionX}
\end{equation} 

\noindent na qual $D$ é a distancia entre o detector e o eixo de rotação (pivô) e $L$ a distância entre o tubo de raios X e o eixo de rotação.

Já a retroprojeção dos dados é dada a partir da inversão dessas equações a fim de calcular os valores da coordenada $(X,Y)$ para cada fatia em $Z$. O procedimento é executado a partir de cada projeção obtida em um determinado ângulo $\theta$ e por fim são somadas a contribuições de cada projeção para cada fatia em $Z$. O fluxograma da Figura \ref{fig:imgCap3FluxogramaBP} ilustra o procedimento que deve ser feito para a reconstrução do volume. Em geral a combinação matemática utilizada para os planos retroprojetados é uma simples soma ou um média ponderada. Técnicas mais sofisticadas para a redução de artefatos de alta atenuação são apresentadas em \citeonline{levakhina2013weighted} e \citeonline{borges2017metal}.

\begin{figure}[H]
	\caption{Fluxograma representando o processo de simples retroprojeção a fim de reconstruir o volume \acs{3D}.}
	\begin{center}
		\includegraphics[scale=0.5, clip, trim= 5.5cm 0cm 6cm 0cm]{imgs/cap3/FluxogramaBP.pdf}
	\end{center}
	\legend{Fonte: do autor, 2019.}
	\label{fig:imgCap3FluxogramaBP}
\end{figure}

%%%%%%%%%%%%%%%%%%%%%%%%%%%%%%%%%%%%%%%%%%%%%%%%%%%%%%%%%%%%%%%%%%%%%%%%%%%%%%%%%%%%%%%%%%%%%%%%%%%%%%%%%%%%%%%%%%%%%%%%%%%%%%											Reconstrução Analítica    														%
%%%%%%%%%%%%%%%%%%%%%%%%%%%%%%%%%%%%%%%%%%%%%%%%%%%%%%%%%%%%%%%%%%%%%%%%%%%%%%%%%%%%%%%%%%%%%%%%%%%%%%%%%%%%%%%%%%%%%%%%%%%%%%

\subsection{Reconstrução Analítica}

As técnicas utilizadas para a reconstrução em equipamentos de tomossíntese são provenientes de \acs{CT}, pelo fato de já estarem consolidadas e serem vastamente conhecidas na literatura. Isso não é diferente quando o assunto é reconstrução de forma analítica. Todo processo vem da teoria da transformada inversa de Radon, que foi discutida com mais detalhes no item \ref{TransformadaRadon2D}. É importante notar que devido ao problema de reconstrução ser mal condicionado, tal como uma aquisição incompleta de dados, a resolução de forma analítica é uma solução aproximada e não leva em conta diversos fatores, como o ruído quântico presente nas projeções. A melhor forma de explicar esse método é partindo de duas dimensões e posteriormente estendê-lo para a terceira dimensão, como é demonstrado nos itens a seguir \cite{mertelmeier2014filtered,xu2014tomographic}.  


\subsubsection{Retroprojeção Filtrada}\label{RetroprojeçãoFiltrada}

Um grande problema relacionado com a simples retroprojeção dos dados do detector é a soma das baixas frequências que ocorre a partir dos diferentes ângulos, como foi demosntrado pela Figura \ref{fig:imgCap3BPMosaico} e \ref{fig:imgCap3BPParcial}. Isso é facilmente observado quando faz-se o uso do \textit{Fourier slice theorem}, também conhecido como \textit{central slice theorem}, ou traduzindo para o português como teorema do corte de Fourier. 

O problema dito acima pode ser observado na Figura \ref{fig:imgCap3FourierSliceTheorem}, caso sejam tomados diferentes ângulos de projeção. A formulação matemática desse teorema é encontrada em \citeonline[p. 61-65]{hsieh2009computed}. Aplicando a transformada de Fourier inversa no domínio da frequência, ao serem adquiridas diversas projeções em toda a faixa de ângulo, é possível recuperar a estrutura espacial do objeto, porém de forma ``borrada'' devido a sobreposição de baixas frequências.     

\begin{figure}[htb]
	\caption{Ilustração do teorema do corte de Fourier.}
	\begin{center}
		\includegraphics[scale=0.75, clip, trim=8.5cm 4cm 10.4cm 5.6cm]{imgs/cap3/FourierSliceTheorem.pdf}
	\end{center}
	\legend{Fonte: do autor, 2019.}
	\label{fig:imgCap3FourierSliceTheorem}
\end{figure}

Com o propósito de contornar esse problema, é aplicado em cada projeção um filtro de rampa \acs{1D}, conhecido como \textit{ramp filter} (\ref{fig:imgCap3FBPFiltersA}), até a frequência de Nyquist $\frac{1}{2 \varDelta x}$, no qual $\varDelta x$ é o tamanho espacial do \textit{pixel}. Esse filtro faz com que as baixas frequências sejam atenuadas mantendo somente as altas, porém devido ao seu formato, o ruído que predomina na alta frequência é amplificado. 

\begin{figure}[htb]
	\centering
	
	\caption{Ilustração (a) do filtro de rampa no domínio da frequência (passa alta) e (b) o mesmo ``janelado'' por uma função Hanning para redução de ruídos (passa faixa).}
	
	\subfloat[]{\includegraphics[scale=0.5, clip, trim=3.5cm 8.5cm 4.1cm 8.6cm]{imgs/cap3/FiltroRampa.pdf}\label{fig:imgCap3FBPFiltersA}}
	\hfil
	\subfloat[]{\includegraphics[scale=0.5, clip, trim=3.5cm 8.5cm 4.1cm 8.6cm]{imgs/cap3/FiltroRampaHanning.pdf}\label{fig:imgCap3FBPFiltersB}}
	\hfil
	
	\legend{Fonte: do autor, 2019.}
	\label{fig:imgCap3FBPFilters}
\end{figure}

Devido a esse impasse é aplicado um ``janelamento'' no filtro, como por exemplo: Hamming, Hanning, Shepp-Logan ou cosseno. A Figura \ref{fig:imgCap3FBPFiltersA} ilustra o filtro rampa no domínio da frequência e a Figura \ref{fig:imgCap3FBPFiltersB} demonstra o mesmo após um ``janelamento'' pela função Hanning. Seguido do procedimento de filtragem, os dados são retomados para o domínio do espaço e então são retroprojetados para a formação da estrutura espacial do objeto \cite{xu2014tomographic}. 

Já para a aplicação em tomossíntese, diversas aproximações devem ser feitas, pois esta técnica abrange somente uma pequena extensão de ângulo com poucas projeções, quando comparada aos equipamentos de \acs{CT}. Outro problema é com a geometria de emissão dos feixes, que são dispostos em formato de meio cone (\textit{half-cone-beam}) \cite{mertelmeier2014filtered}. 

Tendo em vista esses fatores, com uma aproximação da geometria dos feixes, dispondo-os paralelamente, pode-se utilizar o teorema do corte de Fourier para analisar o comportamento das projeções no domínio da frequência. Isso pode ser feito para grandes distâncias entre o tubo e o detector. A Figura \ref{fig:imgCap3FourierSliceTheorem3D} ilustra a utilização do teorema na técnica de tomossíntese para a análise das projeções no domínio da frequência \cite[p. 101-106]{mertelmeier2014filtered}.

\begin{figure}[H]
	\caption{Ilustração do teorema do corte de Fourier aproximado para a técnica de tomossíntese.}
	\begin{center}
		\includegraphics[scale=1.3]{imgs/cap3/FourierSliceTheorem3D.pdf}
	\end{center}
	\legend{Fonte: Adaptado de \citeonline[p. 102]{mertelmeier2014filtered}.}
	\label{fig:imgCap3FourierSliceTheorem3D}
\end{figure} 

O algoritmo de \citeonline[]{feldkamp1984practical} é amplamente utilizado para \acs{CT} e \acs{DBT}. De maneira geral, o método apresenta uma aproximação do algoritmo de \acs{FBP} utilizado em \acs{CT} proveniente da geometria de feixe em formato de leque (\textit{fan-beam}) para uma geometria de feixe cônico (\textit{cone-beam}) \cite{fessler2014fundamentals}. Este algoritmo foi implementado em tomossíntese por \citeonline[]{wu2004comparison}. Basicamente, a implementação da técnica para tomossíntese ocorre em 5 passos \cite[p. 16]{xu2014tomographic}, como descrito a seguir:   

\begin{enumerate}
	\item Transformada de Fourier da projeção em cada linha paralela a trajetória do tubo;
	\item Aplicar o filtro de rampa no domínio da frequência;
	\item Aplicar o filtro de ``janelamento'' para redução de ruído e artefatos;
	\item Transformada inversa de Fourier das projeções filtradas;
	\item Retroprojetar os dados no domínio espacial.   
\end{enumerate} 

%%%%%%%%%%%%%%%%%%%%%%%%%%%%%%%%%%%%%%%%%%%%%%%%%%%%%%%%%%%%%%%%%%%%%%%%%%%%%%%%%%%%%%%%%%%%%%%%%%%%%%%%%%%%%%%%%%%%%%%%%%%%%%											Reconstrução Iterativa    														%
%%%%%%%%%%%%%%%%%%%%%%%%%%%%%%%%%%%%%%%%%%%%%%%%%%%%%%%%%%%%%%%%%%%%%%%%%%%%%%%%%%%%%%%%%%%%%%%%%%%%%%%%%%%%%%%%%%%%%%%%%%%%%%

\subsection{Reconstrução Iterativa}

Como mencionado, os algoritmos de \acs{FBP} têm sido os mais utilizados atualmente em tomossíntese da mama \cite{michell2018role}, pelo fato de serem rápidos e obterem resultados aproximados que são considerados bons \cite{das2011penalized}. Entretanto devido a baixa amostragem no domínio da frequência, esses algoritmos introduzem erros na reconstrução e sofrem dificuldades de confecção dos filtros no domínio de Fourier \cite{xu2015statistical}.

Os métodos iterativos demandam um alto custo computacional para a resolução dos problemas de reconstrução. Devido a isto, os mesmos não eram utilizados por equipamentos comerciais no passado, sendo somente alvos de pesquisas científicas. Com o avanço do poder computacional essas técnicas têm chamado a atenção para aplicações em reconstrução de imagens médicas e demonstram ser promissoras para \acs{DBT} \cite{zeng2010medical,zheng2018detector}.

Para solucionar o problema de maneira iterativa é necessário primeiramente a discretização do modelo físico do sistema, como foi dito na seção \ref{DiscretizaçãodaImagem}. Após essa etapa, faz-se o uso de equações lineares para o modelamento e resolução do problema \cite[p. 125]{zeng2010medical}. Cada \textit{voxel} é denotado por $f_{j} \mid (j=1,2,...,N)$ e cada raio da projeção por $p_{i} \mid (i=1,2,...,M)$. A Figura \ref{fig:imgCap3SistemaMatriz} ilustra um exemplo da retirada de um plano vertical da Figura \ref{fig:imgCap33DGeometry}, onde cada elemento do detector recebe somente um feixe de raio X  e cada \textit{voxel} possui tamanho unitário.  

\begin{figure}[H]
	\caption{Ilustração do princípio de reconstrução iterativa através de equações lineares simples.}
	\begin{center}
		\includegraphics[scale=0.55, clip, trim=10cm 3cm 10cm 2.3cm]{imgs/cap3/SistemaMatriz.pdf}
	\end{center}
	\legend{Fonte: Adaptado de \citeonline[p. 202]{buzug2008computed}.}
	\label{fig:imgCap3SistemaMatriz}
\end{figure} 

Pode-se então relacionar o valor de cada projeção com os valores dos \textit{voxels}, nos quais cada feixe de raio interage, através das seguintes equações lineares:

\begin{equation}
\begin{cases}
f_{1} \, + f_{3} \, = p_{1}, \\ 
f_{2} \, + f_{4} \, = p_{2}, \\ 
\sqrt{2} \, f_{1} \, + \sqrt{2} \, f_{4} \, = p_{3}, \\ 
f_{1} \, + f_{2} \, = p_{4}, \\ 
\end{cases}
\label{eq:eqCap3EquacoesLineares}
\end{equation} 

\noindent sendo também possível reescrever o sistema acima na sua forma matricial, representado pela equação abaixo:

\begin{equation}
p \, = A \, f,
\label{eq:eqCap3MatrizEquacoesLineares1}
\end{equation}

\noindent na qual $p = [p_{1},p_{2},p_{3},...,p_{M}]^{T}$ é um vetor coluna que representa os valores do sinograma no espaço de Radon, ou seja as projeções, {$f = [f_{1},f_{2},f_{3},...,f_{N}]^{T}$ é um vetor coluna que representa os coeficientes de atenuação dos \textit{voxels} no espaço \acs{3D} e $A$ é uma matriz $MxN$, onde cada elemento $a_{ij}$ representa a contribuição que o ``j-ésimo'' \textit{voxel} $f_{j}$ apresenta para a atenuação do feixe de raio X que forma a ``i-ésima'' projeção $p_{i}$, por exemplo, o valor $\sqrt{2}$ em $f_{1}$ e $f_{4}$ para $p_{3}$. Tratando-se da matriz $A$, sua dimensão $M$ significa o número de elementos detectores vezes o número de projeções e $N$ representa o número total de \textit{voxels} do objeto \cite{zeng2010medical,levakhina2014three}.

Como o objetivo da reconstrução é encontrar os coeficientes de atenuação do objeto, deve-se então solucionar $f$. Para isso, se a matriz $A$ é inversível, a imagem reconstruída pode ser facilmente encontrada algebricamente através da equação abaixo:

 \begin{equation}
 f \, = A^{-1} \, p,
 \label{eq:eqCap3MatrizEquacoesLineares2}
 \end{equation}      

\noindent porém na prática esse processo é inviável devido a reconstrução tomográfica ser um problema mal condicionado e ao grande número de equações no sistema, em outras palavras, a matriz $A$ é muito grande e esparsa \cite{levakhina2014three}. Considerando o que foi dito, a melhor alternativa é encontrar uma \textbf{solução aproximada} através de métodos matemáticos de otimização de problemas de uma forma iterativa \cite{buzug2008computed}.  

Para isso, deve-se otimizar uma função $E(x)$, normalmente chamada de função de custo, objetivo ou de energia, com o intuito de encontrar a melhor solução $\hat{x}$, ou a mais aproximada. Assim, é necessário minimizar ou maximizar a função $E(x)$, de acordo com as equações a seguir:
 
\begin{equation}
\hat{x} = \underset{}{\arg\min} \; E(x) \;\;\;\; \text{ou} \;\;\;\; \hat{x} = \underset{}{\arg\max} \; E(x).
\label{eq:eqCap3Minimizacao}
\end{equation} 

Segundo \citeonline[p. 206]{buzug2008computed}, uma solução generalizada para a Equação \ref{eq:eqCap3MatrizEquacoesLineares1} é dada pela minimização de \eqref{eq:eqCap3MinimosQuadrados} dita como norma mínima dos mínimos quadrados, caracterizando-se por um problema de otimização:

\begin{equation}
\chi^{2} \, = \left\| A \, f \, - \,p \right\|^{2},
\label{eq:eqCap3MinimosQuadrados}
\end{equation} 

\noindent onde $E(x) = \chi^{2}$, ou seja, \eqref{eq:eqCap3MinimosQuadrados} é a função de energia. Outras técnicas são encontradas na literatura para a formulação e otimização das funções de custo, tais como métodos de gradientes descendentes, iterativos algébricos e estatísticos, os quais são abordados nos próximos itens.

%%%%%%%%%%%%%%%%%%%%%%%%%%%%%%%%%%%%%%%%%%%%%%%%%%%%%%%%%%%%%%%%%%%%%%%%%%%%%%%%%%%%%%%%%%%%%%%%%%%%%%%%%%%%%%%%%%%%%%%%%%%%%%						               Reconstrução Iterativa (Método Algébrico)    										%
%%%%%%%%%%%%%%%%%%%%%%%%%%%%%%%%%%%%%%%%%%%%%%%%%%%%%%%%%%%%%%%%%%%%%%%%%%%%%%%%%%%%%%%%%%%%%%%%%%%%%%%%%%%%%%%%%%%%%%%%%%%%%%

\subsubsection{Método Algébrico}

Como já foi dito, para a resolver o problema de reconstrução é necessário encontrar a solução para as equações demonstradas em \eqref{eq:eqCap3EquacoesLineares}, porém as aplicações práticas tendem a ter uma enorme quantidade de equações, tornando inviável a resolução por procedimentos usuais \cite[p. 210]{buzug2008computed}. 

Baseado nisso, um método comumente encontrado na literatura é a técnica de reconstrução algébrica (\textit{Algebraic Reconstruction Technique} - \acs{ART}). Esse método visa resolver iterativamente um sistema enorme de equações lineares satisfazendo-as uma a uma, em outras palavras, estima-se uma imagem (coeficientes de atenuação) que solucione uma equação por vez \cite{rangayyan2004biomedical}. 

Segundo \citeonline[p. 211]{buzug2008computed}, esse método idealmente estima que a solução da imagem, i.e. o vetor {$f = [f_{1},f_{2},f_{3},...,f_{N}]^{T}$, seja um ponto no espaço $\mathcal{N}$ dimensional, onde as $M$ equações \eqref{eq:eqCap3EquacoesLineares} sejam hiperplanos que cruzam esse ponto, dado como solução.

Acompanhando a linha do autor, o exemplo abaixo ilustra como é dada a solução do problema através do método especificado. Para o exemplo, o mesmo é simplificado para $N = 2$, ou seja, são considerados somente dois \textit{pixels} e cada um é interceptado por somente um feixe de raio, tendo como resultado duas projeções $M=2$. Através disso, as equações podem ser modeladas seguindo o mesmo raciocínio de \eqref{eq:eqCap3EquacoesLineares}, demonstradas a seguir:

\begin{equation}
\begin{cases}
a_{11}f_{1} \, + a_{12}f_{2} \, = p_{1} \\ 
a_{21}f_{1} \, + a_{22}f_{2} \, = p_{2} \\  
\end{cases}.
\label{eq:eqCap3ARTEquacoesLineares}
\end{equation}

Cada equação, nesse caso, simboliza uma reta no espaço bidimensional, como demonstra a Figura \ref{fig:imgCap3ART}. Uma estimativa inicial $f^{(0)}$ deve ser feita e então esse ponto deve ser projetado perpendicularmente na reta, ou hiperplano, que representa $p_{1}$ para obter a nova imagem $f^{(1)}$. O processo volta a se repetir, projetando $f^{(1)}$ no hiperplano definido pela equação de $p_{2}$ e assim até convergir para a solução desejada $f = (f_{1},f_{2})^{T}$, com um número de $k$ iterações.


\begin{figure}[h]
	\caption{Ilustração geométrica do processo de reconstrução através do método \acs{ART}.}
	\begin{center}
		\includegraphics[scale=1]{imgs/cap3/ART1.pdf}
	\end{center}
	\legend{Fonte: \citeonline[p. 212]{buzug2008computed}.}
	\label{fig:imgCap3ART}
\end{figure} 

A equação abaixo demonstra a fórmula para o cálculo do método \acs{ART}, na qual os \textit{voxels} referentes ao raio $i$ são atualizados ao final do processamento desse mesmo raio, ou seja, a atualização de $\hat{f}$ é feita raio por raio \cite{buzug2008computed, levakhina2014three}:


\begin{equation}
f^{k} = f^{k-1} - \dfrac{(a_{i \, :} \,\, f^{k-1}- p_{i})}{a_{i\, :}\,(a_{i\, :})^{T}}   (a_{i\, :})^{T}   ,  
\label{eq:eqCap3ARTEquacao}
\end{equation}

\noindent na qual $a_{i \, :}$ é um vetor da matriz A contendo todos os elementos j da ``i-ésima'' linha. 

%\begin{equation}
%f^{Próximo} = f^{Atual} - Retroprojeção_{raio}\{\dfrac{Projeção_{raio}(f^{Atual})- Medição_{raio}}{Fator\,Normalização}\}.  
%\label{eq:eqCap3ARTEquacoesSimbolica}
%\end{equation}

Os métodos ditados acima assumem um caso ideal. As projeções dos feixes em sistemas reais sofrem de inconsistências de ruídos e artefatos, bem como na prática o problema é constituído de inúmeras equações. Dado isso, a solução para as equações lineares não é encontrada em somente um ponto, mas sim em uma região que possui múltiplas soluções. Nesse caso uma solução aproximada $\hat{f}$ deve ser estimada no sistema caracterizado como possível e indeterminado \cite[p. 218]{buzug2008computed}. A Figura \ref{fig:imgCap3ARTIndeterminado} exemplifica o caso acima. 

\begin{figure}[h]
	\caption{Ilustração geométrica de como ocorre a reconstrução de um sistema possível indeterminado por meio do método \acs{ART}.}
	\begin{center}
		\includegraphics[scale=1]{imgs/cap3/ART2.pdf}
	\end{center}
	\legend{Fonte: \citeonline[p. 218]{buzug2008computed}.}
	\label{fig:imgCap3ARTIndeterminado}
\end{figure}

Variações do método algébrico, dito acima, são encontradas na literatura. O algoritmo que utiliza a técnica de reconstrução iterativa simultânea (\textit{Simultaneous Iterative Reconstruction Technique} - \acs{SIRT}) é uma modificação do método convencional algébrico que atualiza a estimativa da imagem $f$ somente depois do processamento de todas as projeções, ao contrário do \acs{ART} que atualiza a cada raio \cite{zhang2006comparative,yang2012numerical,zeng2010medical}. Já o método que utiliza a técnica de reconstrução algébrica simultânea (\textit{Simultaneous Algebraic Reconstruction Technique} - \acs{SART}), atualiza $f$ ao final de cada projeção geométrica \cite{zhang2006comparative,levakhina2014three,yang2012numerical}.   

%%%%%%%%%%%%%%%%%%%%%%%%%%%%%%%%%%%%%%%%%%%%%%%%%%%%%%%%%%%%%%%%%%%%%%%%%%%%%%%%%%%%%%%%%%%%%%%%%%%%%%%%%%%%%%%%%%%%%%%%%%%%%%						               Reconstrução Iterativa (Método Estatístico)    										%
%%%%%%%%%%%%%%%%%%%%%%%%%%%%%%%%%%%%%%%%%%%%%%%%%%%%%%%%%%%%%%%%%%%%%%%%%%%%%%%%%%%%%%%%%%%%%%%%%%%%%%%%%%%%%%%%%%%%%%%%%%%%%%

\subsubsection{Método Estatístico}\label{MétodoEstatístico}

Os problemas de reconstrução também podem ser resolvidos por meio de métodos estatísticos que são equivalentes aos problemas de otimização \cite[p. 79]{levakhina2014three}. O algoritmo da máxima verossimilhança (\textit{Maximum Likelihood} - \acs{ML}) é um exemplo de modelamento estatístico. Nesse caso, a função de custo torna-se a função de verossimilhança \cite[p. 77]{levakhina2014three}.

Segundo \citeonline[p. 10]{Fessler2000handbook}, esse método busca estimar um parâmetro, por exemplo os coeficientes de atenuação $(ImgEstimada)$, que maximizam a verossimilhança dado um conjunto de observações $(Medidas)$. Essa estimativa pode ser modelada através da função de verossimilhança representada por meio da Equação \ref{eq:eqCap3ModeloVerossimilhança} e explicada com detalhes a seguir:

\begin{equation}
ImgReconstruida = \underset{}{\arg\max} \;\; l(ImgEstimada \mid Medidas),
\label{eq:eqCap3ModeloVerossimilhança}
\end{equation} 

\noindent onde $l(ImgEstimada)$ é a função de verossimilhança a ser otimizada tendo as medidas das projeções.


De acordo com \citeonline[p. 230]{buzug2008computed}, em um equipamento que utiliza raios X de transmissão, o número de \textit{quanta} emitido pelo tubo e recebido pelo detector segue a distribuição de Poisson, descrita matematicamente por:

\begin{equation}
Prob(Y_{i} = y_{i} \mid \hat{y_{i}}(f) ) = \dfrac{e^{-\hat{y_{i}}(f) } \, [\hat{y_{i}}(f)] ^{y_{i}}}  {y_{i}!},
\label{eq:eqCap3DistribuicaoPoisson}
\end{equation}

\noindent tal que $Y_{i}$ é uma variável aleatória contando o número de \textit{quanta} em cada observação da projeção $y_{i}$. O valor observado $y_{i}$ difere do valor esperado $\hat{y_{i}}$ pelo fato do detector ser um contador de fótons que obedece a distribuição descrita acima, por radiações espalhadas de Comptom, \textit{crosstalk}\footnote{Interferência entre \textit{pixels} vizinhos que ocorrem em detectores indiretos de \acs{DBT}, causados pela difusão da luz no fósforo ou através dos cintiladores \cite{zheng2018detector}.} entre os detectores e também pelo tamanho finito dos mesmos, contrariando as situações ideais da Equação \ref{eq:eqCap3BeerLambert1} \cite[p. 6]{Fessler2000handbook}. Esse valor esperado pode ser calculado segundo uma aproximação da lei de Beer-Lambert, demonstrada abaixo \cite[p. 9]{Fessler2000handbook}:

\begin{equation}
\hat{y_{i}}(f) = b_{i} \, e \, ^{-\hat{p_{i}}(f)} \, + r_{i},
\label{eq:eqCap3BeerLambertDiscreto2}
\end{equation}

\noindent em que $\hat{p_{i}}(f)$ é a função de atenuação que cada raio sofre, $b_{i}$ é o número médio de \textit{quanta} recebido pelo detector caso não exista nenhum objeto entre o tubo e o detector e $r_{i}$ modela os fatores externos mencionados acima. 

A função $\hat{p_{i}}(f)$ é calculada pelo somatório dos produtos da contribuição $a_{ij}$ para a atenuação $f_{j}$ de todos os \textit{voxels} $j$ no percurso do feixe de raio $i$, dada por: 


\begin{equation}
\hat{p_{i}}(f) = \sum_{j=1}^{N} a_{ij} \, f_{j}.
\label{eq:eqCap3BeerEsperancaAtenuacao}
\end{equation}

A probabilidade conjunta de uma amostra aleatória $P$ obter uma observação $p \mid (p = {y_{1},y_{2},y_{3},...,y_{M}})$ das projeções independentes $y_{i}$, dado o valor esperado $\hat{y_{i}}(f)$, é dada pela seguinte equação:   

\begin{equation}
Prob(P = p\mid f ) = \prod_{i=1}^{M}  \dfrac{e^{-\hat{y_{i}}(f)} \, [\hat{y_{i}}(f)]^{y_{i}}}  {y_{i}!},
\label{eq:eqCap3DistribuicaoPoissonConjunta}
\end{equation}

\noindent porém segundo \citeonline[p. 231]{buzug2008computed}, não faz sentido analisar a probabilidade de uma projeção $p$, ou seja uma observação, dado o valor esperado $\hat{y_{i}}(f)$ calculado através dos coeficientes de atenuação $f$, já que o intuito da resolução do problema é estimar os valores desses coeficientes $\hat{f}$. Nesse sentido, pode-se fazer o uso da \textbf{função de verossimilhança} para a resolução do problema.

A \textbf{verossimilhança} diz que uma observação qualquer $Y = y$ é uma evidência que favorece a hipótese A sobre a hipótese B, dada as duas hipóteses inicialmente, se a condição abaixo é satisfeita \cite{morettin2010}:

\begin{equation}
Prob(Y=y_{A}) >Prob(Y=y_{B}).
\label{eq:eqCap3LeiVerossimilhanca}
\end{equation}

Existe uma diferença entre probabilidade e verossimilhança. Quando em uma função de distribuição de probabilidade, e.g. a Equação \ref{eq:eqCap3DistribuicaoPoisson}, o parâmetro fixo é o valor esperado $\hat{y_{i}}(f)$ e o variável é a observação $y_{i}$, esta é denominada de função de probabilidade. Já se a observação é fixa e o parâmetro esperado é variável denomina-se \textbf{função de verossimilhança} \cite{morettin2010}.

Dito esses conceitos junto ao problema encontrado com a Equação \ref{eq:eqCap3DistribuicaoPoissonConjunta}, é plausível a utilização da função de verossimilhança para a resolução do mesmo, tendo como resultado a equação abaixo: 

\begin{equation}
L(f \mid p) = \prod_{i=1}^{M}  \dfrac{e^{-\hat{y_{i}}(f)} \, [\hat{y_{i}}(f)]^{y_{i}}}  {y_{i}!},
\label{eq:eqCap3DistribuicaoPoissonConjuntaVerossimilhanca1}
\end{equation} 
  
Sendo assim, a finalidade da função acima é de variar os valores de $f$ buscando o valor ótimo de verossimilhança $L$, dado um conjunto de observações $p$ \cite[p. 231]{buzug2008computed}. A busca pelo valor máximo da equação está intimamente ligada com o conceito de verossimilhança ditado acima, que busca a melhor hipótese $\hat{f}$ para os valores de $f$ dado o conjunto de observações $p$, como ilustra a figura abaixo. 


\begin{figure}[H]
	\caption{Ilustração de busca pela máxima verossimilhança.}
	\begin{center}
		\includegraphics[scale=0.4]{imgs/cap3/Verossimilhanca.pdf}
	\end{center}
	\legend{Fonte: do autor, 2019.}
	\label{fig:imgCap3Verossimilhanca}
\end{figure}   

Substituindo a Equação \ref{eq:eqCap3BeerLambertDiscreto2} na Equação \ref{eq:eqCap3DistribuicaoPoissonConjuntaVerossimilhanca1} e fazendo o uso da função logaritmo a fim de simplificar os cálculos e por conveniência, tem-se a Equação \ref{eq:eqCap3DistribuicaoPoissonConjuntaVerossimilhanca2}, desprezando constantes independentes de $f$ \cite[p. 10]{Fessler2000handbook}. 

\begin{equation}
l(f \mid p) = \log (L(f \mid p)) = \sum_{i=1}^{M} \, y_{i} \log (b_{i} \, e \, ^{-\hat{p_{i}}(f)} \, + r_{i}) \,-\,(b_{i} \, e \, ^{-\hat{p_{i}}(f)} \, + r_{i}),
\label{eq:eqCap3DistribuicaoPoissonConjuntaVerossimilhanca2}
\end{equation} 

\noindent onde a equação acima é a log-verossimilhança para \acs{CT} de transmissão e reescrevendo-a de maneira mais simples abaixo, podendo fazer uma analogia com a Equação \ref{eq:eqCap3ModeloVerossimilhança}.

\begin{equation}
\hat{f} = \underset{f\geq 0}{\arg\max} \; l(f \mid p),
\label{eq:eqCap3VerossimilhançaFinal}
\end{equation}

Ainda de acordo com \citeonline[p. 11]{Fessler2000handbook}, devido ao problema de reconstrução tomográfica ser mal condicionado, a maximização de $l(f)$ por si só leva a resultados de imagens ruidosas. Isso porque existe um grande conjunto de mapas de atenuação que se encaixam bem, como o resultado da equação. Então, segundo o autor, a função de verossimilhança por si só não consegue encontrar o melhor resultado. Uma das soluções encontradas para esse problema, que é vastamente conhecida na literatura, é a introdução de um termo de regularização na função objetivo, ou seja, incorporar informações já conhecidas para limitar o espaço de soluções de um vasto conjunto para um subconjunto menor. Então a maximização da sua nova forma ``penalizada'' é feita para encontrar a melhor solução contida nesse subconjunto, como é demonstrado abaixo:     

\begin{equation}
\hat{f} = \underset{f\geq 0}{\arg\max} \; \Phi(f), \;\; \text{onde} \;\; \Phi(f) = l(f) - \beta R(f),
\label{eq:eqCap3VerossimilhançaPenalizada}
\end{equation}

\noindent ou simbolicamente:

\begin{equation}
(\text{Função \textit{a posteriori}}) = (\text{Função Log-Verossimilhança}) - \beta (\text{Função \textit{a priori}}).
\label{eq:eqCap3VerossimilhançaPenalizadaSimbolica}
\end{equation}

Nessa notação, $R(f)$ modela a função de regularização do sistema, o fator $\beta$ controla a força de influência dessa função e $l(f)$ é a função objetivo formulada anteriormente. O algoritmos que introduz essa técnica é conhecido como \ac{PML}. 

A formulação de \eqref{eq:eqCap3VerossimilhançaPenalizada} pode ser dada também a partir do modelo Bayesiano, ou teorema de Bayes, demonstrado abaixo, que utiliza de conhecimentos de probabilidade \textit{a priori} para a estimativa de parâmetros \textit{a posteriori}, dado as probabilidades condicionais \cite[p. 144]{zeng2010medical}. Esses algoritmos são conhecidos como métodos \textit{Bayesianos} ou \ac{MAP}. O cálculo da estimativa \acs{MAP} ocorre através do teorema de Bayes:

\begin{equation}
Prob(f \mid p) = \dfrac{Prob(p \mid f) Prob(f)}{Prob(p)},
\label{eq:eqCap3TeoremaBayes}
\end{equation}  

\noindent tal qual,  aplicando o logaritmo na equação acima, tem-se \eqref{eq:eqCap3TeoremaBayesLog}, onde o elemento que não depende do coeficiente de atenuação $f$ é desprezado, gerando \eqref{eq:eqCap3TeoremaBayesLogSimplificada}:

\begin{equation}
\log(Prob(f \mid p)) = \log(Prob(p \mid f)) + \log(Prob(f)) - \log(Prob(p)),
\label{eq:eqCap3TeoremaBayesLog}
\end{equation}

\begin{equation}
\log(Prob(f \mid p)) = \log(Prob(p \mid f)) + \log(Prob(f)),
\label{eq:eqCap3TeoremaBayesLogSimplificada}
\end{equation}

\noindent onde o primeiro termo é o modelo estatístico da aquisição física do sistema, i.e. função de máxima verossimilhança \eqref{eq:eqCap3VerossimilhançaFinal}, e o segundo termo é o modelo estatístico que faz a restrição do sistema para que este possa convergir para um solução apropriada. Simplificadamente tem-se a equação abaixo, sendo possível sua assimilação com a Equação \ref{eq:eqCap3VerossimilhançaPenalizadaSimbolica}.

\begin{equation}
\Phi(f) = l(f) - \beta R(f).
\label{eq:eqCap3TeoremaBayesLogSimplificadaFinal}
\end{equation}

É conhecido da literatura que os mapas de atenuações não variam bruscamente em uma pequena faixa de extensão, ou seja, os níveis de cinza entre os \textit{pixels} vizinhos não alteram significantemente em relação à média local \cite{Fessler2000handbook,buzug2008computed}. Essa informação pode ser considerada como um conhecimento \textit{a priori} sobre a imagem desejada e a partir desse fato, pode ser incorporado uma equação que desestimule a ``rispidez'' sobre a imagem processada, impondo restrições de suavidade. Sendo assim, essa equação pode ser modelada através de funções matemáticas e incorporada como uma restrição para o conjunto de soluções do modelo estatístico.

Um das maneiras mais simples de modelar uma função de restrição de suavidade na imagem é através da discrepância existente entre os valores dos \textit{pixels} vizinhos \cite[p. 12]{Fessler2000handbook}. Esse modelamento pode ser feito de acordo com \citeonline[p. 4]{das2011penalized}:

\begin{equation}
R(f) = \sum_{j=1}^{N}  \sum_{k=1}^{N_{i}} \omega_{jk} \, \psi(f_{j}-f_{k}),
\label{eq:eqCap3DiscrepanciaPixels}
\end{equation}  

\noindent no qual $f$ é o coeficiente de atenuação, $\psi$ mede a discrepância entre $f_{j}$ e $f_{k}$, $N_{i}$ é o número de \textit{voxels} vizinhos, $\omega_{jk} = \omega_{kj}$ define a influência do ``k-ésimo'' \textit{voxel} no ``j-ésimo'' \textit{voxel}. A influência de cada \textit{voxel} em seu vizinho é definida por: $\omega_{jk} = 1$ para os seis \textit{voxels} vizinhos, sendo dois em cada direção do eixo, $\omega_{jk} = 1/\sqrt{2}$ para os diagonais e $\omega_{jk} = 0$ para qualquer outro. A Figura \ref{fig:imgCap3PesosVizinhanca} exemplifica os pesos dados a cada vizinho, em um espaço \acs{2D}, sendo que esses pesos são normalizados.


\begin{figure}[H]
	\centering
	
	\caption{Pesos normalizados atribuídos para a vizinhança (a) quatro e (b) oito.}
	
	\subfloat[]{\includegraphics[scale=0.8]{imgs/cap3/PesoVizinhancaB.pdf}	\label{fig:imgCap3PesosVizinhancaA}}
	
	\subfloat[]{\includegraphics[scale=0.8]{imgs/cap3/PesoVizinhancaA.pdf} \label{fig:imgCap3PesosVizinhancaB}.}
	
	\legend{Fonte: \citeonline[p. 2]{chen2009bayesian}.}
	\label{fig:imgCap3PesosVizinhanca}
\end{figure}   

%\rv{Corta daqui pra baixo}
%
%Ainda no contexto da restrição de suavidade, outra técnica utilizada é o modelamento da função de restrição por meio do Campo Aleatório Markoviano (Markov Random Field - \acs{MRF}), através do uso de distribuição de probabilidades que levam em conta o contexto espacial dos \textit{pixels} \cite[p. 71]{salvadeo2013filtragem}.  
%
%Segundo a definição de \citeonline[p. 8]{li2009markov}, a teoria do Campo Aleatório Markoviano é um ramo da teoria de probabilidade que tem como função analisar as \textbf{dependências espaciais} ou contextuais de dado \textbf{fenômeno físico}. 
%
%Com a finalidade de explicar todo o conceito por trás dessa teoria, é necessário a utilização de notações formais que auxiliarão todo o processo de formulação. Todas essas notações estão contidas no Apêndice \ref{ApendiceA:CamposAleatóriosMarkovianos}, junto com conceitos teóricos importantes a serem definidos. Recomenda-se ao leitor investir um tempo analisando esse capítulo a fim de compreender melhor a teoria descrita com detalhes a seguir.    
%
%Dado todos os conceitos apresentados, um campo aleatório $F$ (\ref{ApendiceA:CampoAleatorio}) em $S$ (\ref{ApendiceA:SitesRotulos}) em relação a sua vizinhança $\nu$ (\ref{ApendiceA:sVizinhancaCliques}) é dito ser de Markov quando o campo atende as duas condições ditadas abaixo:  
%
%
%\begin{enumerate}
%	\item \textbf{Positividade}: $P(f) > 0, \;\; \forall\, f \,\in\, \varPsi$,
%	\item \textbf{Markovianidade}: $P(f_{i} \mid f_{j},\;\; \forall \; j \in S,\; j \neq i) = P(f_{i}\mid f_{\nu_{i}})$,
%\end{enumerate}
%
%\noindent onde $f_{\nu_{i}} = \{f_{i^{'}} \mid i^{'} \in \nu_{i} \} $ é o conjunto de rótulos admitidos para os \textit{sites} na vizinhança de $i$. Em outras palavras, a \textbf{Markovianidade} dita as características locais do campo, onde somente a vizinhança tem influência direta no valor do \textit{site}, ou \textit{pixel} nesse caso. Já a \textbf{positividade}, quando satisfeita, diz que a probabilidade conjunta do campo $P(f)$ é determinada pelas probabilidades condicionais locais do campo \cite{li2009markov}.  
%
%De acordo com \citeonline[p. 4]{won2013stochastic}, os \acs{MRF} em \acs{2D} são uma generalização não-causal das cadeias de Markov \acs{1D} proveniente das análises de sequências. Devido a representação de uma dependência de características locais do campo com a sua vizinhança de um modo não-causal, a probabilidade conjunta não pode ser fatorada\footnote{A fatoração é a conexão entre a probabilidade conjunta e a probabilidade condicional local, segundo \citeonline[p. 18]{won2013stochastic}} em probabilidades das características locais como é feito nas cadeias de Markov \acs{1D}. Tendo isso em mente é necessário a busca de um método que especifique a probabilidade conjunta de um \acs{MRF} por meio de suas características locais, ou seja, probabilidades condicionais locais. 
%
%Isso só foi possível devido a equivalência entre o \acs{MRF} e o Campo Aleatório de Gibbs (\textit{Gibbs Random Field} - \acs{GRF}) dada pelo teorema de \citeonline[]{hammersley1971markov}. Com isso foi possível estabelecer uma relação entre a \textbf{probabilidade conjunta global} (\acs{GRF}) e a \textbf{probabilidade condicional local} (\acs{MRF}) e vice-versa.  
%
%O teorema mencionado acima diz que: \textit{$F$ é um \acs{MRF} em $S$, dado uma vizinhança $\nu$, se e somente se $F$ é um \acs{GRF} em $S$, dado uma vizinhança $\nu$}. Em outras palavras, $F$ é dito ser de Markov somente se a sua probabilidade conjunta $P(f)$ seguir a distribuição de Gibbs (Apêndice \ref{ApendiceB:CamposAleatóriosdeGibbs}).
%
%Diversos modelos distintos de \acs{MRF} podem ser encontrados na literatura \cite{li2009markov,won2013stochastic}, sendo que cada um é utilizado para uma determinada aplicação de acordo com suas funções potenciais. O modelo Gaussiano, denominado \ac{GMRF}, é bastante utilizado em problemas de restauração e reconstrução de imagens \cite{salvadeo2016nonlocal,xu2015statistical,jeng1991compound}. Sua equação de probabilidade condicional local é definida segundo \citeonline[p. 18]{li2009markov}:
%
%\begin{equation}
%P(f_{i} \mid f_{\nu_{i}}) = \dfrac{1}{\sqrt{2 \pi \sigma^{2}_{i}}}  e^{-\frac{1}{2\sigma^{2}_{i}} \, [f_{i}-\mu_{i}-\sum_{i^{'}\in \,\nu_{i}}^{} \beta_{i,i^{'}}(f_{i^{'}} - \mu_{i^{'}}) ]^{2} },
%\label{eq:eqCap3GMRF1}
%\end{equation} 
%
%\noindent onde $\sigma^{2}_{i}$ e $\mu_{i}$ correspondem a variância e a média local respectivamente e $\beta_{i,i^{'}}$ corresponde aos coeficientes de interação dos cliques. 
%
%A probabilidade conjunta é dada de maneira generalizada segundo \citeonline[p. 4]{xu2015statistical}:
%
%\begin{equation}
%P(f) \sim \prod_{i}^{N} \prod_{i^{'}}^{\nu_{i}} e^{\rho(\varDelta_{ii^{'}})},
%\label{eq:eqCap3GMRF2}
%\end{equation} 
%
%\noindent onde: 
%
%\begin{equation}
%\rho(\varDelta_{ii^{'}}) = - \omega_{ii^{'}} \dfrac{\varDelta_{ii^{'}}^{2}}{2\sigma^{2}},
%\label{eq:eqCap3GMRF3}
%\end{equation}
%
%\noindent $\varDelta_{ii^{'}}^{2} = (f_{i}-f_{i^{'}})^{2}$ é a função potencial quadrática e $\omega_{ii^{'}}$ são os pesos da vizinhança, calculados pelo inverso da distância euclidiana entre o \textit{pixel} central e seus vizinhos, dado por:
%
%\begin{equation}
%\omega_{ii^{'}} = \dfrac{1}{((x_{i}-x_{i^{'}})^{2}+((y_{i}-y_{i^{'}}))^{2})^{1/2}}.
%\label{eq:eqCap3GMRF4}
%\end{equation}
%
%No ano de 2005, o trabalho de \citeonline[]{buades2005non} propôs a filtragem de imagens corrompidas por ruído Gaussiano por meio de um filtro da média não-local. Esse filtro explora o fato de que imagens em sua grande maioria possuem um alto grau de redundância. Basicamente, a filtragem é feita através da similaridade entre conjuntos de \textit{pixels}, denominados \textit{patches}, ao invés de ser feita somente com a vizinhança local como era anteriormente, em outras palavras, a remoção do ruído da imagem $I$ se dá através de uma média ponderada $w_{st}$ entre os \textit{patches} similares $P_{s}\, \text{e}\, P_{t} \, \forall \, t \in SW_{s}$. O \textit{patch} $P_{s}$ de um determinado \textit{pixel} $i_{s}$ pode ser caracterizado por uma região quadrada ao entorno daquele \textit{pixel}, e.g. $7x7$. Buscando a redução do custo computacional do método, uma janela de busca $SW$, e.g. $17x17$, é estipulada para a comparação dos \textit{patches}. O valor estimado $\hat{i}_{s}$ do \textit{pixel} livre de ruído é dado por meio de \eqref{eq:eqCap3NLM}, para todo $s$ pertencente a imagem $I$. A Figura \ref{fig:imgCap3Patches} ilustra como é feito o procedimento não-local. 
%
%\begin{equation}
%NLM(\hat{i}_{s}) = \sum_{t \,\in\, SW_{s}}^{} w_{st} \cdot i_{t},
%\label{eq:eqCap3NLM}
%\end{equation}
%
%
%%\begin{figure}[H]
%%	\caption{Ilustração do procedimento não-local baseado na similaridade entre os \textit{patches}.}
%%	\begin{center}
%%		\includegraphics[scale=1.3]{imgs/cap3/Patches.pdf}
%%	\end{center}
%%	\legend{Fonte: \citeonline[p. 2]{salvadeo2016nonlocal}}
%%	\label{fig:imgCap3Patches}
%%\end{figure}
%
%\noindent onde o peso da similaridade $w_{st}$ entre os \textit{patches} é obtido por meio de \eqref{eq:eqCap3NLMw}, calculado pela distância Euclidiana entre os vetores $P_{s}\, \text{e}\, P_{t}$ de forma normalizada. É importante ressaltar que a soma dos pesos dentro da janela deve somar 1, em outras palavras, $\sum_{t\,\in \,SW_{s}}^{}w_{st} = 1$, o que leva a normalização dada pelo denominador da equação abaixo.
%
%\begin{equation}
%w(s,t) = \dfrac{e^{-\|P_{s} - P_{t}\|^{2}/h^{2}}}{\sum_{t\,\in \,SW_{s}}^{}e^{-\|P_{s} - P_{t}\|^{2}/h^{2}}},
%\label{eq:eqCap3NLMw}
%\end{equation}
%
%\noindent onde $h$ é um parâmetro que controla o nível de borramento do filtro. 
%
%Baseado no conceito não-local de similaridade, o trabalho de \citeonline[]{zhang2017applications} propõe a expansão dessa técnica para problemas inversos tais como: remoção de ruído, redução de borramento, reconstrução e reconstrução com alta resolução. O trabalho de \citeonline[]{salvadeo2016nonlocal} explorou os conceitos de \acs{MRF} não-local para remover ruídos de imagens.
%
%Adicionando o conceito não-local no cálculo dos coeficientes de interação dos cliques da equação de \acs{GMRF}, temos a equação \eqref{eq:eqCap3GMRF1} modificada para:
%
%\begin{equation}
%P(f_{i} \mid f_{\nu_{i}}) = \dfrac{1}{\sqrt{2 \pi \sigma^{2}_{i}}}  e^{-\frac{1}{2\sigma^{2}_{i}} \, [f_{i}-\mu_{i}-\sum_{i^{'}\in \,\nu_{i}}^{} \omega(i,i^{'})\beta_{i,i^{'}}(f_{i^{'}} - \mu_{i^{'}}) ]^{2} },
%\label{eq:eqCap3GMRFNL}
%\end{equation}     
%
%
%\rv{Voltar a partir daqui}


Após a formulação da função objetivo e da restrição do sistema é necessário resolver essa equação para a obtenção da imagem reconstruída. Segundo \citeonline[p. 12]{Fessler2000handbook} não existe uma fórmula fechada para a resolução da Equação \ref{eq:eqCap3TeoremaBayesLogSimplificadaFinal} de forma analítica. Desse modo, o uso de algoritmos iterativos é uma alternativa para a solução do problema. Ainda segundo o autor, um algoritmo iterativo primeiro estima uma imagem inicial $f^{(0)}$ como hipótese e então recursivamente busca uma sequência $f^{(1)}, f^{(2)}, \dots , f^{(n)}$ até convergir para o máximo ou mínimo da função. A Figura \ref{fig:imgCap3ProcedimentoIterativo} ilustra o esquemático de um algoritmo iterativo.


\begin{figure}[H]
	\caption{Esquemático do procedimento geral realizado pelos algoritmos de reconstrução iterativa.}
	\begin{center}
		\includegraphics[scale=0.5, clip, trim= 7.5cm 2.8cm 5.3cm 3.5cm]{imgs/cap3/ProcedimentoIterativo.pdf}
	\end{center}
	\legend{Fonte: Adaptado de \citeonline[p. 129]{zeng2010medical}.}
	\label{fig:imgCap3ProcedimentoIterativo}
\end{figure}

Diversos algoritmos para a otimização das funções objetivo são encontrados na literatura \cite{Fessler2000handbook,das2011penalized,zeng2010medical,sidky2014iterative,xu2015statistical,zheng2018detector}. No entanto nesse trabalho somente o algoritmo de Máxima Expectativa (\textit{Expectation Maximization} - \acs{EM}) é discutido.

Considerando a constante $\beta = 0$ da Equação \ref{eq:eqCap3TeoremaBayesLogSimplificadaFinal} elimina-se a informação \textit{a priori} do sistema e então o problema volta a lidar somente com a equação de máxima verossimilhança \eqref{eq:eqCap3DistribuicaoPoissonConjuntaVerossimilhanca2}, ou do inglês \acs{ML}.  

Basicamente, seguindo o trabalho de \citeonline[p. 111]{sidky2014iterative}, a equação do método de \acs{MLEM} é dada por:

\begin{equation}
f^{k} = f^{k-1}  \left(A^{T} \, \dfrac{p}{A \, f^{k-1}}\right)   /    (A^{T}\,1),
\label{eq:eqCap3MLEM}
\end{equation}

\noindent na qual $A$ é a matriz da projeção, a sua transposta $A^{T}$ representa a retroprojeção e o fator de normalização no denominador é calculado fazendo a retroprojeção com valores de $p$ igual a 1.

%Basicamente, o algoritmo de \acs{EM} é constituído por duas etapas \cite[p. 149]{zeng2010medical}. A primeira é a substituição da variável aleatória $y_{i}$ de \eqref{eq:eqCap3DistribuicaoPoissonConjuntaVerossimilhanca2} por sua expectativa, caracterizando o passo ``E''. O próximo passo é maximizar a nova equação, igualando a derivada de $f_{j}$ a zero, sendo essa etapa caracterizada por ``M''. Daí então que é derivado o algoritmo de \acs{MLEM}. A derivação matemática dos passos apresentados pode ser encontrada em \citeonline[p. 148-150]{zeng2010medical}. A equação matemática do algoritmo de \acs{MLEM} é dada por:  \rv{Elias: EM para emissão?}
%
%\begin{equation}
%f_{j}^{Próximo} = \dfrac{f_{j}^{Atual}}{\sum_{i}^{} a_{ij}} \sum_{i}^{} a_{ij} \, \dfrac{y_{i}}{\sum_{j}^{} a_{ij}f_{j}^{Atual}},
%\label{eq:eqCap3MLEM}
%\end{equation}    
%
%\noindent ou de forma simbólicas:  
%
%\begin{equation}
%f^{Próximo} = f^{Atual} \, \dfrac{Retroprojeção\left\{\dfrac{Medição}{Projeção(f^{Atual})}\right\}}{Retroprojeção\{1\}},
%\label{eq:eqCap3MLEMSimbolico}
%\end{equation} 
%
%\noindent onde a retroprojeção do vetor 1 no denominador significa uma constante de normalização do volume.

%Para incorporar a restrição \acs{MAP} no sistema, o algoritmo exposto acima pode ser modificado para sua forma denominada \textit{Green’s one-step late} \cite[p. 151]{zeng2010medical}, adicionando o termo de penalidade no denominador, demonstrado por: 
%
% \begin{equation}
% f_{j}^{Próximo} = \dfrac{f_{j}^{Atual}}{\sum_{i}^{} a_{ij} \, + \beta \dfrac{\partial U(F^{Atual})}{ \partial f_{j}^{Atual}}} \sum_{i}^{} a_{ij} \, \dfrac{y_{i}}{\sum_{j}^{} a_{ij}f_{j}^{Atual}},
% \label{eq:eqCap3MLEMMAP}
% \end{equation}
% 
% \noindent onde $U(F)$ é a função de energia do modelo \acs{MAP}, dado nesse trabalho por \eqref{eq:eqApendiceBDistribuicaoGibbsU1} ou mais precisamente em \eqref{eq:eqCap3GMRFNL} no modelo não-local \acs{GMRF}.


%%%%%%%%%%%%%%%%%%%%%%%%%%%%%%%%%%%%%%%%%%%%%%%%%%%%%%%%%%%%%%%%%%%%%%%%%%%%%%%%%%%%%%%%%%%%%%%%%%%%%%%%%%%%%%%%%%%%%%%%%%%%%%											Redução de Artefatos Metálicos	  												 %
%%%%%%%%%%%%%%%%%%%%%%%%%%%%%%%%%%%%%%%%%%%%%%%%%%%%%%%%%%%%%%%%%%%%%%%%%%%%%%%%%%%%%%%%%%%%%%%%%%%%%%%%%%%%%%%%%%%%%%%%%%%%%%
%\section{Redução de Artefatos de Alta Atenuação} \label{ReduçãodeArtefatosdeAltaAtenuação}
%
%Artefatos inter-planos de objetos com alta atenuação são motivos de pesquisa em \acs{DBT} \cite{hu2008image,levakhina2013weighted,borges2017metal}. Como foi abordado na seção \ref{ProjeçãoeRetroprojeção3D}, a simples retroprojeção calcula as coordenadas de $(X,Y)$ para cada nível das fatias em $Z$ para todas as projeções (\ref{fig:imgCap3wBP2}). 	
%
%\begin{figure}[H]
%	\centering
%	
%	\caption{Ilustração da (a) projeção de objetos em diferentes níveis, seguido da (b) retroprojeção dos mesmos, (c) mostrando como os objetos se dispõem nos planos retroprojetados em um mesmo nível.}
%	
%	\subfloat[Projeção]{\includegraphics[scale=0.6]{imgs/cap3/BpPonderado/wBP1.png}\label{fig:imgCap3wBP1}}
%	\hfil
%	\subfloat[Retroprojeção]{\includegraphics[scale=0.6]{imgs/cap3/BpPonderado/wBP2.png}\label{fig:imgCap3wBP2}}
%	\hfil
%	\subfloat[Foco no processo de retroprojeção]{\includegraphics[scale=0.6]{imgs/cap3/BpPonderado/wBP3.png}\label{fig:imgCap3wBP3}}
%		
%	\legend{Fonte: \citeonline[p. 3]{levakhina2013weighted}.}
%	\label{fig:imgCap3wBPIlustracao}
%\end{figure}
%
%A ocorrência desse tipo de artefato é inevitável devido ao fato da técnica de tomossíntese abranger uma estreita faixa de ângulo junto a um número limitado de projeções \cite{hu2008image}. 
%
%O entendimento desse acontecimento fica mais claro a partir da Figura \ref{fig:imgCap3wBP3}, onde no plano em evidência o círculo vermelho aparece três vezes de forma borrada devido a esse objeto não estar localizado naquele plano. Já o triângulo fica em evidência pois o mesmo está localizado no plano de interesse. É interessante notar que a interferência do círculo se dá pelo aparecimento de cópias de acordo com o número de projeções realizadas e essas cópias se localizam na direção em que cada projeção foi adquirida.
%
%De acordo com \citeonline[p. 4]{levakhina2013weighted}, o princípio do borramento se dá pelo fato de que na retroprojeção dos dados, as estruturas que estão contidas no plano de interesse se coincidem e mutuamente contribuem para aparecerem em foco, e.g. voxel $x_{1}$ na Figura \ref{fig:imgCap3wBP3}, já as estruturas que não pertencem àquele plano não se coincidem e então geram cópias borradas de suas estruturas, e.g. voxel $x_{2}$ na Figura \ref{fig:imgCap3wBP3}.
%
%Como mencionado anteriormente, normalmente a combinação matemática (Figura \ref{fig:imgCap3FluxogramaBP}) utilizada para os planos retroprojetados é uma simples soma ou média. Segundo \citeonline[]{borges2017metal}, uma solução para a diminuição dos artefatos de alta atenuação é a realização de uma média ponderada entre os planos retroprojetados.   
%
%No trabalho de \citeonline[]{borges2017metal} foi encontrado um \textit{patch} como sendo o referência entre os planos e a distância euclidiana entre o escolhido e os outros foi então calculada. Baseado nessa distância, foi estipulado um peso para os \textit{patches} de acordo com a função logística \eqref{eq:eqCap3wBPLogistic}. Após a estipulação dos pesos, foi realizado a média ponderada entre os \textit{pixels} centrais daqueles \textit{patches}. A Figura \ref{fig:imgCap3wBPPesos} ilustra o gráfico resultado da aplicação da função logística para o cálculo dos pesos em função da distância, variando o valor da constante M de inclinação da curva.
%
%\begin{equation}
%w_{x}(d_{x,y}) = 1 - \dfrac{1}{1 + e^{\frac{4,6}{M}(M-2d_{x,y})}} 
%\label{eq:eqCap3wBPLogistic}
%\end{equation}
%
%\begin{figure}[H]
%	\caption{Gráfico da atribuição dos pesos em relação a distância.}
%	\begin{center}		
%	\includegraphics[scale=0.6]{imgs/cap3/wBPPesos.pdf}
%	\end{center}
%	\legend{Fonte: do autor, 2019.}
%	\label{fig:imgCap3wBPPesos}
%\end{figure} 
%



	
