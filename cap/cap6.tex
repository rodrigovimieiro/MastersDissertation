\chapter[Conclusões Preliminares]{Conclusões Preliminares}\label{Capitulo6}

Com o desenvolvimento do presente trabalho até o momento é possível concluir que a técnica de tomossíntese digital da mama possui um enorme campo para o desenvolvimento de pesquisa científica especialmente no ramo da reconstrução de imagens. 

Dentre os tipos de algoritmos estudados, fica evidente que a retroprojeção filtrada (\acs{FBP}) ainda é a mais utilizada por equipamentos comerciais, porém os métodos iterativos são promissores devido a possibilidade de agregação de modelamentos matemáticos de ruído, conhecimentos \textit{a priori}, fácil adaptação a geometria e com os avanços do poder computacional em geral. Grandes empresas já vêm apostando em métodos iterativos para seus respectivos equipamentos.

A aplicação desses métodos foi realizada nesse trabalho, porém alguns algoritmos não obtiveram resultados satisfatórios como era esperado, talvez por falta de pré ou pós processamento. A aplicação do método iterativo estatístico com restrição através da utilização de Campo Aleatório Markoviano não-local não foi possível ainda devido a complexidade teórica envolvida no contexto e problemas na implementação prática das técnicas, mas deve ser objeto de estudo no futuro.

Ainda não estão definidos com exatidão os parâmetros utilizados na geometria dos equipamentos. Faixa de angulação, número de projeções, tipo de movimentação do tubo e angulação do detector, são campos de pesquisas que devem ser explorado a fim de buscar os benefícios da escolha de cada configuração. Para isso é necessário uma plataforma aberta com implementações de algoritmos de reconstrução que possibilite a pesquisa para toda a comunidade acadêmica, que é o objetivo principal desse trabalho.  

Um grande empecilho encontrado no desenvolvimento da pesquisa nessa área é a escassez de recursos de propriedade intelectual como implementações de técnicas e algoritmos. Isso se deve ao fato do ramo ser novo e da proteção existente dentre as grandes empresas envolvidas.       


