\chapter[Trabalhos Futuros]{Trabalhos Futuros}\label{Capitulo8}

Como trabalhos futuros, deve-se levar em consideração o aprimoramento dos algoritmos de reconstrução já desenvolvidos. Outros métodos também podem ser incorporados, tais como os algoritmos iterativos com conhecimentos \textit{a priori} para regularização. É também necessário o estudo de técnicas de reconstrução específicas para equipamentos de tomografia com ângulo limitado.

A extensão da \textit{toolbox} para outras linguagens de programação, como por exemplo Python, é também necessária tendo em vista o aumento da utilização dessa linguagem e pelo fato da mesma ser aberta.

Os métodos de projeção e retroprojeção podem ser acelerados usando técnicas computacionais paralelas em \acs{GPU}, assim como a implementação de projetores do estado da arte.  

Em relação ao modelamento do ruído pós-reconstrução, aplicações na redução do ruído no domínio da reconstrução podem ser avaliadas e comparadas com aquelas aplicadas no domínio das projeções. Além disso, abordagens híbridas podem ser empregadas para alcançar melhores resultados. A análise do ruído também pode ser avaliada em diferentes alturas do volume, para medir a influência do processo de reconstrução no ruído em diferentes fatias de imagem. Por fim, a correlação anisotrópica criada pelo processo de reconstrução pode ser investigada e aplicada nos modelos matemáticos.
