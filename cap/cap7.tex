\chapter[Próximas Etapas e Cronograma]{Próximas Etapas e Cronograma}\label{Capitulo7}

As próximas etapas a serem executadas levam em consideração o aprimoramento dos métodos de reconstrução já desenvolvidos, a aplicação de geometrias de equipamentos comerciais e a implementação do método iterativo estatístico junto com os conhecimentos \textit{a priori} advindos da teoria de Campo Aleatório Markoviano não-local.

A extensão da \textit{toolbox} para outras linguagens de programação, como por exemplo Python, é também necessária tendo em vista o grande aumento na sua utilização e por ser uma linguagem aberta.

Uma outra etapa estabelecida é a implementação de novas técnicas para a redução de artefatos com alta atenuação, bem como o aprimoramento da técnica já implementada no trabalho de \citeonline[]{borges2017metal}. A investigação do não funcionamento da implementação também é necessária. 

Por fim, a publicação de artigos científicos é de extrema importância, assim como a disponibilização de todos os códigos em uma plataforma aberta para a comunidade científica com o intuito de difusão do conhecimento e ampliação de pesquisas na área. 

\colorbox{pink}{Elias: Utilizar métodos específicos para ângulo limitado!}
\colorbox{pink}{Elias: FISTA para otimizar!}
\colorbox{pink}{Alexandre: Compressive sensing reconstrução de sinal usando min norma L0}
\colorbox{pink}{Alexandre: Restrição Tikonov Miller}
\colorbox{pink}{Alexandre: EM-MPM}


\includepdf{docs/Cronograma.pdf}
