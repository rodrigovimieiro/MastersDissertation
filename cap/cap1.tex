\chapter[Introdução]{Introdução}\label{Introdução}


%%%%%%%%%%%%%%%%%%%%%%%%%%%%%%%%%%%%%%%%%%%%%%%%%%%%%%%%%%%%%%%%%%%%%%%%%%%%%%%%%%%%%%%%%%%%%%%%%%%%%%%%%%%%%%%%%%%%%%%%%%%%%%												 	 Motivação    															%
%%%%%%%%%%%%%%%%%%%%%%%%%%%%%%%%%%%%%%%%%%%%%%%%%%%%%%%%%%%%%%%%%%%%%%%%%%%%%%%%%%%%%%%%%%%%%%%%%%%%%%%%%%%%%%%%%%%%%%%%%%%%%
\section{Motivação}
Segundo a \ac{OMS}, o câncer é uma preocupação crescente da saúde pública no âmbito mundial e requer um aumento de atenção, priorização e financiamento. Ainda, segundo a organização, o câncer é a segunda maior causa de morte em todo o mundo, com crescimento de novos casos de 14,1 milhões em 2012 para 21,6 milhões projetados em 2030. A grande preocupação está nos países em desenvolvimento, nos quais esses números crescem mais rapidamente e o gasto estimado anualmente é de 1,16 trilhão de dólares \cite{oms}.

De acordo com os dados do \ac{INCA}, no Brasil, a estimativa indica a ocorrência de 600 mil novos casos para o biênio 2018-2019. Em mulheres, o tipo mais frequente será o de mama\footnote{ Com exceção do câncer de pele não melanoma.} com previsão de 60 mil casos, sendo esse número 21\% do total para o sexo feminino \cite{inca}.

Grandes esforços são necessários para minimizar essas ocorrências. Dentre as ações recomendadas para a diminuição dos casos em geral, está o diagnóstico precoce, o qual deve ser acessível à toda população. O fornecimento de capacitação para as forças de trabalho e o aperfeiçoamento de dados para o auxílio de tomada de decisão é também de suma importância \cite{oms}. 

O câncer de mama possui uma maior ocorrência na população feminina, excetuando-se os casos de câncer de pele não melanoma, conforme as estatísticas do \ac{INCA}. No ano de 2018 foram estimados aproximadamente 627 mil óbitos no mundo, representando a mais elevada causa de morte por câncer em mulheres \cite{oms}.

Dentre as estratégias para o controle e detecção estão o rastreamento e o diagnóstico precoce. A mamografia é o método preconizado para o rastreamento, porém, outras técnicas são utilizadas como: o autoexame e o exame clínico das mamas, o uso de equipamentos de ressonância nuclear magnética, ultrassonografia, termografia e tomossíntese \cite{inca}. A utilização da mamografia como exame para o rastreamento reduz em aproximadamente 20\% a taxa de mortalidade. Essa técnica consiste em examinar mulheres em determinada faixa etária para identificar o câncer de mama antes mesmo de quaisquer sintomas. O exame com fins de rastreamento é indicado para mulheres na faixa etária de 50 a 69 anos em locais com boas condições no sistema de saúde \cite{oms}.

Apesar de ser amplamente usada para rastreamento, o exame de mamografia possui algumas limitações. A maior e mais conhecida, citada na literatura, é a sobreposição de tecidos, que obscurece possíveis lesões, dentre as quais podem haver aquelas malignas, induzindo um diagnóstico errado \cite{vedantham2015digital}. 

A tomossíntese digital da mama (\textit{Digital Breast Tomosynthesis} - \acs{DBT}) é uma técnica tomográfica de ângulo limitado, desenvolvida para minimizar os problemas relacionados à sobreposição de tecidos da mamografia \acs{2D}. Nesse método, múltiplas projeções de raios X da mama são adquiridas em diferentes ângulos, enquanto o tubo se move em uma trajetória fixa pré-definida. Ao final do exame, as imagens radiográficas são processadas para a reconstrução pseudo-\acs{3D} do volume da mama \cite{vedantham2015digital}.  

Diversos métodos de reconstrução de imagem vêm sendo estudados e comparados \cite{wu2004comparison,zhang2006comparative}. No entanto, a elaboração desses algoritmos para tomossíntese mamária é um grande desafio, uma vez que há um limitado número de projeções que são adquiridas com baixas doses de radiação \cite{wu2004comparison}. A geometria de aquisição varia para cada equipamento de \acs{DBT} \cite{vedantham2015digital}. Dessa forma, não há um consenso sobre qual é o número de projeções ideal, ou o melhor ângulo de aquisição, ou ainda qual o algoritmo de reconstrução que deve ser utilizado para o caso da \acs{DBT} \cite{sechopoulos2009optimization}. As técnicas de reconstrução são comumente divididas, de uma maneira geral, em duas categorias: os métodos analíticos e os iterativos. Dentre estes, pode-se citar os algoritmos de retroprojeção (\textit{Back-projection} - \acs{BP}), retroprojeção filtrada (\textit{Filtered Back-projection} - \acs{FBP}) e os iterativos, sendo o método de \acs{FBP} o mais comum em tomossíntese \cite{michell2018role}.  

Algoritmos de reconstrução iterativa (\textit{Iterative Reconstruction} - \acs{IR}) demandam um alto custo computacional. Devido a esse fato, historicamente, os equipamentos comerciais de tomografia por raios X não utilizam esses métodos. Todavia com o avanço do poder computacional, os métodos iterativos vêm sendo  amplamente utilizados por diversos grupos de pesquisas e fabricantes \cite{wu2003tomographic}. Sistemas comerciais de tomossíntese mamária como, por exemplo os fabricados pela Siemens\footnote{\url{www.healthcare.siemens.com}} e pela Hologic\footnote{\url{www.hologic.com}} utilizam \acs{FBP}, no entanto, algoritmos iterativos já são utilizados pelos equipamentos de \acs{DBT} da \ac{GE}\footnote{\url{www.gehealthcare.com}}. 

Dentre os objetivos dos diversos métodos de reconstrução, a redução de artefatos nas imagens reconstruídas pode ser destacada \cite{hu2008image}. Devido a utilização de um feixe cônico e uma baixa amostragem no domínio da frequência dos sistemas de tomossíntese, os algoritmos de \acs{BP} e \acs{FBP}, quando aplicados, introduzem artefatos de objetos fora do foco em imagens que estão em foco no processo de reconstrução \cite[]{levakhina2013weighted, borges2017metal}. Já os métodos iterativos são capazes de agregar no processo de reconstrução o modelamento físico do sistema em geral, além da possibilidade de incluir restrições à convergência do método a partir de conhecimentos \textit{a priori} \cite{xu2015statistical,levakhina2013weighted}.

Esses modelamentos e restrições vêm sendo incorporados nos processos iterativos de tomossíntese com as recentes pesquisas a fim de aprimorar a qualidade das imagens. Informações \textit{a priori} como a similaridade dos \textit{pixels} e a morfologia da imagem foram incorporadas na reconstrução no trabalho de \citeonline{xu2015statistical}. A redução de artefatos de alta atenuação foi incorporada em um algoritmo iterativo, demonstrada no trabalho de \citeonline[]{levakhina2013weighted}. A correlação do ruído e o respectivo borramento nos detectores indiretos foram estudados por \citeonline[]{zheng2018detector}. Foram dados então os primeiros passos para a criação das reconstruções iterativas baseadas em modelamentos (\textit{Model-Based Iterative Reconstruction }- \acs{MBIR}) aplicadas à \acs{DBT}, que já são extensivamente estudadas para o equipamento de tomografia computadorizada (\textit{Computed Tomography} - \acs{CT}).    

De fato, o ruído tem uma importância muito grande em relação a qualidade das imagens de mamografia que são fornecidas para os radiologistas, tendo em vista a detecção do câncer de mama \cite{haus2000screen,huda2003experimental,ruschin2007dose, saunders2007does, samei2007digital, mackenzie2016relationship}. É extremamente relevante conhecer as fontes dessas perturbações para então modelá-las matematicamente e incorporá-las nos métodos computacionais para os devidos propósitos. Remoção de ruído, redução da dose de radiação e função de restrição para algoritmos de reconstrução iterativa são algumas das aplicações nas quais o entendimento dessas perturbações é importante \cite{wu2012dose, romualdo2013mammographic,borges2016method,borges2017pipeline,borges2017method,mackenzie2017characterisation,zheng2018detector}.

%%%%%%%%%%%%%%%%%%%%%%%%%%%%%%%%%%%%%%%%%%%%%%%%%%%%%%%%%%%%%%%%%%%%%%%%%%%%%%%%%%%%%%%%%%%%%%%%%%%%%%%%%%%%%%%%%%%%%%%%%%%%%%												  	Objetivos	    														%
%%%%%%%%%%%%%%%%%%%%%%%%%%%%%%%%%%%%%%%%%%%%%%%%%%%%%%%%%%%%%%%%%%%%%%%%%%%%%%%%%%%%%%%%%%%%%%%%%%%%%%%%%%%%%%%%%%%%%%%%%%%%%

\section{Objetivos}

Esse trabalho tem como objetivo apresentar uma ferramenta de reconstrução de imagens para a tomossíntese da mama e fazer um estudo do comportamento do sinal e do ruído nas imagens reconstruídas.

Especificamente, o desenvolvimento dessa ferramenta visa ampliar as pesquisas em reconstrução tomográfica da mama, visto que a mesma é disponibilizada de maneira \textit{online}\footnote{\url{www.github.com/LAVI-USP/DBT-Reconstruction}} e gratuita. É desejável que o \textit{software} seja capaz de implementar os principais algoritmos de reconstrução de imagens para \acs{DBT}, bem como qualquer geometria de aquisição. Tanto os métodos analíticos quanto os iterativos são incluídos no \textit{software} proposto.

Em trabalhos anteriores, nosso grupo propôs um modelo de função afim para descrever a variância do ruído nas projeções de \acs{DBT}. Neste modelo, o ruído eletrônico é descrito por uma distribuição Gaussiana aditiva e o ruído quântico é descrito por um modelo Gaussiano dependente do sinal com ganho quântico dependente espacialmente \cite{borges2017method,borges2018restoration}. 

Agora, uns dos objetivos desse trabalho é realizar um estudo preliminar do ruído em fatias reconstruídas de \acs{DBT}, a fim de avaliar a adequação desse modelo para imagens pós-reconstrução. Esse modelamento tem a finalidade de fornecer uma base matemática para aplicações que envolvam a remoção de ruído em ambos os domínios, redução de dose de radiação e restrições para algoritmos iterativos de reconstrução.

%%%%%%%%%%%%%%%%%%%%%%%%%%%%%%%%%%%%%%%%%%%%%%%%%%%%%%%%%%%%%%%%%%%%%%%%%%%%%%%%%%%%%%%%%%%%%%%%%%%%%%%%%%%%%%%%%%%%%%%%%%%%%%										     Organização da Monografia														%
%%%%%%%%%%%%%%%%%%%%%%%%%%%%%%%%%%%%%%%%%%%%%%%%%%%%%%%%%%%%%%%%%%%%%%%%%%%%%%%%%%%%%%%%%%%%%%%%%%%%%%%%%%%%%%%%%%%%%%%%%%%%%

\section{Organização da Monografia}

O trabalho é dividido em seu âmbito geral em duas partes, sendo a primeira a respeito do desenvolvimento e validação do \textit{software} de reconstrução e a segunda parte relacionada com a análise e medição do sinal e do ruído em imagens reconstruídas de tomossíntese.

Mais especificamente, o documento é divido em 9 capítulos que visam demostrar todo o trabalho desenvolvido, bem como apresentar os conceitos teóricos fundamentais. 

Primeiramente, no \autoref{Capitulo2}, é feita uma revisão do conteúdo de mamografia digital \acs{2D} e da técnica de tomossíntese da mama. São abordados os fatos históricos, bem como o funcionamento geral de ambas modalidades de exame. São detalhados a física por trás dos equipamentos de \acs{DBT} assim como seus parâmetros geométricos.

Já no \autoref{Capitulo3} é exposto a teoria básica sobre reconstrução, demonstrando os princípios para \acs{2D} e para \acs{3D}. São discutidos os métodos de reconstrução analítica e iterativa. 

No \autoref{Capitulo4}, são expostos os conceitos de ruído em imagens digitais, bem como em exames de mamografia. Formulações matemáticas são feitas para a modelagem do ruído nas projeções e também pós-reconstrução.

Os materiais e métodos utilizados na confecção das duas partes mencionadas são apresentados no \autoref{Capitulo5}  do documento. 

Os resultados obtidos a partir dos experimentos e das avaliações realizadas, bem como as respectivas discussões são descritas no \autoref{Capitulo6}.

Então, no \autoref{Capitulo7}, é feita a conclusão do trabalho e no \autoref{Capitulo8} são apresentados os trabalhos futuros. 

Por fim, no \autoref{Capitulo9} são apontadas as publicações geradas a partir do desenvolvimento desse trabalho.



  

  
